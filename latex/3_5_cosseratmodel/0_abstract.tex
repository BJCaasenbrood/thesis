%!TEX root = ../../thesis.tex
%\graphicspath{{3_chapters/3_chapter/img/}}
\chapterabstract{
This chapter addresses some of the limitations of the previously presented ``\textit{Piecewise-Constant-Curvature}'' (PCC) model in Chapter \ref{chap:PCC}, in particular when infinite-dimensionality of the soft robot's deformable body is assumed. The continuous dynamics of the soft robot are modeled through the differential geometry of Cosserat beams. Using a finite-dimensional truncation, the infinite-dimensional system can be written as a reduced-order port-Hamiltonian (pH) model that preserves desirable passivity conditions. Contrary to prior PCC models, a wide variety of spatial discretizations can be explored that better respect the spatial continuity and continuum mechanics present in soft robots. Given the pH model structure, we derive a stabilizing controller rooted in energy-based techniques and exploits the passivity of the soft robotic system. The model-based controller derived from the reduced Cosserat model produces a local minimizer of closed-loop potential energy that steers end-effector towards a desired configuration. Furthermore, we propose an alternative geodesic setpoint description inspired by the octopus' morphology. We show that stereotypical motions, such as bending wave propagation, can also be achieved via our proposed control scheme with a striking resemblance to its biological counterpart. The effectiveness of the controller is demonstrated through extensive simulations of various soft manipulators.}

% \vspace{-9mm}
% \chapterabstract{
% The infinite-dimensional nature of soft robots has emphasized the difficulty in modeling and control, leading to the classic trade-off between precision and speed. In the last decade, two modeling strategies have dominated the field: \textit{(i)} Finite-Element-Method (FEM) models and \textit{(ii)} soft beam models (\eg, Cosserat models). While FEM enables highly accurate deformations, Lagrangian-based beam models allow for faster computation and ease of controllers design akin to rigid robotics. In this chapter, we propose a mixture between the two modeling approaches by extracting geometric modal information of FEM simulation data. Our approach leads to fast, accurate, and generic low-dimensional models that encode the geometric features and the elasticity of the original soft body into a new strain functional basis -- we call a Geometry-Informed Variable Strain (GIVS) basis. Robustness of the technique is investigated for several systems. Also the nonlinearities with increasing actuation frequency, under environmental contact modeled by signed distance functions, and multi-input pneumatic actuation. Furthermore, we also provide a qualitative comparison between existing strategies, demonstrating that our approach can improve accuracy and speed compared traditional techniques. Also, experiments are performed to highlight the model's transferability to reality.}
