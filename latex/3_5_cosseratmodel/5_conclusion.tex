%!TEX root = ../../thesis.tex
\vspace{-5mm}
\section{Conclusion}
\label{sec:C3:conclusion}
Due to their intrinsic compliance of soft robots, they allow for complex morphological motions that mimic animals in nature. Achieving similar performance to biology highlights the need for more accurate dynamic models and control strategies. In this chapter, we provided a modeling framework for Cosserat beams that leads to a finite-dimensional system in a port-Hamiltonian structure. By exploiting the passivity, an energy-shaping controller was proposed that ensures the closed-loop Hamiltonian is minimal at the desired set-point. The numerical model was developed for a several bio-inspired soft robot (octopus' tentacle and elephant's trunk) with distributed control inputs. The key challenges here regarding both the model as the controller are their ability to capture the hyper-flexibility, deal with inherent under-actuation, and exploit its hyper-redundancy to achieve its control task. Given appropriate controller gains, the model-based controller yields smooth convergence of the soft robot's end-effector while accounting for under-actuation. It was shown that by tuning the controller gains, the intrinsic stiffness of the soft body can be adapted. Furthermore, we investigate the possibility of the proposed approach to mimic nature. By introducing an energy-shaping control law that minimizes the geodesic distance at any point on the curve, we obtain closed-loop soft robots that exhibit stereotypical bending propagation closely resembling those found in octopi. This result further highlights the importance of model-based control in the pursuit of developing robotic systems with bio-like features that explore intrinsic structural flexibility to achieve task efficiency with possibly minimal energy expenditure. Finally, we investigate the affects of reduced-order controllers applied to high-dimensional system. We introduce a quasi-static continuum deformation solver that estimates the reduced controller states such that an energy-shaping technique can be applied. We observe that, irrespective of the truncation, all controllers converge towards a close vicinity of the desired configuration -- even for extremely low-order models. This finding has the potential to reduce the significance of basis span in the effectiveness of model-based control for soft robots. It suggests that soft robots may possess inherent redundancy that can compensate for modeling imperfections. This sheds light on why PCC models, despite their oversimplification, exhibit good efficacy in controlling soft robots. 

There are, however, a few limitations to our approach. The strain parametrization of functional basis does not account for the geometry of the soft robot, meaning some systems require many spatial modes to \textit{accurately} represent the true continuum dynamics. Second, regarding implementation, measuring these spatial modes in an experimental setting is difficult, and future research is required to find a suitable `\textit{soft sensing}' technique that  \textit{(i)} has limited impact on the dynamics, and \textit{(ii)} accounts for the continuity of the elastic body. A possible solution might be the optimal placement of a network of distributed localized sensors, \eg, strain gauges or IMUs. Furthermore, the proposed controller is only suited for set-point regulation or slow-varying references. Exploring (fast)-dynamic control objectives will likely require more research. In particular, controllers that suppress natural resonances of continuum elastic body under fast motion. One could argue that this perhaps fights against the natural dynamics of the soft robot, yet such oscillations might be able to be explored for locomotion or soft manipulators throwing objects rather than traditional pick-and-place strategies.

Given these limitations, future work will focus on \textit{(i)} validating the controller experimentally, and \textit{(ii)} constructing a bases using ``\emph{snapshot decomposition method}'' of FEM-driven data. In particular, the latter two goals could be interesting to explore. 
Both advantages in FEM and Cosserat models, being accurate continuum deformations and computational efficiency; leading to a modeling strategy for ``\textit{optimal}'' finite-dimensional state projection with insightful structures of the passive and active joints.
