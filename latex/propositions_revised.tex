\documentclass[a5paper]{article}

\renewcommand{\normalsize}{\fontsize{9}{9}\selectfont}
\newcommand{\largesize}{\fontsize{10}{9}\selectfont}
\usepackage{geometry}
\usepackage{xcolor}

\geometry{
    a5paper,
    left=2cm,
    right=2cm,
    top=1.0cm,
    bottom=1.0cm,
    twoside=false
}


\pagestyle{empty}

\begin{document}

\begin{center}
\largesize \textbf{Propositions} \\[0.45em]
\normalsize accompanying the dissertation \\[0.45em]
{\textbf{\largesize Design, Modeling, and Control Strategies for Soft Robots}}
\end{center}

% \medskip
\normalsize
\vspace{-1mm}
% \bigskip

\begin{enumerate}
  \setlength\itemsep{0.65em}
  % \item The success of biological systems cannot be solely attributed to their morphology. Instead, it is the interplay between physical structure and coordinated motor control that enables their remarkable functionality and adaptability. The same should apply to soft robots.
  \item The remarkable adaptability and functionality of biological organisms are not merely a consequence of their morphology but emerge from the interplay between structure and coordinated motor control. To mimic such features, soft robots must be inspired from similar principles.
  \begin{flushright}
  \vspace{-2mm}
  \textit{This thesis}
  \vspace{-1mm}
  \end{flushright}

  \item Regarding both the topology and material as unknown properties in the design optimization for soft robotics is essential in discovering new mechanical configurations, which may not be found when either is assumed to be known. 
  \begin{flushright}
  \vspace{-5mm}
  \textit{This thesis, Chapter 3}
  \vspace{-1mm}
  \end{flushright}

  \item Model-based controllers for soft robots must incorporate feedback laws that aim to preserve the intrinsic compliance of soft materials as to enable efficient "\textit{life-like}" motion.
  % \item When designing controllers for soft robots to perform "\textit{life-like}" motions, it is essential to consider that feedback alters intrinsic compliance of the soft materials.
  \begin{flushright}
  \vspace{-5mm}
  \textit{This thesis, Chapter 5}
  \vspace{-1mm}
  \end{flushright}
  
  % \item Low-dimensional controllers can serve as an effective control strategy for soft robots without a significant compromise in performance.
  % Soft robotic control systems should prioritize low-dimensional control that target principal modes of operation, balancing simplicity, efficiency, and performance.
  % Low-dimensional control are more effective than high-dimensional systems for the majority of practical soft robotic applications, as they provide sufficient control capabilities while offering advantages in simplicity, efficiency, robustness, and stability
  % \item Balanced low-dimensional controllers in soft robotics can offer better performance and reliability for most practical applications by capitalizing on simplicity, efficiency, and robustness, that in some cases may outperform their high-dimensional counterpart.
  % \item Though soft robots possess potentially infinite degrees of freedom, the use of reduced-order controllers targeting principle modes of operation proves to be an effective approach by enlisting inherent redundancy to cut through complexity.
  % \item Soft robots, with their seemingly infinite degrees of freedom, can be adeptly managed by reduced-order controllers targeting principle modes of operation that enlist inherent redundancy to cut through complexity.

  % \item Though soft robots possess seemingly infinite degrees of freedom, an implementation of reduced-order control that focus on a few primary modes presents, leveraging on the robots' natural redundancy to cut through the complexity.
  \item Soft robots, with their seemingly infinite degrees of freedom, can be adeptly managed by reduced-order controllers that target only a few principle modes of operation.%, focusing on the robots' inherent redundancy to cut through the complexity.
  \begin{flushright}
  \vspace{-5mm}
  \textit{\small This thesis, Chapter 5 and 6}
  \vspace{-1mm}
  \end{flushright}

  \item Any publication on soft robots would benefit from answering the ``\textit{Why soft?}'' question, instead of adopting a ``\textit{Soft for soft's sake}'' philosophy. 
  \begin{flushright}
  \vspace{-2mm}
  \textit{Inspired by ``Hard questions for soft robotics'' \\ written by Hawkes et al., Science (2021)}
  \vspace{-1mm}
  \end{flushright}  

  % \item More soft roboticists should consider exploring also other principles of actuation aside from fluidics that simulate advancements in bandwidth, scalability, and autonomy.
  % \item Although fluidics has been the cornerstone of progress in soft robotics, other actuation principles should be consider for substantial gains in bandwidth, scalability, and autonomy.
  \item While fluidic actuation has been instrumental in the development of soft robotics, its dominance may limit the progress in areas such as bandwidth, scalability, and autonomy.

  % \item A common and precise language must be set at the start of an interdisciplinary scientific collaboration.
  % \item An interdisciplinary collaboration requires a common language first, and then common goals.
  % \item The convergence of various scientific disciplines within a collaboration relies heavily on the synchronization of terminology.
  \item For scientific disciplines to truly converge in a collaborative context, a common terminology must be prioritized first.

  % \item In order to enhance transparency, reduce the likelihood of fraud, and promote replicability in an era of data-driven research, academic publishers should adopt code and data sharing policies that facilitate automated workflows for easy replication.
  \item Academic publishers, aiming to improve transparency, reduce the potential for fraud, and encourage replicability within the realm of data-driven research, ought to establish code and data-sharing policies that are designed to enable automated workflows.

  % \item The increasing effort to present scientific findings in academic literature as incredible ironically makes them more \textit{'incredible'}.
  % \item Incredible sensational claims, like the phrasing ``\textit{The first}'' proposed in academic literature or its covering media, will ironically render such works \textit{`incredible'} to the public eye.
  % \item Sensationalization of academic literature or its media coverage through incredible claims, like the phrase ``\textit{The first}'', can unintentionally make it \textit{'incredible'} to the public eye
  % \item The increase sensationalization of science media shows that for academic writing liveliness of exposition is as necessary as lucidity.
  \item The increasing sensationalization of the media regarding science underlines that in academic writing, ``\textit{the liveliness of the exposition is as necessary as lucidity}.''
  \begin{flushright}
    \vspace{-5mm}
    \textit{Inspired by ``Popularisation and Sensationalism''\\ in Nature's newsletter (1924)}
    \vspace{-1mm}
  \end{flushright}  

  \item Most of our frustrations with others stems from the fact that we have an innate inability to understand other people fully.

  \item Personal achievements bear little fruits at the cost of friendship.

  % \item Unlike sit-standing desks, regularly refilling your mug can be a simple yet effective means to counteract a lifestyle of minimal physical movement.

\end{enumerate}

\vspace{2mm}
\vfill
\begin{flushright}
Brandon Caasenbrood \\[0.15em]
Eindhoven, January 2024
\end{flushright} 

\begin{center}
\textcolor{black}{
% \scriptsize These propositions are regarded as opposable and defendable, and have been approved as such by the promotors prof.\ dr.\ H. Nijmeijer and dr.\ A.Y. Pogromsky.
% }
\end{center}

\clearpage

\begin{center}
\largesize \textbf{Stellingen} \\[0.45em]
\normalsize behorende bij het proefschrift \\[0.45em]
{\textbf{\largesize Design, Modeling, and Control Strategies for Soft Robots}}
\end{center}

\begin{enumerate}
\setlength\itemsep{0.23em}
\item Het opmerkelijke aanpassingsvermogen en functionaliteit van biologische organismen worden niet alleen bepaald door hun morfologie, maar ontstaan uit de interactie tussen structuur en de gecoördineerde motorische controle. Om deze eigenschappen te evenaren, moet het ontwerp van soft robots geïnspireerd worden door vergelijkbare principes.
\begin{flushright}
  \vspace{-5mm}
  \textit{Dit proefschrift}
  \vspace{-1mm}
  \end{flushright}

\item Het beschouwen van zowel de topologie als het materiaal als onbekende eigenschappen bij het optimaliseren van het ontwerp voor soft robots, is essentieel om nieuwe mechanische configuraties te ontdekken die mogelijk niet worden gevonden wanneer een van beide als bekend wordt verondersteld.
\begin{flushright}
  \vspace{-5mm}
  \textit{Dit proefschrift, Hoofdstuk 3}
  \vspace{-1mm}
  \end{flushright}

% \item Gereduceerde numerieke modellen bieden een betere bruikbaarheid ten opzichte van gesloten-vorm modellen bij het beschrijven van de continue dynamiek van zachte robots.
% \begin{flushright}
% \vspace{-5mm}
% \textit{Dit proefschrift, Hoofdstuk 4}
% \vspace{-1mm}
% \end{flushright}

% \item Modelgebaseerde regeling kan worden gebruikt om \textit{'levenlijke'} dynamische gedragingen in zachte manipulatoren na te bootsen, maar alleen als de feedback de fysieke zachtheid niet significant verstoort.

\item Model-gebaseerde regelaars voor soft robots moeten feedback bevatten die de intrinsieke flexibiliteit van de zachte materialen behouden, om efficiënte `\textit{levensechte}' bewegingen te realiseren.
\begin{flushright}
\vspace{-2mm}
\textit{Dit proefschrift, Hoofdstuk 5}
\vspace{-1mm}
\end{flushright}

%\item Hoewel soft robots schijnbaar oneindig veel vrijheidsgraden bezitten, is de implementatie van laag-dimensionale regelaars die zich richten op primaire werkingsmodi een effectieve strategie, gebruikmakend van de natuurlijke redundantie van de robot om de complexiteit te doorbreken.
\item Soft robots, met hun oneindig vele vrijheidsgraden, kunnen bekwaam worden aangestuurd door gereduceerde regelaars die zich enkel richten op een paar principiële werkingsmodi.
\begin{flushright}
\vspace{-2mm}
\textit{Dit proefschrift, Hoofdstuk 5 en 6}
\vspace{-1mm}
\end{flushright}  

\item Elke publicatie over soft robots zou baat hebben bij het beantwoorden van de vraag ``\textit{Waarom soft?}'', in plaats van het aannemen van een ``\textit{soft omwille van soft}'' filosofie.
%  
\begin{flushright}
\vspace{-2mm}
\textit{Gebaseerd op "Hard questions for soft robotics" \\ geschreven door Hawkes et al., Science (2021)}
\vspace{-1mm}
\end{flushright}  

% \item Meer soft robotici zouden moeten overwegen om andere actuatie principes te verkennen dan alleen fluïdica, die snellere vooruitgang mogelijk maken op het gebied van bandbreedte, schaalbaarheid en autonomie.
\item Hoewel fluïdica fundamenteel is geweest voor de ontwikkeling van soft robotica, kan de overheersende focus hierop de progressie op vlakken als bandbreedte, schaalbaarheid en autonomie hinderen.

% \item Een gebrek aan een gemeenschappelijke en precieze taal binnen een interdisciplinaire wetenschappelijke samenwerking ondermijnt het leggen van een stabiele basis die nodig is voor succes.
% \item Zonder een gemeenschappelijke en nauwkeurige taal binnen een interdisciplinaire wetenschappelijke samenwerking, ontbreekt het aan een stabiel fundament die noodzakelijk is voor succesvol teamwork.
\item Om wetenschappelijke disciplines echt samen te laten komen in een samenwerkingsverband, moet eerst een gemeenschappelijk terminologie ontwikkeld worden.

\item Academische uitgevers die transparantie willen verbeteren, de kans op fraude willen verkleinen en replicatie binnen data-gedreven onderzoek willen stimuleren, zouden beleid voor het delen van code en data moeten opstellen dat geautomatiseerde werkprocessen mogelijk maakt.

% \item De toenemende inspanning om wetenschappelijke bevindingen in de academische literatuur als ongelofelijk te presenteren, maakt ze ironisch genoeg meer ongeloofwaardig.
% \item Ongelooflijke sensationele beweringen, bijvoorbeeld de formulering ``\textit{De eerste}'', geponeerd in academische literatuur of media, zullen ironisch genoeg dergelijke werken ongeloofwaardig maken in publieke ogen.
\item De toenemende sensatiezucht van de media over de wetenschap benadrukt dat bij het academisch schrijven, ``\textit{de levendigheid van de expositie net zo noodzakelijk is als de helderheid.}''
\begin{flushright}
  \vspace{-1.75mm}
  % \textit{Gebaseerd op "The First Room-Temperature Ambient-Pressure Superconductor" geschreven door Lee et al., Arxiv (2023)}
  \textit{Gebaseerd op ``Popularisation and Sensationalism''\\ in een nieuwsartikel van Nature (1924)}
  \vspace{-1mm}
  \end{flushright}  

\item De meeste van onze frustraties met anderen komen voort uit het feit dat we van nature niet in staat zijn om anderen volledig te begrijpen.

% \item Regelmatig je mok bijvullen is effectiever dan een zit-sta bureau.

\item Persoonlijke prestaties ten koste van vriendschap werpen weinig vruchten af.

\end{enumerate}

\vfill
\begin{flushright}
Brandon Caasenbrood \\[0.15em]
Eindhoven, January 2024
\end{flushright} 

%% %% Apart from the name and title of the supervisor, the following text is
%% %% dictated by the promotieregelement.
\begin{center}
% \textcolor{black}{
% \scriptsize Deze stellingen worden opponeerbaar en verdedigbaar geacht en zijn als zodanig goedgekeurd door de promotoren prof.\ dr.\ H. Nijmeijer en dr.\ A.Y. Pogromsky.
% }
\end{center}

%% }

\end{document}


