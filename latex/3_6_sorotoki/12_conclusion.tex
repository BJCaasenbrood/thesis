
\section{Conclusion and future work}
\label{sec:C5:conclusion}
This paper introduces the \texttt{Sorotoki} is an open-source toolkit in {MATLAB} that provides a comprehensive and modular programming environment to address the complex interdependencies associated with the design and control of soft robots. The toolkit consists of seven Object-Oriented classes that are designed to work together to solve a wide range of soft robotic problems. We hope the versatility and flexibility of \texttt{Sorotoki} make it a valuable resource for many researchers and practitioners in the field of soft robotics. The toolkit has been demonstrated to be effective through a range of case studies encompassing a broad range of issues within the field of soft robotics, including inverse design of soft actuators, passive and active soft locomotion, object manipulation with soft grippers, meta-materials, model reduction, model-based control of soft robots, and shape estimation. A unique aspect of the software package is that it does not follow the traditional linear relationship between the complexity of soft robotics systems and the length of code needed to represent them. Instead, complex system behavior can be effectively modeled using a minimal number of lines of code. The \texttt{Sorotoki} software package is particularly notable for its ability to succinctly represent complex soft robotics systems. Despite the intricacy of soft robotics, the accompanying software package is highly effective in modeling complex system behavior with minimal code, making it accessible to individuals with limited programming knowledge. Furthermore, the toolkit provides access to four open-hardware soft robotic systems that can be easily fabricated using commercially available 3D printers.

Nevertheless, the framework presents opportunities for improvement and can be expanded to make it more comprehensive, extensive, and to achieve faster simulation times. This is particularly significant in the field of control, where computational performance is a recurring challenge. The software package \texttt{Sorotoki} addresses this challenge by converting {MATLAB} code to its equivalent in \texttt{c++} using the compilation method. Real-time computation is attainable, yet the produced \code{.mex} functions frequently face limitations due to inadequate memory allocation and a lack of parallel processing. This can result to subpar performance. An alternative solution could be to explore other languages with better computational performance, such as \texttt{Python} (via \texttt{pybind}) and \texttt{Julia}. Especially those with a stronger (open-access) community. %An early \texttt{Julia}-port of \texttt{Sorotoki} is found at \texttt{github.com/BJCaasenbrood/Sorotoki.jl}, which has already shown performance improvements.

On a concluding note, we would like to emphasize that any form of contribution is greatly appreciated. Our framework is envisioned as a collaborative effort between and for the soft robotics community. Researchers who are interested in contributing are welcome to reach out to the authors. Different forms of contributions are possible and more information can be found on the online repository. Additionally, while we strive to maintain the framework to be as error-free as possible, in case of exceptions during execution or any concerns regarding accuracy, please do not hesitate to contact us or through the issue tracker on the repository.


% On a concluding note, we would like to emphasize that any form of contribution is greatly appreciated. Our framework is envisioned as a collaborative effort between and for the soft robotics community. Researchers who are interested in contributing are welcome to reach out to the authors. Different forms of contributions are possible and more information can be found on the online repository. Additionally, while we strive to maintain the framework to be as error-free as possible, in case of exceptions during execution or any concerns regarding accuracy, please do not hesitate to contact us or through the issue tracker on the repository.
