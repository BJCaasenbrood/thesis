%!TEX root = ../../thesis.tex
\vspace{-3mm}
\section{Outline of the thesis}
\label{sec:intro:outline}
This thesis discusses the design, modeling and control of soft robotic systems. Including this introductory materials, the thesis consists of seven chapters. 

Chapter \ref{chap: history} presents a historical overview of the field of soft robotics, complementing the earlier introduction. Chapter \ref{chap: design} introduces the design algorithm for soft actuators aiming to address the inverse design problem. The chapter begins with a brief introduction to continuum mechanics applied to three-dimensional deformation of hyperelastic materials, followed by its numerical implementation using finite elements. Subsequently, numerical optimization procedures are introduced to solve the inverse design within the context of fluidic soft actuation. Chapter \ref{chap:PCC} follows with the second objective of the thesis, which is modeling for control. Instead of volumetric soft robotic models, lower-dimensional soft beam models are introduced, specifically tailored for fast and accurate model-based controllers. The chapter primarily focuses on PCC soft beam models. Chapter \ref{chap:BeyondPCC} addresses the limitations of the PCC model and extends upon it. The chapter formulates a finite-dimensional port-Hamiltonian modeling approach for soft beams, utilizing spatial shape functions to discretize the modal flexibilities of the soft robot. From here, energy-shaping controllers are introduced, enabling shape control for underactuated soft robots. Chapter \ref{chap:Sorotoki} presents the culmination of all theoretical material presented in the thesis, consolidating it into a concise and user-friendly toolkit called \texttt{Sorotoki}. The chapter provides an overview of the included programming tools for the design, modeling, and control of soft robots. It also introduces the Data-driven Variable Strain (DVS) approach, which leads to efficient low-dimensional models. The chapter further showcases a variety of soft robotic systems fabricated using Additive Manufacturing technology, including designs derived from topology optimization techniques. Finally, Chapter \ref{chap: conclusions} concludes the main body of the thesis by summarizing the research deliverables of the previous chapters and providing a list of recommendations that could shape future work.
\\

\textbf{Note to the reader:} Chapters 3-6 are derived from research articles that have been either published or are intended for submission, allowing for autonomous reading. The beginning of each chapter includes a reference to the respective research paper.
% This thesis, however, provide some minor modifications to these works, either in the context of mathematical or material improvement, or connection between other chapters. An overview of these modifications can be found as a supplementary chapter at the end of the thesis in the chapter named \textit{Modifications}. 

\vfill
% \afterpage{
%   \hspace{-10mm}
% \begin{tabular}{m{0.5cm}|c|c|}
%   & Chapter 1 & \textbf{Introduction} \\ \hline
%   \;\trot{Part \RNum{1}\;} & Chapter 2  & \Centerstack{\\ \textbf{Contribution \RNum{1}}: \\ Development of efficient algorithms, applicable to the general \\ design of soft actuators, that solve the inverse design problem: \\Given  a desired morphological motion, what is the according \\ (soft) material distribution within the  design domain to realize \\the desired joint motion or displacement?\\ \\ \textbf{Contribution \RNum{2}}:} \\ \hline
%   \;\trot{Part \RNum{2}\;} & \Centerstack{Chapter 3\\ Chapter 4\\ Chapter 5} &  \\ \hline
% \end{tabular}
% }

% \\ of soft actuators, that solve the inverse design problem: Given \\ a desired morphological motion, what is the \\according (soft) material distribution withing the \\design domain to realize the desired joint motion?