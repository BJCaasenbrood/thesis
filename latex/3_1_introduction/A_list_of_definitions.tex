\chapter{List of definitions}
In this appendix, we present a compilation of the definitions and terminologies proposed in Chapter \ref{chap: introduction} and Chapter \ref{chap:Sorotoki}. It should be noted that these definitions may diverge from those commonly used in literature. Nevertheless, we consider them essential to address the multi-disciplinary perspective inherent in the scientific field of soft robotics. \\

\noindent \textbf{Soft materials} are a class of homogeneous materials with a Young's modulus (\ie, the modulus of elasticity) typically lower than $E \le 10^9$ \si{\pascal}. Following, the word 'soft' or 'softness' refers to the collection of mechanical properties that are often associated with these low moduli materials. \\

%
\noindent \textbf{Soft robotics} is a subclass of robotics with purposefully designed compliant actuators embedded into their soft body whose goal is to enable the robot control over its ability to perform bio-inspired behavior. \\

\noindent \textbf{Soft actuators} are controllable flexible actuation units of the constitute soft robot that through external stimuli are responsible for natural motion within the system or a change in its compliance. \\

\noindent \textbf{(Proprioceptive) soft sensors} are flexible measurements units embedded into the soft robotic body that through external stimuli measure the (local) changes of the system. Softness here implies that the sensor minimally alters the global mechanical behavior of the robot. \\

\noindent \textbf{Hyper-elasticity} is a branch of continuum mechanics that deals with the behavior of materials that exhibit large elastic deformations. In hyper-elastic materials, the stress-strain relationship is nonlinear and the material response depends on the deformation history. Hyper-elasticity is an important concept in the field of soft robotics, as many soft robots are made from hyper-elastic materials and their behavior must be accurately modeled for effective design and control.  \\

\noindent \textbf{Hyper-redundancy}  refers to a design principle in robotics where the number of Degrees of Freedom (DoF) in a robot is significantly greater than what is required for a given task. This redundancy allows the robot to perform the same task in multiple ways, providing increased flexibility and adaptability.
\\

\noindent \textbf{Under-actuation} is a term used to describe a system whose Degrees of Freedom (DoF) cannot be matched (anywhere) by the number of inputs. In other words, there exist joint configurations, or whole sets of configurations, where the robot cannot realize all joint acceleration via the input alone. Although continuum-bodied motions enable increased flexibility and adaptability in the soft robot's movements, but also presents challenges in controlling the system since not all setpoint are globally reachable. Since only a finite number of actuators can be embedded in a continuum, but possess infinite DoF, soft robots are intrinsically underactuated systems \\

\noindent \textbf{Under-sensed} refers to the situation where the robot's sensors are not able to capture all the relevant information about its environment or its own state. This can lead to incomplete or inaccurate perception of the world, which in turn can affect the robot's ability to perform effectively. Undersensing can occur due to limitations in the sensing hardware, environmental factors such as occlusions or lighting conditions, or the complexity of the task itself. Similar to under-actuation, only a finite number of sensors can be embedded in a continuum, and thus soft robots are intrinsically undersensed. \\