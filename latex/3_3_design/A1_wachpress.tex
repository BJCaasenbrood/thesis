\chapter{Appendices to Chapter 3}
\vspace{-10mm}
\section{Wachspress shape functions for polygons}
\label{app:C3:wachpress}
The finite element method involves the reduction of an infinite-dimensional problem to a finite-dimensional one through the utilization of local spatial interpolation. This is achieved by assigning states to the exterior of the finite element and employing specialized basis functions to locally interpolate the continuum. These shape functions must be tailored towards the chosen elemental discretization, and thus researchers often rely on common element types, including triangular elements (\eg, \texttt{Tet3} or \texttt{Tet6}) and quadrilateral elements (\eg, \texttt{Quad8}) \cite{Holzapfel2002,Kim2018,Bendsoe2003,Renaud2011,Duriez2013}. In the present work, we investigate polygonal elements called \texttt{PolyN}, resulting in a finite element mesh composed of heterogeneous mesh elements.

To find a suitable shape function for these \texttt{PolyN} elements, we explore a technique called `\textit{barycentric coordinates}' \cite{Floater2014Jun,Floater2015May}. Let $P_k \subset \R^2$ be a convex polygon viewed as an open set, which is spanned by a set of vertices $\{\vB_i\}_{i=1}^k$ (anti-clockwise ordering) where $k$ is the degree of the polygon. Then, any function $N_i: P_k \to \R$ is considered a barycentric coordinate if for all $\sigmaB \in P_k$, it holds that 
%
\begin{itemize}
    \vspace{-2mm}
    \setlength\itemsep{0.1em}
    \item It is strictly positive, \ie, $N_i(\sigmaB) > 0$, 
    \item Its sum is equal to one, \ie, $\sum_{i=1}^k N_i(\sigmaB) = 1$, 
    \item And $\sum_{i=1}^k N_i(\sigmaB) \vB_i \ge \sigmaB$, 
\end{itemize}
A common function that satisfies such conditions for any $k \ge 3$ is called the Wachspress shape function given as follows:
%
\begin{equation}
N^k_i(\sigmaB)  = \frac{w_i(\sigmaB)}{\sum_{j=1}^{k} w_j(\sigmaB)},
\label{eq:app:wachpress}
\end{equation}
%
with $w_i$ the shape interpolation weights given by 
%
\begin{equation}
w_i(\sigmaB)  = \frac{A(\vB_{i-1}, \vB_{i}, \vB_{i+1})}{A(\vB_{i-1}, \vB_{i}, \sigma) A(\vB_{i}, \vB_{i+1}, \sigma)},
\label{eq:app:wachpress_weight}
\end{equation}
%
where $A$ is the signed area spanned by its three arguments (see Figure \ref{app:fig:C3:voronoimeshExplain}). By convention, we use $\vB_{n_p + 1} = \vB_1$ in \eqref{eq:app:wachpress_weight}. From a numerical standpoint, the computation of this area can be achieved in an efficient manner by evaluating \cite{Talischi2012Mar}:
\begin{figure}
\centering
\vspace{-5mm}
\input{./fig/fig_C3_voronoi_explain.pdf_tex}
\caption{(a) Illustration of the triangular areas $A_i(\sigmaB) := A(\vB_{i-1}, \vB_i, \sigmaB)$ used in the computation of \eqref{eq:app:wachpress_weight}. The figure is adopted from Talischi et al. \cite{Talischi2012Mar,Talischi2012}. (b) Example of an isoparametric mapping from natural to global coordinates.\label{app:fig:C3:voronoimeshExplain}}
\vspace{-3mm}
\end{figure}
%
\begin{equation}
A(\vB_{i}, \vB_{i+1}, \sigmaB) = \frac{1}{2} \left| \begin{matrix}
\sigma_1 & \sigma_2 & 1 \\
v_{1,i-1} & v_{2,i-1} & 1 \\
v_{1,i} & v_{2,i} & 1 \\
\end{matrix} \right|.
\end{equation}
%
Since we consider regular polygonal elements where $\vB_i = \left(\cos\left(\frac{2\pi i}{k}\right), \sin\left(\frac{2\pi i}{k}\right)\right)$, it follows that $A(\vB_{i-1}, \vB_i, \vB_{i+1})$ is constant for all $i$ in the summation for \eqref{eq:app:wachpress}, and can therefore be factored out of the expression. By adopting the notation $A_i(\sigmaB) := A(\vB_{i-1}, \vB_i, \sigmaB)$ and $\alpha_i = \left[A_i A_{i+1}\right]^{-1}$, we can simplify the expression for the Wachspress shape function in \eqref{eq:app:wachpress} further to
%
\begin{equation}
N^k_i= \frac{\left[A_{i} A_{i+1}\right]\inv}{\sum_{j=1}^k \left[A_{j} A_{j+1}\right]\inv} = \frac{\alpha_i }{\sum_{j=1}^k \alpha_i},
\label{eq:app:C3:N}
\end{equation}
%
\begin{rmk}
Let it be clear that \eqref{eq:app:C3:N} is expressed in the natural coordinates $\sigmaB \in P_k$ as opposed to $\XB \in \mathcal{B}_0$. The idea of using reference interpolation functions $N^k_i$ is due to convenience because it is unnecessary to build different shape functions for the elements with an identical topology. The only difference between elements is the mapping from global to natural coordinates. So, given such isoparametric elements, their coordinate transformation and displacement can be expressed by
%
\begin{equation*}
\XB(\sigmaB) \approx \sum^{k}_{i=1} N_{i}^k(\sigmaB) \XB_i^e;  \quad \quad
\dB(\sigmaB,T) \approx \sum^{k}_{i=1} N_{i}^k(\sigmaB) \xB_i^e, 
\end{equation*}
%  
where $\XB_{i}^{e} = (X^e_1,\, X^e_2)^\top$ and $\x_{i}^{e} = (x^e_1,\, x^e_2)^\top$ are the nodal coordinates and the nodal displacement vector related to the $i$-th node of the element $\mathcal{V}_e$, respectively. We provided an illustration of the isoparametric mapping using a polygonal element of degree $k=5$ in Figure \ref{app:fig:C3:voronoimeshExplain}.
\end{rmk}

Finally, the element-wise derivative of the Wachspress shape function with respect to the natural coordinates is given by \cite{Talischi2012Mar}:
%
\begin{equation}
\frac{\p N^k_i}{\p \sigma_j}= \frac{1}{\sum_{q=1}^k \alpha_q} \left(\frac{\p \alpha_i}{\p \sigma_j} - N_{i}^k \sum^{k}_{q=1} \frac{\p \alpha_q}{\p \sigma_j} \right).
\label{eq:app:C3:dNds}
\end{equation}
%
Again, note that the \eqref{eq:app:C3:dNds} is given in terms of the natural coordinates $\sigmaB = [\sigma_1, \sigma_2] \in P_k$. Yet, we are interested in the partial derivatives with respect to the global coordinates, for instance, the deformation gradient $\FB = \boldsymbol{I} + \nabla_0 \dB$, or the Piola stress tensor $\ten{S} = \frac{\partial \Psi}{\partial \ten{E}}$. Let us introduce the Jacobian matrix of the isoparametric mapping as
%
\begin{align}
\JB & = \frac{\p \XB}{\p \sigmaB} = \sum_{i=1}^k \XB_i^{e} \left(\frac{\p N^k_i}{\p \sigma_1}, \; \frac{\p N^k_i}{\p \sigma_2} \right).
\label{eq:app:C3:jacobian}
\end{align}
%
Then, using \eqref{eq:app:C3:jacobian}, it can be shown that the spatial derivatives of shape functions are related by the Jacobian matrix as $\p N/\p \XB = (\p N/\p \sigmaB) \JB\inv$ \cite{Kim2018}. In this context, spatial derivatives can be calculated using an isoparametric reference element and then converted to the global coordinate frame. This approach allows for faster computation of the global system matrices as each element in the mesh follows the same numerical procedure and thus enables parallel computation.