\chapter{Hardware information of VEAB pressure control board}
\blankfootnote{This chapter is based on: B.J. Caasenbrood, A.Y. Pogromsky, and H. Nijmijer. \textit{Sorotoki: a MATLAB toolkit for design, modeling, and control of soft robots} (\textit{in preparation for submission}). }

\vspace{-10mm}
\chapterabstract{
In this chapter, we present \texttt{Sorotoki}, an open-source toolkit in MATLAB\textsuperscript{\scriptsize\textregistered} that offers a comprehensive suite of tools for the design, modeling, and control of soft robots. The complexity involved in researching and building soft robots often stems from the interconnectedness of design and control aspects, which are rarely addressed together as a unified problem. To address the complex interdependencies in soft robotics, the \texttt{Sorotoki} toolkit provides a comprehensive and modular programming environment composed of seven Object-Oriented classes. These classes are designed to work together to solve a wide range of soft robotic problems, offering versatility and flexibility for its users. We provide a comprehensive overview of the \textit{Sorotoki} software architecture to highlight its usage and capacities. The details and interconnections of each module are thoroughly described, collectively explaining how to gradually introduce modeling complexity for various soft robotic scenarios. The effectiveness of \textit{Sorotoki} is also demonstrated through a range of case studies, including novel problem scenarios and established works widely recognized in the soft robotics community. These case studies cover a broad range of research problems in the field of soft robotics, including: inverse design of soft actuators, passive and active soft locomotion, object manipulation with soft grippers, meta-materials, model reduction, model-based control of soft robots, and shape estimation. Additionally, the toolkit provides access to four open-hardware soft robotic systems that can be fabricated using commercially available 3D printers. 
}


\begin{figure}[t]
    \centering
    \vspace{-5mm}
    \input{./pdf/thesis-figure-F-veabboard.pdf_tex}
    \caption{}
    \vspace{-3mm}
    \end{figure}