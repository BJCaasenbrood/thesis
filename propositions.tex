\documentclass[a5paper]{article}

\renewcommand{\normalsize}{\fontsize{9}{9}\selectfont}
\newcommand{\largesize}{\fontsize{10}{9}\selectfont}
\usepackage{geometry}
\usepackage{xcolor}

\geometry{
    a5paper,
    left=2cm,
    right=2cm,
    top=1.0cm,
    bottom=1.0cm,
    twoside=false
}


\pagestyle{empty}

\begin{document}

\begin{center}
\largesize \textbf{Propositions} \\[0.45em]
\normalsize accompanying the dissertation \\[0.45em]
{\textbf{\largesize Design, Modeling, and Control Strategies for Soft Robots}}
\end{center}

% \medskip
\normalsize
\vspace{-1mm}
% \bigskip

\begin{enumerate}
  \setlength\itemsep{0.75em}

  % \item The success of biological systems cannot be solely attributed to their morphology. Instead, it is the interplay between physical structure and coordinated motor control that enables their remarkable functionality and adaptability. The same should apply to soft robots.
  \item The remarkable adaptability and functionality of biological organisms are not merely a consequence of their morphology but emerge from the interplay between structure and coordinated motor control. To mimic such features, soft robots must be inspired from similar principles.
  \begin{flushright}
  \vspace{-2mm}
  \textit{This thesis}
  \vspace{-1mm}
  \end{flushright}

  % \item Exploring a broad material spectrum during topology design optimization are detrimental to finding solutions that can otherwise not be observed when fixing material a-priori during a design procedure. (this thesis, Chapter 3)
  % \item Considering material as a spectrum in optimization-driven design for soft robots is key to finding solutions that are otherwise hidden when viewing the material as a given.    
  
  % \item Regarding topology and material as unknown properties in the design optimization for soft robotics is key for uncovering new mechanical solutions that may not be found when either is assumed to be known. 
  \item Regarding both the topology and material as unknown properties in the design optimization process for soft robots is essential in discovering new mechanical configurations, which may not be found when either is assumed to be known. 
  \begin{flushright}
  \vspace{-5mm}
  \textit{This thesis, Chapter 3}
  \vspace{-1mm}
  \end{flushright}

  % \item Reduced-order numerical models provide superior usability over closed-form models when describing the continuum dynamics of soft robots.
  % \begin{flushright}
  %  \vspace{-2mm}
  % \textit{This thesis, Chapter 4}
  % \vspace{-1mm}
  % \end{flushright}

  % \item Model-based control can be explored to mimic "\textit{life-like}" dynamic behaviors in soft manipulators, only when the model's limitation are properly considered and the control gains carefully selected.   
  % \item Model-based control can be explored to mimic '\textit{life-like}' dynamic behaviors in soft manipulators, only when the control feedback does not significantly disrupt the physical softness.
  % \item Achieving ``\textit{life-like}'' dynamics in soft manipulators through model-based control requires methods that preserve the robot's inherent softness without saturating it with rigid control feedback.
  % By effectively incorporating model-based control, soft robots can achieve lifelike dynamic behaviors while preserving their inherent physical softness, through thoughtful consi of control feedback.
  \item To achieve ``\textit{life-like}'' dynamic trajectories in soft robots through model-based control, it is crucial to preserve the intrinsic softness which
  \begin{flushright}
  \vspace{-2mm}
  \textit{This thesis, Chapter 5}
  \vspace{-1mm}
  \end{flushright}
  
  % \item Modal truncation, or the choice of modal basis, play a small role in the end-effector precision of model-based controlled soft robots, but are important for full-body precision. 
  % \begin{flushright}
  % \vspace{-5mm}
  % \textit{this thesis, Chapter 5}
  % \vspace{-1mm}
  % \end{flushright}
  
  % \item Reduced-order beam models exploring data-driven techniques offer a viable solution to the speed-precision paradigm in controlling soft continuum manipulators.
  
  \item Low-dimensional controllers can serve as an effective control strategy for soft robots without a significant compromise in performance.
  \begin{flushright}
  \vspace{-2mm}
  \textit{This thesis, Chapter 5 and 6}
  \vspace{-1mm}
  \end{flushright}

  % \item The distinction between rigid and soft robotics has created barriers that hinder researchers from exploring and collaborating on topics beyond their respective scopes.

  \item Any publication on soft robots would benefit from answering the ``\textit{Why soft?}'' question, instead of adopting a ``\textit{Soft for soft's sake}'' philosophy. 
  %
  \begin{flushright}
  \vspace{-2mm}
  \textit{Inspired by ``Hard questions for soft robotics'' \\ written by Hawkes et al., Science, 2021.}
  \vspace{-1mm}
  \end{flushright}  
  
  % \item Soft roboticist should be comfortable stepping away from fluidic actuation, and instead explore actuation principles that enable for higher bandwidths and scalability.
  \item More soft roboticists should consider exploring other actuation principles aside from fluidics that allow for faster advancements in bandwidth, scalability, and autonomy."

  % \item A common and precise language lays foundation to the success of an interdisciplinary scientific collaboration.
  \item A lack of a common and precise language within an interdisciplinary scientific collaboration undermines the establishment of a strong foundation necessary for success.

  % \item In an era of data-driven research, adopting open data repositories and standardized code-sharing should be the norm to enhance transparency in scientific publishing.

  % \item Embracing open-access workflows and code-sharing platforms by scientific publishers for data analysis, numerical simulation, and figure generation will enhance transparency and reproducibility in academia, thereby reducing the likelihood of fraud.

  % \item In an era of data-driven research, publishers should promote code sharing policies that enable computing workflows for data analysis, simulation, and figure generation as to improve the transparency and reproducibility. Thereby, also reducing the likelihood of academic fraud.

  % \item In an era of data-driven research, academic publishers should advocate code and data sharing policies that simplify replication through automated workflows. This, in turn, does not only enhance transparency but also reduces the likelihood of fraud.
  \item In order to enhance transparency, reduce the likelihood of fraud, and promote replicability in an era of data-driven research, academic publishers should adopt code and data sharing policies that facilitate automated workflows for easy replication.

  % Despite concerns of misuse and complexity, advocating for code sharing policies that promote replication through automated workflows in academic publishing could enhance transparency and reduce fraud.

  % \item Like calculators and dictionaries, local Large Language Models (LLMs) are ultimately tools and should one day be embraced by schools, universities, industries and thus indirectly academia.

  \item The increasing effort to present scientific findings in academic literature as incredible ironically makes them more \textit{'incredible'}.

  % \item Large Language Models, like ChatGPT, should be considered as dangerous as teaching assistants (TAs) with minor subject knowledge.
  
  \item Most of our frustrations with others stems from the fact that we have an innate inability to understand other people fully.

  % \item Frequently refilling your mug is more effective than a sit-standing desk.
  \item Frequently emptying your mug, as opposed to sit-standing desks, offers an effective solution to the sedentary work environment.

  % \item An effective talk should emphasize on the theme by having a strong introduction. 

  % \item Friendship bears fruits sweeter than personal achievement.
\end{enumerate}

\vspace{3mm}
\vfill
\begin{flushright}
Brandon Caasenbrood \\[0.15em]
Eindhoven, January 2024
\end{flushright} 

\begin{center}
\textcolor{black}{
% \scriptsize These propositions are regarded as opposable and defendable, and have been approved as such by the promotor prof.\ dr.\ H. Nijmeijer, and copromotor dr.\ A.Y. Pogromsky.
}
\end{center}

\clearpage

\begin{center}
\largesize \textbf{Stellingen} \\[0.45em]
\normalsize behorende bij het proefschrift \\[0.45em]
{\textbf{\largesize Design, Modeling, and Control Strategies for Soft Robots}}
\end{center}

\begin{enumerate}
\item De opmerkelijke aanpassingsvermogen en functionaliteit van biologische organismen worden niet alleen bepaald door hun vorm, maar ontstaan uit de interactie tussen structuur en gecoördineerde motorische controle. Om deze kenmerken na te bootsen, moeten zachte robots geïnspireerd worden door vergelijkbare principes.
\begin{flushright}
  \vspace{-5mm}
  \textit{Dit proefschrift}
  \vspace{-1mm}
  \end{flushright}

\item Het beschouwen van zowel de topologie als het materiaal als onbekende eigenschappen is essentieel in het optimalisatieproces voor het ontwerpen van zachte robots, om nieuwe mechanische configuraties te ontdekken die mogelijk niet worden gevonden wanneer wordt aangenomen dat een van beide bekend is.
\begin{flushright}
  \vspace{-3mm}
  \textit{Dit proefschrift, Hoofdstuk 3}
  \vspace{-1mm}
  \end{flushright}

% \item Gereduceerde numerieke modellen bieden een betere bruikbaarheid ten opzichte van gesloten-vorm modellen bij het beschrijven van de continue dynamiek van zachte robots.
% \begin{flushright}
% \vspace{-5mm}
% \textit{Dit proefschrift, Hoofdstuk 4}
% \vspace{-1mm}
% \end{flushright}

\item Modelgebaseerde regeling kan worden gebruikt om \textit{'levenlijke'} dynamische gedragingen in zachte manipulatoren na te bootsen, maar alleen als de feedback de fysieke zachtheid niet significant verstoort.
\begin{flushright}
\vspace{-3mm}
\textit{Dit proefschrift, Hoofdstuk 5}
\vspace{-1mm}
\end{flushright}

\item Laag-dimensionale controllers kunnen dienen als een effectieve regelstrategie voor zachte robots zonder significante compromis in prestaties.
\begin{flushright}
\vspace{-3mm}
\textit{Dit proefschrift, Hoofdstuk 5 en 6}
\vspace{-1mm}
\end{flushright}  

\item Elke publicatie over zachte robots zou baat hebben bij het beantwoorden van de vraag "\textit{Waarom zacht?}", in plaats van het aannemen van een "\textit{Zacht omwille van zacht}" filosofie.
%  
\begin{flushright}
\vspace{-3mm}
\textit{Gebaseerd op "Hard questions for soft robotics" \\ geschreven door Hawkes et al. (2021)}
\vspace{-1mm}
\end{flushright}  

\item Soft robotici moeten zich comfortabel voelen om weg te stappen van fluidische actuatoren en in plaats daarvan actuatie principes te verkennen die hogere bandbreedtes en schaalbaarheid mogelijk maken.

\item Een gemeenschappelijke en duidelijke taal legt de basis voor het succes van interdisciplinaire wetenschappelijke samenwerking.

\item In een periode van data-gedreven onderzoek zouden wetenschappelijke uitgevers beleid moeten bevorderen voor het delen van data en code, dat replicatie vergemakkelijkt via geautomatiseerde workflows. Dit verbetert niet alleen transparantie, maar vermindert ook de kans op fraude.

\item De toenemende inspanning om wetenschappelijke bevindingen in de academische literatuur als ongelofelijk te presenteren, maakt ze ironisch genoeg meer ongeloofwaardig.

\item De meeste van onze frustraties met anderen komen voort uit het feit dat we van nature niet in staat zijn om andere mensen volledig te begrijpen.

\item Regelmatig je mok bijvullen is effectiever dan een zit-sta bureau.

% \item Vriendschap werpt zoetere vruchten af dan persoonlijke prestaties.

\end{enumerate}

\vfill
\begin{flushright}
Brandon Caasenbrood \\[0.15em]
Eindhoven, January 2024
\end{flushright} 

%% %% Apart from the name and title of the supervisor, the following text is
%% %% dictated by the promotieregelement.
%% \begin{center}
%% Deze stellingen worden opponeerbaar en verdedigbaar geacht en zijn als zodanig goedgekeurd door de promotoren prof.\ dr.\ A.\ van Deursen and dr.\ A.\ Zaidman.
%% \end{center}

%% }

\end{document}


