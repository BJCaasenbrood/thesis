\chapter[Conclusions and recommendations]{Conclusions and \\ recommendations}
\label{chap: conclusions}
\thispagestyle{empty}

\section{Conclusions}

\noindent \textbf{Design synthesis of soft actuators. } In Chapter \ref{chap:topo}, the thesis proposes a new framework for the synthesis of soft robot topologies from hyperelastic materials. Generally speaking, the design process of soft robots is complicated by material nonlinearities and numerical implementation issues. To address these challenges, we introduced a nonlinear finite element method for polygonal elements. Our approach provides a more accurate representation of the physical nature of soft robotics, including hyperelasticity and pneumatic actuation, compared to previous research. Numerical analysis indicates that our framework can produce meaningful and insightful material layouts for developing pressure-driven soft robots from soft materials. Additionally, the proposed framework can be adapted to express other actuation principles in soft robotics, such as dielectric polymers or thermal expansion with minor modifications.

\noindent \textbf{Modeling for control of soft manipulators.}

\noindent \textbf{Software developement.} This thesis has introduced \textit{Sorotoki}, an open-source toolkit in MATLAB that provides a comprehensive and modular programming environment for the design, modeling and control of soft robots. The toolkit consists of a libary of Object-Oriented classes designed to solve a wide range of problems in the field of soft robotics, including inverse design of soft actuators, passive and active soft locomotion, object manipulation with soft grippers, model reduction, model-based control of soft robots, and shape estimation. Most notably is its ability to succinctly represent complex soft robotics systems with minimal code, making it accessible to individuals with limited programming knowledge. In combinations with software, the toolkit also presents four open-hardware soft robotic systems that can be easily fabricated using commercially available 3D printers, futher enabling soft robotic technology for the community. Overall, \textit{Sorotoki} is a versatile and flexible resource that can benefit many researchers and practitioners in the field of soft robotics.

\section{Recommendations}
In the present thesis, several research challenges have been addressed. However, as research is an ongoing endeavor, unexplored areas of research persistently emerge. In this section, we propose recommendations based on the chapters within this thesis. The author's recommendations for future research are summarized in Table \ref{}, categorized as short-term, medium-term, and long-term work.

\textit{(Inverse design)} In Chapter \ref{chap:topo}, the possibility of gradient-based design for soft robots through topology optimization has been discussed. It has been demonstrated that relatively simple techniques can be utilized to identify "optimal" structures based on desired motion. However, due to computational limitations, only optimizations of planar mechanical structures have been considered. Therefore, further steps are necessary to convert the topology into an actual structure. Additionally, it should be noted that fabrication constraints have not been taken into account during optimization. However, these limitations create many possible opportunities for future research regarding generative design. 
\par Firsy, the optimization techniques presented in this thesis can be readily applied to the three-dimensional domain. It is highly recommended to extend these techniques to the aforementioned domain, as it simplifies the post-processing of the derived topological structure. It should be noted, however, that the computational complexity of finite element models increases with $\mathcal{O}(n^3)$ with $n$ the global DOF of the mesh. Given this computational increase, exploring GPU parrallel computation or cluster computation are highly encouraged for future research. Alternatively, one potential option is to consider adopting the Material Point Method (MPM). The MPM offers several advantages over FEM, including the ability to handle larger mesh distortions, and better conservation of mass and momentum. Additionally, complex geometries can be handled more eclasily without requiring remeshing. Although MPM is computationally more intensive than FEM, there exist software packages that offer massively parallel simulation on multi-GPU (Graphics Processing Unit) architectures, which have already been applied to the co-design optimization of soft robots for locomotion. 
\par Secondly, one intriguing area of research can be found in multi-material topology optimization. Soft robots often face the issue of limited force transmissibility without significant deformations in their supporting structures. Multi-material investigation offers a promising solution, where material diversity, inspired by nature, can result in a wider range of mechanical compliance, ultimately benefiting potential applications. Moreover, recent progress in multi-material printing technology has facilitated the consolidation of soft, rigid, and flexible conductive materials. This provides a prospect to incorporate skeletal function along with proprioceptive sensing capabilities that can be directly integrated into the optimization loop. Generally speaking, 

\textit{(Modeling for control)} 
\newcommand{\tabitem}{~~\llap{\textbullet}~~}
\newcolumntype{C}[1]{>{\centering\arraybackslash}m{#1}}
\begin{table}
    \centering
\begin{tabular}{|C{4cm}|l|}
    \hline
    \multirow{3}{*}{Short term (<1 year)}   & \tabitem Bananis baidg ndgidog idngd kdnfd \\ 
                                            & \tabitem Bananis baidg ndgidog idngd kdnfd \\ 
                                            & \tabitem Bananis baidg ndgidog idngd kdnfd \\ 
    \hline
  \end{tabular}
\end{table}

\section{Future outlook}
\textit{(Chicken-and-egg problem in model-based control of soft robots)} 

\textit{(Synergy between design and control)} 

\textit{(Importance of sensing and its relation towards learning)} 