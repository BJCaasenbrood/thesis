\chapter[Conclusions, recommendations, and outlook]{Conclusions, recommendations, and future outlook}
\label{chap: conclusions}
\thispagestyle{empty}
In recent years, the scientific community has shown an increasing interest in soft robotics owing to its utilization of soft materials that augment the dexterity, mobility and resilience of robots. The field has demonstrated potential for various applications in both industrial and academic settings, such as reliable and safe manipulation, adaptable gripping, and environmental exploration under uncertainties. Despite the advancements made in soft robotics, there is still a considerable gap between this field and traditional rigid robotics as well as biological systems. As highlighted in the introduction, successful implementation of embodied intelligence in soft robotics necessitates the holistic consideration of the entire embodiment of the bio-inspired system, which harmonizes design and control aspects of soft robots. This dissertation is focused on the development of new tools for enhancing the design, modeling, and control of soft robots, with the aim of improving their overall performance. In this section, we provide an overview of the key contributions of the thesis, along with recommendations for future research directions.

\section{Conclusions}
% Soft robotics as a scientific field has grown significantly, with 
% In this work, we have developed new tools for the design, modeling, and control of soft robotic systems that aim to  push performance and adaptability of said systems further than previous ad-hoc methods. As discussed in the introduction of the thesis, pushing performance in soft robotics is challenging due to the unpredicedent challenges concerning accurate prediction of soft materials under static or dynamic deformations. This ultimately leads to systems with subpar performance as design and control strategies rely on modeling as the  framework. 
% The main objective of the thesis therefore reads:

% \objective{Development and analysis of new systematic tools balanced between the design, modeling, and control of soft continuum robots, whose goal is to achieve similar performance as rigid robotics and eventually biology.}

% The conclusion of the thesis 
% We acknowledge that there are open subproblems that fall outside the scope of this thesis. Accordingly, we provide recommendations for future research avenues in the concluding section of this dissertation.
% This brings us to the first research questions (R1):

% \begin{center}
% \textit{How do we design fluidic-interacting mechanical structures made from soft materials that deform according to a user-defined morphological pattern?}
% \end{center}

\textbf{Design synthesis of soft actuators. } Chapter \ref{chap: design} of the thesis proposes a new framework for synthesis of soft robot topologies from hyperelastic materials. The optimal design of actively deformable soft structures is a complex task due to the presence of numerous nonlinearities. These nonlinearities arise from material properties and fluid-structure interactions resulting in state-dependent loads. It is crucial to consider these effects when designing such structures, as by doing so, we can ensure that the final soft topology is both efficient and effective in meeting its intended motion requirements. Furthermore, ad-hoc approaches may not suffice not only due to time constraint, or due to the increasing demand in joint complexity for soft robotic systems.

We address the challenges associated with designing fluid-driven soft structures by exploring the nonlinear Finite Element Method (FEM) using polygonal elements in conjunction with gradient-based topology optimization. The presented approach involves numerical free-form optimization to identify optimal topologies for fluidic soft actuation with predefined motion criteria set by the user. By exploring the mesh connectivity and volumetric dilation of polygonal elements, we have demonstrated that our approach provides a precise representation of fluid-structure interactions in soft robots, which is currently lacking in traditional structural optimization techniques. Through numerical analysis, we show that the optimization framework can generate meaningful material compositions that aid in the development of pressure-driven soft robots. An interesting observation is that the optimized topologies align with prevalent design often utilized in soft robotics, including the bellows and the well-known PneuNet actuator. While certain differences may be subtle, they highlight the significance of optimal design in soft robotics. For instance, our method results in concave bellow configurations that efficiently increase the actuator's stroke; or alternatively, the bellows in the PneuNet actuator adopt a tear-shaped form to replace the necessity for an inextensible layer at the base. 

%Additionally, the proposed framework of volume-based actuation can be adapted to express other actuation principles in soft robotics, such as dielectric polymers or thermal expansion with minor modifications.

% \begin{itemize}
%   \setlength\itemsep{-.1em}
%   \item {How do we derive dynamic models that offer a reasonable trade-off between the accuracy and its applicability for control?}
%   \item {Can we adopt certain control philosophy from rigid robotics to soft robotics?} 
% \end{itemize}

\textbf{Modeling for control of soft manipulators.} Chapter \ref{chap:PCC} addresses the necessity of bridging the gap between modeling and control-oriented research in soft robotics. The chapter proposes developing accurate dynamic models for continuum soft robots to explore model-based control. However, such dynamic models tailored towards continuum soft robots must maintain real-time performance for transferring model-based control to physical systems. To achieve this goal, we use a minimal set of coordinates related to the differential geometry of spatial curves to express high-dimensional continuum deformation, building upon existing PCC formulations. To improve computational efficiency further, we propose a reduced-order integration scheme for fast forward dynamic simulations. Through numerical benchmarking, we demonstrate its ability to achieve real-time performance. In this study, we introduce high-fidelity FEM data to enhance the representation of state-dependent compliance resulting from hyperelasticity and structural influence in our models. We also introduce simplified models capable of capturing viscoelastic behavior, specifically creep. Our aim is to offer a more precise depiction of soft robots' compliant nature when compared to conventional Hookean options. To qualitatively evaluate our modifications, we conduct experimental measurements that highlight a significant improvement in accuracy for static and dynamic conditions.

The resulting continuum dynamical model shows good correspondence with physical soft robots and enables real-time simulations with various degrees of motion. Due to the similarity in model structure between soft robots and rigid robots, conventional control theory can be utilized. We introduce an adaptive controller that offers excellent tracking performance, even when faced with parameters uncertainties, \eg, material parameters and tip-loads. Moreover, the controller progressively improves the accuracy of these estimations over time, subject to the condition that the unknown parameters are persistently excited. 

In Chapter \ref{chap:BeyondPCC}, the focus is on soft robots and their ability to mimic complex morphological motions found in nature. A modeling framework for Cosserat beams is presented, which leads to a finite-dimensional system in a port-Hamiltonian structure. An energy-shaping controller is proposed that ensures the closed-loop Hamiltonian is minimal at the desired set-point. The numerical model is developed for several bio-inspired soft robots, such as an octopus' tentacle and an elephant's trunk, with distributed control inputs. The key challenges are capturing hyper-flexibility, dealing with inherent under-actuation, and exploiting hyper-redundancy to achieve control tasks. The model-based controller yields smooth convergence of the soft robot's end-effector while accounting for under-actuation. It was shown that by tuning the controller gains, the intrinsic stiffness of the soft body can be adapted, resulting in significantly different quasi-static joint solutions of the set-point problem. However, there are limitations to this approach, such as the strain parametrization of functional basis not accounting for the geometry of the soft robot, making it difficult to accurately represent true continuum dynamics. Additionally, measuring spatial modes in an experimental setting is challenging, and future research is required to find a suitable 'soft sensing' technique. The proposed controller is suited for set-point regulation or slow-varying references. Overall, the mobility of the Cosserat model paired with energy-based control has a close resemblance to biological motion.
% \begin{itemize}
%   \item{To enable better designs and controllers for soft robots, how can we bridge the interdisciplinary aspects intrinsic to this field?}
% \end{itemize}

\textbf{Software development for the soft robotics community.} In Chapter \ref{chap:Sorotoki}, the thesis presents \texttt{Sorotoki}, an MATLAB toolkit that provides a comprehensive and modular programming environment for the design, modeling, and control of soft robots; and  serves as a collection of all prior material presented in the dissertation. The toolkit consists of a library of Object-Oriented classes designed to solve a wide range of problems in the field of soft robotics, including inverse design of soft actuators, passive and active soft locomotion, object manipulation with soft grippers, model reduction, model-based control of soft robots, and shape estimation. Most notably is its ability to succinctly represent complex soft robotics systems with minimal code, makes it highly accessible towards individuals with limited programming knowledge. The chapter also presents a stable modeling platform for soft robotic systems that accommodates various shape functions designed for unique joint mobility imposed by the soft robot's design. Previous works in modeling literature often select these functions without proper consideration, such as using polynomial bases (Chapter \ref{chap:BeyondPCC}). This thesis proposes a geometric modal decomposition approach that extracts geometric strain modes from higher-fidelity FEM simulations to construct generic low-dimensional models that accurately encode the features and elasticity of soft bodies. This approach introduces a new strain parametrization called the "\textit{Data-driven Variable Strain}" (DVS) basis. Finally, the toolkit also presents four open-hardware soft robotic systems that can be easily fabricated using commercially available 3D printers, further enabling soft robotic technology for the community. Overall, \texttt{Sorotoki} is a versatile and flexible resource that can benefit many researchers and practitioners in the field of soft robotics.

\section[Recommendations]{Recommendations}
\textbf{Three-dimensional inverse design:}  Chapter \ref{chap: design} presents the possibility for gradient-based design for soft robots through topology optimization. It has been demonstrated that relatively simple techniques can be utilized to identify locally "\textit{optimal}" structures based on desired morphology. However, due to computational limitations, only optimizations of planar mechanical structures have been considered. Therefore, further steps are necessary to convert the topology into a functional soft actuator structures. One recommendation for improving the optimization techniques discussed in this thesis is to expand their application to three-dimensional domains. This extension is strongly advised, as it facilitates the post-processing of the resulting topological structure, which can be more easily transferred to a 3D-printing platform. It is important to acknowledge that the computational complexity of finite element models experiences an increase of $\mathcal{O}(n^3)$ as $n$, the global degrees of freedom (DOF) of the mesh, grows. Given this significant computational burden, it is highly recommended that future research explores the use of GPU parallel computation or cluster computation. Alternatively, another potential avenue to consider is the adoption of the Material Point Method (MPM). The MPM presents certain advantages over FEM, such as its ability to manage larger mesh distortions and better conservation of mass and momentum. Furthermore, complex geometries can be handled more explicitly without necessitating remeshing. Despite the fact that MPM is computationally more demanding than FEM, there exist software packages providing parallel computation on multi-GPU (Graphics Processing Unit) architectures, which have already been implemented in the co-design optimization of soft robots for locomotion. 

\textbf{Extension towards active and multi-materials optimization:} Another intriguing enhancement to Chapter \ref{chap: design} can be found in the area of multi-material topology optimization. One common issue with soft robots is that their supporting structures experience limited force transmissibility without significant deformations. Conducting research on multi-materials offers a promising solution, where the use of diverse materials, inspired for instance by natural muscular-skeletal systems, can result in a broader range of mechanical operation, eventually benefiting potential applications such as locomotion. Additionally, advancements in multi-material printing technology have made it easier to consolidate soft, rigid, and flexible conductive materials, or even mixtures thereof. This presents an opportunity to incorporate skeletal function along with proprioceptive sensing capabilities that can be directly integrated into the optimization loop.

\textbf{Free-floating soft bodies and application towards locomotion} In Chapter \ref{chap:PCC} and \ref{chap:BeyondPCC}, we present dynamic modeling strategies for soft continuum manipulators, but the approach is also applicable to various other slender soft structures. We showed in Chapter \ref{chap:Sorotoki} that the approach is applicable to soft grippers, \eg, seminal works like \cite{Sinatra2019Aug,Suzumori1992}, and the bending fingers of a soft robot hand. 

A significant constraint of the modeling methodology proposed in this thesis is that the soft beam's structural foundation yield zero velocity and acceleration inherently. Consequently, this limitation (partially) restricts applicability of the model to soft robotic systems designed for terrestrial or aquatic locomotion; or soft grippers on fast moving mobile platforms (\eg, drones) used in agricultural operations, \eg, tomato picking. Other potential applications include academic study-cases that relate to control theory, such as swing-up control for soft pendulums. This modeling restriction arises from the model's projection into the desirable Lagrangian formulation by means of premultiplication of the geometric Jacobian. This effectively eliminates the boundary conditions at the foundation (\ie, the constraint wrenches) in the model and thereby ensuring that the states evolution always satisfy the appropriate kinematic constraints. 

In order to address this issue, it is necessary to incorporate the rigid-body dynamics of the center of mass (CoM) of the soft structure under general motion depicted by an inertial coordinate frame affixed at a certain location on the body. Although the selection of this inertial coordinate frame can be chosen arbitrary, opting for the base of the beam constitutes a reasonable choice inline with prior chapters. Again, we can represent this as an element of $\textrm{SE}(3)$. The equation of motions can be written using a Newton-Euler formulation, akin to the approach presented in Chapter \ref{chap:BeyondPCC}. Consequently, this necessitates a separate solver that calculates the inertia tensor, which accounts for alterations in the center of mass during deformation, and the resultant inertial wrench responsible for the "rigid"-body motion. Theoretically, it is possible to streamline the spatial integration of Lagrangian matrices and rigid-body inertial dynamics into a singular Matrix Differential Equation (MDE), with the objective of accelerating the computational procedure. However, numerical efficiency of such approaches must be explored in future research to even enable online model-based control.

\textbf{Robustness of modal discretization for soft beam models, a Chicken-Egg problem.} Although the prospect of expanding into the realm of locomotion is fascinating, it reveals a challenge that is not immediately apparent. Namely, there is no guarantee that a chosen modal discretization is reflective of the spatial dynamics of the soft beams. Since in many cases robotics application the environment is highly unstructured, there are infinitely many possible force wrenches of different amplitudes and frequency that can act as an disturbance to the system. Due to our inadequate understanding of the soft robotic system interacting with the unknown environment, it is difficult to select \textit{a-priori} an appropriate strain basis. This challenge has been previously discussed in Chapter \ref{chap:BeyondPCC}, where it was referred to as an Chicken-and-Egg (CaE) problem, primarily from the viewpoint of the closed-loop controller. The dynamic feedback loop may trigger the excitation of strain modes that have not been accounted for in the reduced-order model-based control, hence it might paradoxically introduce its own errors. This thus creates a circular reasoning of cause-and-effect when implementing model-based control. Furthermore, ascertaining whether the closed-loop system can excited such modes through rigorous mathematical analysis is challenging.

A recommendation for future research is therefore an in-depth investigation of the effects of ... As described in Chapter \ref{chap:Sorotoki}, the DVS strain method has been shown to effectively capture unmodeled disturbances as point-contacts on multiple locations of a soft body. This finding may suggest that these models harbor a degree of robustness disturbances. Our hypothesis is that soft body may acts as spatial filters that structurally penalize high-order modes, thus a sufficiently large collection of ordered strain modes will mitigate the CaE problem.  

\noindent \textbf{Soft sensing applied to control}

\newpage
\section{Future outlooks on soft robots}

\subsection*{Why not not-so-soft robots?}

\subsection*{Faster soft robots: fluidics, tendons, magnetics or di-electrics?}

\subsection*{Design and control: two sides of the same coin?}

\subsection*{Control philosophy for soft robots}

% \newcommand{\tabitem}{~~\llap{\textbullet}~~}
% \newcolumntype{C}[1]{>{\centering\arraybackslash}m{#1}}
% \begin{table}
%     \centering
% \begin{tabular}{|C{4cm}|l|}
%     \hline
%     \multirow{3}{*}{Short term ($\sim$ 1 year)}   & \tabitem Extension towards 3D topology optimization\newline  \\ 
%                                             & \hspace{4mm} and multi/active material optimization. \\ 
%                                             & \tabitem Bananis baidg ndgidog idngd kdnfd \\ 
%     \hline
%   \end{tabular}
% \end{table}

% %\section{Future horizons of soft robotics}
% \textbf{Chicken-and-egg problem in conventional modeling} 
% \lipsum[1-2]

% \textbf{Synergy between design and control}
% \lipsum[1-2]

% \textbf{Importance of sensing and its relation towards learning in control}
% \lipsum[1-2]