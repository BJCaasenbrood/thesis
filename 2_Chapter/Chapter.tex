%!TEX root = ../thesis.tex
\graphicspath{{2_Chapter/}}
\chapter{Modeling of Continuum Soft Robots -- Piece-wise Constant case}
{\let\thefootnote\relax\footnote{{\color{black}This chapter is based on: \\
\textit{1.} Caasenbrood, B. J., Pogromsky, A. Y., and Nijmeijer, H. (2020). \textbf{Dynamic modeling of hyper-elastic soft robots using spatial curves.} IFAC-PapersOnLine, 53, 9238–9243.\\[0.25em] \textit{2.} Caasenbrood, B. J., Pogromsky, A. Y., and Nijmeijer, H. (2021). \textbf{Dynamic Modeling of Hyper-elastic Soft Robots through Differential Geometry of Curves} Soft Robotics, 2021. (under review).
}}}
\label{chap:2}
\chaptermark{Chapter title for header}

\begin{chapter-abstract}
The motion complexity and use of exotic materials in soft robotics call for accurate and computationally efficient models intended for control. To reduce the gap between material and control-oriented research,we build upon the existing Piecewise-Constant Curvature framework by incorporating hyper-elastic and visco-elastic material behavior. In this work, the continuum dynamics of the soft robot are derived through the differential geometry of spatial curves, which are then related to Finite-Element data to capture the intrinsic geometric and material nonlinearities. To enable fast simulations, a reduced-order integration scheme is introduced to compute the dynamic Lagrangian matrices efficiently, which in turn allows for real-time (multi-link) models with sufficient numerical precision. By exploring the passivity and using the parametrization of the hyper-elastic model, we propose a passivity-based adaptive controller that enhances robustness towards material uncertainty and unmodeled dynamics -- slowly improving their estimates online. As a study case, a fully 3D-printed soft robot manipulator is developed, which shows good correspondence with the dynamic model under various conditions, e.g., natural oscillations, forced inputs, and under tip-loads. The solidity of the approach is demonstrated through extensive simulations, numerical benchmarks, and experimental validations.
\end{chapter-abstract}

\clearpage
\Materialtrue % enable chapter thumbs if disabled on chapter title page
%
%\section{Introduction}\label{chaptertitle:sec:introduction}
\sectionmark{Introduction}
%!TEX root = /home/brandon/Documents/phd/thesis/thesis.tex
\section{Continuous kinematics for soft robots}
By using the equality of mixed partials, we may invoke that $\frac{\p}{\p t} (g') = \frac{\p }{\p \sigma} (\dot{g})$ holds for any instance in space and time. Accordingly, substitution of relations \eqref{eq:eta} and \eqref{eq:xi} into this commutative relation leads to
\begin{align}
\dot{g}\xi + g\dot{\hat{\xi}}  = g'\hat{\eta} + g\hat{\eta}',
\end{align}
which implies
\begin{equation}
g\hat{\eta} \hat{\xi} + g\dot{\hat{\xi}}  = g\hat{\xi}\hat{\eta} + g\hat{\eta}'.
\end{equation}
Multiplying both sides with $g^{-1}$ and rearranging the equality, we find
\begin{equation}
\hat{\eta}' = -(\hat{\xi}\hat{\eta} - \hat{\eta} \hat{\xi}) + \dot{\hat{\xi}},\label{eq:eta_prime}
\end{equation}
where we can recognize, in the parenthesis, the Lie bracket of $\xi$ and $\eta$. The Lie bracket $[\hat{\xi},\hat{\eta}]$ is also an element of Lie algebra $\se{3}$, and thus it may be alternatively expressed in $\R^6$ as the adjoint action between $\xi$ onto $\eta$, namely $\ad_{\xi} \eta: \R^6 \mapsto \R^6$ (see \cite{Spong2006} and \cite{Traversaro2016}). Therefore, the velocity kinematics in \eqref{eq:eta_prime} can be written in vector representation as
\begin{equation}
\eta' = -\ad_\xi \eta + \dot{\xi}.
\label{eq:eta_prime_R6}
\end{equation}
By taking the time derivative of \eqref{eq:eta_prime_R6} and combining the previous results, the continuous kinematic model for the configuration, velocity, and acceleration can be written as system of first-order partial differential equation (PDE) of the form
\begin{equation}
\frac{\p}{\p \sigma}\begin{pmatrix}\; g \;\\  \; \eta \; \\ \; \dot{\eta} \; \end{pmatrix} = \begin{pmatrix} \; g \hat{\xi} \\ \; -\ad_\xi \eta + \dot{\xi} \\ \; -\ad_{\dot{\xi}} \eta - \ad_{{\xi}} \dot{\eta} + \ddot{\xi} \;\end{pmatrix}.
\label{eq:cont_kin_pde}
\end{equation}
For a general case, the boundary conditions of PDE in \eqref{eq:cont_kin_pde} should satisfy $g(0,t) = g_0$, $\eta(0,t) = \eta_0$ and $\dot{\eta}(0,t) = \dot{\eta}_0$. However, in case of a manipulator whose base is spatially fixed, the boundary conditions should satisfy $g(0,t) = g_0$, and $\eta(0,t) = \dot{\eta}(0,t) = 0_6$. Notice that if the strain fields $\xi$, $\dot{\xi}$, and $\ddot{\xi}$ are known, the partial differential equation in \eqref{eq:cont_kin_pde} simply becomes a first-order ordinary differential equation (ODE), which can be easily solved using numerical methods.

\clearpage
%!TEX root = ../../thesis.tex
\section{Continuum dynamic model}
\noindent As mentioned previously, soft robots are composed of soft bodies that may be regarded as a continuum body with (theoretically) infinitely many degrees-of-freedom (DOF). In this section, we aim to derive a compact and computationally efficient model that envelops the continuous dynamics of a soft robot through a small set of generalized coordinates $\vec{q}\in\mathcal{Q}$ and their respective generalized velocities $\dot{\vec{q}}(t)\in\mathbb{R}^n$ with $n$ the number of active joint variables. We base {the modeling framework on the work of Mochiyama et al.\cite{Mochiyama2003} who outlined a theoretical foundation for continuum manipulators. Their work is extended upon by including extensibility, serial-chaining of multiple soft-links, pneumatic actuation, and the introduction of nonlinear and time-dependent material behavior. Earlier modeling strategies addressing similar issues can be found in from Godage et al. \cite{Godage2015,Godage2016}, Della Santina et al. \cite{Santina2020,Santina2020b,Santina2020Pcc}, Renda et al. \cite{Renda2018}, and Boyer et al. \cite{Boyer2021}. Leveraging from the aforementioned works, the continuous dynamics of a soft robot manipulator can be written in the familiar Lagrangian form:
%
\begin{equation}
\mat{M}(\vec{q}) \ddot{\vec{q}} + \vec{h}(\vec{q},\dot{\vec{q}}) = \vec{Q}^{\text{nc}},
\end{equation}
%
where $\mat{M}(\vec{q}) \in \R^{n\times n}$ denotes the generalized inertia matrix, $\vec{h}(\vec{q},\dot{\vec{q}}) \in \R^{n}$ a vector of nonlinear state-dependent force contributions. In this work, a similar modeling framework is adopted; however, we propose an extension to incorporate FEM-driven data to more accurately reflect the underlying continuum mechanics -- in particular hyper-elasticity; and we propose a numerical scheme that allows for fast computation of the continuous dynamics. For completeness, we will recapitulate on the modeling approach here.
%
\begin{figure}[!t]
\hspace{4mm}
\centering
%\includegraphics[width = 0.35\textwidth]{CaasenbroodFig2}
%\caption{Bishop frame representation for the soft robot manipulator expressed in the generalized coordinate vector $q(t)$ and the mapping $p$ describing the backbone curve of the manipulator in $\R^3$. For sake of geometric insight, some additional geometric variables (i.e. $\kappa$, $\phi$ and $\beta$) are also illustrated. \label{fig:configuration}}
\end{figure}
%
\subsection{Kinematics of elastic continuum bodies}
\noindent To represent the hyper-flexible configuration of the soft robot, let us consider a smooth spatial curve that passes through the geometric center of the continuously deformable body, as shown in Figure \ref{fig:configuration}. {In literature, this curve is called} the '\textit{backbone curve}' as it simplifies the three-dimensional deformation imposed by distributed forces acting on the elastic body. The arc-length of the backbone corresponds to the extensible length of the soft robot denoted by the variable $l(t) \in [l_{-},l_{+}]$ which we assume bounded $l_{+} \ge l \ge l_{-}$, and let $L$ be a constant denoting the {total unstressed} length of the soft robot. Next, let us introduce a spatial variable
$\sigma \in \Xs$ that belongs to the one-dimensional material domain of the backbone curve, i.e., $\Xs = [0,\, L]$. {Let it be clear that the spatial variable $\sigma$ represents the arc-length of a material coordinate along the undeformed material domain of the soft robot manipulator.}

Given each material coordinate, we wish to find a suitable low-dimensional joint representation $q(t)$ such that the position vector $^0p$ anywhere on the continuous backbone can be written as a mapping from generalized coordinates and space into $\mathbb{R}^3$:
%
\begin{equation}
^0\vec{p}:  \Xs \times \mathcal{Q}(t) \to \mathbb{R}^3;
\end{equation}
%
and similarly the rotation matrix $^0\mat{\Phi}(\sigma,\vec{q})$ by a mapping from the generalized coordinates and space into $\SO{3}$:
%
\begin{equation}
^0\mat{\Phi}: \Xs \times \mathcal{Q}(t) \to \SO{3}, \label{eq:phi_map}
\end{equation}
%
where {$\SO{3}$ denotes the special orthogonal group for rotations about the origin of $\R^3$}, and $n = \dim(\vec{q})$ the state dimension. Under this notion, the position vectors $^0p(q,0)$ and {$^0p(q,L)$} relate to the base and the end-effector of the soft robot, respectively. {Please note that left-sided superscript are used to indicate the frame of reference.} The set of all points on the backbone $\mathcal{P} = \left\{^0p \in \mathbb{R}^3\, |\, \sigma \in \Xs \right\}$ draws a possible {spatial} configuration of the soft robot given {a time instance $t \in \mathbb{T}$ on a finite horizon $\mathbb{T} = [0,T]$}.
%
\begin{intermezzo}
Despite the inherent flexibility in soft robotics, it is sometimes sufficient to express the kinematics according to the Piecewise Constant Curvature (PCC) condition. Mathematically, it implies that the curvature of the continuous body satisfies $\kappa(q,\sigma_1) = \kappa(q,\sigma_2)$ for a neighboring region of points $\sigma_1,\sigma_2 \subseteq \Xs$. As a result, this condition allows us to describe the full forward kinematics with a significantly reduced set of generalized coordinates, mitigating kinematic complexity in the model. Numerous works employ PCC models \cite{Falkenhahn2015,Katzschmann2019,Tatlicioglu2007,Marchese2016,Godage2016,Santina2020Pcc}, and depending on the degrees of elasticity, the PCC condition has been proven to be consistent for various soft robotic systems.
\end{intermezzo}
%
{Following this Piecewise Constant Curvature (PCC) description, let us assign a coordinate frame that twists minimally along the backbone -- a Bishop frame \cite{Bishop1975}-- parametrized by the following generalized coordinate vector:}
%
\begin{equation}
\vec{q} = \begin{pmatrix}
\,\varepsilon & \kappa_x & \kappa_y\,
\end{pmatrix}^\top \in \mathcal{Q},
\label{eq:coordinate}
\end{equation}
%
\noindent where {$\varepsilon \in \R$ is the elongation strain}, and $\kappa_x,\,\kappa_y\in\mathbb{R}$ are the curvatures or angular strains in $x$-$z$ and $y$-$z$ plane, respectively; and $\mathcal{Q} \subset \R^3$ is an admissible space on which $q$ evolves.It is worth mentioning that the joint description above is somewhat related to Renda. et al. \cite{Renda2018} who proposed a Piece-wise Constant Strain (PCS) parametrization with the exception of including the twist along the tangent.

By exploring the differential geometry of the smooth backbone curve similar to Mochiyama et al.\cite{Mochiyama2003}, we can express the spatial change of the position vector $^0 \vec{p}(0,\vec{q})$ and the orientation matrix $^0\mat{\Phi}(q,\sigma)$ for each material point $\sigma$ along the smooth backbone by
%
\begin{align}
\renewcommand*{\arraystretch}{2}{}
\frac{\partial \,^0\!\mat{\Phi}}{\partial \sigma}(\sigma,\vec{q}) & = \, ^0\mat{\Phi}(\sigma,\vec{q})\,\left[\mat{\Gamma} (\sigma,\vec{q}) \right]_{\times}, \label{eq:change_phi} \\
\frac{\partial \,^0\! \vec{p}}{\partial \sigma}(\sigma,\vec{q}) & = \, ^0\mat{\Phi}(\vec{q},\sigma) \, \vec{U}(\sigma,\vec{q}), \label{eq:change_p}
\end{align}
%
where $[\vec{\Gamma}]_\times \in \sog{3}$ is a skew-symmetric matrix composed of the entries of the vector $\vec{\Gamma} \in \R^3$, and $\vec{U}\in \R^3$ a vector representing the tangent along the extensible backbone. The vectors $\vec{\Gamma}$ and $\vec{U}$ are vectors that define the differential geometry of the backbone, which are unique entries that lives in the tangent space of the rigid-body transformation group $\SE{3}$. Given the Bishop parametrization as described by \eqref{eq:coordinate}, these geometric entities yield
%
\begin{equation}
\vec{\Gamma} = \begin{pmatrix} -\kappa_y \\ \kappa_x \\ 0  \end{pmatrix}; \quad \quad \quad \vec{U} = \begin{pmatrix} \,\, 0 \,\, \\ \,\, 0 \,\, \\ \, \,\varepsilon \,\, \end{pmatrix} + \vec{U}_0,
\end{equation}
%
with $\vec{U}_0 = (0,0,1)^\top$ the unit-tangent. Now, given an initial configuration of backbone's base, i.e., $^0 \mat{\Phi}(0,\vec{q}) = \vec{\Phi}_0$ and $^0 \vec{p}(0,\vec{q}) = 0_3$, we can now solve for the position and orientation for each material coordinate $\sigma$ along the backbone:
%
\begin{align}
^0\mat{\Phi}(\sigma,\vec{q}) & = \vec{\Phi}_0\exp(\sigma [\vec{\Gamma}(\vec{q})]_\times), \label{eq:phi_exact} \\
^0\vec{p}(\sigma,\vec{q}) & = \int_0^\sigma\,^0\mat{\Phi}(\eta,\vec{q})\, \vec{U}(\vec{q}) \; d\eta, \label{eq:pos_vector}
\end{align}
%
where $\exp: \sog{3} \to \SO{3}$ is the exponential map. Let it be clear that the closed-form solutions \eqref{eq:phi_exact} and \eqref{eq:pos_vector} form the forward configuration kinematics of the backbone curve. To express the forward velocity kinematic, let  $\vec{V}(\sigma,\vec{q},\dot{\vec{q}}) = \left(^\sigma \vec{\omega}^\top,^\sigma \vec{v}^\top \right)^\top \in \R^6 \cong \se{3}
$ be the aggregate of the angular velocity and linear velocity components relative to an inertial frame at $\sigma$ (the frame of reference is denoted by a left superscript), where the space $\se{3}$ denotes the Lie algebra of $\SE{3}$. The velocity twist is computed by the following integration procedure:
%
\begin{equation}
 \vec{V}(\sigma,\vec{q},\dot{\vec{q}}) = \Ad_{\mat{g}(\sigma,\cdot)}\inv \int_0^\sigma \Ad_{\mat{g}(\eta,\cdot)}\, J^*\! \dot{q}\;d\eta
 \,=:\, J(q,\sigma) \dot{q}, \label{eq:vel_cont}
\end{equation}
%
where $\Ad_g: \SE{3} \to \mathbb{R}^{6\times 6}$ denotes the adjoint transformation matrix regarding the rigid body transformation $g \in \SE{3}$ that maps local velocities (i.e., twist) to a frame located at $\sigma$, and $J^*$ a constant joint-axis matrix. The joint-axis matrix for an extensible and bendable PCC segment parametrized by the Bishop parameters is given by
%
\begin{equation}
\renewcommand*{\arraystretch}{1}{}
J^* := \left(\dfrac{\p \Gamma}{\p q}^\top \; \dfrac{\p U}{\p q}^\top \right)^\top = \begin{pmatrix}
\,0 & 0 & 0 & 0 & 0 & 1 \, \\
\,0 & 1 & 0 & 0 & 0 & 0 \,  \\
\,-1 & 0 & 0 & 0 & 0 & 0 \,  \\
\end{pmatrix}^\top. \label{eq:joint-axis-matrix}
\end{equation}
%
Although we based the forward kinematics on the work of Mochiyama et al.\cite{Mochiyama2003}, the derived expression for the velocity twist in \eqref{eq:vel_cont} is analogous to the work of Renda et al.\cite{Renda2018,Renda2020}, and Boyer et al. \cite{Boyer2010,Boyer2021}. Please also note that \eqref{eq:vel_cont} gives rise to the geometric manipulator Jacobian $J(q,\sigma)
$ that defines the mapping from joint velocities to the velocity twist for a particular material point $\sigma$ on the continuous body. In continuation, let us also introduce the acceleration twist\cite{Boyer2021,Mochiyama2003,Renda2018} -- obtained through time differentiation of \eqref{eq:vel_cont}:
%
\begin{align}
\dot{V}(q,\dot{q},\ddot{q},\sigma) & = J \ddot{q} + \Ad_{g(\cdot,\sigma)} \inv \int_0^\sigma \Ad_{g(\cdot,\eta)}
\ad_{V(\cdot,\eta)} \, J^*\! \dot{q}\;d\eta \notag \\
& := J(q,\sigma)\ddot{q} + \dot{J}(q,\dot{q},\sigma) \dot{q},
\label{eq:acceleration}
\end{align}
%
where $\ad_{V} \in \mathbb{R}^{6\times 6}$ denotes the adjoint transformation regarding the velocity twist $V \in \se{3}$. The reader is referred to Appendix A for more detailed expressions on the adjoint transformations.
%
\subsection{Euler-Lagrange equations}
\noindent Given the forward kinematics in \eqref{eq:phi_exact}, \eqref{eq:pos_vector}, \eqref{eq:vel_cont} and \eqref{eq:acceleration}, we can shift our attention to formulating the finite-dimensional dynamics of the soft robot. Our goal here is to write the spatio-temporal dynamics of the hyper-elastic soft robot as a second-order ODE into the Lagrangian form:
%
\begin{equation}
\frac{d}{d t}\left(\frac{\partial \mathcal{L}}{\partial \dot{{q}}}\right) - \frac{\partial \mathcal{L}}{\partial {q}} = {Q}^{\nc}, \label{eq:euler_largrange}
\end{equation}
%
\noindent where $\La({q},\dot{q}) := \T(q,\dot{q}) - \mathcal{U}(q)$ is the Lagrangian function, $\T \in \Rp$ and $\mathcal{U}\in \R$ the kinetic and potential energy, respectively; and $Q^{\nc} \in \mathbb{R}^n$ a vector of generalized non-conservative forces. To apply the Lagrangian formalism to a continuum dynamical system, regard an infinitesimal slice of the continuum body for each material coordinate $\sigma$ along the backbone curve. Given this notion, we embody this infinitesimal slice with an inertia tensor $
\mathcal{M} = \text{blkdiag}(\rho I_3,\mathcal{J_\sigma})$ with $\rho = m/L$ the line-density and $J_\sigma$ a tensor for the second moment of inertia. The kinetic energy can be obtained through spatial integration of its respective kinetic energy densities\cite{Boyer2010,Mochiyama2003,Tatlicioglu2007}, i.e., $\mathfrak{T} = \frac{1}{2}V^\top \M V
$:
%
\begin{align}
\mathcal{T}({q},\dot{{q}}) & = \frac{1}{2}\int_\Xs {V}({q},\dot{q},\sigma)^\top\,\mathcal{M}\,{V}({q},\dot{{q}},\sigma) \; d \sigma,
 \notag \\
& =  \frac{1}{2} \dot{q}^\top \int_\Xs  J({q},\sigma)^\top\,\mathcal{M}\, J({q},\sigma) \; d \sigma \, \dot{q}, \notag \\
& = \frac{1}{2}\dot{q}^\top M(q) \dot{q}. \label{eq:kinetic_energy}
\end{align}
%
Note that expression for the kinetic energy naturally gives rise to the generalized inertia matrix $M(q)$ of the Lagrangian model. By substitution of the kinetic energy into the Euler-Lagrange equation \eqref{eq:euler_largrange}, we find $M(q)\ddot{q} + C(q,\dot{q})q$ where $C(q,\dot{q})$ denotes the Coriolis matrix. Instead of computing the Coriolis matrix through the conventional Christoffel symbols\cite{Murray1994}, we adopt a computational scheme by Garofalo et al. \cite{Garofalo2013} used for serial-chain rigid manipulators, in which we replaced the finite summation of $N$ rigid-bodies by a spatial integration over the continuum domain $\Xs$:
%
\begin{multline}
C(q,\dot{q}) = \int_\Xs J(q,\sigma)^\top \C_{V(q,\dot{q},\sigma)}J(q,\sigma)\; + \\ J(q,\sigma)^\top \M \dot{J}(q,\dot{q},\sigma) \; d \sigma,\label{eq:coriolis}
\end{multline}
%
where $\mathcal{C}_{V} = -\mathcal{C}_{V}^\top :=  \M \ad_{V}  - \ad_{V} ^\top \M$ is a skew-symmetric matrix. The computation above is slight different from existing literature\cite{Boyer2021,Renda2020} to ensure that the matrix $\dot{M} - 2C$ is skew-symmetric; the so-called the passivity condition\cite{Murray1994} for Euler-Lagrange systems (see Appendix B for proof). The importance of this property will become apparent later in the energy-based controller design. Lastly, the potential energy is given by sum of gravitational potential energy and internal elastic potential, i.e., $\mathcal{U}({q}) = \mathcal{U}_g({q}) + \mathcal{U}_e({q})
$. Since gravitational potential energy density is \rewritten{given} by $\mathfrak{U}_g = -\rho\,^0p(q,\sigma) \gamma_g$ with $\gamma_g \in \R^3$ is a vector of body accelerations, the potential energy related to gravity is obtained by spatial integration of their respective energy densities:
%
\begin{equation}
\mathcal{U}_g({q}) = - \rho \int_\Xs \,^0p(q,\sigma)^\top \gamma_g \; d \sigma.
\label{eq:potential_energy_grav}
\end{equation}
%
\noindent To model the hyper-elastic nature, lets introduce two nonlinear stiffness functions for both stretching and bending, denoted by $k_e: \R \mapsto \Rsp$ and $k_b: \R \mapsto \Rsp$, respectively. These functions allow us to describe a collective elastic behavior imposed by the hyper-elastic materials and the continuum-bodied deformation. It shall be clear that these entities are unique to the soft robot's geometry and soft material choice, and thus finding a suitable candidate model requires further analysis. Later, we will sculpt these nonlinear stiffness functions through Finite Element Methods (FEM). For now, we assume that these analytical nonlinear stiffness functions are known, and thus the (hyper)-elastic potential energy takes the form
%
\begin{equation}
\mathcal{U}_e({q}) = \int_0^{\varepsilon} k_e(\eta) \,\eta \; d \eta + \int_0^{\beta(q)} k_b(\eta)\, \eta \; d \eta,
\label{eq:potential_energy_elas}
\end{equation}
%
where $\varepsilon$ is the elongation strain, and $\beta({q}) = \kappa L (\varepsilon + 1)$ is the bending angle with the total curvature of the soft segment $\kappa = \sqrt{{\kappa_x}^2 + {\kappa_y}^2}$ (see Figure \ref{fig:configuration}).
\subsection*{Overall dynamics}
\noindent Finally, by combining \eqref{eq:euler_largrange}, \eqref{eq:kinetic_energy}, \eqref{eq:coriolis}, \eqref{eq:potential_energy_grav}, and \eqref{eq:potential_energy_elas}, the continuum dynamics of the soft robot can be casted into the familiar closed form \cite{Santina2020Pcc,Boyer2021,Renda2018,Godage2016} similar to aforementioned model (1):
%
\begin{align}
M({q})\,\ddot{{q}} + {C}({q},\dot{{q}})\,\dot{{q}} + P({q},\dot{q}) + G({q}) & = \tau(u,\delta), \label{eq:dynamic_model}
\end{align}
%
\noindent where $P = d \mathcal{U}_e/d q + R\dot{q}$ is a vector of generalized forces imposed by the deformation of the soft materials with $R \in \R^{n\times n}$ the Rayleigh damping matrix, $G = \p \mathcal{U}_g/\p q$ a vector of generalized gravitational forces, and $u \in \R^m$ the control input with the index $m$ the number of pressure inputs. The generalized input vector is chosen of the form: $\tau(u,\delta) = H u + \delta$ with $H: \R^m \mapsto \R^n$ a mapping from the input space to the joint actuation space, and $\delta(t)$ an external disturbance (e.g., unmodelled material uncertainties).
%
\begin{remark}
Given the context of manipulators, a possible disturbance $\delta(t)$ could be an external mass applied to the tip of the soft robot. Given the kinematic relations in \eqref{eq:vel_cont} and \eqref{eq:acceleration}, one can describe the disturbance (modeled here as a point-mass located at $L$) by a state-dependent vector:
%
\begin{equation}
\delta_m = m_\delta \floor{J(\cdot,L)}_3^\top\left({\normalfont \Ad}_{g(\cdot,L)}\inv\gamma_g + \floor{\dot{V}(\cdot,L)}_3 \right),
\label{eq:delta_payload}
\end{equation}
%
where $\floor{\cdot}_3$ extracts the last three rows of a matrix or vector, and $m_\delta > 0$ the applied mass to the end-effector. It is worth recalling that the acceleration twist can be computed through the geometric Jacobian and its time derivative, i.e., $\dot{V} = J\ddot{q} + \dot{J}\dot{q}$. Indeed, the PCC condition for a soft body can only accurately describe the true dynamics if external forces produced by mass $m_\delta$ do not excessively exceed the intrinsic elastic balancing forces $P(q)$. Alternatively, a soft body can be modeled using multiple PCC curves of smaller size, similar to standard Finite Element discretization.
\end{remark}

The actuation mapping $H$ depends on the geometry, placement, and orientation of the (pneumatic) soft actuators. Since the pneumatic chambers are aligned parallel to the backbone curve and are equally spaced along the circumference, we propose the following ansatz:
%
\begin{equation}
H: = \begin{pmatrix} \alpha_{\varepsilon} & \hdots & \alpha_{\varepsilon} \\ -\alpha_{\kappa} \cos(\phi_1) & \hdots & -\alpha_{\kappa} \cos(\phi_m) \\ \alpha_{\kappa} \sin(\phi_1) & \hdots & \alpha_{\kappa} \sin(\phi_m) \end{pmatrix},
\label{eq:mapping_H}
\end{equation}
%
where $\alpha_{\varepsilon},\alpha_{\kappa} > 0$ are system parameters representing the effective transferal of differential pressure to joint forces, and $\phi_i = (i-1)\cdot\tfrac{2\pi}{m}$ the angular inter-distance between the $m$-number of pneumatic bellows. \revthree{Please note that the parameters $\alpha_{\varepsilon}$ and $\alpha_{\kappa}$ are dependent on the bellow area and radius from the bellow to the backbone curve.}
%


\clearpage
%!TEX root = /home/brandon/Documents/phd/thesis/thesis.tex
\section{Continuous kinematics for soft robots}
By using the equality of mixed partials, we may invoke that $\frac{\p}{\p t} (g') = \frac{\p }{\p \sigma} (\dot{g})$ holds for any instance in space and time. Accordingly, substitution of relations \eqref{eq:eta} and \eqref{eq:xi} into this commutative relation leads to
\begin{align}
\dot{g}\xi + g\dot{\hat{\xi}}  = g'\hat{\eta} + g\hat{\eta}',
\end{align}
which implies
\begin{equation}
g\hat{\eta} \hat{\xi} + g\dot{\hat{\xi}}  = g\hat{\xi}\hat{\eta} + g\hat{\eta}'.
\end{equation}
Multiplying both sides with $g^{-1}$ and rearranging the equality, we find
\begin{equation}
\hat{\eta}' = -(\hat{\xi}\hat{\eta} - \hat{\eta} \hat{\xi}) + \dot{\hat{\xi}},\label{eq:eta_prime}
\end{equation}
where we can recognize, in the parenthesis, the Lie bracket of $\xi$ and $\eta$. The Lie bracket $[\hat{\xi},\hat{\eta}]$ is also an element of Lie algebra $\se{3}$, and thus it may be alternatively expressed in $\R^6$ as the adjoint action between $\xi$ onto $\eta$, namely $\ad_{\xi} \eta: \R^6 \mapsto \R^6$ (see \cite{Spong2006} and \cite{Traversaro2016}). Therefore, the velocity kinematics in \eqref{eq:eta_prime} can be written in vector representation as
\begin{equation}
\eta' = -\ad_\xi \eta + \dot{\xi}.
\label{eq:eta_prime_R6}
\end{equation}
By taking the time derivative of \eqref{eq:eta_prime_R6} and combining the previous results, the continuous kinematic model for the configuration, velocity, and acceleration can be written as system of first-order partial differential equation (PDE) of the form
\begin{equation}
\frac{\p}{\p \sigma}\begin{pmatrix}\; g \;\\  \; \eta \; \\ \; \dot{\eta} \; \end{pmatrix} = \begin{pmatrix} \; g \hat{\xi} \\ \; -\ad_\xi \eta + \dot{\xi} \\ \; -\ad_{\dot{\xi}} \eta - \ad_{{\xi}} \dot{\eta} + \ddot{\xi} \;\end{pmatrix}.
\label{eq:cont_kin_pde}
\end{equation}
For a general case, the boundary conditions of PDE in \eqref{eq:cont_kin_pde} should satisfy $g(0,t) = g_0$, $\eta(0,t) = \eta_0$ and $\dot{\eta}(0,t) = \dot{\eta}_0$. However, in case of a manipulator whose base is spatially fixed, the boundary conditions should satisfy $g(0,t) = g_0$, and $\eta(0,t) = \dot{\eta}(0,t) = 0_6$. Notice that if the strain fields $\xi$, $\dot{\xi}$, and $\ddot{\xi}$ are known, the partial differential equation in \eqref{eq:cont_kin_pde} simply becomes a first-order ordinary differential equation (ODE), which can be easily solved using numerical methods.

\clearpage
%!TEX root = /home/brandon/Documents/phd/thesis/thesis.tex
\section{Continuous kinematics for soft robots}
By using the equality of mixed partials, we may invoke that $\frac{\p}{\p t} (g') = \frac{\p }{\p \sigma} (\dot{g})$ holds for any instance in space and time. Accordingly, substitution of relations \eqref{eq:eta} and \eqref{eq:xi} into this commutative relation leads to
\begin{align}
\dot{g}\xi + g\dot{\hat{\xi}}  = g'\hat{\eta} + g\hat{\eta}',
\end{align}
which implies
\begin{equation}
g\hat{\eta} \hat{\xi} + g\dot{\hat{\xi}}  = g\hat{\xi}\hat{\eta} + g\hat{\eta}'.
\end{equation}
Multiplying both sides with $g^{-1}$ and rearranging the equality, we find
\begin{equation}
\hat{\eta}' = -(\hat{\xi}\hat{\eta} - \hat{\eta} \hat{\xi}) + \dot{\hat{\xi}},\label{eq:eta_prime}
\end{equation}
where we can recognize, in the parenthesis, the Lie bracket of $\xi$ and $\eta$. The Lie bracket $[\hat{\xi},\hat{\eta}]$ is also an element of Lie algebra $\se{3}$, and thus it may be alternatively expressed in $\R^6$ as the adjoint action between $\xi$ onto $\eta$, namely $\ad_{\xi} \eta: \R^6 \mapsto \R^6$ (see \cite{Spong2006} and \cite{Traversaro2016}). Therefore, the velocity kinematics in \eqref{eq:eta_prime} can be written in vector representation as
\begin{equation}
\eta' = -\ad_\xi \eta + \dot{\xi}.
\label{eq:eta_prime_R6}
\end{equation}
By taking the time derivative of \eqref{eq:eta_prime_R6} and combining the previous results, the continuous kinematic model for the configuration, velocity, and acceleration can be written as system of first-order partial differential equation (PDE) of the form
\begin{equation}
\frac{\p}{\p \sigma}\begin{pmatrix}\; g \;\\  \; \eta \; \\ \; \dot{\eta} \; \end{pmatrix} = \begin{pmatrix} \; g \hat{\xi} \\ \; -\ad_\xi \eta + \dot{\xi} \\ \; -\ad_{\dot{\xi}} \eta - \ad_{{\xi}} \dot{\eta} + \ddot{\xi} \;\end{pmatrix}.
\label{eq:cont_kin_pde}
\end{equation}
For a general case, the boundary conditions of PDE in \eqref{eq:cont_kin_pde} should satisfy $g(0,t) = g_0$, $\eta(0,t) = \eta_0$ and $\dot{\eta}(0,t) = \dot{\eta}_0$. However, in case of a manipulator whose base is spatially fixed, the boundary conditions should satisfy $g(0,t) = g_0$, and $\eta(0,t) = \dot{\eta}(0,t) = 0_6$. Notice that if the strain fields $\xi$, $\dot{\xi}$, and $\ddot{\xi}$ are known, the partial differential equation in \eqref{eq:cont_kin_pde} simply becomes a first-order ordinary differential equation (ODE), which can be easily solved using numerical methods.
%\blindtext
%
%
%\clearpage
%
%\section{Section title}\label{chaptertitle:sec:sectiontitle}
%\sectionmark{Section title for header}
%
%\blindtext
