%!TEX root = ../thesis.tex
%%%% ABSTRACT ******************************************************************
\chapter*{Dissertation Title \\ and Abstract} % Try to keep within approx 350 words / one page
\addcontentsline{toc}{chapter}{Abstract}
\markboth{Abstract}{Abstract}
\vspace{-12mm}
\begin{center}
\rule{\textwidth}{.75pt}\vspace*{1mm}
\textbf{{\Large \maintitle} \\[1.0em]}
\text{B.J. Caasenbrood (Brandon) \quad Date: \today}
\rule{\textwidth}{.75pt}
\end{center}
\vspace*{2ex}

In the past two decades, the field of soft robotics has kindled a major  scientific interest among many disciplines of engineering. Contrary to rigid robots, soft robots explore \emph{soft materials} that significantly enhance the robot’s dexterity, enable a rich family of motion primitives, and enhance environmental robustness regarding contact and impact with major safety merits. The main inspiration for soft robotic systems stems from biology with the aim to achieve similar performance and dexterity as biological creatures. Since its inception, soft robotics has exemplified its potential in diverse industrial areas such as safe robotic manipulation, adaptive grasping, aquatic and terrestrial exploration of uncertain environments, rehabilitation, and the bio-mimicry of many animals including birds, fish, elephants, octopuses, and various invertebrates. By exploring the uncharted merits of soft materials and soft actuation, soft robotics has placed the first steppingstones towards achieving biological performance in next-generation robotics.

\par Although some significant leaps have been made towards bridging biology and robotics, there exist major scientific challenges that hinder the advancement of the field. In particular: (I) the Design and (II) Modeling of soft robotic systems. Traditional design of rigid robotics emphasizes on maximum structural rigidity and weight minimization, as to allow for fast, repeatable motion with negligible structural flexibility. Soft robotics, on the other hand, primarily rely on minimal structural rigidity for motion -- so called \emph{hyper-flexibility}. Especially since soft materials undergo large nonlinear mechanical responses under actuation, which leads to highly nonlinear kinematic relations for the robot's workspace. Using traditional engineering principles for soft materials is perhaps obsolete and automated computer-aided design principles for soft robotics might mandate the next steps for the field.
As for modeling, its innate infinite-dimensionality poses fundamental problems for model-based controllers. Besides, as these systems are composed of soft materials, large deformations lead to nonlinear mechanical responses that exotic to classic robotic theory. As a result, in terms of performance, soft robots are easily outclassed by their rigid counterparts nowadays and consequently lack the transferability to industry. The diligence of achieving similar precision and speed to current state-of-the-art robots, and ultimately nature, stresses the paramount importance on design, modeling and control tailored for soft robotics.

\par This thesis will address the design synthesis of soft robots as well as the development of model-based controllers for a subclass  -- soft continuum manipulators.

\par In the first part of this thesis, we present a novel framework for synthesizing the design of soft robotics with various types of soft actuation, \eg, hydraulics and tendons, but primarily pneumatic actuation. Contrary to traditional design methods, such as bio-mimicry, a gradient-based topology optimization is employed to find the optimal soft robotic structure given a user-defined objective function (i.e., desired morphology). Two difficulties are addressed here. First, pressure-based topology optimization is challenging since the adaptive topology changes the pneumatic load at each optimization step. To deal with this issue, we exploit the facial connectivity in the mesh tesselation to efficiently simulate the physics involving pneumatic actuation akin to soft robotic systems. The second issue is describing the hyper-elastic nature of soft materials. Here, nonlinear Finite Element Method (FEM) simulations are explored such that large deformations of hyper-elastic materials can be described accurately. The proposed optimization-driven algorithm is used to obtained a diverse class of morphologies: soft bending actuators, soft artificial muscles, soft grippers, and rotational soft actuators.

Lastly, the

Summarizing, this thesis contains several new techniques on design and mode-based control for the increasingly fast evolving field of soft robotics. Specifically, n

% for instance: Which control strategies are suited for soft robotics? How do we find a reasonable trade-off between the model accuracy and their applicability for control? Can we effectively exploit the intrinsic morphologies through control? Despite these challenges, some significant milestones have been achieved in regards to the development of accurate and computationally efficient dynamic models \cite{Renda2018,Duriez2013,Santina2020,Boyer2020,Grazioso2019,Stramigioli2009}. We believe that speakers on these areas of research could help define and better understand the current challenges in control-oriented modeling of soft robots and possibly paving the way to new and innovative control strategies \cite{Luca1998,Franco2020,Angelini2018,Fagiolini2020,Monje2008,Monje2007}.

% \par In this workshop, we aim to unite various researchers interested in modeling and control of soft robots, in particular those with different areas of expertise and key insights into control. To broaden the horizon on control-oriented modeling of soft robots and its application, recognized experts will cover the new and state-of-the-art developments in soft robotic. Additionally, we also extend towards speakers with key insights into alternative control approaches. To this end, we like to further promote this multi-disciplinary branch of robotics in the control community. We plan to organize the workshop into two sessions: Modeling of Soft Robots (S1), Control Applications for Soft Robots (S2). During these sessions, we hope to foster active discussions and promote the exchange of ideas between experts, younger researchers, and students working in different fields. To achieve this, the workshop will offer a wide variety of formats: \~4 keynote talks (35 minutes including Q\&A), \~6 invited talks (25 minutes including Q\&A), and \~6 student talks (10 minutes including Q\&A) and \~3 poster sessions. The student talks will be selected among the most exciting results submitted to the technical session.


\vspace*{11pt}\noindent
\textbf{Keywords:} \ \ Soft Robotics, Continuum Robots, Design Optimization, Finite-dimensional Modeling, Energy-based Control.
