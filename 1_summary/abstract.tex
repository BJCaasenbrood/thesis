%!TEX root = ../thesis.tex
%%%% ABSTRACT ******************************************************************

\chapter*{\vspace{-45mm}\\ Abstract} % Try to keep within approx 350 words / one page
\addcontentsline{toc}{chapter}{Abstract}
\markboth{Abstract}{Abstract}
\vspace{-15mm}
\begin{center}
\rule{\textwidth}{.75pt}\vspace*{1mm}
\textbf{{\Large \maintitle} \\[1.0em]}
\text{B.J. Caasenbrood (Brandon) \quad Date: \today}
\rule{\textwidth}{.75pt}
\end{center}
%\vspace*{ex}

In the past two decades, the field of soft robotics has kindled a major interest among many scientific disciplines. Contrary to rigid robots, soft robots explore \emph{soft materials} that significantly enhance the robot’s dexterity, enable a rich family of motion primitives, and enhance environmental robustness regarding contact and impact that benefactors human-robot safety. The main inspiration for soft robotic systems stems from biology with the aim to achieve similar performance and dexterity as biological creatures. Since its inception, soft robotics has exemplified its potential in diverse industrial areas such as safe robotic manipulation, adaptive grasping, aquatic and terrestrial exploration subject to environmental uncertainty, rehabilitation, and the bio-mimicry of many animals including birds, fish, elephants, octopuses, and various invertebrates. By exploring the uncharted merits of soft materials and soft actuation, soft robotics has placed the first steppingstones towards achieving biological performance in next-generation robotics.

\par Although some significant leaps have been made towards bridging biology and robotics, there exist major scientific challenges that hinder the advancement of the field. In particular: (I) the Design and (II) Modeling of soft robotic systems. Traditional design of rigid robotics emphasizes on maximum structural rigidity and weight minimization, as to allow for fast, repeatable motion with negligible structural flexibility. Soft robotics, on the other hand, primarily rely on minimal structural rigidity for motion -- so-called \emph{hyper-flexibility}. Furthermore, as soft materials undergo large nonlinear mechanical responses paired with distributed actuation, expressing the robot's workspace often leads to highly nonlinear kinematic descriptions. Using traditional engineering principles for soft materials is perhaps outdated and computer-assisted design principles for soft robotics might mandate the next steps for the field, especially with the recent advances in Additive Manufacturing (AM).
As for modeling, its innate infinite-dimensionality poses fundamental problems for model-based controllers. An important question arises during the modeling of such soft robots; \emph{how to deal with the trade-off between accuracy and computational efficiency?}. Besides, the presence of soft materials imbue the system with nonlinear mechanical responses that are perhaps alien to standard robot modeling. As a result, in terms of closed-loop performance, soft robots nowadays are easily outclassed by their rigid counterparts due to a lack of modeling knowledge. The diligence of achieving similar precision and speed to current state-of-the-art robots, and ultimately nature, stresses the paramount importance on design, modeling, and control tailored for soft robotics.

\par This thesis will address the generative design strategies for soft robots as well as model-based control strategies for a subclass  -- soft continuum manipulators.

\par \vspace*{2pt} In the first part of this thesis, we present a novel framework for synthesizing the design of soft robotics with various types of soft actuation, \eg, tendons, hydraulics, and pneumatics. Contrary to traditional design, such as bio-mimicry, a gradient-based topology optimization is explored to find sub-optimal soft robotic morphologies that satisfy user-defined motion criteria. Two difficulties are addressed here. First, pressure-based topology optimization yields distributed adaptive loadings that changes at each optimization step; and second capturing the hyper-elastic nature of soft materials.
%To deal with this issue, we exploit the facial connectivity in the mesh tesselation to efficiently simulate the physics involving pneumatic actuation akin to soft robotic systems.
A Finite Element Method (FEM) solver is proposed such that the physics under large nonlinear deformations of hyper-elastic materials and pneumatic actuation are accurately preserved. The optimization-driven algorithm yields generative designs for a diverse set of soft morphologies: soft rotational actuators, soft artificial muscles, and soft grippers. By assembly of  smaller soft sub-components, a full soft robot can be developed and through AM of flexible materials the feasibility is validated.

\par The second part of the thesis will focus on the model-based control of soft continuum manipulators, where the emphasis lies on the efficiency and accuracy in low-dimensional models. The continuous dynamics of the soft robot are modeled through the differential geometry of spatial curves. Using a finite-dimensional truncation, the system can be written as a reduced port-Hamiltonian model that preserves desirable control condition, \eg, passivity. However, this modeling techniques introduces gaps between the underlying material mechanics and control-structured dynamic model. Since useful information is attainable through FEM a-priori, new system identification tools are proposed that give inside into the dominant dynamic modes, the hyper-elasticity, and the reachable workspace spanned by soft materials and actuation. The approach yields accurate low-dimensional models with real-time control capabilities but also gives physical insight into optimal sensor placement applicable to proprioceptive sensing.

Following, the thesis treats the development of model-based controllers that can be employed in various control scenarios, \eg, motion planning, set-point stabilization, tracking, and grasping; akin to rigid robotics. The stabilizing controller utilizes an energy-based formulation, providing robustness even when faced with material uncertainties. The controller's effectiveness is demonstrated in simulation for various soft robotic systems that share a close resemblance to biology.

Lastly, the thesis will implement the proposed computationally efficient systematic strategies for the design and control on physical soft robotic systems, including soft grippers, soft manipulators, and soft exoskeletons. As a concluding remark, this thesis contains several new techniques for design and model-based control of the increasingly fast evolving and multi-disciplinary field of soft robotics.

% for instance: Which control strategies are suited for soft robotics? How do we find a reasonable trade-off between the model accuracy and their applicability for control? Can we effectively exploit the intrinsic morphologies through control? Despite these challenges, some significant milestones have been achieved in regards to the development of accurate and computationally efficient dynamic models \cite{Renda2018,Duriez2013,Santina2020,Boyer2020,Grazioso2019,Stramigioli2009}. We believe that speakers on these areas of research could help define and better understand the current challenges in control-oriented modeling of soft robots and possibly paving the way to new and innovative control strategies \cite{Luca1998,Franco2020,Angelini2018,Fagiolini2020,Monje2008,Monje2007}.

% \par In this workshop, we aim to unite various researchers interested in modeling and control of soft robots, in particular those with different areas of expertise and key insights into control. To broaden the horizon on control-oriented modeling of soft robots and its application, recognized experts will cover the new and state-of-the-art developments in soft robotic. Additionally, we also extend towards speakers with key insights into alternative control approaches. To this end, we like to further promote this multi-disciplinary branch of robotics in the control community. We plan to organize the workshop into two sessions: Modeling of Soft Robots (S1), Control Applications for Soft Robots (S2). During these sessions, we hope to foster active discussions and promote the exchange of ideas between experts, younger researchers, and students working in different fields. To achieve this, the workshop will offer a wide variety of formats: \~4 keynote talks (35 minutes including Q\&A), \~6 invited talks (25 minutes including Q\&A), and \~6 student talks (10 minutes including Q\&A) and \~3 poster sessions. The student talks will be selected among the most exciting results submitted to the technical session.

\vspace*{5pt}\noindent
\textbf{Keywords:} \ \ Soft Robots, Flexible Robots, Design Optimization, Continuum Mechanics, Reduced-order Modeling, Model-based Control, Additive Manufacturing.
