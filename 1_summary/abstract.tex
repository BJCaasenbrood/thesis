%!TEX root = ../thesis.tex
%%%% ABSTRACT ******************************************************************
\chapter*{\vspace{-45mm}\\ Abstract} % Try to keep within approx 350 words / one page
\addcontentsline{toc}{chapter}{Abstract}
\markboth{Abstract}{Abstract}
\vspace{-15mm}
\begin{center}
\rule{\textwidth}{.75pt}\vspace*{1mm}
\textbf{{\large \maintitle} \\[0.15em]
\textnormal{\small{B.J. Caasenbrood (Brandon)}}
%\blankfootnote{}
}
\rule{\textwidth}{.75pt}
\end{center} 

\vspace{-2mm}
In the past two decades, the field of soft robotics has sparked significant interest among many scientific disciplines. Contrary to rigid robots, soft robots explore soft materials that significantly enhance the robot's dexterity, enable a rich family of motion primitives, and enhance environmental robustness regarding contact and impact. Since its inception, soft robotics has exemplified its potential in diverse areas such as safe manipulation, adaptive grasping, exploration under environmental uncertainty, rehabilitation, and the bio-mimicry of many animals. By exploring the uncharted versatile nature of soft materials, soft robotics places the first steppingstones towards achieving biological performance in modern-day's robotics. This thesis aims to further the advances in soft robotics by addressing some of the open multi-disciplinary challenges within this young field of research. 
%This thesis aims to further advances in soft robotics by addressing the open multi-disciplinary challenges within this young field of research.

Although soft materials harbor many advantages akin to biology, which are difficult to achieve for rigid robotics, it also roots many fundamental problems. First is the issue of soft robotic design. Traditional robotic design emphasizes high structural rigidity and weight minimization -- a well-established practice in engineering. On the other hand, soft robotic design relishes minimal structural rigidity for motion, leading to complex, highly nonlinear relations between the input and output. Besides, distributed soft actuation, imparted by gravitational and inertial forces acting on the continuum elastic body, introduce joint mobilities that are in many cases uncontrollable nor aligned with the control objective, \eg, precise grasping and manipulation. Since describing the underlying continuum mechanics and applying such mathematical theory to systematic design is challenging, a large number of soft robotic systems are still developed \textit{ad hoc}. 

Second, a direct duality of the previous challenge is dealing with the innate infinite-dimensionality from a control perspective -- particularly with model-based feedback in mind. The transition from rigid to flexible has introduced a new control paradigm: the trade-off between precision and speed in a numerical setting. Not only is control theory for soft robotics in stages of inception, but deriving accurate and numerically efficient model-based controllers is challenging due to large nonlinear deformations of the soft robotic continuum.

In light of these challenges, this thesis proposes a set of systematic tools with theoretical and experimental applications for $(i)$ 
the structural design and fabrication of continuum-deformable soft actuators optimized for user-defined joint motion, $(ii)$ the development of efficient dynamic models of soft continuum manipulators, and $(iii)$ applying such mathematical models to model-based controllers for a subclass of (pneumatic) soft continuum manipulators and soft grippers.
%the transfer of control theory standard in rigid robotics to soft robotics.

The first part of the thesis addresses the design problem by proposing novel computer-automated design algorithms for developing efficient soft actuators. These algorithms account for the underlying continuum mechanics described by a set of partial differential equations, which respect the aforementioned nonlinearities between the input and output motion. Tailoring a user-defined objective to a desired motion and control reachability, an implicit representation of the optimal soft material distribution can be found within a fixed design space. Several generative designs for a diverse subset of soft actuation morphologies are produced including, but not limited to, soft rotational actuators, soft artificial muscles, and soft grippers. In what follows, an optimal design for a soft robotic manipulator with an adaptive gripper is synthesized; and through Additive Manufacturing (AM) of printable flexible material, the sim-to-real boundary is passed. The proposed approach does not only accelerate design convergence but also builds upon the vast library of soft robot morphologies currently unexplored in literature.

The second part of the thesis addresses the question of modeling for control applicable to a class of soft robotic systems -- most notably soft continuum manipulators. The thesis proposes a reduced-order modeling strategy for soft robotics, whose dynamics are derived through the differential geometric theory on spatial beams. Besides discussing earlier modeling strategies, the thesis also proposes a new strain-based parametrization approach that ensures the structural information and the underlying continuum mechanics are preserved when synthesizing the reduced-order beam models -- a possible solution to the aforementioned control paradigm of precision vs. speed. To enhance numerical performance further, spatio-temporal integration schemes are also proposed that exploit the geometric structure of such soft beam models, resulting in real-time simulation with sufficient numerical precision purposefully tailored for control.

The third part of the thesis treats the development of model-based controllers that can be employed in various control scenarios akin to control for traditional rigid robotics, \eg, inverse kinematics and motion planning, set-point stabilization, trajectory tracking, and multi-point grasping of objects. The stabilizing controller is rooted in an energy-based formulism, providing robustness even when faced with material uncertainties. The controller's effectiveness is demonstrated both in simulation and experiments for various soft robotic systems that share a resemblance to biology, \eg, the elephant's trunk or the tentacle of an octopus.

The main contribution of the thesis is a collection of multi-disciplinary tools compressed into one general framework for the design, modeling, and control of a class of soft robots, ranging from the theoretical to the experimental domain.

\vspace*{5pt}\noindent
\textbf{Keywords:} \ \ Soft Robots, Hyper-redundant Robots, Design Optimization, Continuum Mechanics, Reduced-order Modeling, Model-based Control, Additive Manufacturing.
%}
