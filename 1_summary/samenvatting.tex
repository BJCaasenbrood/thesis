%*********************************************************************************%
\chapter*{Samenvatting}
\addcontentsline{toc}{chapter}{Samenvatting}
\markboth{Samenvatting}{Samenvatting}
\vspace{-10mm}
In de afgelopen twee decennia heeft het veld van de zachte robotica veel interesse opgewekt binnen breed spectrum van wetenschappelijke disciplines. In tegenstelling tot rigide robots verkennen zachte robots zachte materialen die de behendigheid van de robot aanzienlijk verbeteren, een rijke collectie van bewegingsprimitieven mogelijk maken en de omgevingsbestendigheid ten aanzien van contact en impact vergroten. Sinds de oorspong heeft de zachte robotica heeft het haar potentieel aangetoond in diverse gebieden zoals veilige manipulatie, adaptief grijpen, verkenning onder omgevingsonzekerheid, revalidatie en de biomimetica van vele dieren. Door de veelzijdige aard van zachte materialen te verkennen, legt de zachte robotica de eerste stappen naar het bereiken van biologische prestaties in de moderne robotica. Deze scriptie heeft als doel de vooruitgang in de zachte robotica verder te bevorderen door enkele van de open multidisciplinaire uitdagingen binnen dit jonge onderzoeksgebied aan te pakken.

Hoewel zachte materialen, zoals systemen in de biologie, veel voordelen hebben, die soms moeilijk te bereiken zijn voor rigide robotica, brengt het ook veel fundamentele problemen met zich mee. Het eerste probleem is het ontwerp van zachte robots. Traditioneel robotica-ontwerp legt de nadruk op hoge structurele stijfheid en gewichtsminimalisatie - een goed doordachte discipline in de engineering. Aan de andere kant houdt het ontwerp van zachte robots van minimale structurele stijfheid voor beweging, wat leidt tot complexe, zeer niet-lineaire relaties tussen input en output. Bovendien leiden gedistribueerde zachte activering, toegepast door zwaartekracht- en traagheidskrachten die op het continu elastische lichaam werken, tot gewrichtsmobiliteiten die in veel gevallen niet te controleren zijn of niet zijn afgestemd op de controle-doelstelling, zoals nauwkeurig grijpen en manipulatie. Omdat het beschrijven van de onderliggende continue mechanica en het toepassen van dergelijke wiskundige theorie op systematisch ontwerp uitdagend is, worden nog steeds een groot aantal zachte robotsystemen \textit{ad hoc} ontwikkeld.

Ten tweede vormt de directe dualiteit van de vorige uitdaging het omgaan met de intrinsieke oneindige-dimensionaliteit vanuit een controleperspectief - met name met modelgebaseerde feedback in gedachten. De overgang van rigide naar flexibel heeft een nieuw controleparadigma geïntroduceerd: de afweging tussen precisie en snelheid in een numerieke omgeving. Niet alleen bevindt de controletheorie voor zachte robotica zich in de beginfase, maar het afleiden van nauwkeurige en numeriek efficiënte modelgebaseerde controllers is uitdagend vanwege de grote niet-lineaire vervormingen van de zachte roboticacontinuüm.

Gezien deze uitdagingen stelt deze scriptie een reeks systematische tools voor met theoretische en experimentele toepassingen voor $(i)$ het structurele ontwerp en de fabricage van continuüm-deformeerbare zachte actuators geoptimaliseerd voor door de gebruiker gedefinieerde gezamenlijke beweging, $(ii)$ de ontwikkeling van efficiënte dynamische modellen voor zachte continuüm manipulatoren, en $(iii)$ het toepassen van wiskundige modellen op modelgebaseerde controllers voor een subklasse van (pneumatische) zachte continuüm manipulatoren en zachte grijpers.

Het eerste deel van deze scriptie richt zich op het ontwerpprobleem door het voorstellen van nieuwe computer-geautomatiseerde ontwerpalgoritmen voor de ontwikkeling van efficiënte zachte actuators. Deze algoritmen houden rekening met de onderliggende continue mechanica die wordt beschreven door een set partiële differentiaalvergelijkingen, die de eerder genoemde niet-lineariteiten tussen de input en output beweging respecteren. Door een door de gebruiker gedefinieerd doel aan te passen aan een gewenste beweging en controlebereik, kan een impliciete representatie van de optimale zachte materiaalverdeling worden gevonden binnen een vast ontwerpruimte. Verschillende generatieve ontwerpen voor een diverse subset van zachte actuator-morfologieën worden geproduceerd, waaronder, maar niet beperkt tot, zachte rotatie-actuators, zachte kunstmatige spieren en zachte grijpers. Vervolgens wordt een optimaal ontwerp voor een zachte robotmanipulator met een adaptieve grijper gesynthetiseerd. Door middel van Additive Manufacturing (AM) van printbaar flexibel materiaal wordt de grens tussen simulatie en realiteit overschreden. De voorgestelde aanpak versnelt niet alleen het convergeren van het ontwerp, maar bouwt ook voort op de enorme bibliotheek van zachte robot-morfologieën die momenteel onontdekt zijn in de literatuur.

Het tweede deel van de scriptie richt zich op de modellering voor controle die van toepassing is op een klasse van zachte robotica-systemen - met name zachte continuüm manipulatoren. De scriptie stelt een modelleerstrategie voor met gereduceerde orde voor zachte robotica, waarvan de dynamica worden afgeleid door middel van differentiële geometrische theorie op ruimtelijke balken. Naast het bespreken van eerdere modelleerstrategieën stelt de scriptie ook een nieuwe spanning-gebaseerde parameterisatiebenadering voor die ervoor zorgt dat de structurele informatie en de onderliggende continue mechanica behouden blijven bij het synthetiseren van de gereduceerde balkmodellen - een mogelijke oplossing voor het eerder genoemde controleparadigma van precisie versus snelheid. Om de numerieke prestaties verder te verbeteren, worden ook spatio-temporele integratieschema's voorgesteld die de geometrische structuur van dergelijke zachte balkmodellen benutten, wat resulteert in real-time simulatie met voldoende numerieke precisie die specifiek is afgestemd op controle.

Het derde deel van de scriptie behandelt de ontwikkeling van op modellen gebaseerde controllers die kunnen worden gebruikt in verschillende controle scenario's vergelijkbaar met de controle voor traditionele rigide robotica, bijvoorbeeld inverse kinematica en bewegingsplanning, set-point stabilisatie, traject volgen en multi-point grijpen van objecten. De stabiliserende controller is geworteld in een op energie gebaseerde formulering, die robuustheid biedt, zelfs wanneer er sprake is van materiaalonzekerheden. De effectiviteit van de controller wordt zowel in simulatie als in experimenten aangetoond voor verschillende zachte robotica-systemen die een gelijkenis vertonen met de biologie, bijvoorbeeld de slurf van een olifant of de tentakel van een octopus.

De belangrijkste bijdrage van de scriptie is een verzameling multidisciplinaire tools gecomprimeerd in één algemeen framework voor het ontwerp, de modellering en de controle van een klasse van zachte robots, variërend van het theoretische tot het experimentele domein.

\vspace*{11pt}\noindent
\textbf{Trefwoorden:} \ \ Zachte robots, hyper-redundante robots, ontwerpoptimalisatie, continuümmechanica, model-gebaseerd regelen, 3D-printen.

