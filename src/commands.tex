%!TEX root = ../thesis.tex

% Texts or
\newcommand{\ie}{\textit{i.e.}}
\newcommand{\eg}{\textit{e.g.}}
\newcommand{\blank}{\,\cdot\,}
\renewcommand{\emph}[1]{'\textit{#1}'}
\newcommand{\sorotoki}{\textup{\texttt{SOROTOKI}} }
\newcommand{\matlab}{\textup{\texttt{MATLAB}} }
\newcommand{\data}[1]{(\raisebox{-1.0pt}{\textcolor{#1}{\,\Large{\textbf{-}\!\textbf{-}}}})}
\newcommand{\dashdata}[1]{(\raisebox{-1.02pt}{\textcolor{#1}{\hspace{0.1em}\Large{\textbf{-\vspace{-0.1em}-}}}})}
\newcommand{\dotdata}[1]{(\raisebox{-1.0pt}{\textcolor{#1}{\,\Large{\textbf{-\vspace{0.1em}$\cdot$}}}})}
\newcommand{\ldata}[1]{\raisebox{-1.02pt}{\textcolor{#1}{\,\Large{\textbf{-}\!\textbf{-}}}}}
\newcommand{\ldatanum}[2]{\raisebox{-1.02pt}{\textcolor{#1}{\,\Large{\textbf{-}\!\!\textbf{-}}}}\textcolor{#1}{#2}}

\newcommand{\trot}[1]{\begin{turn}{90} #1 \end{turn}}
\newcommand{\RNum}[1]{\uppercase\expandafter{\romannumeral #1\relax}}

% Equations/math
\newcommand{\be}{\begin{equation}}
\newcommand{\ee}{\end{equation}}
\newcommand{\benn}{\begin{equation*}}
\newcommand{\eenn}{\end{equation*}}
\newcommand{\bea}{\begin{eqnarray}}
\newcommand{\eea}{\end{eqnarray}}
\newcommand{\beann}{\begin{eqnarray*}}
\newcommand{\eeann}{\end{eqnarray*}}
\newcommand{\ba}{\begin{align}}
\newcommand{\ea}{\end{align}}
\newcommand{\bpm}{\begin{pmatrix}}
\newcommand{\epm}{\end{pmatrix}}
\newcommand{\bbm}{\begin{bmatrix}}
\newcommand{\ebm}{\end{bmatrix}}
\newcommand{\bc}{\begin{center}}
\newcommand{\ec}{\end{center}}
\newcommand{\pwr}[1]{\cdot10^{\textrm{#1}}}

% Symbols and annotations
%\newcommand{\fB}{\boldsymbol{f}}
\newcommand{\maT}{\text{ma}}
\newcommand{\qR}{\mathrm{q}}
\newcommand{\RBB}{\mathbb{R}}
\newcommand{\rmsT}{\text{rms}}
\newcommand{\xBF}{\mathbf{x}}
\newcommand{\interior}{\operatorname{int}}

% Tikz figures
\newcommand{\SF}{1}                 % Scaling factor
\newcommand{\TS}{\normalsize}       % Text size
\newcommand{\lw}{0.7pt}             % Line width
\newcommand{\TSTick}{\small}        % Text size axis labels
\newcommand{\axislabels}[2]{\foreach \x/\y/\s in {#1} {\node[#2,inner sep=1mm] at (\x,\y) {\TSTick $\s$};}}
\newcommand{\wheel}[3]{ \draw[line width=\lw] (#1,#2) circle (#3);
                        \fill[bottom color=MRblue!60!black!80,top color=MRblue!10] (#1,#2) circle (#3-0.5*\lw);
                        \fill[color=MRblue!30] (#1,#2) circle (#3-\lw);
                        \fill[top color=MRblue!60!black!80,bottom color=MRblue!10] (#1,#2) circle (0.7*#3);
                        \fill[color=MRblue!30] (#1,#2) circle (0.7*#3-\lw);
                        \fill[color=black] (#1,#2) circle (0.1*#3);}

\newcommand{\CoM}[3]{\filldraw[inner color=white,outer color=black!7!white,draw=black,line width=\lw] (#1,#2) circle (#3+0.5*\lw);
                     \begin{scope}[xshift=#1,yshift=#2]
                        \clip(-#3,0) -- (0,0) -- (0,#3) -- (#3,#3) -- (#3,0) -- (0,0) -- (0,-#3) -- (-#3,-#3) -- cycle;
                        \fill[inner color=black!50!white,outer color=black] (0,0) circle (#3);
                     \end{scope}}

\definecolor{MRdarkblue}{RGB}{16,9,88}      % Define a set of colors to be used throughout thesis
\definecolor{MRred}{RGB}{152,0,0}
\definecolor{MRgreen}{RGB}{0,146,69}
\definecolor{MRblue}{RGB}{53,153,204}
\definecolor{MRorange}{RGB}{220,85,30}
\definecolor{MRlightgreen}{RGB}{217,224,33}
\definecolor{MRyellow}{RGB}{255,214,0}
\definecolor{MRgrey}{gray}{0.95}
\usetikzlibrary{arrows}
\usetikzlibrary{patterns}
\usetikzlibrary{decorations.markings}
\usetikzlibrary{shadings}
\usetikzlibrary{shapes}

% Counters
\newcounter{TermNum}        % Counter for the terminology
\newcounter{ContNum}        % Counter for the contributions
\renewcommand{\theContNum}{\Roman{ContNum}}
\renewcommand{\theTermNum}{\Roman{TermNum}}

% Other
\newcommand\blankfootnote[1]{%
  \let\thefootnote\relax\footnotetext{#1}%
  \let\thefootnote\svthefootnote%
}
\newcounter{numfootnote}
\newcommand\numfootnote[1]{%
    \stepcounter{numfootnote}%
    \newcommand{\thefootnote}{\thenumfootnote}%
    \footnote{#1}
}

\newcommand{\disclaimer}{\\[\baselineskip]  A detailed list of the differences between this chapter and the article on which it is based is provided in the %\hyperref[chap: Modifications]{\emph{Modifications}}
\emph{Modifications} chapter of this thesis.}
\newcommand{\itemheader}[1]{~\\ \noindent\textbf{#1.}\ \ }
\newcommand{\itemheaderNewpage}[1]{\newpage \noindent\textbf{#1.}\ \ }
\newcommand{\contribution}[2]{\refstepcounter{ContNum}#2 \vspace*{2.1mm}\begin{tcolorbox}[colback=black!2!white,colframe=black!20!white] \textbf{Contribution \Roman{ContNum}.} {\em #1} \end{tcolorbox}\vspace*{2.1mm}}
\newcommand{\objective}[1]{\begin{tcolorbox}[colback=black!2!white,colframe=black!20!white] {\em #1} \end{tcolorbox}}
\newcommand{\terminology}[2]{\refstepcounter{TermNum}#2 \vspace*{2.1mm}\begin{tcolorbox}[colback=black!2!white,colframe=black!20!white] \textbf{Terminology \theTermNum.} {\em #1} \end{tcolorbox}\vspace*{2.1mm}}
\newcommand{\cover}[1]{\ifprint{}\else\includepdf[pages=-]{#1}\cleardoublepage\fi}

\newcommand\tcircle[1]{%
  \raisebox{-0.25pt}{%
    \textcircled{\fontsize{8pt}{0}\selectfont #1}%
  }%
}

\newenvironment{Nomen}
    {\vspace*{-3mm}\begin{center}
    \begin{longtable}{p{.1\textwidth} p{.93\textwidth}}
    }
    {
    \end{longtable}
    \end{center}\vspace*{-1.2cm}
    }

\newcommand{\AddSymbol}[2]{#1 & #2 \\}