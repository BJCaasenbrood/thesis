%!TEX root = C:\Users\s118759\Documents\GitHub\ThesisSoftRobotics\main.tex
\section{Lie Group Theory for Robotics}
The analytical tools used in this work are derived from Lie group theory. Here, we give a brief preliminary on the basics of Lie groups and their associated Lie algebras whose properties will be used later for deriving the kinematic and dynamic model applicable to a set of soft robotic systems. 

The Lie group encompasses the concepts of `group' and `smooth manifold' in a unique embodiment: a Lie group $\mathcal{G}$ is a smooth manifold whose elements satisfy the group axioms. Within the perspective of robotics, the Lie group is viewed as a smooth surface on which the states of the system evolve, that is, the manifold describes or is defined by constraints imposed on the state. The smoothness of the manifold implies there exists a unique tangent space for each point on the manifold. The tangent space of the Lie group at the identity is called the Lie algebra, and it allows us to perform algebra computation concerning the Lie group.

