\section{Modern soft robotics boom in academia}
The precise date of the emergence of soft robotics as a prominent academic field is uncertain, but it is widely believed to have occurred in the mid-2000s. Since then, the field has undergone significant evolution and diversification, with research branching into areas such as active meta-materials, Additive Manufacturing (AM), morphological optimization, model-based control, and learning-based control. In the subsequent sections, we will examine several ongoing research topics in the field of soft robotics. These topics primarily concern the design and control of soft robots. It should be emphasized that while these problems may seem distinct, they are often interconnected. For instance, the structural design of a soft robot influences its workspace and consequently affects the feasibility of \textit{a-priori} defined control objectives. This section aims to underscore such scientific interconnections and demonstrate the essential components required to address common paradigms.

\section{Tailoring design of fluidic soft actuation}
\label{sec:C2:conventional_actuation}
As the name "soft robots" arises from their use of soft materials, it follows that design and fabrication using soft materials play a huge role in their technological development. Contrary to rigid robots, many soft robots explore whole body movement rather than localized regions undergoing motion called \textit{"joints"}. In classic robotics, robots are composed of a countable number of rigid links and joints \cite{Spong2006, Murray1994, Corke2011}, either arranged in series or parallel. Together, they span a workable range of motion called the operational workspace \cite{Spong2006} (see Chapter \ref{chap: introduction}). Focusing on rigid manipulators, whose base is often structurally fixated, their workspace can be obtained through a system of kinematics, often derived through a set of geometrical equalities. Rigid manipulators often have a bounded workspace (assuming actuation limits). In robotic locomotion, similar kinematic descriptions can be obtained for the legs and feet, with the exception of an additional free-floating base. In these cases, however, the workspace is of less interest; rather, the different possibilities of \textit{"gait cycles"} that arise from the link-joint configuration and actuator dynamics determine the system's success for locomotion.

%However, contrary to manipulators, an additional global free-floating frame is introduced (\eg, a coordinate frame at the center of mass) that describe the global coordinates of the full robotic body. As such, for robotic locomotion, the workspaces are often unbounded and thus carry less importance. Nonetheless, their structural layout of joints and links play a crucial role in energy consumption. High efficiency in the cyclic exchange in potential and kinetic energy is hugely beneficial to the duration of locomotion. In any case, either manipulators and locomotion machines, the topological layout of the links and joints are of paramount importance.
Returning to soft robots, the terms such as \textit{joints}, \textit{workspace} and \textit{gait cycles} also apply here. However, the high flexibility of soft robots allows for many non-restricted joint displacements, which makes deriving closed-form mathematical descriptions challenging. The shape of the workspace and locomotion patterns are mainly influenced by the geometry of the soft actuator, its flexibility modes, and how forces are transferred within the continuum soft body. Controlling the motion within soft actuation - reducing parasitic mobility and tailoring motion based on structural geometry - is an active topic in soft robotics research for decades. \\

\begin{figure}[!t]
  %\vspace{-2mm}
  \centering
  %\hspace{2mm}
  \includegraphics*[width=\textwidth]{./pdf/thesis-figurex-1-2.pdf}
  %% This file was created by matlab2tikz.
%
%The latest updates can be retrieved from
%  http://www.mathworks.com/matlabcentral/fileexchange/22022-matlab2tikz-matlab2tikz
%where you can also make suggestions and rate matlab2tikz.
%
\begin{tikzpicture}

\begin{axis}[%
width=0.975\textwidth,
height=0.227\textwidth,
at={(0\textwidth,0\textwidth)},
scale only axis,
axis on top,
clip=false,
xmin=0,
xmax=3000,
tick align=outside,
y dir=reverse,
ymin=0,
ymax=700,
axis line style={draw=none},
ticks=none,
axis x line*=bottom,
axis y line*=left
]
\addplot [forget plot] graphics [xmin=0.5, xmax=2849.5, ymin=0.5, ymax=650.5] {fig_actuation_types-1.png};
\node[right, align=left]
at (axis cs:262.5,780) {\small (a)};
\node[right, align=left]
at (axis cs:910,780) {\small (b)};
\node[right, align=left]
at (axis cs:1401,780) {\small (c)};
\node[right, align=left]
at (axis cs:1904.5,780) {\small (d)};
\node[right, align=left]
at (axis cs:2455.5,780) {\small (e)};
\end{axis}
\end{tikzpicture}%
  \vspace{-6mm}
  \caption{Various examples of continuum-bodied joint motions in modern soft actuation. (a) Soft actuator undergoing contraction by Yang et al. \cite{Yang2016}. (b) Set of serial-chain of bending soft actuator (STIFF-FLOP) by Cianchetti et al. \cite{Cianchetti2013Nov,Cianchetti2014} (c) Soft tentacle composed of twisting soft actuators. (d) Vine-inspired soft actuators capable of growth by Hawkes et al. \cite{Hawkes2017}. (e) Soft manipulator composed of bending and twisting soft actuators through laminate materials by Kim et al. \cite{Kim2019Aug}.}
  %\vspace{-6mm}
  \label{fig:C0:actuationtypes}
\end{figure}

\vspace{-4mm}
\textbf{A: Engineering principles in soft fluidic actuators}. In the past decade, researchers have developed various techniques for exploiting the high elasticity of soft materials for controllable actuation. One key development, with similar working principles to pneumatic muscle groups (see \cite{Mckibben,Morin1953} or Figure \ref{fig:C0:mckibben}), are Soft Pneumatic Actuators (SPAs). A few examples are shown in Figure \ref{fig:C0:actuationtypes}. SPAs undergo mechanics similar to McKibben actuators \cite{Mckibben} or Morin actuators \cite{Morin1953}, yet they encompass a diverse collection of motion besides uniaxial. Examples include contraction and elongation \cite{Yang2016}, axial growth \cite{Hawkes2017}, bending \cite{Mosadegh2014,Galloway2016,Marchese2016}, helical and twisting, and a hybridization of all the aforementioned motions \cite{Kim2019Aug}. An example of soft actuators capable of contraction is the Vacuum-Actuated Muscle-inspired Pneumatic (VAMP) structures by Yang et al. (2016, \cite{Yang2016}). Their work proposes a tailored geometrical structure embedded into a soft elastomer medium that is highly sensitive towards buckling. When subjected to a sufficiently large negative differential pressure, the internal structure undergoes a (reversible) mechanically unstable leading to uniaxial contraction, as seen in Figure \ref{fig:C0:actuationtypes}. Their work is inspired by a similar buckling behavior of patterned elastomer \cite{Bertoldi2008,Mullin2007,Shim2013Aug} subjected to axial loads. These muscle-inspired vacuum soft actuators are fast, produce a stable, repeatable motion; and more importantly, explore structural geometry to reduce parasitic motion. An example of soft bending actuators is the STIFF-FLOP system \cite{Cianchetti2013Nov}. Similar to the ORM system \cite{BibEntryOrm2019Sep}, it has three pressure chambers embedded into a soft cylindrical-shaped elastomer. To prevent ballooning, inextensible rings are placed orthogonal to the deformable backbone. Its design is also reminiscent of \cite{Suzumori1992,Suzumori1991}. Hawkes et al. \cite{Hawkes2017} developed a soft manipulator inspired by the growing behavior of vines. Kim et al. \cite{Kim2019Aug} used laminates that adhere to the volumetrically expanding soft body to govern the motion trajectory through bending and twisting.

\textbf{B: Exploring optimization and evolutionary algorithms}. Besides designing through engineering principles, optimization in soft robotics has been gaining momentum in recent years. Wang et al. \cite{Wang2020Nov} used topology optimization to find the optimal design for a cable-driven soft gripper (Figure \ref{fig:C0:optztypes}a). Similarly, Tian et al. \cite{Tian2020May} explored topology optimization for ferromagnetic soft grippers. Besides soft grippers, evolutionary design algorithms are also employed for soft mobile crawlers and swimmers. Joachimczak et al. \cite{Joachimczak2014Jul,Joachimczak2015} explored an evolutionary search algorithm with the purpose of automatically designing complex morphologies and controllers of multicellular, soft-bodied robots (Figure \ref{fig:C0:optztypes}c). Hu et al. \cite{Hu2019taichi} used a differential physics simulator called \texttt{DiffTachi} that efficiently computes gradient information for each simulation timestep. The gradient information can then be fed into neural network controllers to solve, for instance, the appropriate gait cycles required in the locomotion of soft crawlers, see Figure \ref{fig:C0:optztypes}d.

\begin{figure}[!t]
  \vspace{-3mm}
  %\centering
  %\hspace{2mm}
  %% This file was created by matlab2tikz.
%
%The latest updates can be retrieved from
%  http://www.mathworks.com/matlabcentral/fileexchange/22022-matlab2tikz-matlab2tikz
%where you can also make suggestions and rate matlab2tikz.
%
\begin{tikzpicture}

\begin{axis}[%
width=0.975\textwidth,
height=0.197\textwidth,
at={(0\textwidth,0\textwidth)},
scale only axis,
axis on top,
clip=false,
xmin=0.5,
xmax=3213.5,
tick align=outside,
y dir=reverse,
ymin=0.5,
ymax=650.5,
axis line style={draw=none},
ticks=none,
axis x line*=bottom,
axis y line*=left
]
\addplot [forget plot] graphics [xmin=0.5, xmax=3213.5, ymin=0.5, ymax=650.5] {./fig/fig_optimization_types-1.png};
\node[right, align=left]
at (axis cs:92,780) {\small (a)};
\node[right, align=left]
at (axis cs:551,780) {\small (b)};
\node[right, align=left]
at (axis cs:1365.5,780) {\small (c)};
\node[right, align=left]
at (axis cs:2430.5,780) {\small (d)};
\node[right, align=left]
at (axis cs:2960.5,780) {\small (e)};
\end{axis}
\end{tikzpicture}%
  \includegraphics*[width=\textwidth]{./pdf/thesis-figure-1-3.pdf}
  \vspace{-6mm}
  \caption{Optimization for design and motion of soft robots. (a) Soft gripper by Wang et al. \cite{Wang2020Nov}. (b) Evolutionary algorithms for multicellular soft-bodied robots by Joachimczak et al. \cite{Joachimczak2014Jul,Joachimczak2015}. (c) \texttt{DiffTachi} result for soft crawler by Hu et al. \cite{Hu2019taichi}. (d) Voxel-based optimization for Xenobots \cite{Kriegman2019}.}
  \label{fig:C0:optztypes}
  \vspace{-4mm}
\end{figure}

\vspace{-2mm}
\section{Gaining performance using feedback control}
\label{sec:C0:modelcontrol}
As the inherent properties of soft materials bring forth many benefits, such as adaptability, hyper-redundancy, and passivity with respect to the environment, they also hinder progress in model-based controllers. Earlier, we touched upon this subject with the rise of kinematic and dynamic models for hyper-redundant robotics in the late 80s to early 90s. Chirikjian et al. \cite{Chirikjian1992} provided a kinematic framework for hyper-redundant manipulators with applications to motion planning (see Figure \ref{fig:C0:modeltypes}a). Here, the elastic backbone is approximated using a modal formulation. Such modeling frameworks are one-to-one transferable to soft continuum manipulators. Mochiyama et al. \cite{Mochiyama1998,Mochiyama2003} extended this work to a dynamics formulation, even providing Lyapunov-based control strategies for shape regulation (Figure \ref{fig:C0:modeltypes}b). However, both modeling frameworks were computationally inefficient, lacking transferability to real-time control. The root problem stems from the fact that soft continuum robots, belonging in their exact formulation to the field of continuum mechanics, lead to infinite-dimensional models often expressed as Partial Differential Equations (PDEs). Rigid multi-body systems, like robot manipulators or mobile robots, on the other hand, have convenient Ordinary Differential Equation (ODE) structures as they are rooted in Lagrangian or Newtonian mechanical principles. Rigid-body models were (and still are) fast computationally, and their literature on controller design is vast and well-established \cite{Murray1994,Corke2011,Spong2006}. The computational issues in early continuum robots may be reflected by the literature gap between the 1990s and 2010s. \\

\textbf{A: Modern control-oriented models for soft robots.} In the past decade, significant steps have been made to address the issues of infinite dimensionality \cite{DellaSantina2021}. The key is to formulate a finite-dimensional approximation of the soft robot's dynamics such that they can be written as standard ODEs. Reduced-Order Models (ROMs) have paved the path for model-based controllers for soft robots, whose reduced formulations are both tractable and precise. In the years following its academic boom, many different assumptions and model approximations have emerged to address the issue of control.
%
\begin{figure}[!t]
  %\vspace{-3mm}
  \centering
  %\hspace{-8mm}
  \includegraphics*[width=1.0\textwidth]{./pdf/thesis-figure-1-4.pdf}
  %% This file was created by matlab2tikz.
%
%The latest updates can be retrieved from
%  http://www.mathworks.com/matlabcentral/fileexchange/22022-matlab2tikz-matlab2tikz
%where you can also make suggestions and rate matlab2tikz.
%
\begin{tikzpicture}

\begin{axis}[%
width=0.975\textwidth,
height=0.202\textwidth,
at={(0\textwidth,0\textwidth)},
scale only axis,
axis on top,
clip=false,
xmin=0.5,
xmax=3135.5,
tick align=outside,
y dir=reverse,
ymin=0.5,
ymax=650.5,
axis line style={draw=none},
ticks=none,
axis x line*=bottom,
axis y line*=left
]
\addplot [forget plot] graphics [xmin=0.5, xmax=3135.5, ymin=0.5, ymax=650.5] {./fig/fig_model_types-1.png};
\node[right, align=left]
at (axis cs:205,780) {\small (a)};
\node[right, align=left]
at (axis cs:818.5,780) {\small (b)};
\node[right, align=left]
at (axis cs:1399,780) {\small (c)};
\node[right, align=left]
at (axis cs:2008,780) {\small (d)};
\node[right, align=left]
at (axis cs:2702,780) {\small (e)};
\end{axis}

\begin{axis}[%
width=1.103\textwidth,
height=0.271\textwidth,
at={(-0.064\textwidth,-0.051\textwidth)},
scale only axis,
xmin=0,
xmax=1,
ymin=0,
ymax=1,
axis line style={draw=none},
ticks=none,
axis x line*=bottom,
axis y line*=left
]
\end{axis}
\end{tikzpicture}%

  %\vspace{-6mm}
  \caption{Popular modeling strategies for soft robotics. (a) Hyper-redundant modeling description through tensegrity by Chirikjian \cite{Chirikjian1992,Chirikjian1994}). (b) Analytical continuum beam description by Mochiyama \cite{Mochiyama1999,Mochiyama2003}). (c) Augmented rigid-body model subjected to PCC kinematics by Katzschmann et al. \cite{Katzschmann2019}. (d) Geometric Cosserat model subjected to PCS kinematics by Renda et al. \cite{Renda2018}. (e) Diamond-shaped soft robot manipulator controlled using the FEM-based \texttt{SOFA} software by Duriez et al. \cite{Duriez2013} and related \cite{Coevoet2017,Goury2018}.}
  \label{fig:C0:modeltypes}
  \vspace{-5mm}
\end{figure}
%
\begin{rmk}
 The reduced-order formulations for soft robotics are primarily applicable to slender soft robots, leading to a focus on soft robot manipulators in control-oriented studies. This approach is well-motivated given that many soft robots have one dominant physical dimension compared to the other two \cite{DellaSantina2021}. As such, this thesis primarily focuses on the modeling and control of soft manipulators rather than a broader scope.
\end{rmk}
%
\par
Focusing on soft robot manipulators, a popular choice of finite-dimensional reduction is the so-called Piecewise Constant Curvature (PCC) soft beam model. The PCC modeling approach is by far the most adopted in the soft robotics community \cite{Webster2010}. As the name implies, the soft robot is modeled as an elastically deformable beam with all strains but curvature neglected. Examples of this approximation include \cite{Falkenhahn2015,Marchese2016,Katzschmann2018,Katzschmann2019,Runge2017,Franco2020,Webster2010,DellaSantina2020a}. Highlighting a few, Katzschmann et al. \cite{Katzschmann2018} proposed to connect the PCC formulation to an augmented rigid robot dynamical model with parallel elastic actuation (see Figure \ref{fig:C0:modeltypes}c). A similar approach was proposed earlier by Falkenhahn et al. \cite{Falkenhahn2015} and applied to Festo's bionic arm \cite{Grzesiak2011}. Although such lumped models may seem like a major oversimplification, the proposed model allows sufficient speed and accuracy such that model-based feedback is applicable. This formulation was also employed later in an adaptive sliding mode control scheme \cite{Kazemipour2022May} for the SoPra soft arm \cite{Toshimitsu2021Sep}. Following, Renda et al. \cite{Renda2018} extended the PCC model to Piecewise Constant Strain (PCS) formulation. This formulation allowed for all strain if and only if considered spatially piece-wise constant (Figure \ref{fig:C0:modeltypes}d). Rooted in $\SE{3}$ geometry of the Cosserat approach \cite{Simo1986}, it provided a closer relation with the rigid body geometry of traditional robotics. The formulation also extends to soft manipulators with fluidic actuation \cite{Renda2017Aug, Till2019}. The PCS model was later employed in feedforward controllers by Thuruthel et al. \cite{Thuruthel2018Nov} using model-based policy learning algorithms. To improve efficiency, a recurrent neural network was trained using an offline PCS model. Grazioso et al. \cite{Grazioso2019} explored a similar path of geometric Cosserat beams using helical strain functions. Nevertheless, Constant Strain (CS) models have severe limitations. They often do not originate from continuum mechanics and thus are only applicable in restrictive settings. Although computationally performance might surpass continuous models, due to intrinsic kinematic restrictions, they are unable to capture important continuum phenomena, like buckling, environmental interaction, or wave propagation.

In response to its limitations, many researchers continued their search for efficient and more generalizable alternatives. Della Santina et al. \cite{DellaSantina2020} proposed a polynomial description to describe the continuum dynamics, a description analogous to \cite{Chirikjian1992}. In their work, they expressed the curvature function of the soft robot in terms of a standard polynomial basis. Not only can an exact infinite-dimensional formulation of the problem be obtained (in theory), but truncation at any level is easily changed. The technique is also widely used for flexible-link robot manipulators to capture small vibrations; see DeLuca et al. \cite{DeLuca2016Jul}. Della Santina et al. also showed that PCC-rooted assumptions as control output produce a minimum-phase system \cite{DellaSantina2020} -- a fundamental stepping-stone for nonlinear control. Following, Boyer et al. \cite{Boyer2021} extended upon their prior Cosserat models \cite{Renda2018,Renda2020} and presented a tractable and generalizable beam model for slender soft manipulators. A similar approach to \cite{DellaSantina2020} was followed, but all strains are discretized using a finite set of strain basis functions. Renda et al. \cite{Renda2020} improved computation by introducing a two-stage Gauss quadrature \cite{Zanna1999} to derive the Magnus expansion \cite{Hairer2002}. Other examples of the Cosserat beam descriptions, but more focused on the continuum mechanics rather than control, are the work of Gazzola et al. \cite{Gazzola2018}. Their work allowed for efficient Cosserat beam models suitable for self-collision, thus providing various simulations for twirling and coiling of beams under increasing torsional loads.

Another popular alternative, better suited for general soft robotic systems like soft mobile robots, is reduced-order finite element models \cite{Duriez2013,Coevoet2017,Coevoet2017Feb,Goury2018,Thieffry2017,Thieffry2020,Tonkens2021May,Katzschmann2019Apr,Wu2021Feb,Zhang2017} or neural-network trained using offline FEM simulations \cite{Fang2020Dec}. Starting from high-order FEM data (e.g., state dimensions of the order $10$k) that capture the whole workspace spanned by the network of soft actuators, Proper Orthogonal Decomposition (POD) techniques are employed to drastically reduce the state dimension of the soft robot model. These techniques can even retain external loads (e.g., contact and friction) with precision as long as they are included in the offline data set \cite{Goury2018}. By far, this FEM-driven method has shown the most success in the experimental control regime.

\textbf{B: Soft robot simulation and programming environments}.
The rapid development of soft robotic models in recent years has also increased the demand for (open-access) software packages, especially since many of the aforementioned ROM models require an advanced level of mathematical understanding. In an attempt to help the soft robotics community, many researchers have provided open-source, documented simulators interwoven in their soft robotics research. A popular FEM-based software on soft robotic modeling and control is \texttt{SOFA} by Duriez et al. \cite{Duriez2013,Coevoet2017}. \texttt{DiffTachi} by Hu et al. \cite{Hu2019taichi, Hu2019Oct} explores differential simulations to produce soft machines capable of locomotion. Bern et al. \cite{Bern2022,Bern2019} developed \texttt{SoftIK}, a software for soft deformable plushy robots. Among beam or rod-based models, there exist many options. Examples include \texttt{Elastica} (or \texttt{pyElastica} \cite{Tekinalp2022}) by Gazzola et al. \cite{Gazzola2018,Zhang2019}, \texttt{TMTDyn} by Sadati et al. \cite{Sadati2020}, \texttt{SimSOFT} by Grazioso et al. \cite{Grazioso2019}, and \texttt{SoRoSim} by Mathew et al. \cite{Mathew2021Jul} based on the work of Renda et al. \cite{Renda2020} and Boyer et al. \cite{Boyer2021}. The \texttt{Sorotoki} toolkit by Caasenbrood et al. \cite{SorotokiCode}, an open-source software package presented as a part of this thesis, explores a combination of FEM models and soft beam models. The toolkit, written in Matlab, aims to bridge the gaps between design, modeling, and control of (hyper-elastic) soft robots.

\textbf{C: Closing the loop in soft robotics}. Following the many developments in computational efficiency of reduced-order models (and accordingly the advances in soft sensing), academic research in model-based or model-free control for soft robots is significantly growing since early 2019. Note that, feed-forward controllers for soft manipulators have been proposed years prior, \eg, \cite{Falkenhahn2015May,Falkenhahn2015,Thuruthel2017Oct,Satheeshbabu2019May}.

So far, the PCC model has been more intensively validated experimentally than other models. In Della Santina et al. \cite{DellaSantina2020a}, the augmented rigid-body PCC model was used to design a closed-loop controller for a continuous soft manipulator, presenting two architectures designed for dynamic trajectory tracking and surface following. Prior work is provided in \cite{Katzschmann2019}. A similar approach was followed by Milana et al. \cite{Milana2021} and applied to an artificial soft cilia. They showed that soft bending actuators could mimic the asymmetric motion of the cilia through model-based control. Cao et al. \cite{Cao2021Apr} explored a reduced analytical model \cite{Wang2019Apr}, which is somewhat equivalent to a linear pendulum model (apart from quadratic terms in the potential force), to develop robust tracking controllers without velocity observers. Their controller was tested experimentally on a soft PneuNet actuator. Wang et al. \cite{Wang2022Mar} developed a computed torque controller (see \cite{Spong2006}) using the augmented rigid-body PCC model and applied it to a soft Honeycomb Pneumatic Network Arm \cite{Jiang2016Dec}. Franco et al. (2020, \cite{Franco2020,Franco2020Jan}) used a port-Hamiltonian modeling framework akin to the rigid-body PCC model (\ie, three-link pendulum) and applied such principles to energy-shaping controllers. The performance of their controller was assessed via simulations and experiments on two soft continuum prototypes. On a side note, Franco et al. \cite{Franco2022May} also developed an energy-shaping control law together with nonlinear observers for the control of soft growing robots \cite{Hawkes2017} with pneumatic actuation subject to the (ideal) gas laws.

As mentioned previously, the PCC model has significant limitations that raise questions about the usability, dexterity, and robustness of its control derivation. Although primarily focused on simulation, higher-order dynamics have been used for the development of feedback controllers in soft manipulators. Della Santina et al. (\cite{DellaSantina2021}) developed swing-up controllers for a soft pendulum modeled by the affine curvature models (i.e., a polynomial curvature model \cite{DellaSantina2020} of order $k = 2$). Their approach mirrors the path of the classic control problem of inverted pendulums in the 90s and early 2000s \cite{Spong1996,Spong1996a,Ortega1998,Shiriaev1999Dec}. Later, Weerakoon et al. \cite{Weerakoon2021Dec} extended their work by introducing a revolute base. A common control problem here is under-actuation \cite{Tedrake2022,Spong2006,Murray1994} -- implying that not all control actions can be realized to steer the configuration space to a desired position. In the work of Borja et al. \cite{Borja2022Apr}, they developed a general control framework that can stabilize soft manipulators based on potential energy shaping using the affine curvature model. Their work showed that some linear matrix inequalities can be derived based on the gradient of the potential energy related to the passive and active states, such that local stability can be proven. In layman's terms, elasticity must dominate the forces resulting from gravity in the underactuated states for (potential) energy-shaping controllers (and mostly others) to work.In summary, the development of closed-loop control for multi-modal soft robotic deformations is a promising area of research. However, at present, such control mechanisms are limited to simulated environments and have yet to be fully realized in practical applications.

\section{Summary}
In summary, we have provided a comprehensive overview of the evolution of soft robotics, from its inception in the 1950s to current research trends. The chapter traces the origins of soft robotics in pneumatic muscles and explores the emergence of novel concepts and technologies that have facilitated the development of increasingly sophisticated soft robots through the utilization of exotic material properties. We highlight significant milestones in the field, including the creation of the first soft gripper, integration of robotics with soft actuation, and introduction of design, modeling, and control principles for these systems. Additionally, the chapter discusses key challenges facing the field and serves as a basis for standardizing terminology.