%!TEX root = ../../thesis.tex
\clearpage
\section{Research objectives and contributions}
In this section, the research problem of the thesis is divided into three research challenges for the given topics: design synthesis of soft actuators, modeling of soft robot manipulators, and control of soft robot manipulators. For each research challenges, a contribution of this thesis is presented. \\

\noindent (\textbf{Design synthesis of soft actuators}): As presented by abundance of literature on soft robotic design, either rooted in engineering principles or optimization, achieving an optimal structural geometry that fully accounts for hyper-elasticity in soft materials is no easy feat. Unlike their rigid counterparts and many biological systems for that matter, any external inputs will result in parasitic motion in soft actuators; mainly as distributed continuum deformations that are antagonistic to the input. These parasitic motions -- or better phrased \textit{passive joint displacements} -- lead to imprecisions in the mechanical operation and lost of mechanical efficiency. The challenge of design soft actuators by exploring the intrinsic flexibility of the system is a new 
% \\
\par The main principle of any compliant mechanical device can be described rigorously through continuum mechanics.

\contribution{Development of efficient algorithms, applicable to the general design of soft actuators, that solve the inverse design problem: Given a desired motion and input, what is accordingly the optimal (soft) material distribution within a design domain to realize such joint motion?}{}

Following

\contribution{Fabrication of an array of computer-optimized pneumatic soft actuators through Additive Manufacturing, whose collective assembly can be explored for soft robotic manipulation.}{}

\noindent (\textbf{Modeling of soft robot manipulators}):