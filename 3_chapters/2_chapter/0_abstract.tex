%!TEX root = ../../thesis.tex
\chapterabstract{
%The motion complexity and use of exotic materials in soft robotics call for accurate and computationally efficient models intended for control. To reduce the gap between material and control-oriented research, we build upon the existing Piecewise-Constant Curvature framework by incorporating hyper-elastic and visco-elastic material behavior. 
In this work, the continuum dynamics of the soft robot are derived through the differential geometry of spatial curves, which are then related to Finite-Element data to capture the intrinsic geometric and material nonlinearities. To enable fast simulations, a reduced-order integration scheme is introduced to compute the dynamic Lagrangian matrices efficiently, which in turn allows for real-time (multi-link) models with sufficient numerical precision. By exploring the passivity and using the parametrization of the hyper-elastic model, we propose a passivity-based adaptive controller that enhances robustness towards material uncertainty and unmodeled dynamics -- slowly improving their estimates online. As a study case, a fully 3D-printed soft robot manipulator is developed, which shows good correspondence with the dynamic model under various conditions, e.g., natural oscillations, forced inputs, and under tip-loads. The solidity of the approach is demonstrated through extensive simulations, numerical benchmarks, and experimental validations.}
