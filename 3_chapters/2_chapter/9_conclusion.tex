%!TEX root = ../../thesis.tex
\noindent In this chapter, we aimed to reduce the gap between modeling and control-oriented research in soft robotics. First, the dynamic models that describe the continuum-bodied motions need to be sufficiently accurate, and second the model must retain real-time performance to be applicable in control. By building upon the existing PCC models, we express the continuum deformation using a minimal set of coordinates related to the differential geometry of spatial curves; and explored FEM-based data to model the hyper-elasticity. To retain numerical efficiency, a reduced-order integration scheme is developed that efficiently computes the entries of the Lagrangian model through a Matrix-Differential equation; resulting in a continuum dynamical model for soft manipulators with real-time capabilities at minimal lost in numerical precision. 

The dynamic model has been extensively corroborated through simulations and experimental results. Not only does the dynamic models allow for real-time simulations with systems with various degrees of motion, it show good correspondence with the true physical soft robot. Furthermore, a passivity-based adaptive controller is proposed that provides good tracking performance even in the face of parameter uncertainties. The adaptive controller enables online estimation of the hyper-elastic stiffness and external loads, which further enhances the robustness toward modeling uncertainty undoubtedly present in soft robotics. In future work, we wish to further explore FEM-driven data for the parametrization of the spatial shape functions -- extending beyond constant-curvature, and employ the proposed controller methods to multi-link soft robots.
