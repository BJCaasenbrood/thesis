%!TEX root = ../../thesis.tex
\noindent We previously expressed the position and velocity kinematics as explicit functions of the generalized coordinates (i.e., Bishop parameters) and their time-derivatives. This explicit dependency stems from the PCC conditions inferring the curvature is non-varying along the spatial domain $\Xs$, i.e., $\kappa(q,\sigma) = \kappa(q)$. Although sufficient for some cases, the condition is generally restrictive, and to some extent inconvenient, since the inclusion of multiple links demands piece-wise integration of the kinematics \eqref{eq:pos_vector}, \eqref{eq:phi_exact}, \eqref{eq:vel_cont}, and \eqref{eq:acceleration}. Rather than separation of integration, we can extend this PCC description by using piece-wise continuous spatial function to distinguishes multiple soft-bodied links along the continuous body of the soft robot. The idea of parametrization through shapes functions has been explored earlier by Chirikjian et al.\cite{Chirikjian1994,Chirikjian1992}, and later by Boyer et al. \cite{Boyer2021}, Della Santina et al. \cite{Santina2020b}. A similar discontinuous shape function series was used by Berthet-Rayne et al. \cite{Berthet2021} to pursue multi-body dynamics for growing continuum robots; and proposed by Chirikjian \cite{Chirikjian1992} for hyper-redundant robots earlier.

Following the aforementioned works, let us parameterize the the geometric vectors $\Gamma$ and $U$ for a $N$-link soft robot through the product of a basis of orthonormal functions $\!\{s_i\}_{i \in \N}$ and the Bishop parametrization as follows
%
\begin{align} \Gamma(q,\sigma) & = \sum^N_{i=1} s_i(\sigma) \ceil{J^*}_3
\,\tilde{q}_i, \label{eq:theta_extent} \\ U(q,\sigma) & = \sum^N_{i=1} s_i(\sigma)
\floor{J^*}_3\,\tilde{q}_i + U_0, \label{eq:h_extent} \end{align}
%
where $J^*$ is the joint-axis matrix as in \eqref{eq:adjoint_matrix}, the mathematical operators $\ceil{\cdot}_3$ and $\floor{\cdot}_3$  extract the first or last three rows of a matrix, respectively;  $\tilde{q}_i$ the joint variables of the $i$-th link, and $s_i: \Xs \mapsto \{0,1\}$ is a piece-wise continuous shape function, whose purpose is to be non-zero for a given interval on $\Xs$.
The new generalized coordinate vector becomes the aggregate of all joint variables of the multi-body soft robotic system $q =  \left(\tilde{q}_1^\top,\,\tilde{q}_2^\top,...,\,\tilde{q}_N^\top \right)^\top$ with the vector $\tilde{q}_i = (\varepsilon_{i},\, \kappa_{x,i},\,\kappa_{y,i})^\top$ relating to the Bishop parametrization of the $i$th-link. Given \eqref{eq:theta_extent} and \eqref{eq:h_extent}, we may now rewrite the velocity-twist as
%
\begin{equation} V(q,\dot{q},\sigma) = \Ad_g^{-1}
\int_0^\sigma \Ad_g J^* S(\sigma) \; d\sigma \dot{q} := J(q,\sigma) \dot{q}
\label{eq:vel_vec_dis} \end{equation}
%
where $S = (s_1,\,s_2,\,...,s_N) \otimes I_n$ is an unitary selection matrix derived from the basis of piece-wise continuous shape functions $\!\{s_i\}_{i=1}^N$. To be less ambiguous about this selection matrix $S$, lets consider a spatial coordinate $\sigma_2 \in [L_1,L_1+L_2]$ that lies on the spatial interval of the second link. Consequently, the operation $S(\sigma_2) {q} = {\tilde{q}}_2$ returns the corresponding joint variable of the second link. This selection of
generalized coordinates follows analogously for other links along the serial-chain of the soft manipulator. We provided a small library of piece-wise continuous shape functions upto $1 \le N \le 8$ links under \texttt{./src/pwf} on the open repository\cite{Caasenbrood2021}.
Now, substitution of the discontinuous variation of the geometric Jacobian in \eqref{eq:vel_vec_dis} into \eqref{eq:kinetic_energy} leads to the dynamic model of a $N$-link soft robot manipulator in the
Lagrangian form similar to \eqref{eq:dynamic_model}.
