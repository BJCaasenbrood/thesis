%!TEX root = ../../thesis.tex
Traditional robots are made from rigid and dense materials that ensure accurate and repeatable motions. While rigid robotics excel at fast and precise motion, their structural rigidity lacks the compliance and mechanical robustness needed for safe and passive interaction in an unknown environment. Soft robotics is a field of robotics that aims to improve the motion complexity and environmental robustness that is generally lacking its rigid counterpart. To further promote these topics in robotics, researchers aim to mimic living creatures by developing bio-inspired robots with similar morphologies and mechanical properties \cite{Falkenhahn2015,Suzumori1991,Godage2015,Godage2016,Marchese2014,Kriegman2019}.
In soft robotics, the hyper-flexible and continuum-bodied structure provide them with a rich family of motion primitives. Besides bio-mimicry, soft robotics has proven to be a prominent alternative for rigid robotics with a variety of applications, e.g., manipulation and adaptive grasping \cite{Galloway2016}, untethered locomotion and exploration through uncertain environments \cite{Marchese2014,Choi2011,Pilz2020}, rehabilitation \cite{Polygerinos2015}, and even minimal-invasive surgery \cite{Li2017a,Cianchetti2014}. Although popularity of the field has increased exponentially in recent years, one of the first soft robots date back already to the early 1990's, e.g., the work of Suzumori et al. \cite{Suzumori1991}. Yet, despite years of soft robotics research, their intrinsic hyper-flexible nature still possesses numerous challenges on modeling and control.

One major challenge in modeling is that the soft robot's elastic body undergoes large, continuous deformation. Since its inception, numerous works have addressed the kinematics for soft continuum robots\cite{Jones2006,Mochiyama2003}; yet, its original framework stems from to hyper-redundant robotics nearly a decade earlier\cite{Chirikjian1992,Chirikjian1994}. Similar to soft robots, hyper-redundant robots exploit their high joint redundancy to achieve a wider ranges of tasks (e.g., shape control and collision avoidance) besides end-effector manipulation. To some extent, soft robots can be seen as the successor to hyper-redundant robots in which rigid mechanical joints or links are substituted with hyper-flexible soft elements. As a result, their dynamics involve a continuously deformable inertial body rather than the classical notion of rigid bodies. As such, conventional modeling approaches cannot be applied directly to these continuously deformable robots, stressing the importance of novel modeling strategies. In this respect, the dynamics of a continuously deformable soft robot are in theory of an infinite-dimensional nature. This paradigm shift has further emphasized the challenges in control-oriented modeling of soft robots; as their physical description are often more suited for a Partial Differential Equations (PDEs) rather than Ordinary Differential Equations (ODEs).

Recently, some significant steps have been made towards formulating reduced-order ODE models for elastic continuum soft robots, paving a path towards model-based controllers. Perhaps one of the most popular techniques of spatial reduction is the so-called Piece-wise Constant Curvature (PCC) model. The PCC model assumes that the continuum shape can be described using a number of spatially-constant curves which are parametrized using a set of generalized coordinates. Although PCC models can be seen as a significant oversimplification of true continuum mechanics at hand, these models have proven to be remarkably viable for various control applications. Besides its use in inverse kinematic control \cite{Marchese2014,Marchese2016,Jones2006}, PCC models have also shown to be suitable for feedforward controllers as demonstrated by Falkenhahn et al. \cite{Falkenhahn2015}; and more recently, closed-loop feedback controllers by Della Santina et al. \cite{Santina2020,Katzschmann2019}. Although the aforementioned works utilize the lumped-mass description, others have employed PCC models with uniform mass distribution \cite{Renda2018,Godage2015,Godage2016,Tatlicioglu2007} and current models even extend beyond the constant curvature \cite{Mochiyama2003,Chirikjian1994,Santina2020b}. However, in the face of significant external loading or (distributed) contact with the environment, the PCC assumption is relatively conservative and leads to undesired kinematic constraints on the continuum deformation. Besides, these models often need additional identification to model the compliance as they do not originate from continuum mechanical framework.

On the other hand, Finite-Element Method (FEM) models do originate from continuum mechanics and due to their PDE description provide a more accurate representation of deformations; and are particularly suited to deal with geometric and material nonlinearities. Duriez et al.\cite{Duriez2013} and related works\cite{Coevoet2017,Largilliere2015} showed that reduced-order FEM models can play an important role in closed-loop control -- allowing accurate volumetric deformation and hyper-elastic behavior. Although such real-time simulations for FEM-based models are possible, a significant state-reduction is required to ensure sufficient computational speed. In the process, FEM-based models often loose desirable control properties, e.g., passivity preservation, which might play an important role in control. An alternative modeling strategy is the recently emerging geometrically-exact Cosserat-beam model. Similar to the PCC models, the Cosserat models have the merit benefit that they can be structured into a standard Lagrangian form -- the basis for robotics control theory. Rooted in a geometric method for describing the continuum mechanics using Lie theory\cite{Simo1986}, Boyer et al. \cite{Boyer2010, Boyer2021} proposed a geometrically-exact modeling framework for Cosserat beams using nonlinear parametrization of the strain field. Other examples include the work of Renda et al.\cite{Renda2018,Renda2020} providing various options for Piecewise-Constant Strain (PCS) and Variable Strain modeling approaches. Although recent variants of the Cosserat models offer good computational performance \cite{Till2019,Grazioso2019}, its use in model-based control is slowly upcoming.

In this respect, the topic of reduced-order modeling of soft robots is an active area of research. Yet, a challenge that is frequently overlooked in control-orientated research is the anisotropic material behavior, mechanical saturation, and more importantly, the nonlinear and possibly time-varying nature of the highly hyper-elastic soft materials \cite{Falkenhahn2015, Mochiyama2003, Till2019, Tatlicioglu2007}. This is further amplified by the fact that soft robots are known for their diversity in elastic materials and corresponding morphologies. Mustaza et al.\cite{Mustaza2019} proposed modified nonlinear Kelvin–Voigt material model to embody the complex material behavior of silicone-composite manipulators (so-called STIFF-FLOP actuators). A similar silicone composite actuator was experimentally validated by Sadati et. al \cite{Sadati2021} who proposed a novel modeling approach with an appendage-dependent Hookean model and viscous power-law to describe nonlinear and time-dependent material effects, respectively. Both nonlinear material models show good correspondence with physical soft robots under various dynamic conditions, yet they lack general transferability to the soft robots with different geometries -- intrinsically captured by FEM-driven models. As of today, there are little control-oriented models that both offers geometry and material versatilely similar to FEM-models and the control convenience similar to spatial curve models.

Ultimately, the strong nonlinearities paired with its continuous nature encourage the use of model-based controllers. Nevertheless, regarding the aforementioned model-based control approaches \cite{Santina2020,Katzschmann2019,Falkenhahn2015}, the stability and performance of the closed-loop system could be undermined by uncertainties in physical parameters or unmodelled dynamics. Particularly for state-feedback linearization (e.g., inverse dynamic), as the inversion of inaccurately estimated systems could lead to poor performance and even instability. Adaptive control \cite{Slotine1988,Morgan1977} or energy-based controllers \cite{Ortega1998} might offer the needed robustness towards material uncertainties and unmodelled dynamics. Unfortunately, up till now, the  applicability of adaptive and energy-based control techniques on soft robotics is scarcely explored. Franco et al. \cite{Franco2020} used an adaptive energy-based controller that compensate for external disturbances on the end-effector, yet these controller can be extended to include various slowly-varying material uncertainties, e.g., hyper-elasticity and viscosity.

The contributions of the work are two-fold. First, to derive a finite-dimensional dynamic model of a continuum soft robot, where we briefly recapitulate on existing modeling technique for soft robot manipulators. To address the issue of infinite-dimensionality, we explore the PCC condition that allows for a low-dimensional description of the continuum dynamics. Although such modeling approaches have been thoroughly developed, we will address two issues that will aid the development of model-based controllers. We aim to bridge the gap between the PCC model and the underlying continuum mechanics by matching the quasi-static behavior to a Finite-Element-driven model (FEM); and we propose a reduced-order integration scheme using Matrix-Differential Equations (MDEs) to compute the spatio-temporal dynamics in real-time. Preliminary results of this work were shown in Caasenbrood et al.\cite{Caasenbrood2020}.
%

Second, in regards to the FEM-based hyper-elastic modeling and the possible presence of unmodelled dynamics (e.g., material uncertainties or external loads on the end-effector), a passivity-based adaptive controller is proposed that enhances robustness towards material uncertainties and unmodelled dynamics in closed-loop, slowly improving their estimates online. All source code is made publicly available at Caasenbrood et al.\cite{Caasenbrood2021} \highlight{(see the open software repository)}

% To summarize, the contributions of this work include:
%
% \begin{itemize}
% \item An extension to PCC dynamic models for continuum soft robots such that hyper- and visco-elasticity can be incorporated through FEM-driven data.
% %
% \item An efficient numerical solver for the continuum dynamics using a mix of Matrix-Differential Equations (MDE) and implicit time-integration -- allowing for real-time simulations of multi-link soft manipulators.
% %
% \item Experimental validation of the proposed hyper-elastic soft robot model under various dynamic conditions: unforced, forced, and external tip-loads.
% %
% \item A passivity-based adaptive controller to improve robustness towards uncertain material parameters and unmodelled residual dynamics.
% %
% \item An open-source modeling software\cite{Caasenbrood2021} written in MATLAB to perform real-time simulations of soft robots, and allow for model-based controller design.
% \end{itemize}
% %
% This work is organized as follows. We first briefly detail our physical soft robotic system, followed by a modeling framework to assess the continuum dynamics of hyper-elastic soft robot. Next, we describe the material identification through FEM-driven data. Then, we discuss the passivity-based controller design and its adaptive law to estimate the hyper-elastic parameters online. Conclusively, the numerical simulations and experimental validation results will be presented, followed by a brief conclusion.

\section{Design and fabrication}
By using additive manufacturing, we developed a soft and flexible robot manipulator that is suitable for pick-and-place application. The 3-DOF soft robot can be seen in Figure \ref{fig:soft_robot}. The soft robot manipulator in this work is loosely inspired by the elephant whose trunk-appendage consist mainly of parallel muscles without skeletal support. The anatomy of elephant's trunk provides an excellent study case, as they naturally exhibit continuum-body bending and moderate elongation\cite{Falkenhahn2015,Jones2006,Tatlicioglu2007}.
%
Similar to the earlier soft robotic designs \cite{Suzumori1991,Falkenhahn2015}, the developed soft robot can undergo three-dimensional movement by inflation or deflation of embedded pneumatic bellow network. The soft robot can achieve bending in any preferred direction by differential pressurization of each channel ($<$0.1 MPa). Whereas, simultaneous pressurization accomplishes moderate elongation.

The soft robot is exclusively composed of a printable, flexible thermoplastic elastomer (Young's modulus $\le$ 80 MPa), which intrinsically promotes softness and dexterity. The elastomer material is developed explicitly for Selective Laser Sintering (SLS), a 3-Dimensional (3D) printing method that uses a laser to solidify powdered material. The main advantage of SLS printing over other techniques is that the printed parts are fully self-supported, which allows for complex and highly detailed structures. It should be mentioned that the layer-by-layer material deposition will introduce undesired anisotropic mechanical effects. To mitigate anisotropy, the bellows are printed orthogonal to the printing plane, thereby ensuring mechanical symmetry. For the majority of this work, the 3D-printed soft robot in Figure \ref{fig:soft_robot} will form the basis of the dynamical model. The 3D-model is made available at the open repository \cite{Caasenbrood2020}.
