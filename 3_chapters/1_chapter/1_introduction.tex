%!TEX root = ../../thesis.tex
The field of soft robotics has attracted the interest of many researchers from different backgrounds. Soft robots use compliant and hyper-elastic materials, while the use of rigid materials is minimized. The introduction of soft materials into robotics greatly expanded the field of application for robotics. For example, due to their dexterity and environmental robustness, soft robots are often used in medical applications \cite{Polygerinos2015, Yap2015, Asbeck2015Nov}, adaptive grasping \cite{Galloway2016, Hughes2016}, and locomotion in uncertain environments \cite{Drotman2017}. Unlike its rigid counterpart, soft robots undergo large continuum-bodied motion that, to some extent, resembles morphologies found in nature. These morphologies arise by virtue of the low compliance in soft materials and, more importantly, the structural layout of the soft robot. As of today, many of the fundamental engineering principles in rigid robotics, like design, actuation, sensing, and control, are often not applicable to soft robotics systems. Since its inception, most of these engineering problems have remained challenging or unresolved.

Although the diversity in soft robotics is significant, ranging from adaptive grippers to soft manipulators, most topologies in soft robotics can be associated with nature or engineered geometries for minimal compliance (e.g., bellow shapes). Soft robots often mimic living creatures and their morphologies, e.g., the tentacle of an octopus \cite{Galloway2016, Wehner2016}, or the trunk of an elephant \cite{Drotman2017}. Hypothetically, the abundance of bio-mimicry in soft robotics might be associated with the design complexity of developing robots from soft materials. The large number of degrees-of-freedom and exotic mechanical nature of soft robots makes design significantly challenging, and consequently, the design process can be iterative and time-consuming \cite{Wehner2016}. Therefore, it becomes potentially advantageous to use computational tools that assist or develop appropriate soft robotic topologies given a set of user-defined requirements, like desired motion or force.

In the past, researchers have made efforts to finding morphologies through mathematics, in particular through evolutionary algorithms. The concept of automated creature designs was first introduced by Sims \cite{Sims1994}, who showed that, given a set of basic geometries, locomotive organisms could be generated from evolutionary algorithms. These virtual organisms resembled biological morphologies to some extent; however, the complexity of the material layout was limited. More recent work involving the synthesis of virtual soft robots includes Cheney et al. \cite{Cheney2013}, who successfully produced intricate locomotive morphologies using artificial neural networks and multi-material parameter spaces of active and passive soft voxels. Other work involving morphological synthesis includes \cite{Bern2019, Morzadec2019,Diepen2019}. Unfortunately, the synthesis of morphologies from previous approaches, though novel, remains only in ideal simulated environments. An accurate representation of the nonlinear material properties in soft robotics can be challenging, and in favor of computational efficiency, little detail is spent on the nonlinear nature governing soft materials. Besides, these evolutionary frameworks typically involve a network of `activation' cells or voxels that perform ideal volumetric deformation, biologically resembling muscle functionality while unfortunately lacking resemblance to conventional actuation in soft robotics, e.g., pneumatics, dielectrics, and smart metal alloys (SMA).

Reviewing previous methods, a more efficient approach for solving the optimal morphology might be founded in topology optimization. Topology optimization is the general formulation of a material distribution problem for mechanical solids, where density-based topologies arise throughout an iterative (non-convex) optimization procedure. The synthesis of compliant mechanisms through topology optimization is investigated thoroughly \cite{Sigmund2015, Gain2013, Luo2015}; however, its application to soft robotics is relatively unexplored \cite{Zhang2018,Zolfagharian2019}. Yet, to obtain meaningful topologies for soft robotics, two problems need to be addressed. Since soft robots undergo large deformations, it becomes necessary to describe the nonlinear geometrical deformations accurately. Inherent to significant deformation of soft materials is the importance of nonlinear material behavior, like hyperelasticity. Another concern is the design-dependency of the external forces, in our case, the pneumatic loads. This class of structural problems is more challenging than traditional problems since the load is continuously interacting with the adaptive interface during the iterative optimization process \cite{Wang2016, Vasista2013}. It should be mentioned that the use of compressed air or pressurized fluid is a popular actuation approach in soft robotics.

In this work, we present a novel framework for generating topologies of soft robotics. Contrary to biometry or convectional designs, finding the (optimal) material layout of the soft robot is accomplished through a gradient-based nonlinear topology optimization, where the distribution of soft materials is optimized given a user-defined objective. Our main contributions include the description of nonlinear geometrical deformation and pneumatic loading. We exploit the connectivity properties in polygonal meshes such that synchronized volumetric contraction or expansion of a group of polygonal elements can artificially mimic the geometrical loads in pneumatic actuation. The advantages of our framework in comparison to other literature are: ($i$) a better representation of pneumatic actuation in soft robotics; ($ii$) improved design convergence in contrast to evolution-based optimization methods. To our knowledge, our approach of pressure-driven nonlinear topology optimization is new for soft robotics, and its application could easily extend to other soft robotic systems. %The computational framework detailed in this work is made publicly available at \cite{Caasenbrood2019}.

The remainder of the paper is structured as follows. In section \ref{chap:fem}, we will discuss the continuum mechanics for hyper-elastic materials, followed by a description of the optimization scheme for soft robotics. In section \ref{chap:results}, we propose a numerical example for developing a soft robotic structure to illustrate the effectiveness of our approach.
