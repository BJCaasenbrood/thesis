%!TEX root = ../../thesis.tex
Before detailing the energy-equivalent Cosserat models, we briefly touch upon the Non-Linear Finite Element Method (NLFEM) to solve the continuum dynamics of a general class of soft robots. Although the section might be familiar for some readers, our aim here is to develop an understanding of dynamic NLFEM models and how they relate to Cosserat models.

\subsection{Deformation and Lagrangian strain}
Let us start by introducing the notion of continuum deformation. Let $\config{\Vs}$ be the undeformed body of the soft robot which we refer to as the \textit{reference body} indicated by subscript $\config{(\cdot)}$. As these external forces induce motion in the continuum, the reference will transition from an initial to a new configuration, denoted by $\Vs$ (usually $\Vs \subseteq \R^{\text{nsd}}$ with $1 \le \text{nsd} \le 3$ the spatial dimension). For the sake of simplicity, lets assume $\text{nsd} = 3$. Following, the motion (\ie, the collection of configurations) of the reference for a given time $t \in [0,T]$ can be characterized by the smooth mapping $\vec{\phi}^{(t)}: \config{\Vs} \to \Vs$ \cite{Kim2018,Laursen2001}.

Given this description, we can describe the local displacement of the continuum relative to its reference configuration by
%
\begin{equation}
\dB(\sB,t) := \phiB(\sB,t) - \sB,
\label{eq:C3:displacement}
\end{equation}
%
where $\vec{\sigma} \in \config{\Vs}$ is a material point belonging to the reference body. As $\phiB^{(0)} = \Id$, it simply follows that $\vec{d}(\vec{\sigma},0) = \vec{0}_{3} \; \forall \vec{\sigma} \in \config{\Vs}$ by definition. Essentially, this implies that the internal (elastic) potential energy is zero for $t = 0$ (\ie, unstressed configuration). Throughout this work, we always consider the unstressed configuration of the system as initial condition. Classically, for continuum mechanics problems, the aim is finding an admissible (quasi-static) solution to the displacement fields \eqref{eq:C3:displacement} that satisfies the boundary condition on the surface $\p \config{\Vs}$. Assuming \eqref{eq:C3:displacement} to be sufficiently smooth, we can introduce the fundamental measure of nonlinear deformation called the time-variant
\textit{deformation gradient tensor} \cite{Kim2018}:
%
\begin{align}
\FB(\sB,t)&:=\frac{\p\dB}{\p\sB}(\sB,t) +\mat{I}_3,
\label{eq:C3:deformationgradient}
\end{align}
%
where $\mat{I}_3$ is the identity matrix. The deformation gradient possesses some useful properties explored in this work later. First, since a sub-volume of the continuum body $\Vs$ cannot collapse to a point, it follows that $J := \det{\FB} > 0$ and $\mat{F}\inv$ exists for all $t$. The volume change $J$ can be explored to introduce incompressibility constraints in the hyper-elastic material model. Second, the deformation gradient can be factorized into $\mat{F} = \PhiB \LambdaB
$ with $\LambdaB \succ 0$ the right-handed stretch tensor and $\mat{\Phi} \in \SO{3}$ a homogeneous rotation matrix belonging $\SO{3}$ \cite{Kim2018,Smith2018}. This is particularly interesting here as this decomposition allow us to tie the continuum body to the geometric formulation of Cosserat beam models in terms of the Lie group $\SE{3}$ -- a group of rigid-body transformations, similar to the Cosserat description.

\begin{rmk}
In most literature, the geometric Cosserat beam models do not describe the local volumetric deformation, and any transformation on the body can be represented geometrically by $\SE{3}$. For FEM, however, any local volume of the continuum body can be represented by the group of transformations $\text{Sim}(3)$:
%
$$
\Sim{3} := \left\{ g \in \begin{pmatrix} \PhiB \LambdaB & \dB \\ \vec{0}^\top_3 & 1 \end{pmatrix} \; : \; \PhiB \in \SO{3}, \det{\LambdaB} > 0  \right\}
$$
%
\end{rmk}

\subsection{Energy formulation on the continuum}
The elastic behavior of a deformable body is often specified in terms of a hyper-elastic energy density $\mat{F} \mapsto \Psi(\mat{F})$. There are numerous constitutive models that represent various hyper-elastic descriptions: Neo-Hookean, Rivlin-Mooney, and Yeoh \cite{Kim2018,Goury2018,Duriez2013}; yet, in this work, we explore a modified Neo-Hookean material model \cite{Smith2018,Kim2018} that depends exclusively on the first strain invariant $I_c = \text{trace}(\mat{F}^\top\!\! \mat{F})$ and the volumetric change $J = \det{\mat{F}}$:
%
\begin{equation}
\Psi_\text{NH}(\mat{F}) = \frac{\mu}{2} \left({I}_c - 3 \right) + \frac{\lambda}{2} \left({J} - 1 \right)^2,
\label{eq:C3:neohookean}
\end{equation}
%
where $\mu$ and $\lambda$ are called the Lam\'{e} material coefficients. The total energy of the the continuum dynamical system (i.e., Hamiltonian) is given by $\Hm = \Kf + \Uf$, being the sum of the kinetic energy and potential energy of the continuum body. Given the infinite dimensional continuum formulations \eqref{eq:C3:displacement} and \eqref{eq:C3:deformationgradient}, the kinetic energy and the potential energy formulation yield, respectively:
%
\begin{align}
\Kf & = \iiint_{\config{\Vs}}  \frac{\rho_0}{2} {\frac{\p \vec{d}}{\p t}}^\top\!\!\! (\vec{\sigma},t)  \, \frac{\p \vec{d}}{\p t}(\vec{\sigma},t)\, \; d\vec{\sigma},
\label{eq:C3:kinetic}
\\[0.5em]
\Uf & = \iiint_{\config{\Vs}} \Psi_{\text{NH}}(\mat{F}(\vec{\sigma},t)) - \rho_0 \vec{d}^\top\!\!(\vec{\sigma},t) \vec{a}_g  \; d\vec{\sigma},
\label{eq:C3:potential}
\end{align}
%
where $\config{\rho}$ is a density field w.r.t. the reference body, and $\vec{a}_g$ an external acceleration potential acting on the volume (e.g., gravity or magnetic field). Please note that integration bounds for the kinetic and potential energy belong to the undeformed domain of the reference body.

\subsection{Continuum dynamics on the finite-dimensional space}
In the previous section, all continuum formulations are infinite dimensional. In practice, however, these continuum domains are discretized using \textit{finite elements} in which $\config{\Vs}$ is subdivided into smaller (convex) regions. The basics of the finite element method explores this concept to approximate the continuum domain using a discretized representation $\config{\Vs} \cong \{{\mathcal{V}}^{(n)} \}^{k}_{n=1}$ with $k$ the number of finite elements. Now, instead of expressing the local displacement field exactly, an interpolation scheme is used of (piece-wise) elemental shape functions that map the nodal displacements $\vec{x}_n$ related to the $n$-th finite element to a local approximation of the displacement $
[\vec{d}]_k(\vec{\sigma},\xB(t))\cong \vec{d}(\vec{\sigma},t)$ around the neighborhood $\vec{\sigma} \in \mathcal{V}^{(n)}$ for the element $n$:
%
\begin{align}
[\vec{d}]_{k}(\vec{\sigma},\xB) & = \sum_{i=1}^{n_p}\thetaB_i(\sB) \xB_{i,n}(t) := \ThetaB(\vec{\sigma}) \xB_n,
\label{eq:C3:reduced_displacement}
\end{align}
%
where $\{\vec{\theta}_i\}_{i=1}^{n_p}$ is a set of (orthonormal) elemental shape matrix functionals, and $N_p$ the number of points spanned by the convex hull of the subvolume $\mathcal{V}^{(n)}$. Throughout this work, we will use the mathematical operator $[\,\cdot\,]_k$ to denote the $k$-th order approximation of an infinite-dimensional vector field. Clearly if the number of elements $k \to \infty$, albeit \textit{impossible} numerically, the span of each elements collapses to a point, and therefore the approximate becomes exact (i.e., $\lim_{k \to {\infty}}[\vec{d}]_k =  \vec{d})
$. Now, the aggregate of all nodal displacements form the state representation of the finite-element model, denoted by $\vec{q}$. Substituting \eqref{eq:C3:reduced_displacement} into \eqref{eq:C3:deformationgradient}, we obtain the reduced-order deformation gradient tensor related to the $n$-th finite element:
%
\begin{align}
[\mat{F}]_k(\vec{\sigma},\xB) & = \frac{d \ThetaB_n } {d \vec{\sigma}}(\vec{\sigma}) \xB_n(t) + \mat{I}_3 &  \forall \vec{\sigma} \in \mathcal{V}^{(n)},
\label{eq:C3:reduced_gradient}                                        \\[0.75em]
%
[\dFB]_k(\vec{\sigma},\xB) & = \frac{d \ThetaB_n } {d \vec{\sigma}}(\vec{\sigma}) \dxB_n(t) & \forall \vec{\sigma} \in \mathcal{V}^{(n)}.
\label{eq:C3:reduced_gradient_diff}
\end{align}
%
\begin{rmk}
Let it be clear that in order to preserve the invertibility of the deformation gradient $[\boldsymbol{F}]_k$, it is essentially that $\frac{\p \ThetaB_n}{\p \sB}$ has to well-defined for all $\sB$ inside the finite element domain $\mathcal{V}^{(n)}$. As such, we assume that interpolation functions belong to $\ThetaB_n \in \mathcal{C}^{1}$,\ie, all class-1 differentiable functions.
\end{rmk}
%
Finally, inserting both reduced-order approximations \eqref{eq:C3:reduced_displacement} and \eqref{eq:C3:reduced_gradient} into energy formulations \eqref{eq:C3:kinetic} and \eqref{eq:C3:potential}, and applying Hamilton's principle of least action, we obtain a finite-dimensional dynamic model that describes the evolution of the continuum body
$\Vs(t)$:
%
\begin{equation}
\Sigma_{\text{FEM}}: \; \begin{pmatrix} \dxB \\[0.5em] \dot{\vec{\mu}} \end{pmatrix} = \begin{pmatrix} \mat{0} & \mat{I} \\[0.5em] - \mat{I} & -\mat{R}\end{pmatrix} \begin{pmatrix} \nabla_{\!\xB} \Hm \\[0.6em] \nabla_{\!\vec{\mu}}\Hm \end{pmatrix} + \begin{pmatrix} \mat{0} \\[0.5em] \mat{G} \vec{u} \end{pmatrix} ,
\label{eq:C3:modelFEM}
\end{equation}
%
where the Hamiltonian is $\Hm(\xB,\vec{\mu}) = \tfrac{1}{2} \vec{\mu}^\top \mat{M}\inv \vec{\mu} + \Uf(\vec{q})$ with $\vec{p} := \mat{M}\dot{\vec{q}}$ the generalized momenta, $\mat{R}\succ 0$ a (constant) Rayleigh damping matrix, $\mat{G}(\vec{q})
$ a nonlinear generalized input matrix, and the partial derivatives of the Hamiltonian are computed as follows:
%
\begin{align}
\nabla_{\!\vec{q}}\Hm  = \vec{f}^{\text{e}}(\vec{q}) + \vec{f}^{\text{g}}; \quad
\nabla_{\!\vec{\mu}}\Hm = \mat{M}\inv\vec{\mu}.
\end{align}
%
with $\vec{f}^{\text{e}}$ and $\vec{f}^{\text{g}}$ the (hyper)-elastic and gravitational conservative forces, respectively. The generalized dynamic entries are assembly in the following procedure:
%
\begin{align}
\mat{M} & \cong \bigoplus^{k}_{i=1} \left[\iiint_{\mathcal{V}^{(n)}} \rho_0 {\mat{\Theta}}_i^\top {\mat{\Theta}}_i \; d\vec{\sigma} \right],\\[.25em]
%
\vec{f}^{\text{e}} & \cong \bigoplus^{k}_{i=1} \left[\iiint_{\mathcal{V}^{(i)}} {\mat{B}}_i^\top {\left(\mat{S}_i \right)}^{\vee} \; d\vec{\sigma} \right], \\[.25em]
%
\vec{f}^{\text{g}} & \cong \bigoplus^{k}_{i=1} \left[ \iiint_{\mathcal{V}^{(i)}} \rho_0 \mat{\Theta}_i^\top \vec{a}_g \; d\vec{\sigma} \right].
\end{align}
%
where the operator $\oplus_i^{k}$ denotes the assembly of $k$ number of elemental matrices into one global matrix, $\mat{B}_i$ a nonlinear displacement-strain matrix related the $i$-th finite element \cite{Kim2018}, $\ten{S} = \p^2 \Psi/\p \mat{F}^2
$ the second Piolla-Kirchoff stress tensor, $(\cdot)^\vee$ a column vector representation of a tensor (Voight notation). To solve dynamical system above, we use a so-called Newmark-Beta method which is detailed briefly in Appendix ??. The Newmark-Beta coefficients are chosen as $\beta_1 = \frac{1}{2}$ and $\beta_2 = \frac{1}{4}$, which is said to be unconditionally stable.
%
\subsection{Newmark-$\beta$ method for pH-systems}
The Newmark-$\beta$ method is an implicit numerical integration scheme extensively used to solve high-dimensional dynamic problem. First, let us subdivide the time domain such that $[t_0,t_1,...,T]$ with timestep $\dt = t_{i+1} - t_i$. Then, given the initial conditions for \eqref{eq:C3:modelFEM}, we wish to compute the evolution $\xB(t_i)$ and $\vec{\mu}(t_i)$. For conciseness, let us write the discrete states of the FEM model as $\vec{x}(t_i) = \vec{x}^{(i)}$. Through the extended mean value theorem, we can formulate the general Newmark-beta scheme as
%
\begin{align}
{\vec{p}}^{(i+1)} & = {\vec{p}}^{(i)} + \dt \left[(1-\beta_1)\dot{\vec{p}}^{(i)} + \dt \beta_1 \dot{\vec{p}}^{(i+1)} \right], \\
%
{\vec{x}}^{(i+1)} & = {\vec{x}}^{(i)}  +  \dt \mat{M}\inv \left[ \vec{p}^{(i)} + (\tfrac{1}{2}-\beta_2)\dot{\vec{p}}^{(i)} + \beta_2 \dot{\vec{p}}^{(i+1)} \right],
\end{align}
%
where $\beta_1 \ge \frac{1}{2}$ and $\beta_2 \ge \frac{1}{2}$. Now, in the expressions above only the forward-time acceleration $\dot{\vec{p}}^{(i+1)}$ is the unknown, hence we conveniently write as $\vec{w} = \dot{\vec{p}}^{(i+1)}$. By substituting these into the dynamic flow \eqref{eq:C3:modelFEM}, we retrieve an implicit equality:
%
\begin{multline}
\vec{w} = -\nabla_{\!\vec{q}}\Hm(\vec{w}) -\mat{R}\nabla_{\!\vec{p}}\Hm(\vec{w}) + \vec{G}(\vec{w})\vec{u}^{(i+1)}
\end{multline}
%
with $A(\vec{w}) = \left[\mat{M} + \beta_1 \dt \mat{R} + \beta_2 \dt^2 \mat{K}_{T}(\vec{w}) \right]\mat{M}\inv$. The matrix $\mat{K}_T$ denotes the state-dependent tangent stiffness of the hyper-elastic material model $\Psi_{\text{NH}}$.
