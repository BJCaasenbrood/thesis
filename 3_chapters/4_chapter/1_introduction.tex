%!TEX root = ../../thesis.tex
Soft robotics -- an acronym for soft material robotics -- is an established scientific field of designing, modeling and controlling a class of elastic continuum robots. Through the use of elasticity and softness embedded in the robot's structure, intrinsic motion and adaptability can be achieved in ways previously unseen in rigid robotics. Naturally, this leads to robotic systems with beneficial hyper-redundant and compliant properties, yet it imposes significant challenge on its modeling and control. The field has sparked the interest of many scientific disciples in the past decade, most particularly the robotics control community. The introduction of soft robotics has lead to a paradigm shift in control theory for robotics; namely, the control of elastic robots with theoretically infinite-dimensional Degrees-of-Freedom (DOF).

As of today, two popular modeling strategies for continuum dynamics of soft robots dominate the field: \textit{i}) Finite-Element-Method (FEM) models and \textit{ii}) and beam models. The FEM formulation is a well-known techniques of employing spatial discretization to PDE problems such that its approximate can be solved numerically \cite{Kim2018,Coevoet2017}. With roots in continuum mechanics, FEM has the merit benefit of easily in-cooperating constitutive material laws, nonlinear geometric deformations, and distributed loads, expressed in terms of its tesselation. In practice, however, these models demand considerable computational effort, rendering online predictions and model-based control often infeasible. To enable fast simulation, researchers \cite{Coevoet2017,Duriez2013,Goury2018} typically explore model-reduction techniques in which the full state representation (i.e., all nodal displacements) is projected onto a reduced-order representation with minimal lost in accuracy. Although it is shown that such projection methods are suitable for model-based soft robot control, they often loose desirable properties, e.g., passivity and the physical interpretation of the states.

In contrast, virtual beam models, e.g., Constant Curvature \cite{Katzschmann2019}, Non-Constant Curvature \cite{Santina2020}, and Cosserat beams \cite{Boyer2019}, have roots in Lagrangian-based modeling principles, leading to reduced-order ODE systems similar to the dynamic models of (serial-elastic) rigid robot. Here the continuum dynamics of the soft robot is modeled as spatial curve that passes through the geometric center of its body; whose differential geometry is then discretized through functional basis of shapes.

Given a brief overview of the aforementioned modeling strategies,
we propose a hybrid modeling strategy that benefits from the high-accuracy deformation adapted from FEM-models and the ease of controller implementation similar to beam models. To highlight the effectiveness of our approach, a control architecture is proposed in which model-based controllers arisen from the hybrid method are employed on the full-order FEM model.

Our main contributions include:
\begin{itemize}
\item A POD-projection method for the functional strain basis of soft robotic beam models through FEM-driven data,
\item Preservation of nonlinear geometric deformation, and hyper-elastic material behavior for beam models,
\item Numerical and experimental validation of proposed reduced-order model-based controller.
\end{itemize}
