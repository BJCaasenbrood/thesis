%!TEX root = ../../thesis.tex
In this section, we detail the numerical simulations of the port-Hamiltonian model in \eqref{eq:C3:model_ph} together with the energy-shaping controller in \eqref{eq:C3:control}. For all numerical simulations, we consider a truncation degree of the ROM model is $k = 8$.

Due to the partial differential nature, we have to employ a nested ODE routine to recover the trajectories for $\vec{q}$ and $\vec{p}$. First, we employ an implicit trapezoidal solver with a fixed stepsize of $dt = 30 $ ms to solve \eqref{eq:C3:model_ph}. At each time increment, we have to evaluate the dynamic matrices \eqref{eq:C3:lag_M}-\eqref{eq:C3:lag_tau}. To efficiently compute these dynamic entities, we solve the spatial integration problem over the material domain $\Xs$ by using a second-order Runge-Kutta solver. The stepsize for the spatial solver is $d\sigma = 5$ mm. All simulation examples and underlying source code are provided publicly on the \sorotoki toolkit
\cite{Caasenbrood2020}. Here, the numerical integrations of the system matrices \eqref{eq:C3:lag_M}, \eqref{eq:C3:lag_C}, \eqref{eq:C3:lag_G} and \eqref{eq:C3:lag_tau} are performed using a so-called Matrix-Differential solver (see \cite{Caasenbrood2022}). The simulations are performed on a modern machine (Ryzen 7-5800H, 3.2GHz).

For the soft robotic simulations, we have chosen a linear isotropic Hookean material with shear constraints. %Although hyper-elastic material models are possible, e.g, Neo-Hookean or Yeoh, it is considered out of the scope of this work (see \cite{Kim2018,Renaud2011}). 
Given these material properties, the inertia tensor and the stiffness tensor become diagonal matrices:
%
\begin{align*}
\mat{\mathcal{M}} & = \blkdiag{ \rho_0 \mat{\mathcal{J}},\, \rho_0 A \mat{I}_3}; \\[0.45em]
\mat{\mathcal{K}} & = \blkdiag{\mu_1 \mat{\mathcal{J}},\, E A,\, \mu_2  A,\,\mu_2  A},
\end{align*}
%
where the (average) cross-sectional is $A = \pi r^2$ for a disc radius $r$, and $\mat{\mathcal{J}}$ the second moment of area. The damping tensor chosen as $\mat{\mathcal{D}} = \zeta \mat{\mathcal{K}}$ with damping coefficient $\zeta$. The visco-elastic matrices are precomputed using \eqref{eq:C3:stiff_mat} and \eqref{eq:C3:damp_mat}.

\subsection{Soft robot manipulator inspired by octopus' tentacle}
In the first study-case, we consider a soft robotic arm that is loosely inspired by the tentacles of an octopus. To introduce the under-actuation typically present in soft robotics, we have chosen an actuation matrix $\mat{G} = \blkdiag{\mat{I}_5, \mat{O}_3}$ such that only the first five modes are actively controllable. The system  properties can shown in Table \ref{tab:C3:parameters1}.
%
\begin{table}[t]
  \vspace{-0.25cm}
  \caption{Parameters setting for the numerical solver, the soft manipulator, and the energy-based controller.}\label{tab:C3:parameters1} \centering
  \begin{tabular}{l|c|c|l}
    Parameter description & Symbol & Value & Unit \\
    \midrule
    Finite horizon time & $T $ & $10$ & s \\
    Intrinsic length & $L $ & $ 120$ & mm \\
    Cross-section radius & $r $ & $ 8.25$ & $\text{mm}$ \\
    Uniform density & $\rho_0 $ & $ 1250$ & $\text{kg}\text{m}^{-3}$ \\
    %Gravitational acceleration & $a_{g} $ & $ 9.81$ & $\text{m}\text{s}^{-2}$ \\
    Young's modulus & $E $ & $ 25$ & $\text{MPa}$ \\
    Shear modulus & $\mu_1 $ & $ 10 $ & $\text{MPa}$ \\
    Constraint modulus & $\mu_2 $ & $ 15 $ & $\text{GPa}$ \\
    Rayleigh coefficient & $\zeta $ & $ 0.40 $ & - \\
    \bottomrule
  \end{tabular}
  \vspace{-3mm}
  \end{table}
%
The soft robot is subjected to the energy-based controller in \eqref{eq:C3:control}, where the control gains are tuned to produce a smooth transient: $\lambda_1 = 0.01$ and $\lambda_2 = 0.001$. The artificial spring stiffness is chosen as $\mat{k}_p = \blkdiag{0.01\cdot \mat{I}_3, \mat{I}_3}$. Lastly, the desired configuration of the end-effector is chosen as follows:
%
\begin{equation*}
\mat{g}_d = \begin{pmatrix} \mat{I}_3 & \mat{r}_d \\ \vec{0}_3^\top & 1 \end{pmatrix} \quad \text{with} \quad \mat{r}_d = \begin{pmatrix}  0.04 \\ 0.00 \\ -0.01  \end{pmatrix}.
\end{equation*}
%
The numerical results of the closed-loop system are shown in Figure \ref{fig:octarm1} and \ref{fig:octarm1_states}. It is worth mentioning that these simulation run at $\pm120$ \si{\hertz} real-time.$^{1}$ \blankfootnote{$^{1}$Real-time bandwidth is determined by the ratio between finite horizon and CPU's computation time, \ie, $f \approx T/T_{\textrm{sim}}$,which is affected most by spatial stepsize due to explicit integration.} 

Figure \ref{fig:octarm1} shows the evolution of the continuous deformation along the soft robotic body, whereas Figure \ref{fig:octarm1_states} shows the modal coefficients $\vec{q}(t)$ and generalized momenta $\vec{p}(t)$. As can be seen, the end-effector of the soft robot manipulator slowly converges to the desired set-point $\mat{g}_d \in \SE{3}$. Although the control gains could be increased to promote a faster transient, it was observed that high gains lead to undesired (propagating) oscillations of the flexible structure. A possible solution might be to introduce negative damping to the controller Hamiltonian $\Hm_d$, to overcome the soft robot's structural damping.

For the second simulation run, we modified the control gains to highlight an interesting property of the proposed controller. To be more specific, we increase the controller gains to $\lambda_1 = \lambda_2 = 0.1$. The numerical results for the increased controller gains are shown in Figure \ref{fig:octarm2} and Figure \ref{fig:octarm2_states}. Although the control goal and the initial conditions are chosen identical, the soft robot converges to a different configuration -- albeit, a shape with less '\textit{complexity}'. The cause of less complicated bending patterns has two origins. First, increasing the control gains also artificially impacts the structural stiffness of the soft robot, resulting in soft robot with a higher perceived stiffness. Second, by increasing the stabilizing term $\lambda_2$ in the damped Jacobian inverse \eqref{eq:C3:jacob_damped}, more weight is given towards finding a solution that also minimizes the joint angles $\lVert \vec{q} \rVert_2 $. As intrinsically, more energy is spent to excite higher-order modes in the basis $\{\theta_n\}_{n=1}^k$, the energy-based controller will thus find a minimizer that accounts for the ordering of the shape basis, penalizing higher-order modes. This result indicates that the proposed controller can be effectively tuned alter the structural compliance of the soft robot; and thus could be implemented carefully to preserve '\textit{softness}'. 

\afterpage{
  \vfill
  \begin{figure*}[!t]
    \centering
    \vspace{-12mm}
    \input{./3_chapters/3_chapter/img/fig_C3_3D_srmsoft.tex}
    \vspace{-1mm}
    \caption{Three-dimensional evolution of the soft robot manipulators, slowly converging to the desired set-point $g_d \in \SE{3}$ (indicated by the pink ball). Please observe the morphology that arises which can be related to the motion of an octopus. }
    \label{fig:octarm1}
  \end{figure*}
  \vspace{2mm}
  \begin{figure}[!t]
    \centering
    \vspace{-20mm}
    % This file was created by matlab2tikz.
%
\definecolor{mycolor1}{rgb}{0.00000,0.34510,0.65882}%
\definecolor{mycolor2}{rgb}{0.79216,0.11765,0.17255}%
\definecolor{mycolor3}{rgb}{0.20392,0.65490,0.24706}%
\definecolor{mycolor4}{rgb}{0.93333,0.43922,0.13725}%
\definecolor{mycolor5}{rgb}{0.49412,0.14510,0.51373}%
\definecolor{mycolor6}{rgb}{0.97647,0.67059,0.08235}%
\definecolor{mycolor7}{rgb}{0.24314,0.18039,0.52549}%
\definecolor{mycolor8}{rgb}{0.71373,0.81961,0.14118}%
%
\begin{tikzpicture}

\begin{axis}[%
width=0.37\textwidth,
height=0.3\textwidth,
at={(0\textwidth,0\textwidth)},
scale only axis,
xmin=0,
xmax=6,
xlabel style={font=\color{white!15!black}},
xlabel={time (s)},
ymin=-0.08,
ymax=0.05,
ylabel style={font=\color{white!15!black}},
ylabel={$\vec{q}(t)$},
axis background/.style={fill=white},
xmajorgrids,
ymajorgrids,
ylabel style={yshift=-9.5pt}
]
\addplot [color=mycolor1, line width=1.5pt, forget plot]
  table[row sep=crcr]{%
0	0.00100000000000033\\
0.0999999999999996	0.000732535763058095\\
0.2	0.000833074040049731\\
0.983333333333333	0.00293974513825557\\
1.88333333333333	0.00385980750118531\\
3.56666666666667	0.00456401284455943\\
5.98333333333333	0.00467530988040199\\
};
\addplot [color=mycolor2, line width=1.5pt, forget plot]
  table[row sep=crcr]{%
0	0.00100000000000033\\
0.0666666666666664	0.000815426863866264\\
0.166666666666667	2.29049321793795e-05\\
0.533333333333333	-0.000711551703824753\\
1.08333333333333	-0.000502410285309729\\
2.2	0.000837531682200243\\
4.61666666666667	0.00391868974278786\\
5.98333333333333	0.00474301868188931\\
};
\addplot [color=mycolor3, line width=1.5pt, forget plot]
  table[row sep=crcr]{%
0	0.00100000000000033\\
0.0499999999999998	0.000862540745709239\\
0.083333333333333	7.9715738317887e-05\\
0.15	-0.00286500293669523\\
0.566666666666666	-0.023039392105896\\
0.7	-0.0277099330252577\\
0.85	-0.0319308929707871\\
1.03333333333333	-0.0360781551803369\\
1.25	-0.0399905513472296\\
1.51666666666667	-0.0437716260847507\\
1.83333333333333	-0.0472059644891329\\
2.2	-0.0501605291248053\\
2.63333333333333	-0.0526554477870347\\
3.16666666666667	-0.0547181046207754\\
3.85	-0.0563324704569999\\
4.78333333333333	-0.0575004877876975\\
5.98333333333333	-0.0581824435438563\\
};
\addplot [color=mycolor4, line width=1.5pt, forget plot]
  table[row sep=crcr]{%
0	0.00100000000000033\\
0.116666666666666	0.00141810417547372\\
0.283333333333333	0.00284520823013867\\
0.6	0.00442456639570743\\
1.05	0.00557974262121252\\
1.78333333333333	0.00638962866368331\\
3	0.00663203285105141\\
5.98333333333333	0.00622013713842939\\
};
\addplot [color=mycolor5, line width=1.5pt, forget plot]
  table[row sep=crcr]{%
0	0.00100000000000033\\
0.116666666666666	0.00120378091308915\\
0.7	0.00750897575729681\\
1.03333333333333	0.00941072600012749\\
1.53333333333333	0.0112112704692127\\
2.23333333333333	0.012708719652073\\
3.23333333333333	0.0138286419884883\\
4.78333333333333	0.0145243875181809\\
5.98333333333333	0.014719913864214\\
};
\addplot [color=mycolor6, line width=1.5pt, forget plot]
  table[row sep=crcr]{%
0	0.00100000000000033\\
0.0333333333333332	0.000198168650838326\\
0.15	-0.000146438128797222\\
0.5	-0.00135936491615585\\
1.21666666666667	-0.00436997869578892\\
1.93333333333333	-0.00624567310612001\\
2.91666666666667	-0.00777048401753788\\
4.25	-0.00880789873058863\\
5.98333333333333	-0.00933423162050939\\
};
\addplot [color=mycolor7, line width=1.5pt, forget plot]
  table[row sep=crcr]{%
0	0.00100000000000033\\
0.0333333333333332	0.000156244427058638\\
0.116666666666666	5.79230952633125e-05\\
0.383333333333334	0.000943852594205374\\
1.13333333333333	0.00334768860949186\\
1.9	0.0045178578491134\\
3	0.00516635855243219\\
5.18333333333333	0.00529648809347449\\
5.98333333333333	0.00527621146404211\\
};
\addplot [color=mycolor8, line width=1.5pt, forget plot]
  table[row sep=crcr]{%
0	0.00100000000000033\\
0.0333333333333332	0.000202593966617037\\
0.133333333333334	-0.000139925918890782\\
0.633333333333334	-0.000290741139219897\\
2.43333333333333	0.000544582580761066\\
5.4	0.00102824900177723\\
5.98333333333333	0.00105769909447329\\
};
\end{axis}

\begin{axis}[%
width=0.37\textwidth,
height=0.3\textwidth,
at={(0.493\textwidth,0.017\textwidth)},
scale only axis,
xmin=0,
xmax=6,
xlabel style={font=\color{white!15!black}},
xlabel={time (s)},
ymin=-0.35,
ymax=0.25,
ylabel style={font=\color{white!15!black}},
ylabel={$\vec{p}(t)$},
axis background/.style={fill=white},
xmajorgrids,
ymajorgrids,
ylabel style={yshift=-9.5pt}
]
\addplot [color=mycolor1, line width=1.5pt, forget plot]
  table[row sep=crcr]{%
0	0\\
0.0166666666666666	-0.0022389882244136\\
0.0333333333333332	-0.000883339665332272\\
0.0499999999999998	0.00341285410763525\\
0.083333333333333	0.0209682250607592\\
0.0999999999999996	0.0299272242976309\\
0.116666666666666	0.0363007852376453\\
0.15	0.0415904232953528\\
0.2	0.0426646798081922\\
0.25	0.0431498198932294\\
0.35	0.0534566241726644\\
0.45	0.0695610657130628\\
0.516666666666667	0.0783721351837778\\
0.6	0.0840852512352104\\
0.683333333333334	0.0863756086052616\\
0.95	0.082181744318417\\
1.06666666666667	0.0775254662450982\\
1.15	0.0745111915692922\\
1.23333333333333	0.0705074658301132\\
1.31666666666667	0.0671763307110664\\
1.43333333333333	0.0615429950795159\\
1.55	0.056630179496759\\
1.66666666666667	0.0511707264730568\\
1.81666666666667	0.045058649382578\\
1.96666666666667	0.0387539192842397\\
2.15	0.0320747958970502\\
2.33333333333333	0.0256257407205949\\
2.68333333333333	0.0155973583919842\\
2.93333333333333	0.00968324432571865\\
3.9	-0.00282964760215965\\
5.01666666666667	-0.00508088103279025\\
5.98333333333333	-0.00363870056156568\\
};
\addplot [color=mycolor2, line width=1.5pt, forget plot]
  table[row sep=crcr]{%
0	0\\
0.0166666666666666	9.43487233691087e-05\\
0.0333333333333332	0.00374222577773953\\
0.0499999999999998	0.0116403389419908\\
0.0666666666666664	0.0249326558468956\\
0.0999999999999996	0.0602875600306225\\
0.116666666666666	0.0754236442710443\\
0.15	0.0946616837032153\\
0.2	0.109267109516779\\
0.25	0.12031478286272\\
0.333333333333333	0.134294774828271\\
0.383333333333334	0.141856581716694\\
0.483333333333333	0.149327112540067\\
0.55	0.147622555565285\\
0.683333333333334	0.135855342230123\\
1.03333333333333	0.106290025946391\\
1.16666666666667	0.0973187930340149\\
1.31666666666667	0.0881915763168086\\
2.15	0.0544228064321208\\
3.3	0.0316672329332013\\
3.86666666666667	0.024388442678271\\
4.61666666666667	0.0167435708339321\\
5.98333333333333	0.00765324475762874\\
};
\addplot [color=mycolor3, line width=1.5pt, forget plot]
  table[row sep=crcr]{%
0	0\\
0.0333333333333332	-0.00601773126125504\\
0.0499999999999998	-0.0175035933402601\\
0.0666666666666664	-0.0403285481139068\\
0.0999999999999996	-0.0957554871255155\\
0.116666666666666	-0.115582672571266\\
0.133333333333334	-0.129906392359716\\
0.166666666666667	-0.146413506962633\\
0.45	-0.234446920538813\\
0.5	-0.243974849565285\\
0.55	-0.247085215228163\\
0.6	-0.246170440825747\\
0.7	-0.238372173543493\\
0.966666666666667	-0.207851034477204\\
1.26666666666667	-0.173275120450096\\
1.53333333333333	-0.145645737710617\\
1.81666666666667	-0.120153280842121\\
2.11666666666667	-0.0974720261209265\\
2.43333333333333	-0.0778494014631992\\
2.81666666666667	-0.0591368039190385\\
3.18333333333333	-0.0453590409972904\\
3.6	-0.033491017145419\\
4.21666666666667	-0.0214186747916214\\
4.8	-0.0140850467474376\\
5.85	-0.00680601249895663\\
5.98333333333333	-0.00622017634046035\\
};
\addplot [color=mycolor4, line width=1.5pt, forget plot]
  table[row sep=crcr]{%
0	0\\
0.0166666666666666	0.00429541078256879\\
0.0333333333333332	0.00263804163440362\\
0.0499999999999998	0.00454590376245179\\
0.116666666666666	0.0210669059432442\\
0.4	0.037309133342168\\
0.5	0.0428483501169445\\
0.6	0.0440376490896828\\
0.783333333333333	0.041890563901938\\
2.5	0.0119629161241201\\
3.26666666666667	0.00582623526262172\\
4.2	0.00224211894312276\\
5.51666666666667	0.000582252719429022\\
5.98333333333333	0.000382688812645249\\
};
\addplot [color=mycolor5, line width=1.5pt, forget plot]
  table[row sep=crcr]{%
0	0\\
0.0166666666666666	0.00210925592747824\\
0.0333333333333332	0.000311438389804408\\
0.0999999999999996	0.013085153903047\\
0.133333333333334	0.0179716480615415\\
0.516666666666667	0.0511382326573706\\
0.633333333333334	0.0540311437642016\\
0.8	0.0530108149606079\\
1.11666666666667	0.0464776135586984\\
2.1	0.0251201124650873\\
2.7	0.0163663308070259\\
3.38333333333333	0.00981924236202847\\
4.61666666666667	0.00383154595069701\\
5.98333333333333	0.00141991166409738\\
};
\addplot [color=mycolor6, line width=1.5pt, forget plot]
  table[row sep=crcr]{%
0	0\\
0.0166666666666666	-0.00217007321632678\\
0.0333333333333332	-0.000185607572342761\\
0.0999999999999996	-0.00158054834730414\\
0.183333333333334	-0.00225648059750316\\
0.266666666666667	-0.00390375806547461\\
0.533333333333333	-0.0126560725221205\\
0.683333333333334	-0.0142168277707135\\
1.66666666666667	-0.00863922636016667\\
4.28333333333333	-0.000133565213567444\\
5.98333333333333	0.000159970381544916\\
};
\addplot [color=mycolor7, line width=1.5pt, forget plot]
  table[row sep=crcr]{%
0	0\\
0.0166666666666666	-0.00167429747212822\\
0.0333333333333332	0.000395387159262128\\
0.0499999999999998	-0.000951895314057261\\
0.0666666666666664	1.12957230911093e-05\\
0.116666666666666	-0.00177407886396708\\
0.166666666666667	-0.00146491844196017\\
0.25	-0.00261097687383671\\
0.333333333333333	-0.00335107132516743\\
0.483333333333333	-0.0057424492321605\\
1.16666666666667	-0.00728877674664208\\
5.98333333333333	-0.000527173648784185\\
};
\addplot [color=mycolor8, line width=1.5pt, forget plot]
  table[row sep=crcr]{%
0	0\\
0.266666666666667	0.000609282752626505\\
0.35	0.000576844080265815\\
0.466666666666667	0.00151396580232976\\
0.583333333333333	0.00166954783330731\\
0.7	0.00219246632700187\\
0.85	0.00211800244887694\\
1.03333333333333	0.00230016187960302\\
1.31666666666667	0.00199878873634241\\
1.73333333333333	0.00159964014130232\\
2.65	0.000559782632032935\\
5.98333333333333	-0.000139371173606406\\
};
\end{axis}
\end{tikzpicture}%
    \caption{The evolution of the modal coefficients and the generalized momenta of the soft robot manipulator. The modal coefficients $\q$ are ordered as follows: $k \in \{\ldatanum{Matlab1}{1},\ldatanum{Matlab2}{2},\ldatanum{Matlab3}{3},\ldatanum{Matlab4}{4},\ldatanum{Matlab5}{5},\ldatanum{Matlab6}{6},\ldatanum{Matlab7}{7},\ldatanum{Matlab8}{8}\}$. Observe that mainly mode 3 is dominant. \label{fig:octarm1_states}}
  \end{figure}
  \clearpage
}

\afterpage{
  \vfill
  \begin{figure*}[!t]
    \centering
    % This file was created by matlab2tikz.
%
\begin{tikzpicture}

\begin{axis}[%
width=0.783\textwidth,
height=0.541\textwidth,
at={(0.121\textwidth,0\textwidth)},
scale only axis,
xmin=0,
xmax=1,
ymin=0,
ymax=1,
axis line style={draw=none},
ticks=none,
axis x line*=bottom,
axis y line*=left,
colorbar style={width=6,xshift=-7.5pt}
]
\end{axis}

\begin{axis}[%
width=0.31\textwidth,
height=0.132\textwidth,
at={(0\textwidth,0.422\textwidth)},
scale only axis,
axis on top,
xmin=0.5,
xmax=1034.5,
tick align=outside,
y dir=reverse,
ymin=0.5,
ymax=458.5,
axis line style={draw=none},
ticks=none,
colorbar style={width=6,xshift=-7.5pt}
]
\addplot [forget plot] graphics [xmin=0.5, xmax=1034.5, ymin=0.5, ymax=458.5] {./fig/fig_C3_3D_srmstiff-1.png};
\end{axis}

\begin{axis}[%
width=0.31\textwidth,
height=0.132\textwidth,
at={(0.34\textwidth,0.422\textwidth)},
scale only axis,
axis on top,
xmin=0.5,
xmax=1034.5,
tick align=outside,
y dir=reverse,
ymin=0.5,
ymax=458.5,
axis line style={draw=none},
ticks=none,
colorbar style={width=6,xshift=-7.5pt}
]
\addplot [forget plot] graphics [xmin=0.5, xmax=1034.5, ymin=0.5, ymax=458.5] {./fig/fig_C3_3D_srmstiff-2.png};
\end{axis}

\begin{axis}[%
width=0.31\textwidth,
height=0.132\textwidth,
at={(0.68\textwidth,0.422\textwidth)},
scale only axis,
axis on top,
xmin=0.5,
xmax=1034.5,
tick align=outside,
y dir=reverse,
ymin=0.5,
ymax=458.5,
axis line style={draw=none},
ticks=none,
colorbar style={width=6,xshift=-7.5pt}
]
\addplot [forget plot] graphics [xmin=0.5, xmax=1034.5, ymin=0.5, ymax=458.5] {./fig/fig_C3_3D_srmstiff-3.png};
\end{axis}

\begin{axis}[%
width=0.31\textwidth,
height=0.132\textwidth,
at={(0\textwidth,0.223\textwidth)},
scale only axis,
axis on top,
xmin=0.5,
xmax=1034.5,
tick align=outside,
y dir=reverse,
ymin=0.5,
ymax=458.5,
axis line style={draw=none},
ticks=none,
colorbar style={width=6,xshift=-7.5pt}
]
\addplot [forget plot] graphics [xmin=0.5, xmax=1034.5, ymin=0.5, ymax=458.5] {./fig/fig_C3_3D_srmstiff-4.png};
\end{axis}

\begin{axis}[%
width=0.31\textwidth,
height=0.132\textwidth,
at={(0.34\textwidth,0.223\textwidth)},
scale only axis,
axis on top,
xmin=0.5,
xmax=1034.5,
tick align=outside,
y dir=reverse,
ymin=0.5,
ymax=458.5,
axis line style={draw=none},
ticks=none,
colorbar style={width=6,xshift=-7.5pt}
]
\addplot [forget plot] graphics [xmin=0.5, xmax=1034.5, ymin=0.5, ymax=458.5] {./fig/fig_C3_3D_srmstiff-5.png};
\end{axis}

\begin{axis}[%
width=0.31\textwidth,
height=0.132\textwidth,
at={(0.68\textwidth,0.223\textwidth)},
scale only axis,
axis on top,
xmin=0.5,
xmax=1034.5,
tick align=outside,
y dir=reverse,
ymin=0.5,
ymax=458.5,
axis line style={draw=none},
ticks=none,
colorbar style={width=6,xshift=-7.5pt}
]
\addplot [forget plot] graphics [xmin=0.5, xmax=1034.5, ymin=0.5, ymax=458.5] {./fig/fig_C3_3D_srmstiff-6.png};
\end{axis}

\begin{axis}[%
width=0.31\textwidth,
height=0.132\textwidth,
at={(0\textwidth,0.024\textwidth)},
scale only axis,
axis on top,
xmin=0.5,
xmax=1034.5,
tick align=outside,
y dir=reverse,
ymin=0.5,
ymax=458.5,
axis line style={draw=none},
ticks=none,
colorbar style={width=6,xshift=-7.5pt}
]
\addplot [forget plot] graphics [xmin=0.5, xmax=1034.5, ymin=0.5, ymax=458.5] {./fig/fig_C3_3D_srmstiff-7.png};
\end{axis}

\begin{axis}[%
width=0.31\textwidth,
height=0.132\textwidth,
at={(0.34\textwidth,0.024\textwidth)},
scale only axis,
axis on top,
xmin=0.5,
xmax=1034.5,
tick align=outside,
y dir=reverse,
ymin=0.5,
ymax=458.5,
axis line style={draw=none},
ticks=none,
colorbar style={width=6,xshift=-7.5pt}
]
\addplot [forget plot] graphics [xmin=0.5, xmax=1034.5, ymin=0.5, ymax=458.5] {./fig/fig_C3_3D_srmstiff-8.png};
\end{axis}

\begin{axis}[%
width=0.31\textwidth,
height=0.132\textwidth,
at={(0.68\textwidth,0.024\textwidth)},
scale only axis,
axis on top,
xmin=0.5,
xmax=1034.5,
tick align=outside,
y dir=reverse,
ymin=0.5,
ymax=458.5,
axis line style={draw=none},
ticks=none,
colorbar style={width=6,xshift=-7.5pt}
]
\addplot [forget plot] graphics [xmin=0.5, xmax=1034.5, ymin=0.5, ymax=458.5] {./fig/fig_C3_3D_srmstiff-9.png};
\end{axis}
\end{tikzpicture}%
    \vspace{-1mm}
    \caption{Three-dimensional evolution of the soft robot manipulators, slowly converging to the desired set-point $g_d \in \SE{3}$ (indicated by the pink ball). Please observe that a different morphology arises due to higher control gains, \ie, $\lambda_1 = \lambda_2 = 0.1$, which is caused by the controller affecting the structural compliance of the soft robot.}
    \label{fig:octarm2}
\end{figure*}
\vspace{2mm}
  \begin{figure}[!t]
    \centering
    % This file was created by matlab2tikz.
%
\definecolor{mycolor1}{rgb}{0.00000,0.34510,0.65882}%
\definecolor{mycolor2}{rgb}{0.79216,0.11765,0.17255}%
\definecolor{mycolor3}{rgb}{0.20392,0.65490,0.24706}%
\definecolor{mycolor4}{rgb}{0.93333,0.43922,0.13725}%
\definecolor{mycolor5}{rgb}{0.49412,0.14510,0.51373}%
\definecolor{mycolor6}{rgb}{0.97647,0.67059,0.08235}%
\definecolor{mycolor7}{rgb}{0.24314,0.18039,0.52549}%
\definecolor{mycolor8}{rgb}{0.71373,0.81961,0.14118}%
%
\begin{tikzpicture}

\begin{axis}[%
width=0.37\textwidth,
height=0.3\textwidth,
at={(0\textwidth,0\textwidth)},
scale only axis,
xmin=0,
xmax=6,
xlabel style={font=\color{white!15!black}},
xlabel={time (s)},
ymin=-0.03,
ymax=0.03,
ylabel style={font=\color{white!15!black}},
ylabel={$\vec{q}(t)$},
axis background/.style={fill=white},
xmajorgrids,
ymajorgrids,
ylabel style={yshift=-9.5pt}
]
\addplot [color=mycolor1, line width=1.5pt, forget plot]
  table[row sep=crcr]{%
0	0.00100000000000033\\
0.0999999999999996	0.00103971588713581\\
0.266666666666667	0.00117549874542888\\
0.383333333333334	0.00151761804318351\\
0.466666666666667	0.0020321988385863\\
0.533333333333333	0.00269484033943268\\
0.6	0.00362287778731041\\
0.666666666666667	0.00480943290937663\\
0.783333333333333	0.00723888784369642\\
0.916666666666667	0.00993813998401816\\
1.01666666666667	0.0116933716352383\\
1.13333333333333	0.0134396186381096\\
1.25	0.0149055578394517\\
1.38333333333333	0.0162993751876677\\
1.53333333333333	0.0175773864308324\\
1.7	0.0187122624033087\\
1.9	0.0197705971571311\\
2.13333333333333	0.0206997678583969\\
2.41666666666667	0.0215278097919391\\
2.78333333333333	0.0222971077704228\\
3.26666666666667	0.0230159563690266\\
3.88333333333333	0.0236495932052421\\
4.61666666666667	0.024112682295744\\
5.46666666666667	0.0243539007222431\\
5.98333333333333	0.0243980090290519\\
};
\addplot [color=mycolor2, line width=1.5pt, forget plot]
  table[row sep=crcr]{%
0	0.00100000000000033\\
0.0666666666666664	0.00105311655584384\\
0.233333333333333	0.00116843807874023\\
0.35	0.00147420323991021\\
0.433333333333334	0.00194860524107288\\
0.5	0.00257916654293755\\
0.566666666666666	0.00350000523156258\\
0.633333333333334	0.00473113614920528\\
0.716666666666667	0.00661780995214567\\
0.9	0.0109615584895115\\
1	0.0129998877149839\\
1.1	0.0147580713623414\\
1.21666666666667	0.0165032287013416\\
1.33333333333333	0.0179768873443331\\
1.46666666666667	0.0193905577530034\\
1.61666666666667	0.0207013978604067\\
1.78333333333333	0.0218775033025445\\
1.96666666666667	0.0228957542957406\\
2.18333333333333	0.0238008789240363\\
2.41666666666667	0.0244890740303729\\
2.68333333333333	0.0249949024252842\\
3	0.0253104501963675\\
3.4	0.0254202569265454\\
4.03333333333333	0.0252824430513252\\
5.51666666666667	0.0249299723896934\\
5.98333333333333	0.0249147141724686\\
};
\addplot [color=mycolor3, line width=1.5pt, forget plot]
  table[row sep=crcr]{%
0	0.00100000000000033\\
0.0666666666666664	0.00102726510297035\\
0.366666666666666	0.000737642414384787\\
0.466666666666667	0.000278847250939407\\
0.533333333333333	-0.000281141456603962\\
0.6	-0.00111083686174762\\
0.666666666666667	-0.002225366175443\\
0.75	-0.00393278505075312\\
0.983333333333333	-0.00890343295805085\\
1.1	-0.0109910669020881\\
1.21666666666667	-0.0127832956910714\\
1.35	-0.0145210133303681\\
1.48333333333333	-0.0159829379758865\\
1.63333333333333	-0.0173562015267317\\
1.8	-0.0186066577840558\\
1.98333333333333	-0.0197124488095266\\
2.2	-0.0207327031007214\\
2.45	-0.0216144159348701\\
2.73333333333333	-0.0223303434251818\\
3.08333333333333	-0.0229181177611375\\
3.5	-0.0233304403725221\\
4.06666666666667	-0.0235919785806953\\
5	-0.023692463835169\\
5.98333333333333	-0.0236837306927438\\
};
\addplot [color=mycolor4, line width=1.5pt, forget plot]
  table[row sep=crcr]{%
0	0.00100000000000033\\
0.35	0.00108931508857779\\
0.483333333333333	0.0008990318734865\\
0.583333333333333	0.000528675970282499\\
0.716666666666667	-0.000277596521174317\\
0.933333333333334	-0.00161510792239294\\
1.11666666666667	-0.00244992502715924\\
1.35	-0.00321670490910098\\
1.63333333333333	-0.00385664825599008\\
1.98333333333333	-0.00435911114960952\\
2.41666666666667	-0.00469427494522368\\
2.98333333333333	-0.00484612622651159\\
4	-0.00480169154302956\\
5.98333333333333	-0.00473770360391157\\
};
\addplot [color=mycolor5, line width=1.5pt, forget plot]
  table[row sep=crcr]{%
0	0.00100000000000033\\
0.083333333333333	0.000994901314072649\\
0.65	0.00126767572963615\\
0.883333333333334	0.00185079932096599\\
1.26666666666667	0.00277363851663992\\
1.65	0.00340005661597242\\
2.15	0.00391672489961614\\
2.83333333333333	0.00431760713006302\\
3.73333333333333	0.00456147046419542\\
5.08333333333333	0.00463591201961755\\
5.98333333333333	0.00462512633114986\\
};
\addplot [color=mycolor6, line width=1.5pt, forget plot]
  table[row sep=crcr]{%
0	0.00100000000000033\\
0.0166666666666666	0.000500185575834422\\
0.0333333333333332	0.000197547747774252\\
0.0666666666666664	-7.3172755499229e-07\\
0.216666666666667	-5.88301115236334e-05\\
0.666666666666667	-0.000294607243819023\\
0.833333333333333	-0.000692236559280524\\
1.2	-0.00194720527996939\\
1.56666666666667	-0.00300107480083689\\
1.93333333333333	-0.00374046675046369\\
2.35	-0.00427927455758859\\
2.85	-0.00463739001026742\\
3.5	-0.00482104718696696\\
4.76666666666667	-0.00482601613888356\\
5.98333333333333	-0.00479599395600161\\
};
\addplot [color=mycolor7, line width=1.5pt, forget plot]
  table[row sep=crcr]{%
0	0.00100000000000033\\
0.0166666666666666	0.0004659622024068\\
0.0333333333333332	0.000153422015163329\\
0.083333333333333	-6.44661669113589e-05\\
0.166666666666667	-0.00010673031102737\\
1.18333333333333	0.000282931758941452\\
2.46666666666667	0.000616973307469237\\
4.98333333333333	0.000962495013447473\\
5.98333333333333	0.000985388099384643\\
};
\addplot [color=mycolor8, line width=1.5pt, forget plot]
  table[row sep=crcr]{%
0	0.00100000000000033\\
0.0166666666666666	0.000466455293326895\\
0.0499999999999998	6.95620327464397e-05\\
0.0999999999999996	-5.22169375170023e-05\\
1.4	-0.000229871689484185\\
2.4	0.000103617059704852\\
3.65	0.000269590345438608\\
5.98333333333333	0.000312306373604798\\
};
\end{axis}

\begin{axis}[%
width=0.37\textwidth,
height=0.3\textwidth,
at={(0.486\textwidth,0\textwidth)},
scale only axis,
xmin=0,
xmax=6,
xlabel style={font=\color{white!15!black}},
xlabel={time (s)},
ymin=-0.2,
ymax=0.2,
ylabel style={font=\color{white!15!black}},
ylabel={$\vec{p}(t)$},
axis background/.style={fill=white},
xmajorgrids,
ymajorgrids,
ylabel style={yshift=-9.5pt}
]
\addplot [color=mycolor1, line width=1.5pt, forget plot]
  table[row sep=crcr]{%
0	0\\
0.0166666666666666	-0.0022389882244136\\
0.183333333333334	0.000434866614895668\\
0.233333333333333	0.00267622205121754\\
0.383333333333334	0.013520859814288\\
0.55	0.0257089047544286\\
0.633333333333334	0.0282938004675897\\
0.683333333333334	0.0287875600668777\\
0.816666666666666	0.025605003747418\\
0.833333333333333	0.0242842044133047\\
0.85	0.0241586353998642\\
0.866666666666667	0.0226830625412795\\
0.883333333333334	0.0225356119121836\\
0.9	0.0209329065210273\\
0.916666666666667	0.0207639502344037\\
0.933333333333334	0.0190596793306197\\
0.95	0.0188692384169205\\
0.966666666666667	0.0170875256387237\\
0.983333333333333	0.0168768205431524\\
1	0.0150405890505487\\
1.01666666666667	0.0148125355851256\\
1.03333333333333	0.0129434729821414\\
1.05	0.0127025741342122\\
1.06666666666667	0.0108209756296889\\
1.08333333333333	0.0105729492731799\\
1.1	0.00869754730816563\\
1.11666666666667	0.00844888956317913\\
1.13333333333333	0.00659672332693084\\
1.15	0.00635430488985644\\
1.16666666666667	0.00454064648885133\\
1.18333333333333	0.00431137887164645\\
1.2	0.00254971392038339\\
1.21666666666667	0.0023402918562514\\
1.23333333333333	0.00064234470140434\\
1.25	0.000459057606719604\\
1.26666666666667	-0.00116515013692098\\
1.28333333333333	-0.00131654912380075\\
1.3	-0.00285861694990164\\
1.31666666666667	-0.00297299606943646\\
1.33333333333333	-0.00442604813457592\\
1.35	-0.00449894654523764\\
1.36666666666667	-0.00585752669089157\\
1.38333333333333	-0.00588517691677648\\
1.4	-0.00714513974805442\\
1.41666666666667	-0.00712446973926806\\
1.43333333333333	-0.00828287102535263\\
1.45	-0.00821149105984276\\
1.46666666666667	-0.0092664797156754\\
1.48333333333333	-0.00914265816607873\\
1.5	-0.0100933713440314\\
1.51666666666667	-0.00991600241108692\\
1.53333333333333	-0.0107624647065432\\
1.55	-0.0105310305115021\\
1.56666666666667	-0.0112740579197776\\
1.61666666666667	-0.011290718295851\\
1.63333333333333	-0.0118320402515728\\
1.68333333333333	-0.0114421280337487\\
1.71666666666667	-0.0113003617875824\\
1.76666666666667	-0.0111939379552348\\
1.81666666666667	-0.0100739108464483\\
1.86666666666667	-0.00935895126453179\\
1.91666666666667	-0.00779004418847684\\
1.96666666666667	-0.00657421975455552\\
2.05	-0.00345689570505847\\
2.1	-0.00174977864537862\\
2.18333333333333	0.00188409444892379\\
2.26666666666667	0.00536794068727708\\
2.38333333333333	0.0107480901412398\\
2.53333333333333	0.0172614319530853\\
2.66666666666667	0.0227161695361202\\
2.81666666666667	0.0281200983205947\\
3.08333333333333	0.0350551958963763\\
3.3	0.038085098333922\\
3.5	0.0389356102112686\\
3.71666666666667	0.038014542137784\\
3.96666666666667	0.0351085583814195\\
4.33333333333333	0.0285572192478494\\
5.26666666666667	0.0100959336076532\\
5.63333333333333	0.00474042999280577\\
5.98333333333333	0.00107451489122834\\
};
\addplot [color=mycolor2, line width=1.5pt, forget plot]
  table[row sep=crcr]{%
0	0\\
0.0166666666666666	9.43487233691087e-05\\
0.0499999999999998	0.00266323847031824\\
0.133333333333334	0.00280950270942704\\
0.183333333333334	0.00363352539983275\\
0.233333333333333	0.00431271697322888\\
0.333333333333333	0.00898670336919949\\
0.383333333333334	0.0142021134234129\\
0.416666666666667	0.0192235705032076\\
0.45	0.0259051811251076\\
0.483333333333333	0.0344165857590815\\
0.516666666666667	0.0446711877554824\\
0.55	0.0564621974876767\\
0.583333333333333	0.0697305374431041\\
0.633333333333334	0.0915504870637092\\
0.7	0.1207082357883\\
0.75	0.139429964740127\\
0.766666666666667	0.145202684774989\\
0.816666666666666	0.158734055580934\\
0.833333333333333	0.162944497895182\\
0.9	0.174463512162525\\
1	0.182934740535887\\
1.01666666666667	0.182945448741529\\
1.03333333333333	0.183992555391628\\
1.05	0.183637921260007\\
1.06666666666667	0.184355786445977\\
1.08333333333333	0.183686939703855\\
1.1	0.184112453280627\\
1.11666666666667	0.183172704097718\\
1.13333333333333	0.183337350932957\\
1.15	0.182163487216103\\
1.16666666666667	0.182094332217769\\
1.18333333333333	0.180717742134026\\
1.2	0.180438249911181\\
1.21666666666667	0.178885859462361\\
1.23333333333333	0.178416559315981\\
1.25	0.176711601665145\\
1.26666666666667	0.176070620116452\\
1.28333333333333	0.174233261817895\\
1.3	0.173436746578419\\
1.31666666666667	0.171484596162736\\
1.33333333333333	0.170547053183263\\
1.35	0.168495574661459\\
1.36666666666667	0.16743013642777\\
1.38333333333333	0.165292986552244\\
1.4	0.164111626293622\\
1.45	0.158341215195596\\
1.46666666666667	0.156960118781275\\
1.51666666666667	0.150796479567845\\
1.56666666666667	0.14523466719378\\
1.61666666666667	0.138667256738977\\
1.66666666666667	0.132693140737806\\
1.71666666666667	0.125901303290626\\
1.76666666666667	0.119684306013703\\
1.81666666666667	0.112811742725733\\
1.86666666666667	0.106492219856411\\
1.91666666666667	0.0996564540484659\\
2	0.0890181749269185\\
2.06666666666667	0.080459244666895\\
2.15	0.0698454601123677\\
2.26666666666667	0.056039853222754\\
2.35	0.0464513338246517\\
2.63333333333333	0.0184429510042712\\
2.73333333333333	0.0101281840029701\\
2.83333333333333	0.00269499758674119\\
2.93333333333333	-0.00385666133291629\\
3.05	-0.0104670440222172\\
3.25	-0.0190626244322889\\
3.45	-0.0246407234985995\\
3.68333333333333	-0.027897012679837\\
3.93333333333333	-0.0282797520080633\\
4.11666666666667	-0.0271080232751686\\
4.51666666666667	-0.021797007526251\\
5.4	-0.00740145319557861\\
5.9	-0.00169843112167989\\
5.98333333333333	-0.00100243746084061\\
};
\addplot [color=mycolor3, line width=1.5pt, forget plot]
  table[row sep=crcr]{%
0	0\\
0.0166666666666666	-0.00269018298324664\\
0.15	-0.00234946916758805\\
0.283333333333333	-0.00968119102209641\\
0.35	-0.017158383871184\\
0.4	-0.0258011385108379\\
0.433333333333334	-0.0332848388386031\\
0.466666666666667	-0.0424504620843313\\
0.5	-0.0532716480720499\\
0.533333333333333	-0.0654189052600582\\
0.566666666666666	-0.0787890488824008\\
0.616666666666667	-0.100593520769396\\
0.683333333333334	-0.130060640614654\\
0.716666666666667	-0.143308385883281\\
0.75	-0.154887209428614\\
0.783333333333333	-0.164602073881984\\
0.816666666666666	-0.172463842904694\\
0.85	-0.178605499064345\\
0.883333333333334	-0.183211784147279\\
0.916666666666667	-0.186475911824892\\
0.95	-0.188578590045077\\
1	-0.190016442933969\\
1.06666666666667	-0.189034944032054\\
1.15	-0.184447778339027\\
1.21666666666667	-0.178998609117921\\
1.28333333333333	-0.172387904477089\\
1.36666666666667	-0.163077150705827\\
1.96666666666667	-0.0905618638509509\\
2.13333333333333	-0.0737253916472378\\
2.3	-0.0589701866344239\\
2.46666666666667	-0.0462317700437964\\
2.63333333333333	-0.0353773869698761\\
2.81666666666667	-0.0254275688742664\\
3	-0.0173516266610667\\
3.2	-0.0104340446667788\\
3.41666666666667	-0.00486304060721299\\
3.65	-0.000734592224785224\\
3.91666666666667	0.00210475585201664\\
4.25	0.00362187795158597\\
4.68333333333333	0.00356362488051154\\
5.5	0.00139753227421657\\
5.98333333333333	0.000251709945841228\\
};
\addplot [color=mycolor4, line width=1.5pt, forget plot]
  table[row sep=crcr]{%
0	0\\
0.0166666666666666	0.00429541078256879\\
0.0333333333333332	0.00189832372630594\\
0.0666666666666664	0.000403839532268968\\
0.233333333333333	0.000187564658904904\\
0.35	0.000850406896700129\\
0.483333333333333	0.00173869991856268\\
0.733333333333333	0.00207613386491268\\
0.866666666666667	-0.000700612214314944\\
1.06666666666667	-0.00407905399829289\\
1.31666666666667	-0.00518048552455319\\
1.55	-0.00477405001989784\\
1.86666666666667	-0.00334491419489247\\
2.7	0.0013513611773881\\
3.26666666666667	0.00303928261333652\\
4	0.00298427419221259\\
5.98333333333333	5.75023437399125e-05\\
};
\addplot [color=mycolor5, line width=1.5pt, forget plot]
  table[row sep=crcr]{%
0	0\\
0.0166666666666666	0.00210925592747824\\
0.0333333333333332	-0.000147156844183094\\
0.0499999999999998	0.000837926849941439\\
0.0666666666666664	-1.2472953727638e-05\\
0.116666666666666	0.000256974722861791\\
0.233333333333333	0.000790885976155842\\
0.366666666666666	0.00260293662901478\\
0.466666666666667	0.00557428129062831\\
0.566666666666666	0.0104484005930834\\
0.7	0.0194562339041946\\
0.816666666666666	0.0272308272186663\\
0.916666666666667	0.0319128386952698\\
1.01666666666667	0.0345107068948618\\
1.15	0.0353659102325192\\
1.31666666666667	0.033708687744519\\
1.65	0.0266535700130639\\
1.96666666666667	0.0196245747247481\\
2.4	0.0123356987845193\\
2.81666666666667	0.00772023519238996\\
3.33333333333333	0.00416948037774389\\
3.85	0.00210750052963338\\
5.1	0.000272070191448037\\
5.98333333333333	-2.46025367545144e-06\\
};
\addplot [color=mycolor6, line width=1.5pt, forget plot]
  table[row sep=crcr]{%
0	0\\
0.0166666666666666	-0.00217007321632678\\
0.0333333333333332	-0.000159653851234509\\
0.0499999999999998	-0.000727749899184893\\
0.0666666666666664	0.000211775608901732\\
0.083333333333333	-0.000355276031553942\\
0.0999999999999996	0.000240934550791216\\
0.133333333333334	0.000215619374428044\\
0.183333333333334	-6.8197891113897e-05\\
0.233333333333333	0.000208557414473454\\
0.316666666666666	0.000173606229210144\\
0.466666666666667	0.000892905049374448\\
0.583333333333333	0.00158823107400252\\
0.733333333333333	0.00266115476383622\\
2.33333333333333	0.000890336837508166\\
2.86666666666667	-0.000689502312122059\\
3.4	-0.00153139658496126\\
4.01666666666667	-0.00157480205028282\\
5.98333333333333	-4.29904703800332e-05\\
};
\addplot [color=mycolor7, line width=1.5pt, forget plot]
  table[row sep=crcr]{%
0	0\\
0.0166666666666666	-0.00167429747212822\\
0.0333333333333332	0.000417759620347624\\
0.0499999999999998	-0.000791321736206996\\
0.0666666666666664	0.00042887209237108\\
0.083333333333333	-0.000491795377544513\\
0.0999999999999996	0.000367448278510984\\
0.116666666666666	-0.000366520016036986\\
0.133333333333334	0.000308800297907474\\
0.15	-0.000300748606040457\\
0.166666666666667	0.000260357373457865\\
0.183333333333334	-0.000259068831690357\\
0.216666666666667	-0.000229600848751232\\
0.266666666666667	0.000148534995909166\\
0.316666666666666	-0.00018408563232164\\
0.366666666666666	4.5924722817503e-05\\
0.45	-0.000225332760310337\\
0.533333333333333	-0.000234008898560845\\
0.75	-0.0012082875121564\\
1.45	-0.00310666822674843\\
2.75	-0.00127374291044813\\
3.96666666666667	-0.000369697683955117\\
5.98333333333333	-2.29901431847424e-06\\
};
\addplot [color=mycolor8, line width=1.5pt, forget plot]
  table[row sep=crcr]{%
0	0\\
0.0999999999999996	0.000154141740392966\\
0.15	-0.000178877712126102\\
0.2	0.000185769810370218\\
0.25	-0.000184100379091667\\
0.3	0.000170437505869536\\
0.35	-0.000169666377166422\\
0.4	0.000140974688159545\\
0.45	-0.000158045074154956\\
0.5	0.000105910520931118\\
0.55	-0.000154999689369717\\
0.6	6.45805427517132e-05\\
0.683333333333334	-0.000198513697788449\\
0.766666666666667	-0.00011400364163805\\
0.85	-0.000442169082920607\\
0.933333333333334	-0.000466529340740429\\
1.05	-0.000855751575456587\\
1.16666666666667	-0.000957516363067512\\
1.3	-0.00116406851985662\\
1.45	-0.00138823642594765\\
1.66666666666667	-0.0013646110276353\\
1.91666666666667	-0.00127159551171463\\
2.36666666666667	-0.000705163126432318\\
3.45	0.000326231655884079\\
4.55	0.000317939309873339\\
5.98333333333333	1.06435199134225e-05\\
};
\end{axis}
\end{tikzpicture}%
    \caption{The evolution of the modal coefficients and the generalized momenta of the soft robot manipulator with the increased controller gains. The modal coefficients $\q$ are ordered as follows: $k \in \{\ldatanum{Matlab1}{1},\ldatanum{Matlab2}{2},\ldatanum{Matlab3}{3},\ldatanum{Matlab4}{4},\ldatanum{Matlab5}{5},\ldatanum{Matlab6}{6},\ldatanum{Matlab7}{7},\ldatanum{Matlab8}{8}\}$. Observe now that the modes 1,2, and 3 are dominant. \label{fig:octarm2_states}}
  \end{figure}
  \clearpage
}
\clearpage
\subsection{Multi-link soft robot inspired by the elephant's trunk}
In the second study-case, we consider a two-link soft robot that is inspired by the trunk of an elephant. A similar soft robotic system is considered in \cite{Falkenhahn2015} (\ie, the elephant-inspired bionic arm by \texttt{Festo}), where mobility of the bio-inspired robotic system is achieved through a pneumatic-network distributed along the continuous body of the robot. Therefore, considering a six-bellow network, the actuation matrix takes the form:
%
\begin{equation*}
\mat{G}(\vec{q})\vec{u} = \sum_{n=1}^6 \left[ \int_\Xs [\mat{J}]_k(\sigma,\vec{q}) \cdot \ten{F}_n(\sigma)\; d\sigma\right] u_i,
\end{equation*}
%
where $\{\ten{F}_n\}_{n=1}^6$ is a set of piece-wise constant wrench functionals related to the pneumatic actuation bellows distributed along the soft robotic body, and $\vec{u} = (u_1,\,u_2,\,u_3,\,u_4,\,u_5,\,u_6)^\top$ a vector of wrench amplitudes. The control input sets $\{u_1,..,u_3\}$ relate to the first link and $\{u_4,..,u_6\}$ to the second link of the robot. Given this input configuration, it also follows that $\rank(\mat{G}(\vec{q})) < \dim(\vec{q})$ for all $\vec{q} \in \R^{6k}$, \ie, underactuated. The system and solver properties are given in Table \ref{tab:C3:parameters2}. We again consider $k=8$ spatial modes. To simulate the effect of the gripper, we added an inertial mass at the end-effector modelled by:
%
$$\tauB_{\textrm{ext}} =  \ten{M}_{\textrm{grip}}\,\JB(\q,L)^\top \left[ \left(\vec{0}_3^\top,\, \aB_g ^\top \Ad^\top_{\gB(\cdot,L)} \right)^\top + \dot{\etaB}(\q,L) \right],$$
%
where $\ten{M}_{\textrm{grip}}$ inertia tensor related to the gripper placed at the end-effector of the robot located at $\sigma = L$. Again we apply the energy-based controller in \eqref{eq:C3:control} to the system, where the control gains are $\lambda_1 = 5$ and $\lambda_2 = 1$, while the artificial stiffness matrix $\mat{k}_p$ is kept identical to previous simulations. Lastly, the desired configuration of the end-effector is chosen as follows:
%
\begin{equation*}
\mat{g}_d = \begin{pmatrix} \mat{\Phi}_d & \mat{r}_d \\ \vec{0}_3^\top & 1 \end{pmatrix} \quad \text{with} \quad \mat{r}_d = \begin{pmatrix}  0.125 \\ 0.100 \\ 0.175  \end{pmatrix} \;\;\;\text{and} \;\;\; \PhiB_d = \textrm{Rot}_y\left(\tfrac{1}{4}\pi \right).
\end{equation*}
%

The numerical results of the closed-loop system are shown in Figure \ref{fig:C3:multilink_3D} and Figure \ref{fig:C3:multilink_states} which could reach a real-time performance of $\pm65$Hz. Figure \ref{fig:C3:multilink_3D} shows the continuous deformation along the soft robotic body. Figure \ref{fig:C3:multilink_states} shows the trajectories of the modal coefficients $\vec{q}(t)$ and the generalized momenta $\vec{p}(t)$. As can be seen, the end-effector of the soft robot manipulator slowly converges to the desired set-point $\mat{g}_d \in \SE{3}$, even when dealing with piece-wise distributed actuation loads applied to the continuous backbone. To describe the discontinuous actuation profiles, however, higher order modes are required as can be seen in Figure \ref{fig:C3:multilink_states}. This might indicate there exist better tailored compact shape bases for this soft robotic system.
%
\afterpage{
  \begin{figure*}[!t]
    \centering
    \vspace{-3mm}
    % This file was created by matlab2tikz.
%
\begin{tikzpicture}

\begin{axis}[%
width=0.862\textwidth,
height=0.595\textwidth,
at={(0.08\textwidth,0.007\textwidth)},
scale only axis,
xmin=0,
xmax=1,
ymin=0,
ymax=1,
axis line style={draw=none},
ticks=none,
axis x line*=bottom,
axis y line*=left,
colorbar style={width=6,xshift=-7.5pt}
]
\end{axis}

\begin{axis}[%
width=0.234\textwidth,
height=0.321\textwidth,
at={(0\textwidth,0.329\textwidth)},
scale only axis,
axis on top,
xmin=0.5,
xmax=522.5,
tick align=outside,
y dir=reverse,
ymin=0.5,
ymax=746.5,
axis line style={draw=none},
ticks=none,
colorbar style={width=6,xshift=-7.5pt}
]
\addplot [forget plot] graphics [xmin=0.5, xmax=522.5, ymin=0.5, ymax=746.5] {fig_C3_3D_srmarm-1.png};
\end{axis}

\begin{axis}[%
width=0.234\textwidth,
height=0.321\textwidth,
at={(0.374\textwidth,0.329\textwidth)},
scale only axis,
axis on top,
xmin=0.5,
xmax=522.5,
tick align=outside,
y dir=reverse,
ymin=0.5,
ymax=746.5,
axis line style={draw=none},
ticks=none,
colorbar style={width=6,xshift=-7.5pt}
]
\addplot [forget plot] graphics [xmin=0.5, xmax=522.5, ymin=0.5, ymax=746.5] {fig_C3_3D_srmarm-2.png};
\end{axis}

\begin{axis}[%
width=0.234\textwidth,
height=0.321\textwidth,
at={(0.749\textwidth,0.329\textwidth)},
scale only axis,
axis on top,
xmin=0.5,
xmax=522.5,
tick align=outside,
y dir=reverse,
ymin=0.5,
ymax=746.5,
axis line style={draw=none},
ticks=none,
colorbar style={width=6,xshift=-7.5pt}
]
\addplot [forget plot] graphics [xmin=0.5, xmax=522.5, ymin=0.5, ymax=746.5] {fig_C3_3D_srmarm-3.png};
\end{axis}

\begin{axis}[%
width=0.234\textwidth,
height=0.321\textwidth,
at={(0\textwidth,0\textwidth)},
scale only axis,
axis on top,
xmin=0.5,
xmax=522.5,
tick align=outside,
y dir=reverse,
ymin=0.5,
ymax=746.5,
axis line style={draw=none},
ticks=none,
colorbar style={width=6,xshift=-7.5pt}
]
\addplot [forget plot] graphics [xmin=0.5, xmax=522.5, ymin=0.5, ymax=746.5] {fig_C3_3D_srmarm-4.png};
\end{axis}

\begin{axis}[%
width=0.234\textwidth,
height=0.321\textwidth,
at={(0.374\textwidth,0\textwidth)},
scale only axis,
axis on top,
xmin=0.5,
xmax=522.5,
tick align=outside,
y dir=reverse,
ymin=0.5,
ymax=746.5,
axis line style={draw=none},
ticks=none,
colorbar style={width=6,xshift=-7.5pt}
]
\addplot [forget plot] graphics [xmin=0.5, xmax=522.5, ymin=0.5, ymax=746.5] {fig_C3_3D_srmarm-5.png};
\end{axis}

\begin{axis}[%
width=0.234\textwidth,
height=0.321\textwidth,
at={(0.749\textwidth,0\textwidth)},
scale only axis,
axis on top,
xmin=0.5,
xmax=522.5,
tick align=outside,
y dir=reverse,
ymin=0.5,
ymax=746.5,
axis line style={draw=none},
ticks=none,
colorbar style={width=6,xshift=-7.5pt}
]
\addplot [forget plot] graphics [xmin=0.5, xmax=522.5, ymin=0.5, ymax=746.5] {fig_C3_3D_srmarm-6.png};
\end{axis}
\end{tikzpicture}%
    \vspace{-9mm}
    \caption{Three-dimensional evolution of the soft robot inspired by the elephant's trunk (whose muscular network is mimicked through six pneumatic bellows), slowly converging to the desired set-point $\vec{g}_d \in \SE{3}$ (\ie, the pink ball). }
    \label{fig:C3:multilink_3D}
  \end{figure*}
  %\vspace{14mm}
  \begin{figure}[!t]
    \centering
    \vspace{-10mm}
    % This file was created by matlab2tikz.
%
\definecolor{mycolor1}{rgb}{0.00000,0.34510,0.65882}%
\definecolor{mycolor2}{rgb}{0.79216,0.11765,0.17255}%
\definecolor{mycolor3}{rgb}{0.20392,0.65490,0.24706}%
\definecolor{mycolor4}{rgb}{0.93333,0.43922,0.13725}%
\definecolor{mycolor5}{rgb}{0.49412,0.14510,0.51373}%
\definecolor{mycolor6}{rgb}{0.97647,0.67059,0.08235}%
\definecolor{mycolor7}{rgb}{0.24314,0.18039,0.52549}%
\definecolor{mycolor8}{rgb}{0.71373,0.81961,0.14118}%
%
\begin{tikzpicture}

\begin{axis}[%
width=0.37\textwidth,
height=0.3\textwidth,
at={(0\textwidth,0\textwidth)},
scale only axis,
xmin=0,
xmax=10,
xlabel style={font=\color{white!15!black}},
xlabel={time (s)},
ymin=-0.004,
ymax=0.01,
ylabel style={font=\color{white!15!black}},
ylabel={$\vec{q}(t)$},
axis background/.style={fill=white},
xmajorgrids,
ymajorgrids,
ylabel style={yshift=-9.5pt}
]
\addplot [color=mycolor1, line width=1.5pt, forget plot]
  table[row sep=crcr]{%
0	9.99999999962142e-06\\
0.25	-1.9634258732637e-05\\
0.316666666666666	-0.00011889501723239\\
0.466666666666667	-0.00034974937035237\\
0.616666666666667	-0.000520996720670297\\
0.75	-0.000613174287103391\\
0.883333333333333	-0.000639429737178787\\
1.01666666666667	-0.000597517708481377\\
1.16666666666667	-0.000478776989314866\\
1.36666666666667	-0.000245271855137119\\
1.68333333333333	0.000203986561702507\\
2.28333333333333	0.00106020965843889\\
2.63333333333333	0.00148791324584074\\
2.98333333333333	0.00184817499330947\\
3.35	0.00215745619246199\\
3.75	0.00242661611856398\\
4.2	0.0026610486373233\\
4.71666666666667	0.00286251357521117\\
5.33333333333333	0.0030351972970184\\
6.1	0.00318099312643483\\
7.08333333333333	0.00329887698982922\\
8.41666666666667	0.00338959944313366\\
10	0.00344508664737297\\
};
\addplot [color=mycolor2, line width=1.5pt, forget plot]
  table[row sep=crcr]{%
0	9.99999999962142e-06\\
0.466666666666667	-5.42600181798747e-05\\
0.933333333333334	-0.000135609183020691\\
1.23333333333333	-0.000117068012054133\\
1.65	-1.63662585173086e-05\\
3.03333333333333	0.000358087067853674\\
3.83333333333333	0.000487785132346374\\
4.85	0.000582877273155091\\
6.31666666666667	0.000649290499589839\\
8.86666666666667	0.000690807358642687\\
10	0.000698180744862498\\
};
\addplot [color=mycolor3, line width=1.5pt, forget plot]
  table[row sep=crcr]{%
0	9.99999999962142e-06\\
1.1	-5.97302680791501e-05\\
1.9	-4.23020265003515e-05\\
4.35	2.89427732571568e-05\\
8.43333333333333	4.94966068345093e-05\\
10	5.12455086187913e-05\\
};
\addplot [color=mycolor4, line width=1.5pt, forget plot]
  table[row sep=crcr]{%
0	9.99999999962142e-06\\
1.15	-6.33747136777885e-05\\
2.3	-2.19828590868332e-05\\
4.4	3.21929963362777e-05\\
8.51666666666667	5.57868117478932e-05\\
10	5.79565893641387e-05\\
};
\addplot [color=mycolor5, line width=1.5pt, forget plot]
  table[row sep=crcr]{%
0	9.99999999962142e-06\\
0.25	-2.25472945487581e-05\\
0.316666666666666	-0.000117528576064174\\
0.533333333333333	-0.000452845955035031\\
0.833333333333334	-0.00086145789448544\\
1.08333333333333	-0.00113895610597048\\
1.35	-0.00136184618068036\\
1.71666666666667	-0.00159592804291542\\
2.23333333333333	-0.0018565011772349\\
2.83333333333333	-0.00209306616982374\\
3.45	-0.00227091142586389\\
4.15	-0.0024068843569367\\
5.03333333333333	-0.00250973011938527\\
6.26666666666667	-0.00258229124701259\\
8.3	-0.00262845124186306\\
10	-0.00264291883257073\\
};
\addplot [color=mycolor6, line width=1.5pt, forget plot]
  table[row sep=crcr]{%
0	9.99999999962142e-06\\
0.433333333333334	-5.08507486607357e-05\\
1.5	-0.000304754725128475\\
2.21666666666667	-0.000368067304856723\\
3.45	-0.000402925130639886\\
6.76666666666667	-0.00040750540525103\\
10	-0.00040536661652979\\
};
\addplot [color=mycolor7, line width=1.5pt, forget plot]
  table[row sep=crcr]{%
0	9.99999999962142e-06\\
1.18333333333333	-6.74707063321733e-05\\
2.25	-8.44764161183065e-05\\
10	-4.63332421016815e-05\\
};
\addplot [color=mycolor8, line width=1.5pt, forget plot]
  table[row sep=crcr]{%
0	9.99999999962142e-06\\
1.46666666666667	-9.27051775256871e-05\\
3.28333333333333	-0.000117443245319038\\
10	-0.000122239508373312\\
};
\addplot [color=mycolor1, dashed, line width=1.5pt, forget plot]
  table[row sep=crcr]{%
0	9.99999999962142e-06\\
0.183333333333334	3.75613182495016e-05\\
0.233333333333333	0.000111128747892764\\
0.266666666666667	0.000253310489346958\\
0.300000000000001	0.000512372462944555\\
0.366666666666667	0.00107589497567595\\
0.449999999999999	0.00168925102038742\\
0.6	0.00270920870560687\\
0.766666666666667	0.00378497709614933\\
0.9	0.00458139066604524\\
1	0.00511349594608035\\
1.1	0.00557067569007863\\
1.2	0.00596147652287904\\
1.31666666666667	0.00635019481783239\\
1.45	0.00672234366593649\\
1.58333333333333	0.00703154531622197\\
1.73333333333333	0.00731768512348907\\
1.9	0.00757282978812235\\
2.1	0.00780879449770921\\
2.31666666666667	0.00799624247446218\\
2.56666666666667	0.00814502033501974\\
2.86666666666667	0.0082531395897405\\
3.21666666666667	0.00831163916119415\\
3.66666666666667	0.00831890909931055\\
4.31666666666667	0.00825792942301895\\
6.15	0.00798591888230682\\
7.63333333333333	0.00781345956204582\\
9.25	0.00769351909867133\\
10	0.00765617621001269\\
};
\addplot [color=mycolor2, dashed, line width=1.5pt, forget plot]
  table[row sep=crcr]{%
0	9.99999999962142e-06\\
0.466666666666667	6.10690486446686e-06\\
1.15	0.000112176632287131\\
2.05	0.0002613539582903\\
2.85	0.000315609383013893\\
4.36666666666667	0.000337322219641223\\
10	0.000352120368605213\\
};
\addplot [color=mycolor3, dashed, line width=1.5pt, forget plot]
  table[row sep=crcr]{%
0	9.99999999962142e-06\\
0.383333333333333	4.04531362896421e-06\\
1.33333333333333	-7.52942860060557e-07\\
3.85	-0.000123682941051584\\
6.76666666666667	-0.000148552450827566\\
10	-0.00015335579626985\\
};
\addplot [color=mycolor4, dashed, line width=1.5pt, forget plot]
  table[row sep=crcr]{%
0	9.99999999962142e-06\\
0.616666666666667	-7.88021543627337e-05\\
1.26666666666667	-0.000172437536365422\\
2.01666666666667	-0.00019943732533001\\
3.83333333333333	-0.000177344024878678\\
8.2	-0.000143305727723586\\
10	-0.000139378999367779\\
};
\addplot [color=mycolor5, dashed, line width=1.5pt, forget plot]
  table[row sep=crcr]{%
0	9.99999999962142e-06\\
0.183333333333334	3.54001978895013e-05\\
0.233333333333333	0.000101430203935493\\
0.266666666666667	0.000227553188528518\\
0.300000000000001	0.000455571250443043\\
0.366666666666667	0.000945699918222687\\
0.449999999999999	0.00146889394027205\\
0.566666666666666	0.00213042702376498\\
0.699999999999999	0.00282627516949496\\
0.816666666666666	0.00337891053289852\\
0.916666666666666	0.00379753109035796\\
1.01666666666667	0.00415172598342295\\
1.11666666666667	0.00443574241745637\\
1.23333333333333	0.00469545680031125\\
1.36666666666667	0.00491993494947529\\
1.51666666666667	0.00510215423502203\\
1.68333333333333	0.00523874057509488\\
1.88333333333333	0.00533541277568084\\
2.13333333333333	0.00538452070990481\\
2.45	0.00537476547637361\\
2.91666666666667	0.0052860002652082\\
5.15	0.00478050346034742\\
6.23333333333333	0.00463877011428515\\
7.6	0.00452816595045746\\
9.46666666666667	0.00444682842326927\\
10	0.00443240675862455\\
};
\addplot [color=mycolor6, dashed, line width=1.5pt, forget plot]
  table[row sep=crcr]{%
0	9.99999999962142e-06\\
0.533333333333333	7.78314165117422e-07\\
1.35	4.40344546248639e-06\\
2.48333333333333	-6.31175408756235e-05\\
4.26666666666667	-0.000159007911499032\\
6.51666666666667	-0.000201423074940976\\
10	-0.000217741631880486\\
};
\addplot [color=mycolor7, dashed, line width=1.5pt, forget plot]
  table[row sep=crcr]{%
0	9.99999999962142e-06\\
0.383333333333333	4.98764000766982e-06\\
1.86666666666667	2.43087557905142e-05\\
4.25	1.31701567429587e-05\\
10	2.60154146403124e-05\\
};
\addplot [color=mycolor8, dashed, line width=1.5pt, forget plot]
  table[row sep=crcr]{%
0	9.99999999962142e-06\\
0.800000000000001	-9.67755703626949e-05\\
1.38333333333333	-0.00014719297617205\\
2.38333333333333	-0.000151678849697134\\
7.95	-0.000116441688177815\\
10	-0.000114785294995201\\
};
\end{axis}

\begin{axis}[%
width=0.37\textwidth,
height=0.3\textwidth,
at={(0.486\textwidth,0\textwidth)},
scale only axis,
xmin=0,
xmax=10,
xlabel style={font=\color{white!15!black}},
xlabel={time (s)},
ymin=-20,
ymax=50,
ylabel style={font=\color{white!15!black}},
ylabel={$\vec{p}(t)$},
axis background/.style={fill=white},
xmajorgrids,
ymajorgrids,
ylabel style={yshift=-9.5pt}
]
\addplot [color=mycolor1, line width=1.5pt, forget plot]
  table[row sep=crcr]{%
0	0\\
0.116666666666667	0.555035788942618\\
0.166666666666664	1.06219003317783\\
0.216666666666669	1.60246343385393\\
0.233333333333334	1.63766225772844\\
0.25	1.46682500858844\\
0.283333333333331	0.193873755963445\\
0.333333333333336	-2.13844897474269\\
0.350000000000001	-2.46040286720136\\
0.366666666666667	-2.54674490807219\\
0.383333333333333	-2.42458007776797\\
0.416666666666664	-1.65804007356524\\
0.466666666666669	0.523754573998374\\
0.533333333333331	4.81764927697563\\
0.633333333333333	13.0172037702731\\
0.866666666666667	32.7806341370626\\
0.966666666666669	39.0890699103559\\
1.05	42.8377185920357\\
1.11666666666667	44.8938621711081\\
1.18333333333333	46.2524022900842\\
1.23333333333333	46.8897112137274\\
1.28333333333333	47.2572497666775\\
1.31666666666667	47.3716367777974\\
1.33333333333334	47.3931737320013\\
1.35	47.3951238582711\\
1.38333333333333	47.3375962167593\\
1.41666666666666	47.2078844025908\\
1.46666666666667	46.8942065673314\\
1.53333333333333	46.2909244586992\\
1.61666666666667	45.2995686083801\\
1.73333333333333	43.5845689935051\\
1.88333333333333	41.0175986247204\\
2.16666666666666	35.6804485358905\\
2.53333333333333	28.8681961246315\\
2.76666666666667	24.9429012800531\\
2.98333333333333	21.6755233962065\\
3.18333333333333	18.9943912696178\\
3.38333333333333	16.6241999341712\\
3.58333333333334	14.5434947220667\\
3.76666666666667	12.8678235520672\\
3.95	11.3910114310935\\
4.13333333333333	10.0918881086443\\
4.31666666666667	8.94976895129363\\
4.5	7.94626211568307\\
4.68333333333334	7.06409498296375\\
4.86666666666667	6.28848370122914\\
5.05	5.60573918729641\\
5.25	4.95334637531487\\
5.45	4.38444655664622\\
5.65	3.88761631429502\\
5.85	3.45303872071123\\
6.05	3.07228728047511\\
6.26666666666667	2.71224636889075\\
6.48333333333333	2.39894967015381\\
6.7	2.12597667692378\\
6.91666666666666	1.88746845108027\\
7.15	1.6638477562358\\
7.4	1.45688981636304\\
7.65	1.27846965726039\\
7.93333333333334	1.10549188680034\\
8.21666666666667	0.958396673995907\\
8.55	0.812861697922131\\
8.91666666666666	0.680756145135383\\
9.31666666666667	0.563392187784018\\
9.76666666666667	0.4576214041303\\
10	0.411614098754434\\
};
\addplot [color=mycolor2, line width=1.5pt, forget plot]
  table[row sep=crcr]{%
0	0\\
0.0166666666666675	-0.043584139914886\\
0.0666666666666664	-0.0719380514879582\\
0.166666666666666	-0.138774275541513\\
0.183333333333334	-0.114044866657027\\
0.199999999999999	-0.0374749714716867\\
0.216666666666667	0.12959031397801\\
0.233333333333333	0.455465142322364\\
0.266666666666667	1.87662812225489\\
0.300000000000001	3.57777975321791\\
0.316666666666666	3.85301520017723\\
0.333333333333334	3.87020671523709\\
0.366666666666667	3.68383344297918\\
0.416666666666666	3.18148817783323\\
0.483333333333333	2.14782903682955\\
0.566666666666666	0.373317067716689\\
0.699999999999999	-3.11419698247251\\
0.916666666666666	-8.77499283375355\\
1.01666666666667	-10.8538644265167\\
1.1	-12.1503148131931\\
1.16666666666667	-12.8900933939312\\
1.23333333333333	-13.4154030008827\\
1.3	-13.765972151472\\
1.35	-13.9347845990472\\
1.4	-14.0343797592155\\
1.43333333333333	-14.0681463374813\\
1.46666666666667	-14.07849135188\\
1.5	-14.0676560738987\\
1.53333333333333	-14.037648510555\\
1.58333333333333	-13.9610575816769\\
1.65	-13.8065803558752\\
1.73333333333333	-13.5446590545671\\
1.83333333333333	-13.1526926220116\\
1.96666666666667	-12.5351174829895\\
2.16666666666667	-11.4892885018322\\
2.8	-8.11370094042524\\
3.03333333333333	-7.03338289062339\\
3.25	-6.14112635294935\\
3.45	-5.41154800874623\\
3.65	-4.7673950625444\\
3.85	-4.20159197160994\\
4.05	-3.70622276692833\\
4.25	-3.27331316648748\\
4.45	-2.89527945458179\\
4.65	-2.56515926815417\\
4.85	-2.27670540782292\\
5.06666666666667	-2.00477900105954\\
5.3	-1.75248691590606\\
5.53333333333333	-1.53583579719666\\
5.78333333333333	-1.33710318036402\\
6.03333333333333	-1.16725847305382\\
6.31666666666667	-1.0040725832953\\
6.6	-0.866498798599066\\
6.93333333333333	-0.731568566528265\\
7.3	-0.610156334977738\\
7.71666666666667	-0.499249008821748\\
8.15	-0.407519975656376\\
8.65	-0.324501277656482\\
9.21666666666667	-0.25250877467723\\
9.88333333333333	-0.189586681874749\\
10	-0.180432891787389\\
};
\addplot [color=mycolor3, line width=1.5pt, forget plot]
  table[row sep=crcr]{%
0	0\\
0.0666666666666664	0.00247796992254834\\
0.199999999999999	-0.0189300841278151\\
0.233333333333333	-0.0544439239156542\\
0.266666666666667	-0.143490608545081\\
0.300000000000001	-0.257903165765081\\
0.316666666666666	-0.271966694671509\\
0.466666666666667	-0.165395981885704\\
0.633333333333333	-0.0744685948239727\\
0.85	0.02044501561973\\
0.949999999999999	0.0403354852576019\\
1.06666666666667	0.0432522947327776\\
1.23333333333333	0.0153051197754301\\
1.56666666666667	-0.0822862910937197\\
1.91666666666667	-0.181725688760805\\
2.21666666666667	-0.240928866925413\\
2.5	-0.270512246032553\\
2.78333333333333	-0.277856666794159\\
3.18333333333333	-0.261434810963506\\
3.9	-0.19842860565867\\
5.01666666666667	-0.106935513697554\\
5.8	-0.0659570414464952\\
7	-0.0310314240386003\\
8.91666666666667	-0.0096357726006655\\
10	-0.00511644748007356\\
};
\addplot [color=mycolor4, line width=1.5pt, forget plot]
  table[row sep=crcr]{%
0	0\\
0.133333333333333	-0.000844211269733108\\
0.216666666666667	-0.0115035947884845\\
0.25	-0.030766375777036\\
0.316666666666666	-0.0825423297463939\\
0.516666666666667	-0.0523845086455594\\
1.1	0.0125115646645408\\
1.68333333333333	0.0212147712315751\\
2.36666666666667	0.0255148064010768\\
3.28333333333333	0.0285923896473204\\
4.01666666666667	0.0252197822470315\\
5.1	0.0177628030217871\\
6.41666666666667	0.0104695733834994\\
9.56666666666667	0.00303094679619598\\
10	0.00256648826946737\\
};
\addplot [color=mycolor5, line width=1.5pt, forget plot]
  table[row sep=crcr]{%
0	0\\
0.0500000000000007	0.177238959247351\\
0.0833333333333339	0.231455655676836\\
0.116666666666667	0.243944006043487\\
0.133333333333333	0.233767122231203\\
0.166666666666666	0.159406917862261\\
0.199999999999999	-0.0608337105446655\\
0.233333333333333	-0.668534265625036\\
0.266666666666667	-2.22024214249582\\
0.366666666666667	-9.26173886841285\\
0.416666666666666	-10.950967399183\\
0.466666666666667	-11.8414970310645\\
0.5	-12.1410747784578\\
0.533333333333333	-12.2767496454885\\
0.550000000000001	-12.2971377754201\\
0.566666666666666	-12.2949311319529\\
0.6	-12.2335705108536\\
0.666666666666666	-11.9926292163129\\
0.75	-11.6909769145835\\
0.800000000000001	-11.5883128572778\\
0.833333333333334	-11.5583255972767\\
0.866666666666667	-11.5544246300905\\
0.966666666666667	-11.5718625289924\\
1	-11.5240836531072\\
1.03333333333333	-11.4140536380437\\
1.1	-11.0498795024764\\
1.36666666666667	-9.36107180318786\\
1.46666666666667	-8.92535880052486\\
1.56666666666667	-8.60114371303335\\
1.66666666666667	-8.35563342241598\\
1.8	-8.09704062113663\\
2.21666666666667	-7.34242896678617\\
2.41666666666667	-6.90556207036078\\
2.68333333333333	-6.25221872930057\\
3.41666666666667	-4.40147611308132\\
3.66666666666667	-3.84408233093755\\
3.91666666666667	-3.34482668497963\\
4.13333333333333	-2.95979062493279\\
4.36666666666667	-2.59311353286655\\
4.6	-2.27243714106076\\
4.83333333333333	-1.99327957204016\\
5.06666666666667	-1.75093978068639\\
5.31666666666667	-1.52704058704006\\
5.56666666666667	-1.33483317305838\\
5.83333333333333	-1.15977360458324\\
6.13333333333333	-0.993773374838387\\
6.45	-0.84793015771054\\
6.76666666666667	-0.726546301836047\\
7.15	-0.606066416584197\\
7.55	-0.504626019387119\\
8	-0.413404953870339\\
8.56666666666667	-0.324522251548462\\
9.23333333333333	-0.246695865794791\\
10	-0.18197470067636\\
};
\addplot [color=mycolor6, line width=1.5pt, forget plot]
  table[row sep=crcr]{%
0	0\\
0.0166666666666675	-0.0435841399142536\\
0.0833333333333339	-0.0578223410321215\\
0.116666666666667	-0.0318257931858952\\
0.15	0.0343076958585389\\
0.183333333333334	0.193426273113138\\
0.216666666666667	0.597976775380959\\
0.25	1.63205082996371\\
0.316666666666666	4.75002225073091\\
0.35	5.07462978219923\\
0.416666666666666	5.43381329827187\\
0.466666666666667	5.58822153269\\
0.516666666666667	5.65989153800199\\
0.550000000000001	5.67486421723795\\
0.6	5.66736625710019\\
0.75	5.60866972796128\\
0.816666666666666	5.61931052118844\\
0.933333333333334	5.65496526957004\\
0.966666666666667	5.64278605036076\\
1	5.60660640523162\\
1.05	5.50025235395723\\
1.15	5.21734970586463\\
1.38333333333333	4.54709411229716\\
1.51666666666667	4.24332719465472\\
1.66666666666667	3.96153505167208\\
1.86666666666667	3.63804335403444\\
2.36666666666667	2.88553568621624\\
2.76666666666667	2.30782748964915\\
3.05	1.93656403441345\\
3.3	1.64569029795508\\
3.53333333333333	1.40755705599321\\
3.76666666666667	1.2010384890147\\
4	1.02390958581304\\
4.25	0.863258649151241\\
4.51666666666667	0.720761625826986\\
4.8	0.596800268363181\\
5.1	0.490705853832466\\
5.43333333333333	0.397085678280273\\
5.8	0.317012257678257\\
6.21666666666667	0.247884268475847\\
6.76666666666667	0.182181538255739\\
7.38333333333333	0.131674583892565\\
8.28333333333333	0.0849571566028455\\
9.58333333333333	0.047825663702092\\
10	0.0402491427690812\\
};
\addplot [color=mycolor7, line width=1.5pt, forget plot]
  table[row sep=crcr]{%
0	0\\
0.0666666666666664	0.00259856772862932\\
0.199999999999999	-0.0214120685359553\\
0.233333333333333	-0.0559491892008985\\
0.266666666666667	-0.138290306654802\\
0.300000000000001	-0.24260859862952\\
0.316666666666666	-0.256717016437893\\
0.449999999999999	-0.187725132993322\\
0.716666666666667	-0.0563897033698506\\
0.966666666666667	0.0984493637993236\\
1.1	0.150680802400139\\
1.25	0.178180313965116\\
1.38333333333333	0.180831433843336\\
1.56666666666667	0.162738282264419\\
2.06666666666667	0.0725262023688238\\
2.41666666666667	0.0204940993191478\\
2.85	-0.0173104218788378\\
3.41666666666667	-0.0357016408708741\\
4.5	-0.0333063512122465\\
10	-0.00371545633366566\\
};
\addplot [color=mycolor8, line width=1.5pt, forget plot]
  table[row sep=crcr]{%
0	0\\
0.133333333333333	-0.0010264820929109\\
0.216666666666667	-0.0118369478490088\\
0.25	-0.0297141546969026\\
0.316666666666666	-0.0769347570726655\\
0.5	-0.0586143075461614\\
0.783333333333333	-0.0405131052436509\\
1.16666666666667	-0.00947656829413113\\
1.66666666666667	-0.0106645966180228\\
2.21666666666667	-0.0291154004867238\\
3.8	-0.0294180357060849\\
4.95	-0.0166062130025466\\
6.06666666666667	-0.00878642472087066\\
9.58333333333333	-0.00145873860478041\\
10	-0.00115751647279616\\
};
\addplot [color=mycolor1, dashed, line width=1.5pt, forget plot]
  table[row sep=crcr]{%
0	0\\
0.0333333333333332	0.0181162152105916\\
0.0999999999999996	0.11249604838541\\
0.133333333333333	0.207368446869426\\
0.166666666666666	0.369800901034388\\
0.199999999999999	0.667898571646964\\
0.233333333333333	1.27326621930927\\
0.300000000000001	3.28535490721656\\
0.316666666666666	3.15587144529245\\
0.366666666666667	2.55938199160953\\
0.383333333333333	2.49344761378566\\
0.4	2.48581955907746\\
0.416666666666666	2.52805747835137\\
0.449999999999999	2.72848091543438\\
0.5	3.24160114608203\\
0.583333333333334	4.44258183970542\\
0.833333333333334	8.35949253864869\\
0.9	9.0713759703447\\
0.966666666666667	9.54604971782724\\
1.01666666666667	9.73759273309519\\
1.06666666666667	9.82555065100589\\
1.1	9.84832784176593\\
1.13333333333333	9.84487240960919\\
1.16666666666667	9.81773378415323\\
1.21666666666667	9.73819749884572\\
1.28333333333333	9.57284557001981\\
1.36666666666667	9.29495990025935\\
1.5	8.74552038201469\\
1.75	7.57391920597789\\
2.08333333333333	6.03903790334496\\
2.31666666666667	5.08309289677934\\
2.51666666666667	4.36088304786243\\
2.71666666666667	3.7289817726291\\
2.91666666666667	3.18224079412286\\
3.1	2.74941249370468\\
3.28333333333333	2.37435858129396\\
3.48333333333333	2.02374145206485\\
3.68333333333333	1.72579413875521\\
3.88333333333333	1.47287569534933\\
4.08333333333333	1.25824027116123\\
4.3	1.06225980022707\\
4.51666666666667	0.897825369454379\\
4.76666666666667	0.740864545219351\\
5	0.620157223242105\\
5.26666666666667	0.507015851051337\\
5.55	0.410103487558365\\
5.9	0.316499435845607\\
6.26666666666667	0.241931725053067\\
6.7	0.176722979975388\\
7.21666666666667	0.122026780731195\\
7.9	0.0752883897259533\\
8.71666666666667	0.0426050073601978\\
10	0.017729485546619\\
};
\addplot [color=mycolor2, dashed, line width=1.5pt, forget plot]
  table[row sep=crcr]{%
0	0\\
0.0166666666666675	-0.00144808023636855\\
0.0999999999999996	-0.083515393654821\\
0.133333333333333	-0.148482447902058\\
0.166666666666666	-0.254488234664539\\
0.199999999999999	-0.434253500563583\\
0.233333333333333	-0.765732857951539\\
0.300000000000001	-1.73315687239637\\
0.333333333333334	-1.45743397014828\\
0.366666666666667	-1.24180348931805\\
0.383333333333333	-1.20824504678191\\
0.4	-1.21764816810517\\
0.433333333333334	-1.3345182655704\\
0.483333333333333	-1.6920542823059\\
0.550000000000001	-2.3919227440371\\
0.683333333333334	-4.1321084651701\\
0.816666666666666	-5.79363699053429\\
0.9	-6.58737495774963\\
0.966666666666667	-7.03513171773882\\
1.01666666666667	-7.254240892967\\
1.08333333333333	-7.42775574821648\\
1.13333333333333	-7.49491814087563\\
1.16666666666667	-7.51356731473884\\
1.2	-7.51367390424357\\
1.25	-7.48359980367282\\
1.3	-7.42236469827396\\
1.36666666666667	-7.30228182945505\\
1.45	-7.10519545944837\\
1.58333333333333	-6.71782584388422\\
1.8	-5.99672358563719\\
2.23333333333333	-4.54681488905701\\
2.46666666666667	-3.85472762415288\\
2.68333333333333	-3.28796556789394\\
2.88333333333333	-2.8304676123116\\
3.08333333333333	-2.43264493502476\\
3.26666666666667	-2.11589089285846\\
3.46666666666667	-1.81711292075371\\
3.66666666666667	-1.56120548251622\\
3.86666666666667	-1.34241676490096\\
4.08333333333333	-1.14127781998577\\
4.3	-0.971605576072706\\
4.53333333333333	-0.818473952883892\\
4.78333333333333	-0.682600442214055\\
5.05	-0.563795708752993\\
5.33333333333333	-0.461368661379879\\
5.68333333333333	-0.36167683764282\\
6.03333333333333	-0.284623108389274\\
6.5	-0.208152653824319\\
7.06666666666667	-0.14369138496199\\
7.8	-0.0902600182907634\\
8.83333333333333	-0.0481664069104788\\
10	-0.0246044317389362\\
};
\addplot [color=mycolor3, dashed, line width=1.5pt, forget plot]
  table[row sep=crcr]{%
0	0\\
0.0500000000000007	0.00182714857773902\\
0.0999999999999996	0.00997279411578589\\
0.199999999999999	-0.00924729398276192\\
0.233333333333333	-0.0866342614788476\\
0.266666666666667	-0.291242242198356\\
0.300000000000001	-0.498192147831036\\
0.316666666666666	-0.525618305029237\\
0.35	-0.520746117667352\\
0.416666666666666	-0.466368850171007\\
0.5	-0.343400664177143\\
0.583333333333334	-0.176664405045608\\
0.983333333333333	0.707737559503768\\
1.08333333333333	0.871963078334996\\
1.16666666666667	0.971606932861969\\
1.26666666666667	1.05422453026699\\
1.36666666666667	1.10547014632956\\
1.48333333333333	1.13493020687801\\
1.6	1.13991362751152\\
1.75	1.11994632826463\\
1.91666666666667	1.07405006100555\\
2.15	0.985145844759286\\
3.26666666666667	0.530232830802936\\
3.61666666666667	0.426347367755422\\
4.03333333333333	0.328892318343307\\
4.45	0.254565269393281\\
4.98333333333333	0.184944057466504\\
5.63333333333333	0.127126666037913\\
6.36666666666667	0.0848586875504047\\
7.33333333333333	0.0512358298343045\\
8.68333333333333	0.0265032679845447\\
10	0.014541823290493\\
};
\addplot [color=mycolor4, dashed, line width=1.5pt, forget plot]
  table[row sep=crcr]{%
0	0\\
0.116666666666667	0.00517186834943928\\
0.183333333333334	0.0228888264238378\\
0.216666666666667	0.0511893055132653\\
0.25	0.119789684786682\\
0.300000000000001	0.262539084348472\\
0.316666666666666	0.262545984983982\\
0.383333333333333	0.224407585372855\\
0.449999999999999	0.222526340900229\\
0.533333333333333	0.245781117065244\\
0.666666666666666	0.309732887447883\\
0.883333333333333	0.418820533325896\\
0.983333333333333	0.44385807548773\\
1.11666666666667	0.443526883414123\\
1.31666666666667	0.418521016125178\\
1.65	0.344495882862876\\
2.21666666666667	0.213772886965735\\
2.61666666666667	0.143875305966185\\
3.05	0.0907046047257438\\
3.56666666666667	0.0505013869382367\\
4.25	0.0215071947715213\\
5.36666666666667	0.00280670835738661\\
6.88333333333333	-0.00290757295819333\\
10	-0.00193923462885692\\
};
\addplot [color=mycolor5, dashed, line width=1.5pt, forget plot]
  table[row sep=crcr]{%
0	0\\
0.0333333333333332	0.0171974607407233\\
0.15	0.140912430213367\\
0.183333333333334	0.242693995345538\\
0.216666666666667	0.487767219604409\\
0.25	1.08676734648603\\
0.283333333333333	1.97493815379028\\
0.300000000000001	2.06424371269492\\
0.333333333333334	1.45804782825219\\
0.383333333333333	0.643916929693766\\
0.433333333333334	0.234864068081581\\
0.483333333333333	0.049501108136397\\
0.516666666666667	-0.000889899451060217\\
0.550000000000001	-0.0136053334774999\\
0.583333333333334	-0.00142934031401154\\
0.65	0.0590875200915484\\
0.733333333333333	0.129057266274771\\
0.783333333333333	0.138818947579436\\
0.816666666666666	0.125250612827175\\
0.866666666666667	0.0723174275092813\\
0.916666666666666	-0.0172283673418843\\
1	-0.227715841828211\\
1.08333333333333	-0.434631280323265\\
1.18333333333333	-0.610208255197021\\
1.41666666666667	-0.948694735224599\\
1.68333333333333	-1.29938766856219\\
1.85	-1.47072233840341\\
1.98333333333333	-1.57178782231504\\
2.11666666666667	-1.6402125576159\\
2.25	-1.67821845986963\\
2.38333333333333	-1.68943555591544\\
2.53333333333333	-1.67504154641068\\
2.68333333333333	-1.63861894308051\\
2.9	-1.5569499219418\\
3.13333333333333	-1.44593792661778\\
3.58333333333333	-1.20851760204136\\
4	-0.99792228686305\\
4.41666666666667	-0.814694025210505\\
4.76666666666667	-0.68453322865464\\
5.18333333333333	-0.556362224366865\\
5.58333333333333	-0.456875569476816\\
6.03333333333333	-0.367545884090774\\
6.58333333333333	-0.283812183762937\\
7.18333333333333	-0.216077177875384\\
7.9	-0.157913149868829\\
8.85	-0.10606950377954\\
10	-0.0668545710120672\\
};
\addplot [color=mycolor6, dashed, line width=1.5pt, forget plot]
  table[row sep=crcr]{%
0	0\\
0.0166666666666675	-0.00144808023605592\\
0.0666666666666664	-0.0343880262819596\\
0.133333333333333	-0.0714910678617464\\
0.166666666666666	-0.104716696072876\\
0.199999999999999	-0.176080270676774\\
0.233333333333333	-0.350243563719168\\
0.283333333333333	-0.873571443707135\\
0.300000000000001	-0.857850022367231\\
0.4	0.295922050257833\\
0.449999999999999	0.526937438380653\\
0.483333333333333	0.602611289115144\\
0.516666666666667	0.636853716846721\\
0.550000000000001	0.641426581829458\\
0.6	0.611281152638664\\
0.683333333333334	0.515737851605852\\
0.783333333333333	0.408817756038131\\
0.85	0.374082675687399\\
0.9	0.371356190823535\\
0.966666666666667	0.392307815871812\\
1.06666666666667	0.432368433010916\\
1.16666666666667	0.431978866203403\\
1.3	0.434285323799372\\
1.4	0.457768103184765\\
1.51666666666667	0.507417167509804\\
1.73333333333333	0.631895836834921\\
1.98333333333333	0.76875383066379\\
2.15	0.835897384009936\\
2.31666666666667	0.879518041068771\\
2.48333333333333	0.900679290027586\\
2.66666666666667	0.901652556037067\\
2.9	0.876539625301774\\
3.2	0.815751744160188\\
3.75	0.670108508388587\\
4.48333333333333	0.483973869259824\\
5.05	0.370069438005059\\
5.65	0.278299576566006\\
6.35	0.201031561495748\\
7.23333333333333	0.135629377424683\\
8.05	0.0957281715290925\\
9.65	0.0502195076689667\\
10	0.0438840042045552\\
};
\addplot [color=mycolor7, dashed, line width=1.5pt, forget plot]
  table[row sep=crcr]{%
0	0\\
0.0500000000000007	0.00127125954513652\\
0.0999999999999996	0.00417561858126447\\
0.183333333333334	-0.0322175535176257\\
0.199999999999999	-0.0517479189696335\\
0.233333333333333	-0.144499707201664\\
0.266666666666667	-0.355200795147187\\
0.300000000000001	-0.571298134491933\\
0.333333333333334	-0.632928107404755\\
0.4	-0.681099238462156\\
0.466666666666667	-0.706549486171117\\
0.516666666666667	-0.711601288183292\\
0.6	-0.692609513493384\\
0.666666666666666	-0.658862729547664\\
0.75	-0.596654390359085\\
0.866666666666667	-0.474474862433368\\
0.983333333333333	-0.316727160097091\\
1.18333333333333	-0.024167074073846\\
1.3	0.105620999275992\\
1.41666666666667	0.201452246240546\\
1.53333333333333	0.268194093087656\\
1.66666666666667	0.315652432583002\\
1.81666666666667	0.341504191024569\\
1.98333333333333	0.346474867637516\\
2.2	0.329784961699001\\
2.6	0.269346235544557\\
3.26666666666667	0.167678833274824\\
3.76666666666667	0.113835708060297\\
4.33333333333333	0.0737098546082375\\
5.06666666666667	0.0431347229077073\\
6.1	0.0214682291869863\\
7.76666666666667	0.00774644200645902\\
10	0.00221557853653032\\
};
\addplot [color=mycolor8, dashed, line width=1.5pt, forget plot]
  table[row sep=crcr]{%
0	0\\
0.116666666666667	0.00393850308281252\\
0.183333333333334	0.0176383130469713\\
0.216666666666667	0.0409293836302744\\
0.25	0.099004423053616\\
0.300000000000001	0.218277037849973\\
0.333333333333334	0.202125393749586\\
0.4	0.162930787927268\\
0.483333333333333	0.14786344222132\\
0.833333333333334	0.110845536986893\\
0.949999999999999	0.0672339400618931\\
1.25	-0.0628177764287194\\
1.45	-0.119486860989433\\
1.68333333333333	-0.163340994319093\\
1.93333333333333	-0.18869330704482\\
2.26666666666667	-0.1962495122511\\
2.73333333333333	-0.177818533972271\\
4.78333333333333	-0.0632784935072142\\
5.83333333333333	-0.0360122468593644\\
8.16666666666667	-0.0117143540741136\\
10	-0.00534183603391369\\
};
\end{axis}
\end{tikzpicture}%
    \caption{The evolution of the modal coefficients and the generalized momenta of the soft robot manipulator inspired by the elephant's trunk. The modal coefficients $\q$ are ordered as follows $k \in \{\ldatanum{Matlab1}{1},\ldatanum{Matlab2}{2},\ldatanum{Matlab3}{3},\ldatanum{Matlab4}{4},\ldatanum{Matlab5}{5},\ldatanum{Matlab6}{6},\ldatanum{Matlab7}{7},\ldatanum{Matlab8}{8}\}$ where the full and dashed lines represent the first and second link, respectively. Observe that mainly modes 1 and 5 are dominant in both links. \label{fig:C3:multilink_states}}
  \end{figure}
\clearpage
}
\clearpage
\begin{table}[t]
\vspace{0.2cm}
\caption{Parameters setting for the numerical solver, the soft manipulator  by the elephant's trunk, and the energy-based controller.}\label{tab:C3:parameters2} \centering
\begin{tabular}{l|c|c|l}
  Parameter description & Symbol    & Value    & Unit                     \\
  \midrule
  Finite horizon time   & $T $      & $20$     & s                        \\
  Intrinsic length      & $L $      & $ 360$   & mm                       \\
  Cross-section radius  & $r $      & $ 25$    & $\text{mm}$              \\
  Uniform density       & $\rho_0 $ & $ 250$   & $\text{kg}\text{m}^{-3}$ \\
  %Gravitational acceleration & $a_{g} $ & $ 9.81$ & $\text{m}\text{s}^{-2}$ \\
  Young's modulus       & $E $      & $ 35$    & $\text{MPa}$             \\
  Shear modulus         & $\mu_1 $  & $ 20 $   & $\text{MPa}$             \\
  Constraint modulus    & $\mu_2 $  & $ 15 $   & $\text{GPa}$             \\
  Rayleigh coefficient  & $\zeta $  & $ 0.45 $ & -                        \\
  \bottomrule
\end{tabular}
\end{table}
%

