%!TEX root = ../../thesis.tex
The field of soft robotics is slowly growing as a prominent successor to conventional rigid robotics. Contrary to rigid robots, soft robots explore `\textit{soft materials}' that significantly enhance the robot's dexterity, inherent safety, enable a rich family of motion primitives, and provide environmental robustness. By fully exploiting soft materials, soft robotics places the first steps towards achieving performance similar to biology \cite{Choi2011,Falkenhahn2015,Marchese2014}. In this work, we primarily focus on a subclass of soft robots called '\textit{soft manipulators}'.

Although significant steps have been taken towards bridging biology and soft robotics, its innate infinite-dimensionality poses substantial challenges on modeling and control. To be more specific, soft robots theoretical allow for infinitely many degrees-of-freedom along their continuously deformable body. This renders them particularly suited for PDE models \cite{Duriez2013,Largilliere2015,Wu2021} rather than the conventional ODE for traditional robotics \cite{Spong2006,Murray1994}. Additionally, their actuation often employs distributed loads (\eg, pneumatics \cite{Falkenhahn2015,Marchese2014} and tendons \cite{Till2019,Wu2021}). Consequently, classical descriptions of rigid links and joints paired with local actuation are no longer viable nor physically representative. This paradigm shift calls for novel control-oriented modeling approaches tailored for hyper-flexible and under-actuated robots.

In the last decade, the field of modeling for soft robotic systems has matured sufficiently and currently their applicability in model-based control is slowly feasible \cite{DellaSantina2021}. To highlight a few: reduced-order finite element models \cite{Duriez2013,Zhang2017,Wu2021}, constant and non-constant curvature approaches \cite{Katzschmann2019,DellaSantina2020}, Cosserat-beam models \cite{Renda2020,Boyer2021}, and learning-based approaches \cite{Bruder2019}. The Piece-wise Constant Curvature (PCC) model -- a popular method of state reduction that assumes piecewise constant strains along the soft robot's body -- has proven to be viable for modeling solution applicable to feedforward controllers \cite{Falkenhahn2015}, and more recently model-based feedback controllers \cite{DellaSantina2020,Katzschmann2019}. Nevertheless, the PCC approach has (severe) limitations. They do not originate from continuum mechanics and thus are only applicable in restrictive settings. Although computationally performance might surpass continuous models, due to intrinsic kinematic restrictions, they are unable to capture important continuum phenomena, like buckling, environmental interaction, or wave propagation.

On the contrary, Cosserat beam-models have shown to capture a wide range of continuum deformations. Cosserat models originate from continuum mechanical PDE description and thus allow a more accurate description of the hyper-flexible nature under large deformations. The computational dynamics of Cosserat beams have been extensively developed by \cite{Simo1986} through Geometrically-Exact finite elements on the Lie group $\SE{3}$; and recently, these models are slowly gaining popularity in the soft robotics community \cite{Renda2018,Renda2020,Boyer2021,Till2019}. Ultimately, the strong nonlinearities paired with the diligence to achieve biological performance encourages Cosserat models for control. Yet, compared to the abundance of PCC soft robotic models, literature on model-based control is scarce.

In this chapter, we aim to highlight the capabilities of Cosserat models for model-based control, in particular energy-based strategies. To this end, a finite-dimensional modeling approach is proposed such that the continuous dynamics can be cast into a port-Hamiltonian (pH) structure. The Lagrangian modeling framework is adopted from \cite{Boyer2021} and \cite{Renda2020}, but modified to suit a pH-structure. The main advantage of pH systems is the common formalism with energy-based control. Through the pH structure, we propose an energy-shaping control law that ensures stabilization of the end-effector of the soft robot. Similar energy-based control strategies can be found in \cite{Franco2020,Schaft2004,Ortega2002,Ortega1998} for rigid-body systems. As a study case, we consider a soft robot manipulator inspired by an octopus tentacle (see Figure 1). With the ability to deform continuously and its distributed muscular system, it is ideal for illustrating the complex morphological motions present in soft robotics. Again, all code is made publicly available under the \sorotoki toolkit on \cite{SorotokiCode}, which builds upon previous work Caasenbrood et al. (2021, \cite{Caasenbrood2021}).

The chapter is organized as follows. Section \ref{sec:chap3_model} will detail a modeling approach for a general class of soft robot manipulators, starting with the Cosserat-beam theory. In Section \ref{sec:chap3_control}, we propose an energy-shaping control strategy. Lastly, we show the effectiveness of energy-based controller through numerical simulation in Section  \ref{sec:chap3_result}, followed by a brief conclusion in Section  \ref{sec:chap3_conclusion}.
