\chapter{Appendices to Chapter 5}
\section{Introduction to Lie group and Lie algebra} 
\label{app:C3:liegroup}
Here we briefly discuss some notation and basic operations on the Lie groups $\SO{3}$ and $\SE{3}$, and their respective Lie algebras $\sog{3}$ and $\seg{3}$. This appendix serves a compact comprehensive introduction in the context of robotics, and the appendix is based on the comprehensive work of Bullo and Murray (1995, \cite{Bullo1995}).

\subsection{Basic definition(s)}
Here, we focus first on the (Lie) group of rigid-body transformations about the origin of $\R^3$ -- denoted by $\SE{3}$. Let $G \subset \SE{3}$ be a matrix Lie group and its respective algebra $\mathfrak{g} \in \seg{3}$ its Lie Algebra. Then, the evolution of a general rigid body under motion with pose $\gB \in G$ can be described using 
%
\begin{equation}
\dot{\gB} = \gB\, \hat{\etaB}^{\textrm{b}} \;  \Longleftrightarrow  \; \dot{\gB} = \hat{\etaB}^{\textrm{s}}\,\gB. \quad \hat{\etaB}^{\textrm{b}},\hat{\etaB}^{\textrm{s}} \in \mathfrak{g}
\end{equation}
%
where the velocity twist relative its body frame is given by $\hat{\etaB}^\textrm{b}$ or to a spatial frame by $\hat{\etaB}^\textrm{s}$. Since $\dot{\gB} = \gB \hat{\etaB}^{\textrm{b}}$ is invariant under left multiplication by constant matrices, we call it \textit{left invariant}; and correspondingly $\dot{\gB} = \hat{\etaB}^{\textrm{s}} \gB $ is said to be \textit{right invariant}. Let it be clear that the geometric strain $\xiB(\sigma,t)$ and velocity twist $\etaB(\sigma,t)$ in \eqref{eq:C3:xi} and \eqref{eq:C3:eta} are thus expressed in the body frame. Next, let us discuss the adjoint actions. For all $\gB \in G$ and any $\XB,\YB \in \mathfrak{g}$, the adjoint action $\Ad_{\gB}$ and the matrix commutator or adjoint action on the algebra $\ad_{\XB}$ are defined as 
%
\begin{align}
\Ad_{\gB} \XB & = \gB \XB \gB \inv \label{eq:C3:A1:adjoint1}\\
\ad_{\XB} \YB & = \left[\XB,\YB\right]  = \XB \YB - \YB\XB. \label{eq:C3:A1:adjoint2}
\end{align}
%
Now, on $\SE{3}$ and $\seg{3}$ we represent a matrix group element $\gB = (\PhiB, \gammaB) \in  \SO{3} \times \R^3 \cong \SE{3}$ and a (velocity) twist vector field $\hat{\etaB} = (\OmegaB,\VB) \in \seg{3}$ using homogenous coordinates,
%
\begin{align}
    \gB  := 
    \begin{pmatrix}
    \; {\PhiB} & \gammaB \; \\
    \vec{0}_3^\top & 1
    \end{pmatrix}; \quad
    %
    \hat{\etaB} := 
    \begin{pmatrix}
    \OmegaB^{\times} & \vec{V} \; \\
    \vec{0}_3^\top & 1
    \end{pmatrix}.
\end{align}
%
where the operator $(\cdot)^{\times}:\R^3 \to \sog{3}$ is defined such that $\x^\times\yB = \x \times \yB$ for all $\x,\yB \in \R^3$, and $(\cdot)^\wedge:\R^6 \to \seg{3}$. Now, representing the geometric twist as a column vector $\etaB^\wedge \to \etaB$, it straightforwardly follows from \eqref{eq:C3:A1:adjoint1} and \eqref{eq:C3:A1:adjoint2} that the adjoint actions can be written in the form:
%
\begin{align}
    \Ad_{\vec{g}}  := 
    \begin{pmatrix}
    \; {\PhiB} & \vec{0}_{3\times3} \; \\
    \gammaB^{\times}\,{\PhiB} & {\PhiB}
    \end{pmatrix}; \quad
    %
    \ad_{\etaB}  := 
    \begin{pmatrix}
    \VB^{\times} & \vec{0}_{3\times3} \; \\
    \vec{\Omega}^{\times} & \VB^{\times} 
    \end{pmatrix}.
\end{align}
The notation above are analogous to the notations used in Chapter 3. \\ \vspace{-3mm}

\subsection{Exponential and logarithmic map}
An important operation in Lie group theory, is the exponential and logarithmic maps that serve as transformations between the groups and their respective algebras. Lets start with the exponential map. Given $\OmegaB^\times\in \sog{3}$ and $\etaB = (\OmegaB,\VB) \in \seg{3}$, the exponential maps for the orientation group $\exp_{\SO{3}}: \sog{3} \to \SO{3}$ and the rigid-body transformation group $\exp_{\SO{3}}: \seg{3} \to \SE{3}$ are given respectively by
%
\begin{align}
\exp_{\SO{3}}(\OmegaB) & = \IB + \sin \lVert \OmegaB \rVert\frac{\OmegaB^\times}{\lVert \OmegaB \rVert} + (1 - \cos \lVert \OmegaB \rVert) \frac{ \OmegaB^\times \OmegaB^\times }{\lVert \OmegaB \rVert^2}; \label{eq:C3:A1:rogdrigues} \\[0.45em]
\exp_{\SE{3}}(\xiB) & =
\begin{pmatrix}
\exp_{\SO{3}}(\OmegaB) & A(\OmegaB) \VB \\[0.45em]\vec{0}_3^\top & 1
\end{pmatrix},
\end{align}
%
where $\lVert \cdot \rVert$ stands for the Euclidean norm and the operator $A(\OmegaB)$ is given By
%
\begin{equation}
A(\OmegaB) = \IB + \left(\frac{1 - \lVert \OmegaB \rVert}{\lVert \OmegaB \rVert}\right) \frac{\OmegaB^\times}{\lVert \OmegaB \rVert} + (1 - \sin \lVert \OmegaB \rVert) \frac{ \OmegaB^\times \OmegaB^\times }{\lVert \OmegaB \rVert^2}.
\end{equation}
%
Note that we have seen equation \eqref{eq:C3:A1:rogdrigues} earlier which is known as the Rogdrigues' formula. Referring to \cite{Bullo1995}, in an open neighborhood of the origin in $G$, we define $\etaB = \log(\gB) \in \mathfrak{g}$ to be the \emph{exponential coordinates} of the group element $\gB$. 
Then, the logarithmic map can be regarded as the local chart of the manifold $G$. 
As such, let $\gB = (\PhiB, \gammaB) \in \SO{3} \times \R^3$ be such that $\trace(\PhiB) \neq -1$. Then, the logarithmic map $\log_{SO{3}}: \SO{3} \to \sog{3}$ is given by
%
\begin{equation}
  \log_{\SO{3}}(\PhiB) = \frac{\theta}{2\sin\theta}(\PhiB - \PhiB^\top), 
  \label{eq:C3:A1:logSO3} 
\end{equation}
%
where the angle $\theta$ satisfies $\cos \theta = \frac{1}{2} \left(\trace(\PhiB)- 1 \right)$ and is bounded by $|\theta| < \pi$. Following \eqref{eq:C3:A1:logSO3}, and logarithmic map $\log_{SE{3}}: \SE{3} \to \seg{3}$ is then given by
%
\begin{equation}
    \log_{\SE{3}}(\gB) = \begin{pmatrix}
        \OmegaB^\times & A\inv(\OmegaB) \gammaB \\[0.45em]\vec{0}_3^\top & 1
        \end{pmatrix},
  \end{equation}
  %
where $\OmegaB^\times = \log_{\SO{3}}(\PhiB)$ and the mapping $A\inv(\OmegaB)$ is given By
%
\begin{equation}
A\inv(\OmegaB) = \IB - \frac{1}{2}\OmegaB^\times + \left[1 - \frac{\lVert \OmegaB \rVert}{2}\cot\left(\frac{\lVert \OmegaB \rVert}{2}\right) \right]  \frac{ \OmegaB^\times \OmegaB^\times }{\lVert \OmegaB \rVert^2},
\end{equation}
%
where $\cot: \R \to \R$ is the co-tangent function.
