%!TEX root = ../../thesis.tex
\section{Energy-based controller for tasks on SE(3)} \label{sec:chap3_control}
Given the reduced port-Hamiltonian model in \eqref{eq:C3:model_ph}, our objective here is to find a controller $\vec{u}$ that ensures $\lim_{t\to\infty} \mat{g}(L,t) = \mat{g}_d$ in which $\mat{g}_d \in \SE{3}$ denotes the desired configuration of the end-effector. To achieve the control objective, the main idea is to reshape the potential energy function of the reduced-order finite-dimensional system using a standard energy-shaping techniques, common to standard port-controlled Hamiltonian models \cite{Schaft2004,Ortega2002}.

%\subsection{Setpoint regulation on $\SE{3}$}
We adopted an energy-based control strategy akin to the works of Franco et al. \cite{Franco2020}, Ortega et al. \cite{Ortega1998,Ortega2002} and Schaft \cite{Schaft2004}. Following the energy-shaping strategy, the model-based nonlinear controller becomes
%
\begin{equation}
\vec{u} = \vec{G}\ginv \left( \nabla_{\vec{q}}\Hm - \nabla_{\vec{q}}\Hm_d \right),
\label{eq:C3:control}
\end{equation}
%
where $\mat{G}\ginv = \bigl(\mat{G}^\top \mat{G}\bigr)\inv \mat{G}^\top$ is the generalized inverse of the actuation map $\mat{G}$, and the desired Hamiltonian in closed loop
$\Hm_d = \frac{1}{2} \vec{p}^\top \mat{M}\inv \vec{p} + \Uf_d$ that satisfies $\textrm{argmin}_{\mat{g}_L} \Uf_d = \mat{g}_d$ with $\mat{g}_L = \mat{g}(L,\cdot)$ the pose of the end-effector. Note that we purposefully omitted any damping injection as the system's intrinsic visco-elastic damping is deemed sufficiently large to guarantee stability. Following the concept of a kinematic feedback controller \cite{Bullo1995,Boyer2021} that artificially mimic an elastic element between the end-effector and the desired configuration in $\SE{3}$, we have choose the gradient of the desired potential energy as
%
\begin{equation}
\grad{\q}\,\Uf_{d}= \lambda_1 \mat{J}^\top\bigl(\mat{J J}^\top + \lambda_2 \mat{I} \bigr)\inv \ten{F}_u,
\label{eq:C3:jacob_damped}
\end{equation}
%
where $\lambda_1 >0$ is a proportional gain, $\lambda_2 > 0$ a controller gain related to the artificial damping of the pseudo-inverse, $\ten{F}_u = \kB_p T_{\textrm{SE}(3)}(\mat{\mathcal{E}})\mat{\mathcal{E}}$ an artificial control wrench with positive definite stiffness matrix $\kB_p$, $\mat{\mathcal{E}} = \log_{\SE{3}} ([\mat{g}_L]_k \inv \mat{g}_d)$ where $\log_{\SE{3}}(\cdot)
$ and $T_{\SE{3}}(\cdot)$ denote the logarithmic map (see Appendix \ref{app:C3:liegroup}) and the tangent-space map, respectively. We refer the reader to Sonneville et al. \cite{Sonneville2014} for the numerical computations of the tangent map on $\SE{3}$. The vector $\vec{\mathcal{E}}$ may be regarded as the geometric error between the homogeneous transformations $\mat{g}_d$ and $\mat{g}_L$ such that if $\mat{g}_d = \mat{g}_L$ will simply yield $\lVert \vec{\mathcal{E}} \rVert_2 = 0$. Furthermore, the controller gains $\lambda_1$ and $\lambda_2$ can be tuned accordingly to tweak the desired transient behavior of the closed-loop system, similar to a classical PD controller.