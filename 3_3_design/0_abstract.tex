%!TEX root = ../../thesis.tex
\vspace{-3mm}
\chapterabstract{In this chapter, we present a novel framework for synthesizing the design of pressure-driven soft robots. Contrary to traditional design methods, a topology optimization scheme is employed to find the optimal soft robotic structure given user-defined motion requirements. To the best of our knowledge, the combination of pressure-driven topology optimization and soft robotics is, as of the date of this thesis, unexplored. Two difficulties are related to this problem. First, pressure-based topology optimization is challenging since the adaptive topology changes the fluidic load at each optimization step. To deal with this issue, we exploit the facial connectivity in polygonal mesh tessellations to efficiently simulate the physics involving fluidic actuation in soft robotics. The second issue is describing the hyper-elastic nature of soft materials. Here, the nonlinear Finite Element Method (FEM) is explored so that large deformations can be accurately described. Numerical investigation shows that the framework can produce meaningful and insightful material layouts with little to no prior knowledge of soft robotic design. We show the existence of new structures, but also familiar soft robotics structures such as the \textit{PneuNet}. Interestingly, when considering a spectrum of soft materials, we demonstrate that parts of the optimal structural topology appear invariant regardless of the choice of materials, while other design aspects are directly related to elasticity, at times even non-monotonically. In short, the proposed framework not only accelerates design convergence but can also extend to the development of new and unexplored soft robot morphologies.}
