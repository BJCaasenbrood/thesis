\section{Element formulation for the Lagrangian strain}
\label{app:C3:straindisplacement} 
When solving finite element numerically, it is often more convenient to utilize matrix-vector notation as opposed to tensor notation. This approach involves representing second-order symmetric tensors using vectors, while fourth-order symmetric tensors are represented using matrices. Given the two-dimensional formulation of the finite element problem described in Chapter \ref{chap: design}, it is possible to define the Green-Lagrange strain tensor and second Piola stress strain tensor, which are both symmetric second-order tensors, as vector in $\R^3$ following the Voigt notation:
%
\begin{align}
\ten{E} = \begin{bmatrix} 
\mathcal{E}_{11} & \cdot & \cdot \\
\mathcal{E}_{12} & \mathcal{E}_{22} & \cdot \\
\mathcal{E}_{13} & \mathcal{E}_{23} & \mathcal{E}_{33} 
\end{bmatrix}
 \;\; \Longrightarrow \;\;
\textrm{voigt}_{\textrm{2D}}\left( \ten{E} \right) &= \left(E_{11},\; E_{22},\, 2E_{12} \right)^\top,
\end{align}
\begin{align}
\ten{S} = \begin{bmatrix} 
\mathcal{S}_{11} & \cdot & \cdot \\
\mathcal{S}_{12} & \mathcal{S}_{22} & \cdot \\
\mathcal{S}_{13} & \mathcal{S}_{23} & \mathcal{S}_{33} 
\end{bmatrix}
    \;\; \Longrightarrow \;\;
\textrm{voigt}_{\textrm{2D}}\left( \ten{S} \right) &= \left(S_{11},\; S_{22},\, S_{12} \right)^\top,
\end{align}
%
where the two-dimensional reduction follows from the plane strain conditions in which $\mathcal{E}_{i3} = \mathcal{E}_{3i} = 0$ for all $i = \{1,2,3\}$; and $\mathcal{S}_{i3} = \mathcal{S}_{3i} = 0$ for $i = \{1,2\}$. For convenience, we will introduce the following $\underline{\ten{E}}:=\textrm{voigt}_{\textrm{2D}}(\ten{E})$ and $\underline{\ten{S}}:=\textrm{voigt}_{\textrm{2D}}(\ten{S})$. Our goal now is to derive an expression for the (nonlinear) strain-displacement matrix $\BB(\dB)$ which relates the displacement field and the variation of the Lagrangian strain as $\delta \underline{\ET} = \BB(\dB) \delta \dB$. To do so, we explore the elemental interpolation using the Wachspress shape function that we derived previously in Appendix \ref{app:C3:wachpress}. The derivation for the Piolla stress is dependent on the constitutive material models, thus we presented its derivation separately in Appendix \ref{eq:app:C3:psi}.

The following vector form of the displacement gradient is defined first
\begin{align}
\textrm{voigt}_{\textrm{2D}} \left(\nabla_0 \dB \right) & = \left( \frac{\p d_1}{\p X_1},\,\frac{\p d_1}{\p X_2},\, \frac{\p d_2}{\p X_1},\, \frac{\p d_2}{\p X_2}\right)^\top \\[0.25em]
\textrm{voigt}_{\textrm{2D}} \left(\FB\right) & = \left( F_{11},\,F_{12},\, F_{21},\,F_{2} \right)^\top \notag \\[0.15em] & = \left( 1 + \frac{\p d_1}{\p X_1},\,\frac{\p d_1}{\p X_2},\, \frac{\p d_2}{\p X_1},\,  1 + \frac{\p d_2}{\p X_2}\right)^\top 
\end{align}
%
Then, the symmetric second-order Lagrangian strain tensor defined as $\ET = \frac{1}{2}(\FB^\top \FB - \boldsymbol{I})$ expressed using the Voigt notation can be calculated by
%
\begin{equation}
 \underline{\ten{E}} = \begin{pmatrix}
\dfrac{\p d_1}{\p X_1} + \dfrac{1}{2} \left(\dfrac{\p d_1}{\p X_1}\dfrac{\p d_1}{\p X_1} + \dfrac{\p d_2}{\p X_1}\dfrac{\p d_1}{\p X_2} \right) \\[0.25em] \\
\dfrac{\p d_2}{\p X_2} + \dfrac{1}{2} \left(\dfrac{\p d_1}{\p X_2}\dfrac{\p d_2}{\p X_1} + \dfrac{\p d_2}{\p X_2}\dfrac{\p d_2}{\p X_2} \right) \\[0.25em] \\
\dfrac{\p d_1}{\p X_2} + \dfrac{\p d_2}{\p X_1} + \dfrac{\p d_1}{\p X_2}\dfrac{\p d_1}{\p X_1} + \dfrac{\p d_2}{\p X_1}\dfrac{\p d_2}{\p X_2} 
 \end{pmatrix}
\end{equation}
%
Then, finally, the nonlinear displacement–strain matrix can be computed as follows
\begin{equation}
\BB_{e} = \left(\BB_1,\,\BB_2,\,\BB_3, ...,\, \BB_{n_p} \right),
\end{equation}
%
\begin{align}
\BB_i = \begin{pmatrix} 
F_{11} \dfrac{\p N_{i}}{\p X_1}, & F_{12}  \dfrac{\p N_{i}}{\p X_2}, &  F_{11}  \dfrac{\p N_{i}}{\p X_2}  \; + F_{12} \dfrac{\p N_{i}}{\p X_1} \\[0.25em] \\
F_{21}  \dfrac{\p N_{i}}{\p X_1},  & F_{22}  \dfrac{\p N_{i}}{\p X_2}, & F_{21}  \dfrac{\p N_{i}}{\p X_2} \; + F_{22}  \dfrac{\p N_{i}}{\p X_1}  
\end{pmatrix}^\top
\end{align}
%
