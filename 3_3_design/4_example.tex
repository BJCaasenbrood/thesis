\section{Numerical Examples}
\label{chap:results}
As an illustrative example of the proposed computational algorithm, we synthesize a topology of a soft robot undergoing bending motion. This morphology is often associated with a popular class of soft robots named `PneuNet' actuators \cite{Polygerinos2015b, Galloway2016, Hughes2016}. PneuNet actuators consist of a set of rectangular pneumatic chambers inside an elastomer medium. When pressurized, these chambers inflate, and the elastomer structure undergoes bending. The goal of this numerical study case is to synthesize a soft robot topology that undergoes bending motion similar to the PneuNet actuator. 

In our example, the following settings are chosen. We consider a rectangular design domain $\mathcal{B}_0$ with dimensions 20$\times$40 mm (as shown in Fig. \ref{fig:topo_result}). The maximum material infill is $V^* = 0.3$. The bottom left corner is fixed, and the right bottom corner is equipped with an artificial spring $k_{out} = 1.0$ N/mm. The objective of the optimization is to maximize the vertical displacement of the left bottom corner in the direction $\boldsymbol{u}_{out}$. It shall be clear that the entries of the vector $\boldsymbol{L}$ are non-zero for the corresponding nodal degrees-of-freedom. The Yeoh parameters are $c_1 = 36$ kPa, $c_2 = 0.25$ kPa, and $c_3 = 0.023$ kPa modelled after the silicone elastomer Dragonskin 10A. Concerning the pneumatic actuation, the adaptive topology is subjected to an artificial pressure parameter $\tilde{p}_e = 0.01$, where the virus element is located at the center of the material domain $\mathcal{B}_0$.

\begin{figure}[!b]
    \centering
    \vspace{-3mm}
    %\includegraphics[height=0.33\textwidth]{./img/topo_bend/resultTopo.eps}
    \caption{Topology optimization results for a soft bending actuator within the design domain of $20 \times 40$ mm. The Yeoh material parameters are $c_1 = 36$ kPa, $c_2 = 0.25$ kPa, and $c_3 = 0.023$ kPa; and $V^* = 0.3$.}
    \label{fig:topo_result}
  \end{figure}

  Using the method described in Algorithm \ref{alg:topology_opt}, an optimized topology is obtained (as shown in Fig. \ref{fig:topo_result}). As can be seen, the computational algorithm provides a new and interesting variation on the well-familiar PneuNet actuator. Contrary to its rectangular predecessor, the optimal structure has a significant resemblance to a bellow-shaped actuator to accommodate bending mobility further. To validate our synthesized topology, a three-dimensional finite element analysis is conducted. First, we use Gaussian radial basis functions (RBFs) to reconstruct a smooth manifold surface from the discretized optimization mesh. Due to spatial symmetry, the spatial reconstruction can be horizontally repeated to construct a full `PneuNet' actuator (see Fig. \ref{fig:filter}). The two-dimensional geometry is then imported into CAD and converted to three-dimensional geometry with enclosed pneumatic chambers. The pneumatic chambers are subjected to a positive differential pressure of $\Delta p = 15$ kPa. As can be illustrated by Fig. \ref{fig:fem_topo}, the finite element analysis verifies that the synthesized soft robot topology accomplishes the desired bending morphology when pressurized.