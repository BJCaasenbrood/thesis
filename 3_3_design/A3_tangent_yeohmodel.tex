\section{Piolla stress tensor and stress-strain tangent for the nearly incompressible Yeoh model} \label{app:C3:yeohmodel}
In the following section, we will derive the second Piolla stress tensor and the stress-strain tangent tensor which are derived from the constitutive material model $\Psi$. Previously, we have introduce the nearly incompressible Yeoh model expressed as
%
\begin{equation}
\Psi(J_1,J) = \sum_{i = 1}^{3} c_i (J_1 - 3)^i + \sum_{j = 1}^{3} \frac{1}{d_j} (J - 1)^{2j},
\label{eq:app:C3:psi}
\end{equation}
%
where $J_1 = J^{-\tfrac{2}{3}} I_1$ with $I_1 = \trace({\ten{E}})$ the first true strain invariant and $J = \det{\FB}$ the volumetric change, and $c_i$ and $d_i$ material parameters. To express the second Piolla-Kirchhoff stress tensor, denoted by $\ten{S}$, and the fourth-order tangent stiffness tensors, $\mathbb{D}$, it is necessary to obtain the partial derivatives of $J$ and $J_1$ with respect to the Green-Lagrangian strain tensor $\ten{E}$.  These derivatives are provided in the works \cite{Kim2018,Renaud2011} as follows: 
%
\begin{equation}
\frac{\p J}{\p \ten{E}} = J \ten{C}\inv, \quad \quad
\frac{\p J_1}{\p \ten{E}} = \IB - \frac{1}{3} \ten{C}\inv.
\end{equation}
%
Taking the derivative of the constitutive material model $\Psi$ with respect to $\ten{E}$, and substitution of the strain invariant derivatives into \eqref{eq:app:C3:psi} leads to 
%
\begin{align}
\ten{S} = 2\beta_{1} J^{-\tfrac{2}{3}}\left[\IB - \frac{I_1}{3}\CT\inv \right] & + \beta_2 J \ten{C}^{-1},
\label{eq:app:C3:secondPiollastress}
\end{align}
%
where $\beta_1 = \sum_{i=1}^3 i {c_i} (J_1 - 3)^{i-1} $ and $\beta_2 = \sum_{j=1}^3 \frac{2j}{d_j} \left(J - 1\right)^{2j - 1}$ are scalar functions of the strain variants. In order to solve nonlinear structures using the finite element method, it is necessary to calculate the stress-strain tangent operator $\mathbb{D}$ in order to construct the tangent stiffness matrix $\KB_T$. First, let us introduce the following tensor operation between two second-order tensors $\AT$ and $\BT$ as: 
%
\begin{equation}
(\AT \otimes \BT)_{ijkl} = A_{ij} B_{kl}, \quad \quad (\AT \,\overline{\underline{\otimes}}\, \BT)_{ijkl} = \frac{1}{2}\left(A_{ik} B_{jl} + A_{il}B_{jk} \right)
\end{equation}
%
for indices $i,j,k,l \in \{1,2,3\}$. Then, the derivative of the second Piolla stress tensor in \eqref{eq:app:C3:secondPiollastress} with respect to the Lagrangian strain $\ten{E}$ is gives the stress-strain tangent tensor $\mathbb{D}$ which is a fourth-order tensor calculated as follows \cite{Renaud2011}:
%
\begin{align}
\mathbb{D} & = 4J^{-\tfrac{4}{3}}\gamma_1 \left[\IB - \frac{I_1}{3}\CT\inv \right] \otimes \left[\IB - \frac{I_1}{3}\CT\inv \right] \notag \\[0.25em]
& - \frac{4}{3}J^{-\tfrac{2}{3}} \beta_1 \left[\ten{C}\inv \otimes \IB + \IB \otimes \ten{C}\inv + \frac{I_3}{3} \ten{C}\inv \otimes \ten{C}\inv - I_1 \ten{C}\inv \overline{\underline{\otimes}} \,\ten{C}\inv \right] \notag \\[0.25em]
& + \gamma_2 J^2 \ten{C}\inv \otimes \ten{C}\inv + \beta_2 J \left[\ten{C}\inv \otimes \ten{C} \inv - 2 \ten{C}\inv \overline{\underline{\otimes}} \ten{C}\inv \right],
\end{align}
%
\newpage
\noindent where $\gamma_1 = \sum_{i=2}^3 i(i-1) {c_i} (J_1 - 3)^{i-2}$ and $\gamma_2 = \sum_{j=1}^3 \frac{2j(2j-1)}{d_j} \left(J - 1\right)^{2j - 2}$ are scalar functions. It is important to note that the stress-strain tangent tensor has major and minor symmetries,\ie, $\mathbb{D}_{ijkl} = \mathbb{D}_{klij}$ and $\mathbb{D}_{jijl} = \mathbb{D}_{jilk}$ \cite{Kim2018,Holzapfel2002}. Hence, by utilizing the Voigt notation, these symmetries can be leveraged to compactly represent the fourth-order stress-strain tangent tensor as an equivalent matrix of size 6 by 6 as follows:
%
\begin{equation}
\DB = \begin{pmatrix}
\mathbb{D}_{1111} & \mathbb{D}_{1122}  & \mathbb{D}_{1133}  & \mathbb{D}_{1123}  & \mathbb{D}_{1113}  & \mathbb{D}_{1112} \\
\cdot & \mathbb{D}_{2222}  & \mathbb{D}_{2233}  & \mathbb{D}_{2223}  & \mathbb{D}_{2213}  & \mathbb{D}_{2212} \\
\cdot & \cdot & \mathbb{D}_{3333}  & \mathbb{D}_{3323}  & \mathbb{D}_{3313}  & \mathbb{D}_{3312} \\
\cdot & \cdot & \cdot  & \mathbb{D}_{2323}  & \mathbb{D}_{2313}  & \mathbb{D}_{2312} \\
\cdot & \cdot & \cdot  & \cdot & \mathbb{D}_{1313}  & \mathbb{D}_{1312} \\
\cdot & \cdot & \cdot  & \cdot  & \cdot  & \mathbb{D}_{1212} 
\end{pmatrix}.
\end{equation}
%
Furthermore, considering the plane strain conditions and utilizing the Voigt notation for two-dimensional problems (refer to Section \ref{app:C3:straindisplacement}), we obtain that
%
\begin{equation}
\underline{\DB} := \textrm{voigt}_{\textrm{2D}}(\DB_0) = \begin{pmatrix}
\mathbb{D}_{1111} & \mathbb{D}_{1122} & 0 \\
\mathbb{D}_{2211} & \mathbb{D}_{2222} & 0 \\
0 & 0 & \mathbb{D}_{1212} 
\end{pmatrix},
\end{equation}
where we use that $\mathbb{D}_{ij12} = 0$ for all $i,j \in \{1,2,3\}$ except $i=1$ and $j=2$. Applying a reduced formulation, we obtain the tangent stiffness matrix by evaluating the Jacobian of the conservative elastic forces with respect to the state displacements $\xB$. The expression for this matrix is as follows:
%
\begin{align}
\KB_T & := \frac{\p}{\p \x}  \left[ \sum_{e=1}^{n} \int_{\mathcal{V}_e} \BB_e^\top \underline{\ten{S}}_e \; dV \right] = \sum_{e=1}^{n} \int_{\mathcal{V}_e} \frac{\p \BB_e^\top}{\p \x} \underline{\ten{S}}_e +  \BB_e^\top \frac{\p \underline{\ten{S}}_e}{\p \ten{E}} \frac{\p \ten{E}}{\p \x} \; dV \notag \\[0.25em]
& = \sum_{e=1}^{n} \int_{\mathcal{V}_e} \frac{\p \BB_e^\top}{\p \x} \underline{\ten{S}}_e +  \BB_e^\top \underline{\DB} \BB_e \; dV.
\end{align}
%
We should mention that the numerical computation of the tangent stiffness matrix is considerable, and often the numerical bottleneck for fast FEM simulation. It is thus often wise to optimize the computation above, for instance, using GPU parallel computation. Alternatively, there exist methods that circumvent extensive computations. The BFGS method is an iterative method that aims to find the minimum (or maximum) of a nonlinear objective function \cite{Holzapfel2002}; or the secant-stiffness method \cite{Kim2018}. Both approximate the inverse Hessian matrix of the objective function by updating it at each iteration, saving computational cost considerably. 