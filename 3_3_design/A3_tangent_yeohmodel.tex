\clearpage
\section{Piolla stress tensor and stress-strain tangent operator for the compressible Yeoh model}
\label{app:C3:yeohmodel}
In section \ref{sec:C2:hyperelastic}, we have introduce the compressible Yeoh model given by 
%
\begin{equation}
\Psi(J_1,J) = \sum_{i = 1}^{3} c_i (J_1 - 3)^i + \sum_{j = 1}^{3} \frac{1}{d_j} (J - 1)^{2j},
\label{eq:app:C3:psi}
\end{equation}
%
where $J_1 = J^{-\tfrac{2}{3}} I_1$ with $I_1$ the first true strain invariant of the Lagrangian strain tensor, and $c_i$ and $d_i$ material parameters. To express the second Piolla-Kirchhoff stress tensor $\ten{S} = \p \Psi/ \p \ten{E}$ and the fourth-order tangent stiffness tensors, \ie, $\mathbb{D} = \p \ten{S}/ \p \ten{E}$, we require the partial derivatives of $J$ and $J_1$ with respect to the Green-Lagrangian strain tensor $\ten{E}$. According to \cite{Kim2018,Renaud2011}, it follows that $\tfrac{\p J}{\p \ten{E}} = J \ten{C}\inv$  and therefore it follows that
%
\begin{align}
\frac{\p J_1}{\p \ten{E}} = \IB - \frac{1}{3} \ten{C}\inv 
\end{align}
%
substitution of the tensor derivatives into \eqref{eq:app:C3:psi}