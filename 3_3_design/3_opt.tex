\section{Nonlinear topology optimization}
\label{chap:topo} 
\subsection{Solid Isotropic Material With Penalization (SIMP)}
The Solid Isotropic Material With Penalization (SIMP) method is a popular material interpolation scheme used in topology optimization \cite{Sigmund2015,Gain2013,Vasista2013}. In the approach, each finite element $e \in \{1,2,...,n\}$ is assigned with a continuous density variable $\rho_e \in [0,1]$ to indicate if an elemental volume is solid ($\rho_e = 1$) or void ($\rho_e = 0$). This leads to a real-valued density field representing the material distribution within a discretized domain $\mathcal{B}_0$, where the global material distribution is represented by the density vector $\vec{\rho} = \left[\,\rho_1,\,\rho_2,...,\,\rho_n\,\right]^\top$. The density vector $\vec{\rho}$ contains the design variables of the optimization scheme. To relate these artificial densities to elasticity, the strain energy of the constitutive material model ${\Psi}$ is then parameterized using $\vec{\rho}$. To improve numerical robustness, a modified SIMP approach is used in which the strain-energy function is artificially modified as follow 
%
\begin{equation}
{\Psi}_e = [\,\varepsilon + (1-\varepsilon)\,{\rho_e}^p\,]\, {\Psi}, \label{eq:simp}
\end{equation}
%
where $p>1$ is the penalty factor for penalizing intermediate densities during the optimization process, and $0 < \varepsilon \ll 1$ is a lower-bound on the material densities. In this work, we choose $p = 4$ and $\varepsilon = 10^{-3}$. Given the artificial elasticity model above, it shall be clear that residual vector function in \eqref{eq:residual} now depends on the nodal displacements and the densities of the adaptive topology, $\boldsymbol{x}$ and $\boldsymbol{\rho}$, respectively.

\subsection{Artificial pneumatic loads by exploring dilation}
The most general principle of actuation in soft robotics involves pneumatic networks that are embedded in the elastic body. Within the context of topology optimization, however, describing internal pressure-based loads is difficult. Contrary to conventional optimization problems with static loads, in a pressure-based problem, the location, direction, and magnitude of the load changes concerning the material distribution at every optimization step. This inherently yields design-dependency in the global force vector. To resolve this difficulty, we exploit the connectivity properties of polygonal tessellations to describe the physics involving pneumatics efficiently. More specifically, isolated regions of void polygonal elements will artificially mimic the geometrical loads in pneumatic actuation by volumetric expansion or contraction.

To identify void regions in the topology, we introduce the notion of a so-called `virus' element whose purpose is to infect neighboring elements with a low elemental density (voids). These virus elements are chosen a-priori and invariant, similar to the input of a pneumatic network. If the elements adjacent to an infected element have a density lower than a specified threshold $\gamma$, that is, $\rho_e < \gamma$, they become infected, and its logic will spread to its adjacent neighborhood. Afterward, each infected element will be influenced by the same pneumatic load. Clearly, as the topological layout of the discretized domain $\mathcal{B}_0$ changes during the optimization procedure, the influence of the pneumatic load will vary accordingly. To better reflect the physics involving pneumatics, facial connectivity must be qualified for infection, where two elements must share a common edge to be adjacent. 

A flood-fill algorithm is used \cite{chartrand1977} to find the affected region of a virus element efficiently. In case of irregular tessellations, like the Voronoi tessellation, the flood-fill algorithm requires an adjacency matrix $\boldsymbol{\Gamma}$, which contains information about the mesh connectivity. The adjacency matrix can be directly computed from the element-node-incidence matrix. Here, the incidence matrix $\boldsymbol{A} \in \N^{n\times m}$ is a sparse unit-matrix with $n$ columns for each element and $m$ rows for each node, where $\boldsymbol{A}_{ij} = 1$ iff node $i$ is incident upon element $j$. Given the incidence matrix, the adjacency matrix $\boldsymbol{\Gamma}\in \N^{n\times n}$ can be computed by
%
\begin{equation}
\boldsymbol{\Gamma} = \boldsymbol{A}\boldsymbol{A}^\top - \text{diag}(\boldsymbol{A}\boldsymbol{A}^\top), \label{eq:adjecency}
\end{equation}
%
where the matrix has non-zero entries $\boldsymbol{\Gamma}_{ij} = 1,2$ iff elements $i$ and $j$ share a common node or edge, respectively. As mentioned earlier, the adjacency will be modified such that edge connectivity is required for adjacency. Secondly, all the rows and columns corresponding to elements that satisfy $\rho_e \ge \gamma$ will be set to zero to ensure those elements are unaffected by the flood-fill. The pseudo-code for the identification of the pneumatic region is provided in Algorithm \ref{alg:floodfill}.

\begin{algorithm}[!t]
    \SetKwInOut{Input}{Input}
    \SetKwInOut{Output}{Output}
    \Input{Tessellation $\mathcal{T}$, set of virus elements $\mathcal{V}$, threshold $\gamma$, material density vector $\boldsymbol{\rho}$}
    \Output{Set of elemental indices $\mathcal{E}$}
    construct $\boldsymbol{\Gamma}  \gets {\textbf{BuildAdjacencyMatrix}}(\mathcal{T})$\;
    modify $\boldsymbol{\Gamma}  \gets {\textbf{EdgeConnectivity}}(\boldsymbol{\Gamma})$\;
    find thresholds $\mathcal{X} = \{\, i \in [1,n] \;|\; \rho_i \ge \gamma \,\}$\;
    set zeros $\boldsymbol{\Gamma}_i = (\boldsymbol{\Gamma}^\top)_i = 0$ for all $i \in  \mathcal{X}$\;
    initialize empty set $\mathcal{E} \gets \emptyset$\;
  \For{$i = 1: \text{numVirus} $}{{}
  $\mathcal{I} \gets {\textbf{doFloodFill}}(\boldsymbol{\Gamma},\mathcal{V}_i)$\;
  %$\text{bool} \gets \textbf{isEnclosed}(\mathcal{I})$\;
        $\mathcal{E} \gets \mathcal{E} \cup \mathcal{I}$\;
  }
    \caption{Find elemental indices subjected to volumetric differential pressure loads \label{alg:floodfill}}
\end{algorithm}

From a mathematical perspective, suppose that Algorithm 1 can be described by a mapping $\phi: \R^n \mapsto \{\N\}$ such that the mapping $\phi(\boldsymbol{\rho})$ returns a set of elemental indices of infected elements $\mathcal{E} \subseteq \{1,2,...,n\}$ that undergo volumetric compression or expansion depending on boundary conditions. Then, the global force matrix $\boldsymbol{f}$ in \eqref{eq:residual} can be assembled from elemental force vectors of the elements belonging to the affected region of the virus element $\mathcal{E}$, that is, 
%
\begin{equation}
%\boldsymbol{f}(\boldsymbol{x},\boldsymbol{\rho}) = \sum_{e \in \mathcal{E}} \tilde{p}_n \boldsymbol{W}_{\!e} \boldsymbol{K}_e \boldsymbol{n}_e, \label{eq:pressureforce}
\boldsymbol{f}(\boldsymbol{x},\boldsymbol{\rho}) = \sum_{e \in \mathcal{E}} \tilde{p}_e \boldsymbol{W}_{\!e} \, \boldsymbol{t}_{e}, \label{eq:pressureforce}
\end{equation}
%
where $\tilde{p}_e$ is a dimensionless parameter representing the magnitude and direction of artificial pneumatic loading, $\boldsymbol{W}_{\!e}$ a diagonal weighting matrix of densities at nodal level and $\boldsymbol{t}_e = \int_{\mathcal{V}_e} \boldsymbol{B}^\top \boldsymbol{D}_e \boldsymbol{\epsilon}_{v} \, dV$ the elemental force vector with $\boldsymbol{\epsilon}_{v} = [1,1,0]^\top$ the volumetric strain and $\boldsymbol{D}_e$ the fourth-order constitutive elasticity tensor in matrix notation \cite{Renaud2011}. We stress that the global force vector in \eqref{eq:pressureforce} represents an artificial pneumatic load, since this method assumes that the pressure loads can be emulated by anisotropic volumetric change of the polygonal elements. The magnitude and direction of this artificial pneumatic load can be controlled by varying the dimensionless parameter $\tilde{p}_e$.


\subsection{Optimization problem}
Given the notion of morphology in soft robotics, which is similar to compliant mechanisms, the general objective is to maximize the output displacement $\boldsymbol{u}_{out}$ on a virtual workpiece modeled by an artificial stiffness $k_{out} > 0$ \cite{Sigmund2015, Gain2013}. The objective function or the (desired) output displacement can be expressed by $\Phi = \boldsymbol{L}^\top\boldsymbol{x}(\boldsymbol{\rho})$, where $\boldsymbol{L} \in \R^{2n}$ a sparse unit-vector composed of nonzero entries for the degrees-of-freedom corresponding to the desired morphology of the soft robot. Given this description of directional output displacement, the topology optimization problem for soft robots can be formulated as
\begin{equation}
\begin{aligned}
\max_{\boldsymbol{\rho}} \quad &  \Phi = \boldsymbol{L}^\top \boldsymbol{x}(\boldsymbol{\rho}), \\
\textrm{s.t.} \quad & c:=\boldsymbol{R}(\boldsymbol{x},\boldsymbol{\rho}) \,= \,0,\\
& g:=\boldsymbol{v}^\top \boldsymbol{\rho}  \le V^*, \\
  &\boldsymbol{\rho} \in \mathcal{P},
  \label{eq:opt}
\end{aligned}
\end{equation}
where $\boldsymbol{R} \in \R^{2n}$ the global residual force vector in its equilibrium state, $\boldsymbol{v} \in \R^{m}$ a constant vector of relative elemental volumes, $V^* \in \Rp$ the maximum material infill, and $\mathcal{P} = \{\boldsymbol{\rho} \in \R^n \; | \; 0  \le \rho_e \le 1 \; \forall e \in [1,n] \}$ a set of feasible material densities that ensure numerical robustness.

\subsection{Solver and sensitivity analysis}
In this section, we discuss the use of a gradient-based optimization solver for the synthesis of the soft robot. Herein, we use the Method of Moving Asymptotes (MMA) proposed by Svanberg \cite{Svanberg1987}. The MMA is similar to other nonlinear programming approaches, like Optimality Criteria (OC) and Sequential Quadratic Programming (SQP), in that it finds an optimal solution to a nonlinear non-convex optimization problem with inequality constraints. The MMA solves a sequence of sub-problems, i.e., convex approximations of the true problem, which are constructed from gradient-based information of the objective function $\Phi$ and their constraints $c$ and $g$. The sensitivity of the inequality constraint in \eqref{eq:opt} is $\p g / \p \boldsymbol{\rho} = \boldsymbol{v}$ since we assume that the elemental volume $\boldsymbol{v}$ is constant. The sensitivity of the objective function is less trivial due to its dependency on the nodal displacements $\boldsymbol{x}(\boldsymbol{\rho})$. Since it is computationally expensive to obtain the displacements $\boldsymbol{x}$ through the Newton-Raphson method, it becomes beneficial to avoid the computation of their sensitivities. Thus, the sensitivities are computed through the adjoint method in which an augmented objective function is considered
\begin{equation}
{\Phi} = \boldsymbol{L}^\top \boldsymbol{x} - \boldsymbol{\lambda}^\top \boldsymbol{R},
\end{equation}

\noindent where $\boldsymbol{\lambda} \in \R^{2n}$ is a constant vector referred to as the adjoint vector. Note that, in case of equilibrium (i.e., $\boldsymbol{R} = 0$), the global residual forces are equivalent to zero; therefore, the adjoint vector can be chosen freely. For the sake of brevity, we denote differentiations by $\p(\cdot)/\p x := (\cdot),_{x}$. Differentiations of the objective function with respect to the elemental densities $\rho_e$ for each finite element $e \in \{1,2,...,n\}$ yields
\begin{align}
\frac{\Phi}{\p {\rho_e}} & = \,\boldsymbol{L}^\top \frac{\x}{\p \rho_e} - \boldsymbol{\lambda}^\top\left( \frac{\p \boldsymbol{R}}{ \p \x} \frac{\p \x}{\p \rho_e} + \frac{\p \boldsymbol{R}}{ \p \rho_e} \right), \notag \\
 & = \left(\boldsymbol{L}^\top - \boldsymbol{\lambda}^\top \boldsymbol{K}_T \right) \frac{\boldsymbol{x}}{\p \rho_e} - \boldsymbol{\lambda}^\top \frac{\p \boldsymbol{R}}{\p \rho_e},  \label{eq:sen_deriv} 
\end{align}

\noindent where the Jacobian of the residual force vector is substituted by the tangent stiffness matrix $\KB_T = \p \RB/ \p {\boldsymbol{x}}$ (see \cite{Kim2018}). By choosing the adjoint vector $\boldsymbol{\lambda}^\top = \boldsymbol{L}^\top(\boldsymbol{K}_T)^{-1}$, the terms involving $\boldsymbol{x},_{\rho_e}$ can be eliminated and thus computation of the gradient becomes feasible. Therefore, the gradient of the objective function can be compactly written as
\begin{align}
\frac{\p {\Phi} }{\p \rho_e} \overset{\eqref{eq:residual}}{=} -\boldsymbol{L}^\top(\boldsymbol{K}_T)^{-1} \left( \int_{\mathcal{B}_0} \boldsymbol{B}^\top \frac{\p \ST}{\p {\rho_e}} \; dV - \frac{ \p \fB}{\p {\rho_e}}  \right). \label{eq:sens_f}
\end{align} 

\noindent However, it shall be clear that the derivation of the loading sensitivity $\boldsymbol{f}\!,_{\rho_e}$ is not straightforward since the global force vector is constructed from nested functions of logic operations (e.g., flood-fill). Therefore, the loading sensitivity is derived numerically using the forward difference method. To reduce computation time, we propose to only compute the sensitivities of the elements at the boundary of the pressure region, since the gradient of the pneumatic region is largest near the boundary of the infected set $\mathcal{E}$. Conclusively, a brief overview of the computational design algorithm for synthesizing pressure-driven soft robots is provided in pseudo-code in Algorithm \ref{alg:topology_opt}. 

\begin{algorithm}[!t]
  \SetKwInOut{Input}{Input}
  \SetKwInOut{Output}{Output}
  \Input{Domain $\mathcal{B}_0$, material $\Psi$, initial $\boldsymbol{\rho}^{(0)}$, infill $V^*$, virus $\mathcal{V}$, output $\boldsymbol{L}$, pressure $\tilde{p}_e$ }
  \Output{Optimal soft robot topology $\boldsymbol{\rho}^*$}
  construct tessellation $\mathcal{T} \gets {\textbf{VoronoiMesher}}(\mathcal{B}_0)$\;
  \While{$\text{convergence} \neq 1 $}{
  update artificial material $\Psi_e$ using \eqref{eq:simp} \;
  find infected set $\mathcal{E}$ using \textbf{Algorithm 1} \;
  build residual forces $\boldsymbol{R}$ using \eqref{eq:residual} and \eqref{eq:pressureforce}\; 
  solve displacements $\boldsymbol{x} \gets \textbf{NewtonRaphson}(\boldsymbol{R})$ \;
  evaluate $g,_{\boldsymbol{\rho}} \gets \p g/\p{{\rho}_e}$ \;
  evaluate $\Phi,_{\boldsymbol{\rho}} \gets\p\Phi/\p{{\rho}_e}$ using \eqref{eq:sens_f} \; 
  update $\boldsymbol{\rho} \gets \textbf{MMA}(\Phi,_{\boldsymbol{\rho}},g,_{\boldsymbol{\rho}},\boldsymbol{\rho})$ \;
  }
  \caption{Computational design algorithm for pneumatic soft robots.\label{alg:topology_opt}}
\end{algorithm}
