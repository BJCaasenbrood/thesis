%!TEX root = thesis.tex
% Packages %%%%%%%%%%%%%%%%%%%%%%%%%%%%%%%%%%%%%%%%%%%%%%%%%%%%%%%%%%%%%%%%%%%%%%%%%%%%%%%%%%%%%%%%
\usepackage[english]{babel}
\usepackage{amsmath}
\usepackage{blindtext}
\usepackage{amssymb}
\usepackage{amsthm}                     % Theorem etc. Load before cleveref
\usepackage{psfrag}                     % \psfrag{tag}[position][psposition][scale][rotation]{text} to overlay
\usepackage{subcaption}                 % \begin{subfigure}, more options than subfig-package
\usepackage{wrapfig}
\usepackage[breaklinks=true,bookmarksnumbered=true,bookmarksdepth=2,bookmarksopen=true,bookmarksopenlevel=0,hidelinks,hypertexnames=false]{hyperref}
\urlstyle{same}                         % use same font type and size for url's
\usepackage{bookmark}
\usepackage{hyperref}
\usepackage{appendix}                   % subappendices environment
\usepackage[nottoc]{tocbibind}          % Bibliography in table of contents
\usepackage{graphicx}
%\usepackage[dvips]{epsfig}
\usepackage[dvipsnames]{xcolor}         % loads a set of 68 cmyk color names
%\usepackage{sectsty}                    % controls for section headers
%\usepackage[normalem]{ulem}             % underlining options
\usepackage{bm}                         % used to get bold greek symbols
\usepackage{algorithm}                  % enables algorithm environment
%\usepackage{algorithmic}               % algorithm formatting package
\usepackage{algpseudocode}              % algorithm formatting package
\usepackage{enumerate}                  % more options for enumerate environment
\usepackage{cite}                      % for better formatting of citations
\usepackage{mathtools}                  % more math operator and relational symbols (\coloneqq) and building blocks (multlined). requires amsmath
\usepackage[perpage]{footmisc}          % resets footnote numbering on each page
\usepackage{rotating}                   % enables rotated text
\usepackage[textwidth = 2cm,textsize = tiny]{todonotes}                  % enables the use of \todo{} \missingfigure{}, and \listoftodos

\usepackage{svg}

% Environment for theorems etc. %%%%%%%%%%%%%%%%%%%%%%%%%%%%%%%%%%%%%%%%%%%%%%%%%%%%%%%%%%%%%%%%%%%
\newtheorem{thm}{Theorem}[chapter]
% \newtheorem{assum}[thm]{Assumption}
% \newtheorem{corollary}[thm]{Corollary}
% \newtheorem{defn}[thm]{Definition}
% \newtheorem{exm}[thm]{Example}
% \newtheorem{lem}[thm]{Lemma}
% \newtheorem{rem}[thm]{Remark}
% \newtheorem{problem}[thm]{Problem}
% \newtheorem{fact}[thm]{Fact}
% \newtheorem{claim}[thm]{Claim}
% \newtheorem{mydescription}[thm]{Description}
% \newtheorem{property}[thm]{Property}
% \newtheorem{pf}[thm]{Proof}

% References %%%%%%%%%%%%%%%%%%%%%%%%%%%%%%%%%%%%%%%%%%%%%%%%%%%%%%%%%%%%%%%%%%%%%%%%%%%%%%%%%%%%%%
% Heading and captions
\newcommand{\chapref}[1]{Chapter~\ref{#1}}
\newcommand{\secref}[1]{Section~\ref{#1}}
\newcommand{\figref}[1]{Figure~\ref{#1}}
\newcommand{\twofigref}[2]{Figures \ref{#1} and \ref{#2}}
\newcommand{\appref}[1]{Appendix~\ref{#1}}
\newcommand{\tabref}[1]{Table~\ref{#1}}

% Text
\newcommand{\itref}[1]{\textit{\ref{#1}}}
\newcommand{\remref}[1]{\textit{Remark~\ref{#1}}}
\newcommand{\assref}[1]{\textit{Assumption~\ref{#1}}}
\newcommand{\defref}[1]{\textit{Definition~\ref{#1}}}
\newcommand{\lemref}[1]{\textit{Lemma~\ref{#1}}}
\newcommand{\thmref}[1]{\textit{Theorem~\ref{#1}}}
\newcommand{\exmref}[1]{\textit{Example~\ref{#1}}}
\newcommand{\algoref}[1]{\textit{Algorithm~\ref{#1}}}
\newcommand{\corref}[1]{\textit{Corollary~\ref{#1}}}

% Mark sections of captions for referring to divisions of figures
\newcommand{\figleft}{{\em (Left)}}
\newcommand{\figcenter}{{\em (Center)}}
\newcommand{\figright}{{\em (Right)}}
\newcommand{\figtop}{{\em (Top)}}
\newcommand{\figbottom}{{\em (Bottom)}}
\newcommand{\captiona}{{\em (a)}}
\newcommand{\captionb}{{\em (b)}}
\newcommand{\captionc}{{\em (c)}}
\newcommand{\captiond}{{\em (d)}}

% Math %%%%%%%%%%%%%%%%%%%%%%%%%%%%%%%%%%%%%%%%%%%%%%%%%%%%%%%%%%%%%%%%%%%%%%%%%%%%%%%%%%%%%%%%%%%%
% Number sets
\newcommand{\Rset}{\mathbb{R}}
\newcommand{\Nset}{\mathbb{N}}
\newcommand{\Zset}{\mathbb{Z}}
\newcommand{\Uset}{\mathbb{U}}

% Space sets
% \newcommand{\As}{\mathcal{A}}
% \newcommand{\Bs}{\mathcal{B}}
% \newcommand{\Cs}{\mathcal{C}}
% \newcommand{\Ds}{\mathcal{D}}
% \newcommand{\Es}{\mathcal{E}}
% \newcommand{\Fs}{\mathcal{F}}
% \newcommand{\Is}{\mathcal{I}}
% \newcommand{\Js}{\mathcal{J}}
% \newcommand{\Ks}{\mathcal{K}}
% \newcommand{\Ls}{\mathcal{L}}
% \newcommand{\Ns}{\mathcal{N}}
% \newcommand{\Ps}{\mathcal{P}}
% \newcommand{\Qs}{\mathcal{Q}}
% \newcommand{\Rs}{\mathcal{R}}
% \newcommand{\Ss}{\mathcal{S}}
% \newcommand{\Ts}{\mathcal{T}}
% \newcommand{\Us}{\mathcal{U}}
% \newcommand{\Vs}{\mathcal{V}}
% \newcommand{\Ws}{\mathcal{W}}
% \newcommand{\Xs}{\mathcal{X}}
% \newcommand{\Zs}{\mathcal{Z}}

% Math operators
%\newcommand{\diag}{\operatorname{diag}}
%\newcommand{\post}{\operatorname{post}}
%\newcommand{\Post}{\operatorname{Post}}

% Better matrices with vertical lines for MIP chapter
\makeatletter
\renewcommand*\env@matrix[1][*\c@MaxMatrixCols c]{%
  \hskip -\arraycolsep
  \let\@ifnextchar\new@ifnextchar
  \array{#1}}
\makeatother

% If toggles %%%%%%%%%%%%%%%%%%%%%%%%%%%%%%%%%%%%%%%%%%%%%%%%%%%%%%%%%%%%%%%%%%%%%%%%%%%%%%%%%%%%%%
% TU/e font family
\newif\iftuefont
%\tuefonttrue
\tuefontfalse

% Thesis vs paper version
\newif\ifthesis
\thesistrue
%\thesisfalse

% Chapter image page on left-hand side of chapter title page
\newif\ifchapterpage
\chapterpagetrue
%\chapterpagefalse


% Font and style %%%%%%%%%%%%%%%%%%%%%%%%%%%%%%%%%%%%%%%%%%%%%%%%%%%%%%%%%%%%%%%%%%%%%%%%%%%%%%%%%%
% Load TU/e font family
\RequirePackage[T1]{fontenc}
\newcommand{\tuefontmeta}{\fontfamily{zmb}\selectfont}
\newcommand{\tuefontscala}{\fontfamily{zsc}\selectfont}

% MAKE new part style
\iftuefont
\newcommand*\partownfont{\tuefontscala\mdseries\bfseries\huge}
\else
\newcommand*\partownfont{\sffamily\mdseries\bfseries\huge}
\fi
\partfont{\partownfont}

% MAKE new chapter style
\iftuefont
\newcommand*\chapfont{\tuefontscala\mdseries\bfseries\LARGE}
\else
\newcommand*\chapfont{\sffamily\mdseries\bfseries\LARGE}
\fi
\chapterfont{\raggedleft\chapfont\sectionrule{3ex}{2pt}{-2ex}{1pt}}
\chapternumberfont{\raggedleft\scshape\chapfont}

% MAKE new section style
\iftuefont
\newcommand*\sectfont{\tuefontscala\mdseries\bfseries\large}
\else
\newcommand*\sectfont{\sffamily\mdseries\bfseries\large}
\fi
\sectionfont{\sectfont}

% MAKE new subsection style
\iftuefont
\newcommand*\subsectfont{\tuefontscala\mdseries\bfseries}
\else
\newcommand*\subsectfont{\sffamily\mdseries\bfseries}
\fi
\subsectionfont{\subsectfont}

% MAKE new subsubsection style
\iftuefont
\newcommand*\subsubsectfont{\tuefontscala\mdseries\bfseries}
\else
\newcommand*\subsubsectfont{\sffamily\mdseries\bfseries}
\fi
\subsubsectionfont{\subsubsectfont}

% Chapter abstract style
\newenvironment{chapter-abstract}% environment name
{% begin code
  \par%\vspace{\baselineskip}
  \noindent
  \begin{itshape}%
  \small
  \noindent\ignorespaces
}%
{% end code
  \end{itshape}\ignorespacesafterend
}

% Change header font
\iftuefont
\fancyhead[LO]{\tuefontscala\itshape\nouppercase\rightmark}
\fancyhead[RE]{\tuefontscala\itshape\nouppercase\leftmark}
\fi

% other
\allowdisplaybreaks
\newcommand*{\QED}{\hfill\ensuremath{\square}}%
\newcommand\xput[2][0.5]{%
    \rule{#1\linewidth}{0pt}\makebox[0pt][c]{#2}\hfill}

%Set tolerance for paragraph stretching. Range 0 to 9999. High means sloppy, much stretching (gives underfull hboxes), low means rigid, no stretching (gives overfull hboxes)
\tolerance=3000
%\sloppy	% sets tolerance to 9999

% Enable referencing description label %%%%%%%%%%%%%%%%%%%%%%%%%%%%%%%%%%%%%%%%%%%%%%%%%%%%%%%%%%%%
\makeatletter
\let\orgdescriptionlabel\descriptionlabel
\renewcommand*{\descriptionlabel}[1]{%
  \let\orglabel\label
  \let\label\@gobble
  \phantomsection
  \edef\@currentlabel{#1}%
  %\edef\@currentlabelname{#1}%
  \let\label\orglabel
  \orgdescriptionlabel{#1}%
}
\makeatother

% Enable chapter thumb index %%%%%%%%%%%%%%%%%%%%%%%%%%%%%%%%%%%%%%%%%%%%%%%%%%%%%%%%%%%%%%%%%%%%%%
\usepackage[contents={},opacity=1,scale=1,color=black]{background}
\usepackage{tikzpagenodes}
\usepackage{totcount}
\usetikzlibrary{calc}

\newif\ifMaterial

\newlength\LabelSize
\setlength\LabelSize{1.5cm}
\newlength\LabelSpace
\setlength\LabelSpace{1.4cm}

\AtBeginDocument{%
\regtotcounter{chapter}
\setlength\LabelSize{\dimexpr\textheight/\totvalue{chapter}\relax}
\ifdim\LabelSize>1.5cm\relax
  \global\setlength\LabelSize{1.5cm}
\fi
}

\newcommand\AddLabels{%
\Materialtrue%
\AddEverypageHook{%
\ifMaterial%
\ifodd\value{page} %
 \backgroundsetup{
  angle=0,
  position={139.08mm,0},
  vshift=25.5mm-2mm-40mm-0.5*\LabelSize-(\thechapter-1)*\LabelSize-(\thechapter-1)*\LabelSpace,
  hshift=0,
  contents={
  \tikz\node[fill=black,minimum height=\LabelSize,minimum width=1.2cm,font=\LARGE\tuefontscala] {\color{white}\textbf{\thechapter}\,\,};
  }%
 }
 \else
 \backgroundsetup{
  angle=0,
  position={-21.75mm,0},
  vshift=25.5mm-2mm-40mm-0.5*\LabelSize-(\thechapter-1)*\LabelSize-(\thechapter-1)*\LabelSpace,
  hshift=0,
  contents={
  \tikz\node[fill=black,minimum height=\LabelSize,minimum width=1.2cm,font=\LARGE\tuefontscala] {\,\,\color{white}\textbf{\thechapter}};
  }%
 }
 \fi
 \BgMaterial%
\else\relax\fi}%
}

\newcommand\RemoveLabels{\Materialfalse}
