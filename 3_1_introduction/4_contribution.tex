%!TEX root = ../../thesis.tex
\vspace{-3mm}
\section{Research objective}
In short, soft robotics is a rapidly emerging subfield within the broader domain of robotics that focuses on the use of soft materials to create systems capable of dexterous bio-inspired morphological behavior akin to animals, capable of locomotion, grasping, and manipulation. Such features are made possible by the unique mechanical properties intrinsic to soft materials and soft fluidic actuation. Although open-loop control strategies have already established bio-like features, soft robotics still lags behind in terms of precision and speed compared to rigid robotics. To address this gap, new design principles for soft fluidic actuation and accurate, fast control-oriented models are believed to be crucial missing components. Furthermore, in the context of biological systems, it is not solely their morphological design that accounts for their success. Rather, it is the intricate interplay between their physical structure and coordinated motor control that enables their functionality. This leads us to the main objective of this thesis: \\[0.15em]

\objective{Development and analysis of new systematic tools balanced between the design, modeling, and control of soft continuum robots, which strive for similar capabilities as rigid robotics and eventually biological systems.}
% \objective{Development and analysis of new systematic tools balanced between the design, modeling, and control of soft continuum robots, whose goal is to streamline the development process of soft robots to be on par with rigid robotics}
%
% \noindent  \vspace{-3mm}
%The motivation for soft robotic technology to eventually offer better performance and efficiency than rigid robots, and eventually biological systems, highlights the need for generalizable solutions to these related issues of design and control, which are inherently interconnected.
%Yet, soft robots presents design, modeling and control challenges and control as inherent nonlinearities, \eg, hyper-elasticity, nonlinear geometric deformation, and fluid-structure dynamics; often require complex physics-based modeling strategies. The potential for soft robotic technology to achieve performance similar to that of biological systems highlights the need for generalizable solutions to issues related to design and control, which are inherently interdependent. 
%The aforementioned statement leads to a few open questions:
% Therefore, the main research question of this thesis is formulated as follows:
% \objective{How can we address the interdependent challenges of designing and controlling soft robots, taking into account the difficulties associated with soft fluidic actuators and developing accurate models for soft manipulators?}
% \vspace{1mm}
\vspace{-4mm}
% \begin{itemize}
%     \setlength\itemsep{-.1em}
%     \item {How do we design fluidic-interacting mechanical structures made from soft materials that deform according to a user-defined morphological pattern?}
%     \item {How do we derive dynamic models that offer a reasonable trade-off between the accuracy and its applicability for control?}
%     \item {Can we adopt classical control methods from rigid robotics to soft robotics?} 
%     \item {Can we effectively explore the intrinsic morphological properties of the soft robots, similar to biology, that are meaningful for modeling and control?}
%     \item{To enable better designs and controllers for soft robots, how can we bridge the interdisciplinary aspects intrinsic to this field?}
% \end{itemize}

\section{Research questions and contributions}
In the subsequent section, we shall discuss the challenges linked to the aforementioned objective and list the contributions made by the thesis. To methodically achieve our research objective, we have divided our research approach into three branches: (I) automated design synthesis of soft fluidic actuators, (II) modeling for control of soft robot manipulators, and (III) the development of software aimed to support the soft robotics community. For each research objective, a number of contributions are listed. \vspace{1mm}

\par \textbf{I: Automated design synthesis of soft actuators}. Our first research focus is on the design aspect of soft robotics. Upon reviewing the existing literature, it can be observed that soft robotic design is typically approached either through hand-driven principles or numerical optimization. Irrespective of the approach, achieving an optimal structural geometry that fully accommodates hyper-redundancies in soft materials remains a challenging task. Unlike their rigid counterparts and many biological systems for that matter, the kinematics are inherently encoded in the topological structure of the soft materials and where actuation is presented in the system. This implies the workspace cannot be characterized analytically in closed form through joint motions stemming from one point, as is commonly done for joints in rigid robotics. It is therefore of paramount importance that the nonlinear behaviors of soft materials, both in hyper-elasticity and nonlinear geometrical deformation, are understood and accounted for during the design process. This brings us to our first research question (R1):
%
\begin{center}
\textit{
How do we design fluidic-interacting mechanical structures made from soft materials that deform according to a user-defined morphological pattern?
}
\end{center}
%
% This deepens the research problem by imposing questions on optimality regarding material choice.
%As presented by abundance of literature on soft robotic design, either rooted in engineering principles or through optimization, achieving an optimal structural geometry that fully accounts for hyper-redundancies in soft materials is no easy feat. Unlike their rigid counterparts and many biological systems for that matter, the kinematics are inherently encoded in the topological structure of the soft materials and where actuation is presented in the system. This implies the workspace cannot be characterized analytically in closed from through joint motions stemming from one point (unlike joints in rigid robotics). Furthermore, any external disturbances can lead to parasitic motion, \ie, distributed continuum deformations, which are often antagonistic to the input and thus cannot be solved via control alone. These parasitic motions, or better phrased \textit{"passive joint displacements"}, arise from the redundant flexibility of the system that are often unaccounted for during design. As such, they should be kept at as minimum as they lead to imprecisions in the mechanical operation and lost of mechanical efficiency (\eg, the balloon effect \cite{Daerden1999}). Furthermore, there exist many elasticity moduli for various options of soft material, ranging from soft gels to hard rubber \cite{Rus2015}. It is therefore of paramount importance that the nonlinear behaviors of soft materials, both in hyper-elasticity and nonlinear geometrical deformation, is understood and accounted for during the design process. This deepens the research problem by imposing questions on optimality regarding material choice.

%\par \textit{(Optimality in soft material design)}. 
The fundamental principles of compliant mechanical devices can be rigorously described through continuum mechanics. In this thesis, we aim to integrate the underlying theory of continuum mechanics with the automated design of efficient soft actuators that enable user-defined objectives. To achieve this, we utilize optimization algorithms that seek to minimize user-specified objective functions by optimizing the layout of soft materials. This approach is rooted in continuum mechanics \cite{Holzapfel2002,Kim2018} and its subsequent discretization through FEM method. The aforementioned issue is commonly referred to as an "inverse design problem" - determining the shape from final deformation, rather than deducing the deformation from the shape. The field of combining continuum mechanics and free-form optimization in compliant structures is a well-established area known as "\emph{topology optimization}" \cite{Bendsoe2003}. However, such techniques are not easily applicable to soft actuation due to the presence of hyper-elasticity and nonlinear geometric deformations in soft materials. Furthermore, fluidic or pneumatic actuation further complicates the optimization process, as these loads become both design and state-dependent. This leads us to our first contribution:
% %
% \begin{figure}[!t]
%   %\centering
%   %\hspace{5mm}
%   %% This file was created by matlab2tikz.
%
%The latest updates can be retrieved from
%  http://www.mathworks.com/matlabcentral/fileexchange/22022-matlab2tikz-matlab2tikz
%where you can also make suggestions and rate matlab2tikz.
%
\definecolor{mycolor1}{rgb}{0.86275,0.89412,0.93725}%
\definecolor{mycolor2}{rgb}{0.29804,0.17255,0.57255}%
\definecolor{mycolor3}{rgb}{0.68235,0.69020,0.70980}%
%
\begin{tikzpicture}

\begin{axis}[%
width=0.801\textwidth,
height=0.331\textwidth,
at={(0.105\textwidth,0.004\textwidth)},
scale only axis,
xmin=0,
xmax=1,
ymin=0,
ymax=1,
axis line style={draw=none},
ticks=none,
axis x line*=bottom,
axis y line*=left
]
\end{axis}

\begin{axis}[%
width=0.279\textwidth,
height=0.362\textwidth,
at={(0\textwidth,0\textwidth)},
scale only axis,
axis on top,
xmin=-0.1,
xmax=1.25,
xtick={0,0.5,1},
xticklabels={\empty},
tick align=outside,
ymin=-0.5,
ymax=1.25,
ytick={-0.4,-0.2,0,0.2,0.4,0.6,0.8,1,1.2},
yticklabels={\empty},
axis line style={draw=none},
ticks=none,
axis x line*=bottom,
axis y line*=left
]
\addplot [forget plot] graphics [xmin=-0.05, xmax=1.25, ymin=-0.5, ymax=1.25] {fig_invdesign-1.png};
\addplot [color=black, dashed, forget plot]
  table[row sep=crcr]{%
0.21	1\\
0.295173289214242	0.991532195620041\\
0.360614496621075	0.977810163344932\\
0.418142259455368	0.959986548839953\\
0.471886979935072	0.937862650434878\\
0.522985542553323	0.911336814912063\\
0.570430743407341	0.881220607329087\\
0.613693356970083	0.848317940550163\\
0.652564947497686	0.813366441581409\\
0.68705250155865	0.777004193400618\\
0.717291975485328	0.739769047614185\\
0.749942239774055	0.691972210792866\\
0.775290627245592	0.647157469802955\\
0.795033357001678	0.604697941835834\\
0.813164743822043	0.556079270801141\\
0.825624491062359	0.512248711047301\\
0.835154837060245	0.465469076714938\\
0.840303166826548	0.424011642532202\\
0.842542893094428	0.381298845317327\\
0.841762985016358	0.343554374302487\\
0.838176919017599	0.301917475728051\\
0.832472337320446	0.266644995924904\\
0.824685388486502	0.232666488848871\\
0.815137145599415	0.200618027500202\\
0.804134684048246	0.170730999313808\\
0.791946040508724	0.143016843910748\\
0.778218130617618	0.116137976028283\\
0.764024483630162	0.0921450863786217\\
0.749353057410406	0.0703479386685352\\
0.733755104909042	0.0496607577279725\\
0.71764607559251	0.0305820098589388\\
0.701252046415901	0.0131963921383742\\
0.685053356193564	-0.00217386681408499\\
};
\addplot [color=black, only marks, mark=*, mark options={solid, black}, forget plot]
  table[row sep=crcr]{%
0.685053356193564	-0.00217386681408499\\
};
\addplot [color=black, line width=3.0pt, forget plot]
  table[row sep=crcr]{%
-0.15	0\\
0.35	0\\
};

\addplot[area legend, draw=none, fill=mycolor1, forget plot]
table[row sep=crcr] {%
x	y\\
-0.15	0\\
-0.15	-0.1\\
0.35	-0.1\\
0.35	0\\
}--cycle;
\node[right, align=left]
at (axis cs:0.02,-0.25) {\small (a)};
\end{axis}

\begin{axis}[%
width=0.279\textwidth,
height=0.362\textwidth,
at={(0.696\textwidth,0\textwidth)},
scale only axis,
axis on top,
xmin=-0.1,
xmax=1.25,
xtick={0,0.5,1},
xticklabels={\empty},
tick align=outside,
ymin=-0.5,
ymax=1.25,
ytick={-0.4,-0.2,0,0.2,0.4,0.6,0.8,1,1.2},
yticklabels={\empty},
axis line style={draw=none},
ticks=none,
axis x line*=bottom,
axis y line*=left
]
\addplot [forget plot] graphics [xmin=-0.05, xmax=1.25, ymin=-0.5, ymax=1.25] {fig_invdesign-2.png};
\addplot [color=black, dashed, forget plot]
  table[row sep=crcr]{%
0.21	1\\
0.244192530179641	0.998960351660961\\
0.278227568247254	0.996491487549144\\
0.310851957290166	0.99282103243068\\
0.342297828028995	0.988055212229345\\
0.372643052387765	0.98229247850009\\
0.40196055775063	0.975614055032036\\
0.430315723496673	0.968087756254422\\
0.459081859520303	0.959218630886533\\
0.486053514029505	0.949929237064135\\
0.512042577526441	0.939992825870722\\
0.53849371483274	0.928688908367215\\
0.563259563426235	0.917184827070021\\
0.585734545273986	0.90604438403161\\
0.60823734459696	0.89392061520427\\
0.628631657350333	0.882333464449473\\
0.649242253824007	0.869693547842639\\
0.669537040868068	0.85635751594729\\
0.68930502875653	0.842495209743257\\
0.72140777639443	0.815670783196885\\
0.741840751370248	0.79820276190575\\
0.761062883903797	0.78070263576285\\
0.779882580889839	0.762335013460291\\
0.797923435673018	0.743532359785675\\
0.815115864671519	0.724421760037603\\
0.831469008875325	0.705041805258556\\
0.847232567246937	0.685046331647295\\
0.86198467371677	0.665133009814635\\
0.876233462409813	0.644481598829865\\
0.88953767845295	0.623904793485784\\
0.902126903926303	0.603020694627006\\
0.913917801827952	0.582045196609194\\
0.924986002533254	0.560850758556501\\
0.93532084538888	0.539496368251922\\
0.944838292879368	0.518295443609127\\
};
\addplot [color=black, line width=3.0pt, forget plot]
  table[row sep=crcr]{%
-0.15	0\\
0.35	0\\
};

\addplot[area legend, draw=none, fill=mycolor1, forget plot]
table[row sep=crcr] {%
x	y\\
-0.15	0\\
-0.15	-0.1\\
0.35	-0.1\\
0.35	0\\
}--cycle;
\node[right, align=left]
at (axis cs:-0,-0.25) {\small (c)};
\node[right, align=left, font=\color{mycolor2}]
at (axis cs:0.85,0.5) {$\Large \boldsymbol{\star}$};
\end{axis}

\begin{axis}[%
width=0.279\textwidth,
height=0.362\textwidth,
at={(0.348\textwidth,0\textwidth)},
scale only axis,
axis on top,
xmin=-0.1,
xmax=1.25,
xtick={0,0.5,1},
xticklabels={\empty},
tick align=outside,
ymin=-0.5,
ymax=1.25,
ytick={-0.4,-0.2,0,0.2,0.4,0.6,0.8,1,1.2},
yticklabels={\empty},
axis line style={draw=none},
ticks=none,
axis x line*=bottom,
axis y line*=left
]
\addplot [forget plot] graphics [xmin=-0.05, xmax=1.25, ymin=-0.5, ymax=1.25] {fig_invdesign-3.png};
\addplot [color=black, dashed, forget plot]
  table[row sep=crcr]{%
0.21	1\\
0.295173289214242	0.991532195620041\\
0.360614496621075	0.977810163344932\\
0.418142259455368	0.959986548839953\\
0.471886979935072	0.937862650434878\\
0.522985542553323	0.911336814912063\\
0.570430743407341	0.881220607329087\\
0.613693356970083	0.848317940550163\\
0.652564947497686	0.813366441581409\\
0.68705250155865	0.777004193400618\\
0.717291975485328	0.739769047614185\\
0.749942239774055	0.691972210792866\\
0.775290627245592	0.647157469802955\\
0.795033357001678	0.604697941835834\\
0.813164743822043	0.556079270801141\\
0.825624491062359	0.512248711047301\\
0.835154837060245	0.465469076714938\\
0.840303166826548	0.424011642532202\\
0.842542893094428	0.381298845317327\\
0.841762985016358	0.343554374302487\\
0.838176919017599	0.301917475728051\\
0.832472337320446	0.266644995924904\\
0.824685388486502	0.232666488848871\\
0.815137145599415	0.200618027500202\\
0.804134684048246	0.170730999313808\\
0.791946040508724	0.143016843910748\\
0.778218130617618	0.116137976028283\\
0.764024483630162	0.0921450863786217\\
0.749353057410406	0.0703479386685352\\
0.733755104909042	0.0496607577279725\\
0.71764607559251	0.0305820098589388\\
0.701252046415901	0.0131963921383742\\
0.685053356193564	-0.00217386681408499\\
};
\addplot [color=black, only marks, mark=*, mark options={solid, black}, forget plot]
  table[row sep=crcr]{%
0.685053356193564	-0.00217386681408499\\
};
\addplot [color=mycolor3, only marks, mark=*, mark options={solid, mycolor3}, forget plot]
  table[row sep=crcr]{%
0.835154837060245	0.465469076714938\\
};
\addplot [color=black, line width=3.0pt, forget plot]
  table[row sep=crcr]{%
-0.15	0\\
0.35	0\\
};

\addplot[area legend, draw=none, fill=mycolor1, forget plot]
table[row sep=crcr] {%
x	y\\
-0.15	0\\
-0.15	-0.1\\
0.35	-0.1\\
0.35	0\\
}--cycle;
\node[right, align=left]
at (axis cs:-0,-0.25) {\small (b)};
\node[right, align=left, font=\color{mycolor2}]
at (axis cs:1.02,0.505) {$\Large \boldsymbol{\star}$};
\end{axis}
\end{tikzpicture}%
%   %\vspace{-7mm}
%   \caption{Schematic illustration of the inverse design problem in soft robotics. (a) Bending behavior of PneuNet actuator under linearly increasing pressure. (b) Desired end-effector position $(\textcolor{deepcolor}{\star})$ outside the robot's workspace, changing the input is not sufficient will not improve objective -- the workspace is \textit{a-priori} encoded into the soft topology. (c) Solution to inverse design problem by finding a soft topology that contains $(\textcolor{deepcolor}{\star})$ in its workspace.}
%   \vspace{-6mm}
%   \label{fig:C0:contribOne}
% \end{figure}
% %
\contribution{Development of efficient algorithms, applicable to the general design of fluidic soft actuators, solving the inverse design problem: Given a desired motion and input, what is accordingly the optimal soft material distribution within a design domain to realize such joint motion?}{}

%\textit{(Fabrication through Additive Manufacturing)}.
After applying such automated algorithms, a plethora of soft actuation systems with varying joint mobility can be developed with ease. However, since these designs originate from simulations, concerns arise regarding their transferability to practical applications. Therefore, we will investigate the transfer of simulation-based designs into functional and feasible soft actuators. Recent advancements in Additive Manufacturing have made it possible to fabricate complex 3D geometries with minimal difficulty and effort. Our second contribution therefore reads:

\contribution{Testing and validation of optimized soft fluidic actuators by exploring Additive Manufacturing methods of printable soft materials.}{}

\textbf{II: Modeling for control of soft manipulators}. The second research path focuses on reduced-order modeling (ROM), aiming to balance precision and speed for control. Parallel to design, modeling for control is a crucial aspect of achieving biological performance. Accurate and fast models are required to achieve this goal. However, the infinite-dimensionality inherent in soft continuum robots poses significant challenges. While the PCC approach and its augmented rigid-body variations have been proposed as a solution, such models do not respect fundamental continuum mechanics. As a result, they impose strict operational constraints on any soft system, such as limiting hyper-elastic nonlinearities or slowing down actuation to prevent dynamic mismatching. This leads to the second and third research questions (R2 and R3) on the topics of modeling and control: %Recent studies \cite{Katzschmann2019,Milana2021,Franco2020} have explored these models and demonstrated exceptional computation speeds.
%
\begin{center}
\textit{
How do we derive dynamic models that offer a reasonable trade-off \\ between the accuracy and its applicability for control? }
\end{center}
\begin{center}
\textit{And, can we adopt classical controllers from rigid to soft robotics?}
\end{center}
%
%\textit{(Accurate control-oriented models through PCC condition)}. 
By building upon the original works of Chirikjian et al. \cite{Chirikjian1992} and Mochiyama et al. \cite{Mochiyama1992} in the 90s, the thesis presents a dynamic modeling formulation for soft manipulators that better preserves their continuum nature. First, to address the issue of infinite-dimensionality, a reduced-order modeling strategy for soft robot manipulators is proposed, whose mathematical framework is based on the differential geometric theory of spatial curves. Such a framework allows for easy transferability to conventional ROM models in soft robotics, like the PCC strain \cite{DellaSantina2020,Katzschmann2019,Falkenhahn2015}. However, the thesis proposes two improvements that are essential for the development of model-based controllers. Inspired by the success of FEM-based models in soft robotics \cite{Duriez2013,Goury2018}, we bridge the gap between the PCC model and the underlying continuum mechanics by matching the quasi-static behavior to FEM data. Second, to enhance computational efficiency, new reduced-order integration schemes are required that compute the spatio-temporal dynamics in real-time, thus enabling controller design. This brings us to our third contribution:
%The proposed reduced-order soft beam model is tested rigorously in simulation; however, for practical control tests demand experimental rigorousness. As such, a 3-DOF soft robot manipulator is developed \textit{ad-hoc} through Additive Manufacturing. The proposed dynamic model is tested under various experimental conditions. For example, natural oscillations, forced inputs, and under tip-disturbances of various inertial mass. Various performance measures are introduced to compare the proposed model objectively with respect to linear elastic alternatives. 

\contribution{Development of computationally fast and accurate dynamic models for soft manipulators composed of hyper-elastic soft materials that are directly applicable to classical control theory akin to rigid robotics.}{}

%\textit{(Models beyond the PCC)}. 
As outlined in Section \ref{sec:C0:modelbasedcontrol}, the PCC method's piecewise continuity imposes limitations on kinematic redundancies, thereby impeding the exploration of the hyper-redundant nature of soft robots, a fundamental characteristic often lacking in rigid robotics. For instance, constant strains alone cannot adequately capture significant nonlinear deflections caused by gravity, nor can they accurately describe environmental interactions. Hence, the research question (R4) that arises is:
%
\begin{center}
\textit{Can we effectively explore the intrinsic morphological flexibilities of soft robots, similar to biology, that are meaningful for modeling and control?}
\end{center}
%
Therefore, the third research objective focuses on relaxing the PCC condition to pursue its true infinite-dimensionality more closely. This is achieved by adopting prior modeling strategies of differential curves, where spatially-varying strain fields are considered and approximated through sets of orthogonal shape functions rather than piecewise representations. Building upon prior models presented in the thesis, the underactuated and hyper-redundant soft system is written in a port-Hamiltonian framework \cite{Schaft2004, Ortega2002}, which enables energy-shaping techniques by modifying the closed-loop potential energy of the system -- a well-known practice in classical robotics \cite{Schaft2004, Ortega1998, Ortega2002}. By exploring the hyper-redundancy in soft robotics, more advanced control objectives can be achieved, allowing for multi-modal shape regulation and full-body grasping. However, spatial discretization in these model-based controllers plays a crucial role in their dexterity to achieve various control tasks. Within this context, the fourth contribution reads:

\contribution{A systematic study of spatial discretization in low-order energy-shaping controllers applied to high-order soft robotic models with a focus on closed-loop performance in shape regulation control.}{} 
%\vspace{-5mm}
%\textit{(Exploiting structural geometry for reduction)}. 
Contributions III and IV provide a stable modeling platform for a variety of possible shape functions tailored to unique joint mobilities in soft robotic systems. Yet, many works in modeling literature choose such functions in \textit{ad-hoc} fashion, \eg, polynomial bases \cite{DellaSantina2020,Boyer2021,Chirikjian1991}. Within this context, the thesis explores a (geometric) modal decomposition approach. Similar to the eigenmode analysis in structural dynamics, geometric strain modes are extracted from higher-order (volumetric) FEM simulation data and used to construct optimal soft beam models. The approach leads to fast, accurate, and generic low-dimensional models that encode the geometric features and elasticity of the true soft body into a new strain parametrization, we call the \textit{"Data-driven Variable Strain"} (DVS) basis. A merit benefit of the approach can be naturally expanded to identify the hyper-elastic material parameters and the actuation map of the reduced beam model, which is often miss in PCC models. Our fifth contribution therefore reads: %In context of robustness, the nonlinearities with increasing actuation frequency, under environmental contact modeled by signed distance functions, and multi-input pneumatic actuation are also considered. The thesis continues with a qualitative comparison between existing strategies, demonstrating that our approach can improve accuracy and speed compared traditional techniques. Our fifth contribution therefore reads:
 
\contribution{A novel method for finite-dimensional model reduction in soft manipulators that explores the mechanical interconnection between structural geometry and flexibility modes of the soft manipulator body.}{}

% Abstract—The infinite-dimensional nature of soft robots has
% emphasized the difficulty in modeling and control, leading to
% the classic trade-off between precision and speed. In the last
% decade, two modeling strategies have dominated the field: (i)
% Finite-Element-Method (FEM) models and (ii) soft beam models
% (e.g., affine curvature or Cosserat models). While FEM enables
% highly accurate deformations, Lagrangian-based beam models
% allow for faster computation and ease of controllers design akin
% to rigid robotics. In this letter, we propose a mixture between
% the two modeling approaches by extracting geometric modal
% information of FEM simulation data. Our approach leads to
% fast, accurate, and generic low-dimensional models that encode
% the geometric features and the elasticity of the original soft
% body into a new strain functional basis – we call a Geometry-
% Informed Variable Strain basis. Robustness of the technique is
% investigated for several systems. Also the nonlinearities with
% increasing actuation frequency, under environmental contact
% modeled by signed distance functions, and multi-input pneumatic
% actuation. Furthermore, we also provide a qualitative comparison
% between existing strategies, demonstrating that our approach can
% improve accuracy and speed compared traditional techniques.
% Also, experiments are performed to highlight the model’s trans-
% transferability to reality.

\textbf{C: Software development}. Our final research branch is centered around interdisciplinary and the transferal of knowledge towards the soft robotics community. Soft robotics is intrinsically multidisciplinary field since it involves the integration of knowledge and techniques from various scientific disciplines, such as materials science, mechanical and electrical engineering, computer science, and biology. Our final research question (R5) therefore reads:

\begin{center}
\textit{To enable better designs and controllers within the soft robotics community, how can we bridge the interdisciplinary aspects that are intrinsic to the field?}
\end{center}

\thumbimageodd{\qrcode[height=1.0cm]{https://bit.ly/3zIvbHt}}{\texttt{Sorotoki}}{0.1mm} 

The thesis integrates all the previously mentioned contributions to form a comprehensive software package for soft robotics, which is referred to as \texttt{Sorotoki}. The name is an acronym for "SOft RObotics TOolKIt". The primary objective of this software is to bridge the gap between various disciplines of soft robotics, including design, modeling, and control. It provides a minimal programming framework that enables users to solve complex problems using minimal lines of code. The toolkit is closely integrated with Contributions I to V of this thesis and is publicly available at \url{https://github.com/BJCaasenbrood/SorotokiCode} \cite{SorotokiCode}. Additionally, the software enables real-time hardware control of fluidics, facilitating the evaluation of model-based controllers on physical systems. We also present open-hardware soft robots that can be manufactured using commercial printing methods. Thus, the final contribution of this thesis involves the creation of this software package.

\contribution{Developement of a versatile, user-friendly, open-source software called \textnormal{\texttt{Sorotoki}} that envelops the presented theory on design, modeling and control of the thesis into one coherent MATLAB\hspace{0.75mm}\textsuperscript{\scriptsize\textregistered} toolkit.}{}
%\vfill