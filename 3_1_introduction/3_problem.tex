\vspace{-3mm}
\section{State-of-the-art solutions in soft robotics}
Driven by the aspiration for bio-like performance in robotics, there have been significant advances in the development of continuum soft robots since the early 2000s. Besides octopi \cite{Renda2018,Chang2022,Mazzolai2019Oct}, various researchers draw inspiration from a plethora of biological organisms such as fish \cite{Katzschmann2018,Marchese2015}, elephants' trunks \cite{Jones2006,Wehner2016,Godage2015}, snakes \cite{Rafsanjani2018Feb,Gazzola2018,Marchese2015}, birds \cite{Gazzola2018,Zufferey2022Dec}, and even the human hand \cite{vanLaake2022Sep,Fras2018Oct}. Researchers have discovered several ways to mimic biology by harmonizing soft materials and robotics. This section offers a brief overview of the state-of-the-art solutions concerning design and control.

\begin{rmk}
\textit{Considering the extensive scope of soft robotics, an additional chapter (Chapter \ref{chap: history}) has been included to provide a more comprehensive overview of its origins dating back to the early 1980s, as well as the various research aspects currently associated with this field. Thus, the following section serves as a preliminary introduction, ensuring brevity in the following section while setting the stage for the research problems relevant to the scope of the thesis.}
\end{rmk}

\subsection{Soft fluidic actuators inspired by muscular hydrostats}
Conventional soft robots are characterized by their continuum-bodied motion that arises from a network of compliant actuators embedded throughout its soft body. There exist many options for such embodied actuation, including tendons \cite{Rucker2011Jul,Renda2017Aug}, chemical reactions \cite{Bartlett2015,Hubbard2021}, light-driven liquid crystals \cite{Vantomme2021,Pilz2020,daCunha2020}, and ferromagnetic materials \cite{Kim2019AugMagnet,Venkiteswaran2019Feb}, and hydrogels \cite{Jiao2022Jun,Lee2020Dec}. However, the most common approach is fluidic actuation \cite{Katzschmann2018,Marchese2015,Overvelde2015Sep,vanLaake2022Sep,Falkenhahn2015,DellaSantina2020a} that mimics muscular hydrostats found in animals. The latter are commonly referred to as \emph{"Soft Fluidic Actuators"} (SFAs), and the majority are designed via human-driven techniques. In this section, we aim to provide a brief explanation of SFA technology and explore the potential benefits of incorporating structural optimization into the design process alongside the common human-driven design techniques.

Soft fluidic actuators are inflatable fluidic channels that are embedded into an elastic soft body. When pressurized fluid is applied, the elastic pressure vessel uniformly distributes the internal stresses along the interior, resulting in motion due to strain relaxation of the surrounding continuum body. Since elasticity is key, silicone rubbers are commonly employed for their notable material properties. Specifically, silicone rubbers exhibit a low Young's modulus, only display material fatigue at high strains exceeding $100\%$, and offer many commercial alternatives. By exploring purposefully designed asymmetrical geometries, often created by hand, predictable motion can be guided \cite{Xavier2022Jun,Rus2015,Hughes2016Nov}. The aforementioned principles are analogous to those of (semi-rigid) compliant mechanisms, which flexible structures that facilitate motion or force transmission through elastic deformation of their components, as opposed to conventional rigid-body joints. Through the exploration of structural geometries or deliberate incorporation of stiffer materials, a diverse range of motion primitives can be achieved which may include bending, twisting, and elongation.

\subsection[Human-driven design versus design optimization]{Systematic design of soft fluidic actuation: human-driven versus design optimization}
A popular example of a soft fluidic actuator (SFA) is the PneuNet actuator, which was proposed by Mosadegh et al. \cite{Mosadegh2014}. This actuator has a consistent linear input-output bending behavior and its design is simple, easy to fabricate, and repeatable, making it a standard in the academic field. To illustrate the capabilities, in Figure \ref{fig:C0:pneunetGrasp}, we present an example of it grasping an aluminum cylinder, inspired by the octopus example mentioned earlier \cite{Sumbre2001Sep}. Despite using rudimentary open-loop control, the system can emulate complex behaviors such as reaching, grasping, and pulling objects closer through manipulation, similar to those exhibited by the octopus. As shown in Figure \ref{fig:C0:pneunetGrasp}, an important property is that the inherent mechanical impedance allows for safe and stable interactions with objects of varying rigidity, even without environmental perception. This contrasts rigid robots interacting with unknown environments, where impedance control appears to be a necessary requirement \cite{Murray1994,DeLuca2016Jul}.

\begin{figure}
\centering
\vspace{-2mm}
\includegraphics*[width=0.975\textwidth]{./pdf/thesis-figurex-1-3-2.pdf}
\caption{\small An illustrative example of a fluidic PneuNet actuator \cite{Mosadegh2014} reaching and grasping a 20 \si{\milli \meter} aluminum cylinder, which is inspired by the morphological grasping motion of the octopus' soft arm \cite{Sumbre2001Sep} as seen in Figure \ref{fig:C0:octopus}. The main body is composed of Dragonskin\texttrademark\, 10A silicone, and 30A for the bottom layer. It is worth noting that despite using open-loop motion control (\ie, linear pressure ramp of 40 \si{\kilo \pascal}), the enveloping properties of soft materials can lead to the emergence of complex "intelligent" behavior. \label{fig:C0:pneunetGrasp}}
\vspace{-5mm}
\end{figure}
    
Given its success in open-loop, common applications for SFA (Soft Fluidic Actuator) technology are therefore soft grippers \cite{Galloway2016, Hughes2016Nov, Ansari2022Sep, Teleshev1981, Sinatra2019Aug} that are useful in handling delicate objects. Pushing the technology further, system dexterity has been extended by incorporating many different SFAs into one single body, similar to muscle groups in animals. Consequently, researchers have explored their potential for enabling highly dexterous robots capable of in-hand manipulation \cite{Suzumori1991, Graule2022}, as well as autonomous terrestrial and aquatic locomotion \cite{Choi2011, Katzschmann2018, Drotman2017, Suzumori1992}. Although SFAs provide a wide range of motion, they require significant amounts of supplied volume inflow, which adversely affects their speed, compactness, and efficiency \cite{Overvelde2015Sep, Rus2015, Xavier2022Jun}. A common nonlinear phenomenon called "\emph{ballooning}" occurs when the elastic membrane of the changes deforms significantly, further increasing the channel's volumetric capacitance. This behavior is particularly evident in soft materials that exhibit strain softening nonlinearities, which intensifies this inefficient expansion. In order to address the aforementioned limitations, various researchers have proposed two potential solutions. The first involves the use of hand-designed geometrical features, such as ribs, which increase structural compliance. The second solution involves the utilization of composites made up of different materials. One example of this approach is demonstrated in the work of Polygerinos et al. \cite{Polygerinos2015, Polygerinos2013}, as well as other similar studies \cite{Fras2018Oct, Suzumori1991, Cianchetti2013Nov}. These studies employ fiber-reinforced SFAs that incorporate fiber weaves to create new soft material composites. The introduction of such weaves results in anisotropic mechanical properties due to the fibers having a low bending-twist modulus but high elongation modulus. Additionally, various weave patterns can be used to steer the deformation characteristics towards a desired kinematic profile \cite{Kim2019Aug, Connolly2017Jan}, or as proprioceptive sensors using magnetic inductance \cite{Felt2019Feb, Felt2015Oct} or strain sensors made from EGaIn (Eutectic Gallium Indium) \cite{Park2012,Tapia2020}. Recently, multi-material 3D-printing approaches have also been employed, combining actuation with integrated sensing \cite{Wolterink2022Oct, Zhou2021Apr}. It is noteworthy that the advent of Additive Manufacturing (AM) technology has significantly contributed to the advancement of soft robotics technology, enabling flexibility in design and materials, surpassing what was previously achievable with subtractive techniques.

%Based on a review by Marchese et al. \cite{Marchese2014}, three common geometries for SFAs are explored: $(i)$ cylindrical, $(ii)$ ribbed, and $(iii)$ pleated. Although cylindrical geometries are the easiest to fabricate,
% Soft actuators from large circumferential strains that increase volume significantly without producing significant motion. This undesirable phenomenon is known as "\emph{ballooning}". Moreover, some soft materials exhibit strain softening nonlinearities that further intensify this inefficient volumetric expansion. Ribbed and pleated SFA designs, however, prevent such large volumetric expansion but require more complicated fabrication methods, such as lost-wax rubber casting \cite{Marchese2015}. 
% Alternatively, Polygerinos et al. \cite{Polygerinos2015} and other related works use of fiber-reinforced SFAs, which embedded single fibers or fiber-sheet matrices. These composite materials have anisotropic mechanical properties that arise from the fibers exhibiting low moduli for bending and twisting, but high for extensibility. Interestingly, the interlacing patterns of the fiber mesh can lead to exclusive homogeneous motion such as pure elongation, twisting and bending \cite{Polygerinos2015,Kamble2022Jan}. %Fibers in soft robotics have also been utilized in sensing applications, \eg, Galloway et al. \cite{Galloway2019} presented a fiber optic shape sensor, while Tapia et al. \cite{Tapia2020} explored the automated routing of a network of stretchable fiber sensors. 
% Connolly et al. \cite{Connolly2017Jan} demonstrated that the placement of inextensible fiber alignment can be optimized to ensure that the soft robot follows a prescribed (quasi-static) kinematic trajectory. 

Like in nature, there are often no unique design solutions when finding the optimal geometrical shape, material composites, and actuation type, or a combination of the aforementioned. While such design freedom has its benefits, it also poses challenges in establishing effective hand-driven design solutions that are tailored towards functionality. Furthermore, predicting structural deformation \textit{a-priori} is challenging and often requires time-consuming numerical simulations. In response, optimization-based solutions are slowly being considered. Cheney et al. \cite{Cheney2013} employed a compositional pattern-producing network (CPPN) to determine the most efficient arrangement of soft material voxels and activation patterns, thereby facilitating the synthesis of soft robots capable of terrestrial and aquatic locomotion in simulation. Following that, Kriegman et al. \cite{Kriegman2020} explored auto-generative processes for a variety of candidate creatures in silico, with the aim of achieving specific motor functions like locomotion, object manipulation, and transport.
% These designs were then implemented as living systems composed of muscular frog cells, resulting in the predicted behaviors being exhibited. 

Although voxel-based \cite{Kriegman2020,Cheney2013} and commonly parametric shape optimization \cite{Coevoet2017,Manns2018Jan,Morzadec2019Apr} approaches have demonstrated success in soft robot design, these methods are inherently imposed by design restrictions that may perhaps limit the design possibilities in soft robotics, especially given the recent advances in Additive Manufacturing (AM). In contrast, topology optimization \cite{Bendsoe2003,Gain2013Dec,Zhang2017Topo,Talischi2012Mar,Vasista2013Jul} (TopOpt) is a versatile technique that enables the design of structures with desired functionality without imposing significant design constraints. Furthermore, TopOpt approaches explore gradient-based optimizers derived from continuum theory that allow accurate descriptions of nonlinear deformations, which are simplified in many schemes for computation speed. Wang et al. \cite{Wang2020Nov} and Tian et al. \cite{Tian2020May} explored TopOpt for soft grippers using tendons and ferromagnets, respectively. Yuhn et al. \cite{Yuhn2023Feb} extended density-based optimization to include time as an additional variable, allowing for simultaneous optimization of structure and movement with gradient-based methods through 4-dimensional TopOpt. However, research on TopOpt applied to fluidic soft actuators is limited, perhaps due to the challenges associated with modeling fluid-structure interaction in adaptive structures.

\vspace{-2mm}
\subsection{Model-based control for soft robots}
\label{sec:C0:modelbasedcontrol}
In a previous discussion, we showed that the open-loop approach can accomplish various complex tasks such as locomotion \cite{Suzumori1992,Choi2011,Katzschmann2018}, grasping \cite{Suzumori1992}, and manipulation \cite{Marchese2016}. While open-loop control has been successfully demonstrated, it ultimately relies on \textit{a priori} system knowledge that determines the control logic based on numerical surrogate models or experimental regression. This, independent of using open-loop strategies, further establishes the importance of modeling, which is nowadays a crucial ingredient for modern robotics \cite{Spong1996,Murray1994,Khatib1987}. %The use of models for control can naturally exploit system information to enhance performance and robustness when executing tasks, and this holds true for soft robotics as well. 
In the early days of soft robotics, model-based control strategies were deemed unfeasible due to the infinite-dimensional nature of these systems \cite{DellaSantina2020}. Over the last two decades, however, a plethora of modeling solutions have surfaced, paving the way for sophisticated model-based controllers, which are expected to push the dexterity of soft robotics systems even beyond what is already achievable in open-loop.
% The necessity for high-precision models allows us to understand and predict such behavior withing small margins of deviation, thus facilitating controllers that achieve the desired behavior without feedback compensation (under conservative conditions). Closed-loop controllers can explore useful model structures to compensate for large deviations, uses feedback to adjust the outputs and maintain a desired state. In recent years, various modeling techniques have gained popularity in soft robotics. 

A common modeling approach is the Finite Element Method (FEM), which involves spatially discretizing continuum solids into a group of "\textit{finite elements}," permitting the underlying Partial Differential Equation to be rewritten as an approximate Ordinary Differential Equation (ODE) \cite{Holzapfel2002,Kim2018}. These can be solved straightforwardly using standard numerical integration (\eg, \texttt{ode45}, \texttt{ode23t} in MATLAB\textsuperscript{\scriptsize\textregistered}).
%By treating the soft body as a continuum mechanical solid, its dynamics can be described by the momentum balance as a Partial Differential Equation (PDE), which are often solved using the Ritz-Galerkin projection method \cite{Kim2018,Holzapfel2002}. The method yields a $n$-dimensional surrogate problem represented as an Ordinary Differential Equation (ODE), which are reminiscent those in rigid robotics, and can be solved straightforwardly using standard numerical integration methods. 
%
While primarily explored for quasi-static behavior, FEM models have demonstrated their efficacy in handling hyperelasticity, geometric nonlinearities due to large deformation, fluid-structure interactions, and other multi-physical domains \cite{Xavier2022Jun,Hughes2016Nov,Smith2022_FEM,Moerman2018,Maas2012}. Motivated by the assumption that quasi-static models provide sufficiently accuracy to describe soft robots under slow-varying dynamics, open-loop forward kinematic controllers are used that inversely search the input space for the desired deformation profile \cite{Marchese2015,Bern2019,Marchese2016}. Alternatively, high-fidelity FEM data can be fed into neural networks approximators \cite{Fang2022Jun,Zheng2020May} or Quadratic Programming (QP) algorithms \cite{Bern2019} that tackle the inverse kinematic problem directly. Yet, the high-dimensionality of these models, with sometimes millions of DoFs, poses a significant challenge when considering dynamic behavior \cite{Goury2018,Duriez2013}, thereby limiting their applicability for closed-loop feedback control. Undoubtedly, the significant interest of soft robotics in both academic and industrial sectors arises from its potential applications in dynamic conditions. For example, one of the key advantage is its ability to withstand high-velocity impacts, making it an attractive option for grippers use over rigid options. Another advantage involves converting kinetic energy into potential energy through reversible elastic deformation, which is challenging to achieve with fully rigid robotic systems.

A viable solution for reducing dimensionality of FEM is found in Proper Orthogonal Decomposition (POD). Here, time-series data is collected, and through singular value decomposition, the principal dynamic modes are identified and combined to construct a reduced linear dynamical model. Not only does this approach improve speed, but it also preserves accurate, robust, and efficient models suited for closed-loop controller design \cite{Goury2018}. Recent extensions \cite{Sifakis2012Aug} preserve even nonlinear deformations and self-contact. The proposed modeling approach is encapsulated in an open-source software called \texttt{SOFA} \cite{Duriez2016, Coevoet2017Feb}, which facilitated a wide range of closed-loop controller designs contributed by an active community in past years \cite{Largilliere2015,Goury2018,Duriez2016,Wu2021Feb,Li2022Feb}. Furthermore, it has been successfully implemented in physical systems, and recently Reinforcement Learning (RL) methods have been explored \cite{Schegg2022}. Alternatively, Koopman system identification tools form a data-driven approach that can be applied directly to experimental data (\eg, measurements from optical markers placed on the exterior of the soft robot), leading to discrete-time dynamical systems. By gathering measured data alone, accurate models of the true system can be identified \cite{Bruder2019,Komeno2022Oct}. These models are often followed by Model-Predictive Control (MPC) approaches \cite{Bruder2020Dec,Bruder2019}. 

% \begin{figure}[!t]
%     \centering
%     \hspace{-2mm}
%     \includegraphics*[width=\textwidth]{./pdf/thesis-figure-1-model.pdf}
%     %% This file was created by matlab2tikz.
%
%The latest updates can be retrieved from
%  http://www.mathworks.com/matlabcentral/fileexchange/22022-matlab2tikz-matlab2tikz
%where you can also make suggestions and rate matlab2tikz.
%
\definecolor{mycolor1}{rgb}{0.62745,0.45882,1.00000}%
\definecolor{mycolor2}{rgb}{0.62869,0.46064,1.00000}%
\definecolor{mycolor3}{rgb}{0.62993,0.46245,1.00000}%
\definecolor{mycolor4}{rgb}{0.63118,0.46426,1.00000}%
\definecolor{mycolor5}{rgb}{0.63242,0.46608,1.00000}%
\definecolor{mycolor6}{rgb}{0.63366,0.46789,1.00000}%
\definecolor{mycolor7}{rgb}{0.63490,0.46970,1.00000}%
\definecolor{mycolor8}{rgb}{0.63614,0.47151,1.00000}%
\definecolor{mycolor9}{rgb}{0.63738,0.47333,1.00000}%
\definecolor{mycolor10}{rgb}{0.63862,0.47514,1.00000}%
\definecolor{mycolor11}{rgb}{0.63987,0.47695,1.00000}%
\definecolor{mycolor12}{rgb}{0.64111,0.47877,1.00000}%
\definecolor{mycolor13}{rgb}{0.64235,0.48058,1.00000}%
\definecolor{mycolor14}{rgb}{0.64359,0.48239,1.00000}%
\definecolor{mycolor15}{rgb}{0.64483,0.48421,1.00000}%
\definecolor{mycolor16}{rgb}{0.64607,0.48602,1.00000}%
\definecolor{mycolor17}{rgb}{0.64732,0.48783,1.00000}%
\definecolor{mycolor18}{rgb}{0.64856,0.48964,1.00000}%
\definecolor{mycolor19}{rgb}{0.64980,0.49146,1.00000}%
\definecolor{mycolor20}{rgb}{0.65104,0.49327,1.00000}%
\definecolor{mycolor21}{rgb}{0.65228,0.49508,1.00000}%
\definecolor{mycolor22}{rgb}{0.65352,0.49690,1.00000}%
\definecolor{mycolor23}{rgb}{0.65476,0.49871,1.00000}%
\definecolor{mycolor24}{rgb}{0.65601,0.50052,1.00000}%
\definecolor{mycolor25}{rgb}{0.65725,0.50234,1.00000}%
\definecolor{mycolor26}{rgb}{0.65849,0.50415,1.00000}%
\definecolor{mycolor27}{rgb}{0.65973,0.50596,1.00000}%
\definecolor{mycolor28}{rgb}{0.66097,0.50777,1.00000}%
\definecolor{mycolor29}{rgb}{0.66221,0.50959,1.00000}%
\definecolor{mycolor30}{rgb}{0.66345,0.51140,1.00000}%
\definecolor{mycolor31}{rgb}{0.66470,0.51321,1.00000}%
\definecolor{mycolor32}{rgb}{0.66594,0.51503,1.00000}%
\definecolor{mycolor33}{rgb}{0.66718,0.51684,1.00000}%
\definecolor{mycolor34}{rgb}{0.66842,0.51865,1.00000}%
\definecolor{mycolor35}{rgb}{0.66966,0.52047,1.00000}%
\definecolor{mycolor36}{rgb}{0.67090,0.52228,1.00000}%
\definecolor{mycolor37}{rgb}{0.67215,0.52409,1.00000}%
\definecolor{mycolor38}{rgb}{0.67339,0.52590,1.00000}%
\definecolor{mycolor39}{rgb}{0.67463,0.52772,1.00000}%
\definecolor{mycolor40}{rgb}{0.67587,0.52953,1.00000}%
\definecolor{mycolor41}{rgb}{0.67711,0.53134,1.00000}%
\definecolor{mycolor42}{rgb}{0.67835,0.53316,1.00000}%
\definecolor{mycolor43}{rgb}{0.67959,0.53497,1.00000}%
\definecolor{mycolor44}{rgb}{0.68084,0.53678,1.00000}%
\definecolor{mycolor45}{rgb}{0.68208,0.53859,1.00000}%
\definecolor{mycolor46}{rgb}{0.68332,0.54041,1.00000}%
\definecolor{mycolor47}{rgb}{0.68456,0.54222,1.00000}%
\definecolor{mycolor48}{rgb}{0.68580,0.54403,1.00000}%
\definecolor{mycolor49}{rgb}{0.68704,0.54585,1.00000}%
\definecolor{mycolor50}{rgb}{0.68828,0.54766,1.00000}%
\definecolor{mycolor51}{rgb}{0.68953,0.54947,1.00000}%
\definecolor{mycolor52}{rgb}{0.69077,0.55129,1.00000}%
\definecolor{mycolor53}{rgb}{0.69201,0.55310,1.00000}%
\definecolor{mycolor54}{rgb}{0.69325,0.55491,1.00000}%
\definecolor{mycolor55}{rgb}{0.69449,0.55672,1.00000}%
\definecolor{mycolor56}{rgb}{0.69573,0.55854,1.00000}%
\definecolor{mycolor57}{rgb}{0.69698,0.56035,1.00000}%
\definecolor{mycolor58}{rgb}{0.69822,0.56216,1.00000}%
\definecolor{mycolor59}{rgb}{0.69946,0.56398,1.00000}%
\definecolor{mycolor60}{rgb}{0.70070,0.56579,1.00000}%
\definecolor{mycolor61}{rgb}{0.70194,0.56760,1.00000}%
\definecolor{mycolor62}{rgb}{0.70318,0.56942,1.00000}%
\definecolor{mycolor63}{rgb}{0.70442,0.57123,1.00000}%
\definecolor{mycolor64}{rgb}{0.70567,0.57304,1.00000}%
\definecolor{mycolor65}{rgb}{0.70691,0.57485,1.00000}%
\definecolor{mycolor66}{rgb}{0.70815,0.57667,1.00000}%
\definecolor{mycolor67}{rgb}{0.70939,0.57848,1.00000}%
\definecolor{mycolor68}{rgb}{0.71063,0.58029,1.00000}%
\definecolor{mycolor69}{rgb}{0.71187,0.58211,1.00000}%
\definecolor{mycolor70}{rgb}{0.71311,0.58392,1.00000}%
\definecolor{mycolor71}{rgb}{0.71436,0.58573,1.00000}%
\definecolor{mycolor72}{rgb}{0.71560,0.58755,1.00000}%
\definecolor{mycolor73}{rgb}{0.71684,0.58936,1.00000}%
\definecolor{mycolor74}{rgb}{0.71808,0.59117,1.00000}%
\definecolor{mycolor75}{rgb}{0.71932,0.59298,1.00000}%
\definecolor{mycolor76}{rgb}{0.72056,0.59480,1.00000}%
\definecolor{mycolor77}{rgb}{0.72181,0.59661,1.00000}%
\definecolor{mycolor78}{rgb}{0.72305,0.59842,1.00000}%
\definecolor{mycolor79}{rgb}{0.72429,0.60024,1.00000}%
\definecolor{mycolor80}{rgb}{0.72553,0.60205,1.00000}%
\definecolor{mycolor81}{rgb}{0.72677,0.60386,1.00000}%
\definecolor{mycolor82}{rgb}{0.72801,0.60568,1.00000}%
\definecolor{mycolor83}{rgb}{0.72925,0.60749,1.00000}%
\definecolor{mycolor84}{rgb}{0.73050,0.60930,1.00000}%
\definecolor{mycolor85}{rgb}{0.73174,0.61111,1.00000}%
\definecolor{mycolor86}{rgb}{0.73298,0.61293,1.00000}%
\definecolor{mycolor87}{rgb}{0.73422,0.61474,1.00000}%
\definecolor{mycolor88}{rgb}{0.73546,0.61655,1.00000}%
\definecolor{mycolor89}{rgb}{0.73670,0.61837,1.00000}%
\definecolor{mycolor90}{rgb}{0.73794,0.62018,1.00000}%
\definecolor{mycolor91}{rgb}{0.73919,0.62199,1.00000}%
\definecolor{mycolor92}{rgb}{0.74043,0.62381,1.00000}%
\definecolor{mycolor93}{rgb}{0.74167,0.62562,1.00000}%
\definecolor{mycolor94}{rgb}{0.74291,0.62743,1.00000}%
\definecolor{mycolor95}{rgb}{0.74415,0.62924,1.00000}%
\definecolor{mycolor96}{rgb}{0.74539,0.63106,1.00000}%
\definecolor{mycolor97}{rgb}{0.74664,0.63287,1.00000}%
\definecolor{mycolor98}{rgb}{0.74788,0.63468,1.00000}%
\definecolor{mycolor99}{rgb}{0.74912,0.63650,1.00000}%
\definecolor{mycolor100}{rgb}{0.75036,0.63831,1.00000}%
\definecolor{mycolor101}{rgb}{0.75160,0.64012,1.00000}%
\definecolor{mycolor102}{rgb}{0.75408,0.64375,1.00000}%
\definecolor{mycolor103}{rgb}{0.75533,0.64556,1.00000}%
\definecolor{mycolor104}{rgb}{0.75657,0.64737,1.00000}%
\definecolor{mycolor105}{rgb}{0.75781,0.64919,1.00000}%
\definecolor{mycolor106}{rgb}{0.75905,0.65100,1.00000}%
\definecolor{mycolor107}{rgb}{0.76029,0.65281,1.00000}%
\definecolor{mycolor108}{rgb}{0.76153,0.65463,1.00000}%
\definecolor{mycolor109}{rgb}{0.76277,0.65644,1.00000}%
\definecolor{mycolor110}{rgb}{0.76402,0.65825,1.00000}%
\definecolor{mycolor111}{rgb}{0.76526,0.66007,1.00000}%
\definecolor{mycolor112}{rgb}{0.76650,0.66188,1.00000}%
\definecolor{mycolor113}{rgb}{0.76774,0.66369,1.00000}%
\definecolor{mycolor114}{rgb}{0.76898,0.66550,1.00000}%
\definecolor{mycolor115}{rgb}{0.77022,0.66732,1.00000}%
\definecolor{mycolor116}{rgb}{0.77147,0.66913,1.00000}%
\definecolor{mycolor117}{rgb}{0.77271,0.67094,1.00000}%
\definecolor{mycolor118}{rgb}{0.77395,0.67276,1.00000}%
\definecolor{mycolor119}{rgb}{0.77519,0.67457,1.00000}%
\definecolor{mycolor120}{rgb}{0.77643,0.67638,1.00000}%
\definecolor{mycolor121}{rgb}{0.77767,0.67819,1.00000}%
\definecolor{mycolor122}{rgb}{0.77891,0.68001,1.00000}%
\definecolor{mycolor123}{rgb}{0.78016,0.68182,1.00000}%
\definecolor{mycolor124}{rgb}{0.78140,0.68363,1.00000}%
\definecolor{mycolor125}{rgb}{0.78264,0.68545,1.00000}%
\definecolor{mycolor126}{rgb}{0.78388,0.68726,1.00000}%
\definecolor{mycolor127}{rgb}{0.78512,0.68907,1.00000}%
\definecolor{mycolor128}{rgb}{0.78636,0.69089,1.00000}%
\definecolor{mycolor129}{rgb}{0.78760,0.69270,1.00000}%
\definecolor{mycolor130}{rgb}{0.78885,0.69451,1.00000}%
\definecolor{mycolor131}{rgb}{0.79009,0.69632,1.00000}%
\definecolor{mycolor132}{rgb}{0.79133,0.69814,1.00000}%
\definecolor{mycolor133}{rgb}{0.79257,0.69995,1.00000}%
\definecolor{mycolor134}{rgb}{0.79381,0.70176,1.00000}%
\definecolor{mycolor135}{rgb}{0.79505,0.70358,1.00000}%
\definecolor{mycolor136}{rgb}{0.79630,0.70539,1.00000}%
\definecolor{mycolor137}{rgb}{0.79754,0.70720,1.00000}%
\definecolor{mycolor138}{rgb}{0.79878,0.70902,1.00000}%
\definecolor{mycolor139}{rgb}{0.80002,0.71083,1.00000}%
\definecolor{mycolor140}{rgb}{0.80126,0.71264,1.00000}%
\definecolor{mycolor141}{rgb}{0.80250,0.71445,1.00000}%
\definecolor{mycolor142}{rgb}{0.80374,0.71627,1.00000}%
\definecolor{mycolor143}{rgb}{0.80499,0.71808,1.00000}%
\definecolor{mycolor144}{rgb}{0.80623,0.71989,1.00000}%
\definecolor{mycolor145}{rgb}{0.80747,0.72171,1.00000}%
\definecolor{mycolor146}{rgb}{0.80871,0.72352,1.00000}%
\definecolor{mycolor147}{rgb}{0.80995,0.72533,1.00000}%
\definecolor{mycolor148}{rgb}{0.81119,0.72715,1.00000}%
\definecolor{mycolor149}{rgb}{0.81243,0.72896,1.00000}%
\definecolor{mycolor150}{rgb}{0.81368,0.73077,1.00000}%
\definecolor{mycolor151}{rgb}{0.81492,0.73258,1.00000}%
\definecolor{mycolor152}{rgb}{0.81616,0.73440,1.00000}%
\definecolor{mycolor153}{rgb}{0.81740,0.73621,1.00000}%
\definecolor{mycolor154}{rgb}{0.81864,0.73802,1.00000}%
\definecolor{mycolor155}{rgb}{0.81988,0.73984,1.00000}%
\definecolor{mycolor156}{rgb}{0.82113,0.74165,1.00000}%
\definecolor{mycolor157}{rgb}{0.82237,0.74346,1.00000}%
\definecolor{mycolor158}{rgb}{0.82361,0.74528,1.00000}%
\definecolor{mycolor159}{rgb}{0.82485,0.74709,1.00000}%
\definecolor{mycolor160}{rgb}{0.82609,0.74890,1.00000}%
\definecolor{mycolor161}{rgb}{0.82733,0.75071,1.00000}%
\definecolor{mycolor162}{rgb}{0.82857,0.75253,1.00000}%
\definecolor{mycolor163}{rgb}{0.82982,0.75434,1.00000}%
\definecolor{mycolor164}{rgb}{0.83106,0.75615,1.00000}%
\definecolor{mycolor165}{rgb}{0.83230,0.75797,1.00000}%
\definecolor{mycolor166}{rgb}{0.83354,0.75978,1.00000}%
\definecolor{mycolor167}{rgb}{0.83478,0.76159,1.00000}%
\definecolor{mycolor168}{rgb}{0.83602,0.76341,1.00000}%
\definecolor{mycolor169}{rgb}{0.83726,0.76522,1.00000}%
\definecolor{mycolor170}{rgb}{0.83851,0.76703,1.00000}%
\definecolor{mycolor171}{rgb}{0.83975,0.76884,1.00000}%
\definecolor{mycolor172}{rgb}{0.84099,0.77066,1.00000}%
\definecolor{mycolor173}{rgb}{0.84223,0.77247,1.00000}%
\definecolor{mycolor174}{rgb}{0.84347,0.77428,1.00000}%
\definecolor{mycolor175}{rgb}{0.84471,0.77610,1.00000}%
\definecolor{mycolor176}{rgb}{0.84596,0.77791,1.00000}%
\definecolor{mycolor177}{rgb}{0.84720,0.77972,1.00000}%
\definecolor{mycolor178}{rgb}{0.84844,0.78154,1.00000}%
\definecolor{mycolor179}{rgb}{0.84968,0.78335,1.00000}%
\definecolor{mycolor180}{rgb}{0.85092,0.78516,1.00000}%
\definecolor{mycolor181}{rgb}{0.85216,0.78697,1.00000}%
\definecolor{mycolor182}{rgb}{0.85340,0.78879,1.00000}%
\definecolor{mycolor183}{rgb}{0.85465,0.79060,1.00000}%
\definecolor{mycolor184}{rgb}{0.85589,0.79241,1.00000}%
\definecolor{mycolor185}{rgb}{0.85713,0.79423,1.00000}%
\definecolor{mycolor186}{rgb}{0.85837,0.79604,1.00000}%
\definecolor{mycolor187}{rgb}{0.85961,0.79785,1.00000}%
\definecolor{mycolor188}{rgb}{0.86085,0.79966,1.00000}%
\definecolor{mycolor189}{rgb}{0.86209,0.80148,1.00000}%
\definecolor{mycolor190}{rgb}{0.86334,0.80329,1.00000}%
\definecolor{mycolor191}{rgb}{0.86458,0.80510,1.00000}%
\definecolor{mycolor192}{rgb}{0.86582,0.80692,1.00000}%
\definecolor{mycolor193}{rgb}{0.86706,0.80873,1.00000}%
\definecolor{mycolor194}{rgb}{0.86830,0.81054,1.00000}%
\definecolor{mycolor195}{rgb}{0.86954,0.81236,1.00000}%
\definecolor{mycolor196}{rgb}{0.87079,0.81417,1.00000}%
\definecolor{mycolor197}{rgb}{0.87203,0.81598,1.00000}%
\definecolor{mycolor198}{rgb}{0.87327,0.81779,1.00000}%
\definecolor{mycolor199}{rgb}{0.87451,0.81961,1.00000}%
\definecolor{mycolor200}{rgb}{0.81905,0.74006,0.99664}%
\definecolor{mycolor201}{rgb}{0.76359,0.66050,0.99328}%
\definecolor{mycolor202}{rgb}{0.70812,0.58095,0.98992}%
\definecolor{mycolor203}{rgb}{0.65266,0.50140,0.98655}%
\definecolor{mycolor204}{rgb}{0.59720,0.42185,0.98319}%
\definecolor{mycolor205}{rgb}{0.54174,0.34230,0.97983}%
\definecolor{mycolor206}{rgb}{0.48627,0.26275,0.97647}%
\definecolor{mycolor207}{rgb}{0.86275,0.89412,0.93725}%
%
\begin{tikzpicture}

\begin{axis}[%
width=0.774\textwidth,
height=0.242\textwidth,
at={(0.118\textwidth,0.003\textwidth)},
scale only axis,
xmin=0,
xmax=1,
ymin=0,
ymax=1,
axis line style={draw=none},
ticks=none,
axis x line*=bottom,
axis y line*=left
]
\end{axis}

\begin{axis}[%
width=0.302\textwidth,
height=0.264\textwidth,
at={(0\textwidth,0\textwidth)},
scale only axis,
xmin=-1,
xmax=1,
xtick={-1,-0.5,0,0.5,1},
xticklabels={\empty},
ymin=-0.5,
ymax=1.25,
ytick={-0.5,0,0.5,1},
yticklabels={\empty},
axis line style={draw=none},
ticks=none,
axis x line*=bottom,
axis y line*=left
]
\addplot [color=mycolor2, line width=1.5pt, forget plot]
  table[row sep=crcr]{%
-0.584958712597518	0.0243477391605708\\
-0.578395869368097	0.0121034471751481\\
};
\addplot [color=mycolor3, line width=1.5pt, forget plot]
  table[row sep=crcr]{%
-0.591262388502971	0.0367274510941034\\
-0.584958712597518	0.0243477391605708\\
};
\addplot [color=mycolor4, line width=1.5pt, forget plot]
  table[row sep=crcr]{%
-0.597304104226123	0.0492370981154573\\
-0.591262388502971	0.0367274510941034\\
};
\addplot [color=mycolor5, line width=1.5pt, forget plot]
  table[row sep=crcr]{%
-0.603081182970708	0.0618711377962995\\
-0.597304104226123	0.0492370981154573\\
};
\addplot [color=mycolor6, line width=1.5pt, forget plot]
  table[row sep=crcr]{%
-0.608591065188497	0.0746239725958389\\
-0.603081182970708	0.0618711377962995\\
};
\addplot [color=mycolor7, line width=1.5pt, forget plot]
  table[row sep=crcr]{%
-0.613831309713301	0.087489952340829\\
-0.608591065188497	0.0746239725958389\\
};
\addplot [color=mycolor8, line width=1.5pt, forget plot]
  table[row sep=crcr]{%
-0.618799594842545	0.100463376728892\\
-0.613831309713301	0.087489952340829\\
};
\addplot [color=mycolor9, line width=1.5pt, forget plot]
  table[row sep=crcr]{%
-0.623493719365896	0.113538497854049\\
-0.618799594842545	0.100463376728892\\
};
\addplot [color=mycolor10, line width=1.5pt, forget plot]
  table[row sep=crcr]{%
-0.62791160354052	0.126709522753344\\
-0.623493719365896	0.113538497854049\\
};
\addplot [color=mycolor11, line width=1.5pt, forget plot]
  table[row sep=crcr]{%
-0.632051290012514	0.13997061597343\\
-0.62791160354052	0.126709522753344\\
};
\addplot [color=mycolor12, line width=1.5pt, forget plot]
  table[row sep=crcr]{%
-0.635910944684117	0.153315902155982\\
-0.632051290012514	0.13997061597343\\
};
\addplot [color=mycolor13, line width=1.5pt, forget plot]
  table[row sep=crcr]{%
-0.639488857526311	0.166739468640785\\
-0.635910944684117	0.153315902155982\\
};
\addplot [color=mycolor14, line width=1.5pt, forget plot]
  table[row sep=crcr]{%
-0.642783443336455	0.180235368085351\\
-0.639488857526311	0.166739468640785\\
};
\addplot [color=mycolor15, line width=1.5pt, forget plot]
  table[row sep=crcr]{%
-0.64579324244061	0.193797621099908\\
-0.642783443336455	0.180235368085351\\
};
\addplot [color=mycolor16, line width=1.5pt, forget plot]
  table[row sep=crcr]{%
-0.648516921340251	0.20742021889658\\
-0.64579324244061	0.193797621099908\\
};
\addplot [color=mycolor17, line width=1.5pt, forget plot]
  table[row sep=crcr]{%
-0.650953273303082	0.221097125951593\\
-0.648516921340251	0.20742021889658\\
};
\addplot [color=mycolor18, line width=1.5pt, forget plot]
  table[row sep=crcr]{%
-0.653101218897673	0.234822282679329\\
-0.650953273303082	0.221097125951593\\
};
\addplot [color=mycolor19, line width=1.5pt, forget plot]
  table[row sep=crcr]{%
-0.654959806471712	0.248589608117041\\
-0.653101218897673	0.234822282679329\\
};

\addplot[area legend, draw=none, fill=mycolor19, forget plot]
table[row sep=crcr] {%
x	y\\
-0.62190600661236	0.269970699862386\\
-0.653101218897673	0.234822282679329\\
-0.696804133207599	0.252102554112939\\
-0.621831963836141	0.10375056113766\\
}--cycle;
\addplot [color=mycolor20, line width=1.5pt, forget plot]
  table[row sep=crcr]{%
-0.656528212573634	0.262393002619035\\
-0.654959806471712	0.248589608117041\\
};
\addplot [color=mycolor21, line width=1.5pt, forget plot]
  table[row sep=crcr]{%
-0.65780574231745	0.276226350559137\\
-0.656528212573634	0.262393002619035\\
};
\addplot [color=mycolor22, line width=1.5pt, forget plot]
  table[row sep=crcr]{%
-0.658791829690624	0.290083523040228\\
-0.65780574231745	0.276226350559137\\
};
\addplot [color=mycolor23, line width=1.5pt, forget plot]
  table[row sep=crcr]{%
-0.659486037804842	0.303958380609673\\
-0.658791829690624	0.290083523040228\\
};
\addplot [color=mycolor24, line width=1.5pt, forget plot]
  table[row sep=crcr]{%
-0.659888059089577	0.317844775979418\\
-0.659486037804842	0.303958380609673\\
};
\addplot [color=mycolor25, line width=1.5pt, forget plot]
  table[row sep=crcr]{%
-0.659997715428361	0.331736556749557\\
-0.659888059089577	0.317844775979418\\
};
\addplot [color=mycolor26, line width=1.5pt, forget plot]
  table[row sep=crcr]{%
-0.659814958237695	0.345627568134172\\
-0.659997715428361	0.331736556749557\\
};
\addplot [color=mycolor27, line width=1.5pt, forget plot]
  table[row sep=crcr]{%
-0.659339868488581	0.359511655688222\\
-0.659814958237695	0.345627568134172\\
};
\addplot [color=mycolor28, line width=1.5pt, forget plot]
  table[row sep=crcr]{%
-0.658572656670642	0.373382668034286\\
-0.659339868488581	0.359511655688222\\
};
\addplot [color=mycolor29, line width=1.5pt, forget plot]
  table[row sep=crcr]{%
-0.657513662698864	0.387234459587944\\
-0.658572656670642	0.373382668034286\\
};
\addplot [color=mycolor30, line width=1.5pt, forget plot]
  table[row sep=crcr]{%
-0.656163355763002	0.401060893280592\\
-0.657513662698864	0.387234459587944\\
};
\addplot [color=mycolor31, line width=1.5pt, forget plot]
  table[row sep=crcr]{%
-0.654522334119697	0.414855843278488\\
-0.656163355763002	0.401060893280592\\
};
\addplot [color=mycolor32, line width=1.5pt, forget plot]
  table[row sep=crcr]{%
-0.652591324827422	0.428613197696812\\
-0.654522334119697	0.414855843278488\\
};
\addplot [color=mycolor33, line width=1.5pt, forget plot]
  table[row sep=crcr]{%
-0.650371183424353	0.442326861307556\\
-0.652591324827422	0.428613197696812\\
};
\addplot [color=mycolor34, line width=1.5pt, forget plot]
  table[row sep=crcr]{%
-0.647862893549325	0.455990758240023\\
-0.650371183424353	0.442326861307556\\
};
\addplot [color=mycolor35, line width=1.5pt, forget plot]
  table[row sep=crcr]{%
-0.645067566506027	0.469598834672761\\
-0.647862893549325	0.455990758240023\\
};
\addplot [color=mycolor36, line width=1.5pt, forget plot]
  table[row sep=crcr]{%
-0.641986440770633	0.483145061515721\\
-0.645067566506027	0.469598834672761\\
};
\addplot [color=mycolor37, line width=1.5pt, forget plot]
  table[row sep=crcr]{%
-0.638620881443099	0.496623437081459\\
-0.641986440770633	0.483145061515721\\
};
\addplot [color=mycolor38, line width=1.5pt, forget plot]
  table[row sep=crcr]{%
-0.634972379642345	0.510027989744198\\
-0.638620881443099	0.496623437081459\\
};

\addplot[area legend, draw=none, fill=mycolor38, forget plot]
table[row sep=crcr] {%
x	y\\
-0.596202069818307	0.516852053453825\\
-0.638620881443099	0.496623437081459\\
-0.672146825637701	0.529556149782669\\
-0.660853050018576	0.363720114397519\\
}--cycle;
\addplot [color=mycolor39, line width=1.5pt, forget plot]
  table[row sep=crcr]{%
-0.631042551845617	0.52335278058557\\
-0.634972379642345	0.510027989744198\\
};
\addplot [color=mycolor40, line width=1.5pt, forget plot]
  table[row sep=crcr]{%
-0.626833139172299	0.536591906025868\\
-0.631042551845617	0.52335278058557\\
};
\addplot [color=mycolor41, line width=1.5pt, forget plot]
  table[row sep=crcr]{%
-0.622346006612508	0.549739500439642\\
-0.626833139172299	0.536591906025868\\
};
\addplot [color=mycolor42, line width=1.5pt, forget plot]
  table[row sep=crcr]{%
-0.617583142200807	0.562789738754478\\
-0.622346006612508	0.549739500439642\\
};
\addplot [color=mycolor43, line width=1.5pt, forget plot]
  table[row sep=crcr]{%
-0.612546656135397	0.575736839031805\\
-0.617583142200807	0.562789738754478\\
};
\addplot [color=mycolor44, line width=1.5pt, forget plot]
  table[row sep=crcr]{%
-0.607238779843195	0.588575065028606\\
-0.612546656135397	0.575736839031805\\
};
\addplot [color=mycolor45, line width=1.5pt, forget plot]
  table[row sep=crcr]{%
-0.60166186499119	0.601298728738863\\
-0.607238779843195	0.588575065028606\\
};
\addplot [color=mycolor46, line width=1.5pt, forget plot]
  table[row sep=crcr]{%
-0.595818382444533	0.613902192913651\\
-0.60166186499119	0.601298728738863\\
};
\addplot [color=mycolor47, line width=1.5pt, forget plot]
  table[row sep=crcr]{%
-0.589710921171817	0.626379873558727\\
-0.595818382444533	0.613902192913651\\
};
\addplot [color=mycolor48, line width=1.5pt, forget plot]
  table[row sep=crcr]{%
-0.583342187098023	0.638726242408537\\
-0.589710921171817	0.626379873558727\\
};
\addplot [color=mycolor49, line width=1.5pt, forget plot]
  table[row sep=crcr]{%
-0.57671500190566	0.650935829375522\\
-0.583342187098023	0.638726242408537\\
};
\addplot [color=mycolor50, line width=1.5pt, forget plot]
  table[row sep=crcr]{%
-0.569832301784607	0.663003224973657\\
-0.57671500190566	0.650935829375522\\
};
\addplot [color=mycolor51, line width=1.5pt, forget plot]
  table[row sep=crcr]{%
-0.562697136131229	0.674923082715136\\
-0.569832301784607	0.663003224973657\\
};
\addplot [color=mycolor52, line width=1.5pt, forget plot]
  table[row sep=crcr]{%
-0.555312666197336	0.686690121479144\\
-0.562697136131229	0.674923082715136\\
};
\addplot [color=mycolor53, line width=1.5pt, forget plot]
  table[row sep=crcr]{%
-0.547682163689579	0.698299127851671\\
-0.555312666197336	0.686690121479144\\
};
\addplot [color=mycolor54, line width=1.5pt, forget plot]
  table[row sep=crcr]{%
-0.539809009319919	0.709744958435326\\
-0.547682163689579	0.698299127851671\\
};
\addplot [color=mycolor55, line width=1.5pt, forget plot]
  table[row sep=crcr]{%
-0.531696691307789	0.721022542128136\\
-0.539809009319919	0.709744958435326\\
};
\addplot [color=mycolor56, line width=1.5pt, forget plot]
  table[row sep=crcr]{%
-0.523348803834634	0.732126882370301\\
-0.531696691307789	0.721022542128136\\
};
\addplot [color=mycolor57, line width=1.5pt, forget plot]
  table[row sep=crcr]{%
-0.514769045451497	0.743053059357941\\
-0.523348803834634	0.732126882370301\\
};

\addplot[area legend, draw=none, fill=mycolor57, forget plot]
table[row sep=crcr] {%
x	y\\
-0.476401234776035	0.734243072221917\\
-0.523348803834634	0.732126882370301\\
-0.541406472299162	0.775514358784875\\
-0.59557355531981	0.618367716704828\\
}--cycle;
\addplot [color=mycolor58, line width=1.5pt, forget plot]
  table[row sep=crcr]{%
-0.505961217440371	0.75379623222282\\
-0.514769045451497	0.743053059357941\\
};
\addplot [color=mycolor59, line width=1.5pt, forget plot]
  table[row sep=crcr]{%
-0.496929222130029	0.764351641177105\\
-0.505961217440371	0.75379623222282\\
};
\addplot [color=mycolor60, line width=1.5pt, forget plot]
  table[row sep=crcr]{%
-0.487677061167091	0.774714609622204\\
-0.496929222130029	0.764351641177105\\
};
\addplot [color=mycolor61, line width=1.5pt, forget plot]
  table[row sep=crcr]{%
-0.478208833743082	0.784880546220743\\
-0.487677061167091	0.774714609622204\\
};
\addplot [color=mycolor62, line width=1.5pt, forget plot]
  table[row sep=crcr]{%
-0.46852873477828	0.794844946930764\\
-0.478208833743082	0.784880546220743\\
};
\addplot [color=mycolor63, line width=1.5pt, forget plot]
  table[row sep=crcr]{%
-0.45864105306314	0.804603397001258\\
-0.46852873477828	0.794844946930764\\
};
\addplot [color=mycolor64, line width=1.5pt, forget plot]
  table[row sep=crcr]{%
-0.448550169358143	0.814151572928129\\
-0.45864105306314	0.804603397001258\\
};
\addplot [color=mycolor65, line width=1.5pt, forget plot]
  table[row sep=crcr]{%
-0.438260554452885	0.823485244369728\\
-0.448550169358143	0.814151572928129\\
};
\addplot [color=mycolor66, line width=1.5pt, forget plot]
  table[row sep=crcr]{%
-0.427776767185283	0.832600276021122\\
-0.438260554452885	0.823485244369728\\
};
\addplot [color=mycolor67, line width=1.5pt, forget plot]
  table[row sep=crcr]{%
-0.417103452421775	0.841492629446247\\
-0.427776767185283	0.832600276021122\\
};
\addplot [color=mycolor68, line width=1.5pt, forget plot]
  table[row sep=crcr]{%
-0.406245338999398	0.850158364867147\\
-0.417103452421775	0.841492629446247\\
};
\addplot [color=mycolor69, line width=1.5pt, forget plot]
  table[row sep=crcr]{%
-0.395207237630666	0.858593642909502\\
-0.406245338999398	0.850158364867147\\
};
\addplot [color=mycolor70, line width=1.5pt, forget plot]
  table[row sep=crcr]{%
-0.383994038772173	0.866794726303673\\
-0.395207237630666	0.858593642909502\\
};
\addplot [color=mycolor71, line width=1.5pt, forget plot]
  table[row sep=crcr]{%
-0.372610710457866	0.874757981540504\\
-0.383994038772173	0.866794726303673\\
};
\addplot [color=mycolor72, line width=1.5pt, forget plot]
  table[row sep=crcr]{%
-0.361062296097943	0.882479880481164\\
-0.372610710457866	0.874757981540504\\
};
\addplot [color=mycolor73, line width=1.5pt, forget plot]
  table[row sep=crcr]{%
-0.349353912244361	0.88995700192029\\
-0.361062296097943	0.882479880481164\\
};
\addplot [color=mycolor74, line width=1.5pt, forget plot]
  table[row sep=crcr]{%
-0.337490746323936	0.897186033101762\\
-0.349353912244361	0.88995700192029\\
};
\addplot [color=mycolor75, line width=1.5pt, forget plot]
  table[row sep=crcr]{%
-0.325478054340038	0.904163771186431\\
-0.337490746323936	0.897186033101762\\
};
\addplot [color=mycolor76, line width=1.5pt, forget plot]
  table[row sep=crcr]{%
-0.31332115854391	0.910887124671138\\
-0.325478054340038	0.904163771186431\\
};

\addplot[area legend, draw=none, fill=mycolor76, forget plot]
table[row sep=crcr] {%
x	y\\
-0.281411330820085	0.887833542216325\\
-0.325478054340037	0.904163771186431\\
-0.325217455975627	0.951158287765415\\
-0.436296359050945	0.827503052164232\\
}--cycle;
\addplot [color=mycolor77, line width=1.5pt, forget plot]
  table[row sep=crcr]{%
-0.301025445076631	0.917353114758419\\
-0.31332115854391	0.910887124671138\\
};
\addplot [color=mycolor78, line width=1.5pt, forget plot]
  table[row sep=crcr]{%
-0.288596361582772	0.923558876676262\\
-0.301025445076631	0.917353114758419\\
};
\addplot [color=mycolor79, line width=1.5pt, forget plot]
  table[row sep=crcr]{%
-0.276039414796809	0.929501660947353\\
-0.288596361582772	0.923558876676262\\
};
\addplot [color=mycolor80, line width=1.5pt, forget plot]
  table[row sep=crcr]{%
-0.263360168103345	0.935178834607241\\
-0.276039414796809	0.929501660947353\\
};
\addplot [color=mycolor81, line width=1.5pt, forget plot]
  table[row sep=crcr]{%
-0.250564239072244	0.940587882370874\\
-0.263360168103345	0.935178834607241\\
};
\addplot [color=mycolor82, line width=1.5pt, forget plot]
  table[row sep=crcr]{%
-0.237657296969751	0.945726407747005\\
-0.250564239072244	0.940587882370874\\
};
\addplot [color=mycolor83, line width=1.5pt, forget plot]
  table[row sep=crcr]{%
-0.224645060246714	0.950592134099966\\
-0.237657296969751	0.945726407747005\\
};
\addplot [color=mycolor84, line width=1.5pt, forget plot]
  table[row sep=crcr]{%
-0.211533294005007	0.955182905658329\\
-0.224645060246714	0.950592134099966\\
};
\addplot [color=mycolor85, line width=1.5pt, forget plot]
  table[row sep=crcr]{%
-0.19832780744329	0.959496688470033\\
-0.211533294005007	0.955182905658329\\
};
\addplot [color=mycolor86, line width=1.5pt, forget plot]
  table[row sep=crcr]{%
-0.185034451283229	0.963531571303526\\
-0.19832780744329	0.959496688470033\\
};
\addplot [color=mycolor87, line width=1.5pt, forget plot]
  table[row sep=crcr]{%
-0.171659115177323	0.967285766494544\\
-0.185034451283229	0.963531571303526\\
};
\addplot [color=mycolor88, line width=1.5pt, forget plot]
  table[row sep=crcr]{%
-0.158207725099474	0.97075761073814\\
-0.171659115177323	0.967285766494544\\
};
\addplot [color=mycolor89, line width=1.5pt, forget plot]
  table[row sep=crcr]{%
-0.144686240719476	0.973945565825611\\
-0.158207725099474	0.97075761073814\\
};
\addplot [color=mycolor90, line width=1.5pt, forget plot]
  table[row sep=crcr]{%
-0.131100652762569	0.976848219326009\\
-0.144686240719476	0.973945565825611\\
};
\addplot [color=mycolor91, line width=1.5pt, forget plot]
  table[row sep=crcr]{%
-0.117456980355229	0.979464285211921\\
-0.131100652762569	0.976848219326009\\
};
\addplot [color=mycolor92, line width=1.5pt, forget plot]
  table[row sep=crcr]{%
-0.103761268358386	0.981792604429246\\
-0.117456980355229	0.979464285211921\\
};
\addplot [color=mycolor93, line width=1.5pt, forget plot]
  table[row sep=crcr]{%
-0.0900195846892294	0.983832145410715\\
-0.103761268358386	0.981792604429246\\
};
\addplot [color=mycolor94, line width=1.5pt, forget plot]
  table[row sep=crcr]{%
-0.0762380176328084	0.985582004532934\\
-0.0900195846892294	0.983832145410715\\
};
\addplot [color=mycolor95, line width=1.5pt, forget plot]
  table[row sep=crcr]{%
-0.0624226731445977	0.98704140651673\\
-0.0762380176328084	0.985582004532934\\
};

\addplot[area legend, draw=none, fill=mycolor95, forget plot]
table[row sep=crcr] {%
x	y\\
-0.0420070668634339	0.953382710916075\\
-0.0762380176328084	0.985582004532934\\
-0.0577002819474253	1.02876654870859\\
-0.208159733769702	0.958118878135949\\
}--cycle;
\addplot [color=mycolor96, line width=1.5pt, forget plot]
  table[row sep=crcr]{%
-0.0485796721452428	0.988209704770646\\
-0.0624226731445977	0.98704140651673\\
};
\addplot [color=mycolor97, line width=1.5pt, forget plot]
  table[row sep=crcr]{%
-0.0347151478086711	0.989086381677411\\
-0.0485796721452428	0.988209704770646\\
};
\addplot [color=mycolor98, line width=1.5pt, forget plot]
  table[row sep=crcr]{%
-0.0208352428447749	0.989671048823275\\
-0.0347151478086711	0.989086381677411\\
};
\addplot [color=mycolor99, line width=1.5pt, forget plot]
  table[row sep=crcr]{%
-0.00694610677786922	0.989963447170092\\
-0.0208352428447749	0.989671048823275\\
};
\addplot [color=mycolor100, line width=1.5pt, forget plot]
  table[row sep=crcr]{%
0.00694610677786922	0.989963447170092\\
-0.00694610677786922	0.989963447170092\\
};
\addplot [color=mycolor101, line width=1.5pt, forget plot]
  table[row sep=crcr]{%
0.0208352428447749	0.989671048823275\\
0.00694610677786922	0.989963447170092\\
};
\addplot [color=white!1!mycolor100, line width=1.5pt, forget plot]
  table[row sep=crcr]{%
0.0347151478086711	0.989086381677411\\
0.0208352428447749	0.989671048823275\\
};
\addplot [color=mycolor102, line width=1.5pt, forget plot]
  table[row sep=crcr]{%
0.0485796721452428	0.988209704770646\\
0.0347151478086711	0.989086381677411\\
};
\addplot [color=mycolor103, line width=1.5pt, forget plot]
  table[row sep=crcr]{%
0.0624226731445977	0.98704140651673\\
0.0485796721452428	0.988209704770646\\
};
\addplot [color=mycolor104, line width=1.5pt, forget plot]
  table[row sep=crcr]{%
0.0762380176328084	0.985582004532934\\
0.0624226731445977	0.98704140651673\\
};
\addplot [color=mycolor105, line width=1.5pt, forget plot]
  table[row sep=crcr]{%
0.0900195846892294	0.983832145410715\\
0.0762380176328084	0.985582004532934\\
};
\addplot [color=mycolor106, line width=1.5pt, forget plot]
  table[row sep=crcr]{%
0.103761268358386	0.981792604429246\\
0.0900195846892294	0.983832145410715\\
};
\addplot [color=mycolor107, line width=1.5pt, forget plot]
  table[row sep=crcr]{%
0.117456980355229	0.979464285211921\\
0.103761268358386	0.981792604429246\\
};
\addplot [color=mycolor108, line width=1.5pt, forget plot]
  table[row sep=crcr]{%
0.131100652762569	0.976848219326009\\
0.117456980355229	0.979464285211921\\
};
\addplot [color=mycolor109, line width=1.5pt, forget plot]
  table[row sep=crcr]{%
0.144686240719476	0.973945565825611\\
0.131100652762569	0.976848219326009\\
};
\addplot [color=mycolor110, line width=1.5pt, forget plot]
  table[row sep=crcr]{%
0.158207725099474	0.97075761073814\\
0.144686240719476	0.973945565825611\\
};
\addplot [color=mycolor111, line width=1.5pt, forget plot]
  table[row sep=crcr]{%
0.171659115177323	0.967285766494544\\
0.158207725099474	0.97075761073814\\
};
\addplot [color=mycolor112, line width=1.5pt, forget plot]
  table[row sep=crcr]{%
0.185034451283229	0.963531571303526\\
0.171659115177323	0.967285766494544\\
};
\addplot [color=mycolor113, line width=1.5pt, forget plot]
  table[row sep=crcr]{%
0.19832780744329	0.959496688470033\\
0.185034451283229	0.963531571303526\\
};

\addplot[area legend, draw=none, fill=mycolor113, forget plot]
table[row sep=crcr] {%
x	y\\
0.204027054479588	0.920545137119871\\
0.18503445128323	0.963531571303526\\
0.21892356552952	0.996090447752228\\
0.0528301576766055	0.989600465640907\\
}--cycle;
\addplot [color=mycolor114, line width=1.5pt, forget plot]
  table[row sep=crcr]{%
0.211533294005007	0.955182905658329\\
0.19832780744329	0.959496688470033\\
};
\addplot [color=mycolor115, line width=1.5pt, forget plot]
  table[row sep=crcr]{%
0.224645060246714	0.950592134099966\\
0.211533294005007	0.955182905658329\\
};
\addplot [color=mycolor116, line width=1.5pt, forget plot]
  table[row sep=crcr]{%
0.237657296969751	0.945726407747005\\
0.224645060246714	0.950592134099966\\
};
\addplot [color=mycolor117, line width=1.5pt, forget plot]
  table[row sep=crcr]{%
0.250564239072244	0.940587882370874\\
0.237657296969751	0.945726407747005\\
};
\addplot [color=mycolor118, line width=1.5pt, forget plot]
  table[row sep=crcr]{%
0.263360168103345	0.935178834607241\\
0.250564239072244	0.940587882370874\\
};
\addplot [color=mycolor119, line width=1.5pt, forget plot]
  table[row sep=crcr]{%
0.276039414796809	0.929501660947353\\
0.263360168103345	0.935178834607241\\
};
\addplot [color=mycolor120, line width=1.5pt, forget plot]
  table[row sep=crcr]{%
0.288596361582772	0.923558876676262\\
0.276039414796809	0.929501660947353\\
};
\addplot [color=mycolor121, line width=1.5pt, forget plot]
  table[row sep=crcr]{%
0.301025445076631	0.917353114758419\\
0.288596361582772	0.923558876676262\\
};
\addplot [color=mycolor122, line width=1.5pt, forget plot]
  table[row sep=crcr]{%
0.31332115854391	0.910887124671138\\
0.301025445076631	0.917353114758419\\
};
\addplot [color=mycolor123, line width=1.5pt, forget plot]
  table[row sep=crcr]{%
0.325478054340038	0.904163771186431\\
0.31332115854391	0.910887124671138\\
};
\addplot [color=mycolor124, line width=1.5pt, forget plot]
  table[row sep=crcr]{%
0.337490746323936	0.897186033101762\\
0.325478054340038	0.904163771186431\\
};
\addplot [color=mycolor125, line width=1.5pt, forget plot]
  table[row sep=crcr]{%
0.349353912244361	0.88995700192029\\
0.337490746323936	0.897186033101762\\
};
\addplot [color=mycolor126, line width=1.5pt, forget plot]
  table[row sep=crcr]{%
0.361062296097943	0.882479880481164\\
0.349353912244361	0.88995700192029\\
};
\addplot [color=mycolor127, line width=1.5pt, forget plot]
  table[row sep=crcr]{%
0.372610710457866	0.874757981540504\\
0.361062296097943	0.882479880481164\\
};
\addplot [color=mycolor128, line width=1.5pt, forget plot]
  table[row sep=crcr]{%
0.383994038772173	0.866794726303673\\
0.372610710457866	0.874757981540504\\
};
\addplot [color=mycolor129, line width=1.5pt, forget plot]
  table[row sep=crcr]{%
0.395207237630666	0.858593642909502\\
0.383994038772173	0.866794726303673\\
};
\addplot [color=mycolor130, line width=1.5pt, forget plot]
  table[row sep=crcr]{%
0.406245338999398	0.850158364867147\\
0.395207237630666	0.858593642909502\\
};
\addplot [color=mycolor131, line width=1.5pt, forget plot]
  table[row sep=crcr]{%
0.417103452421775	0.841492629446247\\
0.406245338999398	0.850158364867147\\
};
\addplot [color=mycolor132, line width=1.5pt, forget plot]
  table[row sep=crcr]{%
0.427776767185283	0.832600276021122\\
0.417103452421775	0.841492629446247\\
};

\addplot[area legend, draw=none, fill=mycolor132, forget plot]
table[row sep=crcr] {%
x	y\\
0.417860158823289	0.794503482875769\\
0.417103452421775	0.841492629446247\\
0.460995321504928	0.858287162139578\\
0.305482013710109	0.916979164139116\\
}--cycle;
\addplot [color=mycolor133, line width=1.5pt, forget plot]
  table[row sep=crcr]{%
0.438260554452885	0.823485244369728\\
0.427776767185283	0.832600276021122\\
};
\addplot [color=mycolor134, line width=1.5pt, forget plot]
  table[row sep=crcr]{%
0.448550169358143	0.814151572928129\\
0.438260554452885	0.823485244369728\\
};
\addplot [color=mycolor135, line width=1.5pt, forget plot]
  table[row sep=crcr]{%
0.45864105306314	0.804603397001258\\
0.448550169358143	0.814151572928129\\
};
\addplot [color=mycolor136, line width=1.5pt, forget plot]
  table[row sep=crcr]{%
0.46852873477828	0.794844946930764\\
0.45864105306314	0.804603397001258\\
};
\addplot [color=mycolor137, line width=1.5pt, forget plot]
  table[row sep=crcr]{%
0.478208833743082	0.784880546220743\\
0.46852873477828	0.794844946930764\\
};
\addplot [color=mycolor138, line width=1.5pt, forget plot]
  table[row sep=crcr]{%
0.487677061167091	0.774714609622204\\
0.478208833743082	0.784880546220743\\
};
\addplot [color=mycolor139, line width=1.5pt, forget plot]
  table[row sep=crcr]{%
0.496929222130029	0.764351641177105\\
0.487677061167091	0.774714609622204\\
};
\addplot [color=mycolor140, line width=1.5pt, forget plot]
  table[row sep=crcr]{%
0.505961217440371	0.75379623222282\\
0.496929222130029	0.764351641177105\\
};
\addplot [color=mycolor141, line width=1.5pt, forget plot]
  table[row sep=crcr]{%
0.514769045451497	0.743053059357941\\
0.505961217440371	0.75379623222282\\
};
\addplot [color=mycolor142, line width=1.5pt, forget plot]
  table[row sep=crcr]{%
0.523348803834634	0.732126882370301\\
0.514769045451497	0.743053059357941\\
};
\addplot [color=mycolor143, line width=1.5pt, forget plot]
  table[row sep=crcr]{%
0.531696691307789	0.721022542128136\\
0.523348803834634	0.732126882370301\\
};
\addplot [color=mycolor144, line width=1.5pt, forget plot]
  table[row sep=crcr]{%
0.539809009319919	0.709744958435326\\
0.531696691307789	0.721022542128136\\
};
\addplot [color=mycolor145, line width=1.5pt, forget plot]
  table[row sep=crcr]{%
0.547682163689579	0.698299127851671\\
0.539809009319919	0.709744958435326\\
};
\addplot [color=mycolor146, line width=1.5pt, forget plot]
  table[row sep=crcr]{%
0.555312666197336	0.686690121479144\\
0.547682163689579	0.698299127851671\\
};
\addplot [color=mycolor147, line width=1.5pt, forget plot]
  table[row sep=crcr]{%
0.562697136131229	0.674923082715136\\
0.555312666197336	0.686690121479144\\
};
\addplot [color=mycolor148, line width=1.5pt, forget plot]
  table[row sep=crcr]{%
0.569832301784607	0.663003224973657\\
0.562697136131229	0.674923082715136\\
};
\addplot [color=mycolor149, line width=1.5pt, forget plot]
  table[row sep=crcr]{%
0.57671500190566	0.650935829375522\\
0.569832301784607	0.663003224973657\\
};
\addplot [color=mycolor150, line width=1.5pt, forget plot]
  table[row sep=crcr]{%
0.583342187098023	0.638726242408537\\
0.57671500190566	0.650935829375522\\
};
\addplot [color=mycolor151, line width=1.5pt, forget plot]
  table[row sep=crcr]{%
0.589710921171817	0.626379873558727\\
0.583342187098023	0.638726242408537\\
};

\addplot[area legend, draw=none, fill=mycolor151, forget plot]
table[row sep=crcr] {%
x	y\\
0.565743567858549	0.595150548485617\\
0.583342187098023	0.638726242408537\\
0.630309480590788	0.637105797418889\\
0.509920501464797	0.751716589689956\\
}--cycle;
\addplot [color=mycolor152, line width=1.5pt, forget plot]
  table[row sep=crcr]{%
0.595818382444533	0.613902192913651\\
0.589710921171817	0.626379873558727\\
};
\addplot [color=mycolor153, line width=1.5pt, forget plot]
  table[row sep=crcr]{%
0.60166186499119	0.601298728738863\\
0.595818382444533	0.613902192913651\\
};
\addplot [color=mycolor154, line width=1.5pt, forget plot]
  table[row sep=crcr]{%
0.607238779843195	0.588575065028606\\
0.60166186499119	0.601298728738863\\
};
\addplot [color=mycolor155, line width=1.5pt, forget plot]
  table[row sep=crcr]{%
0.612546656135397	0.575736839031805\\
0.607238779843195	0.588575065028606\\
};
\addplot [color=mycolor156, line width=1.5pt, forget plot]
  table[row sep=crcr]{%
0.617583142200807	0.562789738754478\\
0.612546656135397	0.575736839031805\\
};
\addplot [color=mycolor157, line width=1.5pt, forget plot]
  table[row sep=crcr]{%
0.622346006612508	0.549739500439642\\
0.617583142200807	0.562789738754478\\
};
\addplot [color=mycolor158, line width=1.5pt, forget plot]
  table[row sep=crcr]{%
0.626833139172299	0.536591906025868\\
0.622346006612508	0.549739500439642\\
};
\addplot [color=mycolor159, line width=1.5pt, forget plot]
  table[row sep=crcr]{%
0.631042551845617	0.52335278058557\\
0.626833139172299	0.536591906025868\\
};
\addplot [color=mycolor160, line width=1.5pt, forget plot]
  table[row sep=crcr]{%
0.634972379642345	0.510027989744198\\
0.631042551845617	0.52335278058557\\
};
\addplot [color=mycolor161, line width=1.5pt, forget plot]
  table[row sep=crcr]{%
0.638620881443099	0.496623437081459\\
0.634972379642345	0.510027989744198\\
};
\addplot [color=mycolor162, line width=1.5pt, forget plot]
  table[row sep=crcr]{%
0.641986440770633	0.483145061515721\\
0.638620881443099	0.496623437081459\\
};
\addplot [color=mycolor163, line width=1.5pt, forget plot]
  table[row sep=crcr]{%
0.645067566506027	0.469598834672761\\
0.641986440770633	0.483145061515721\\
};
\addplot [color=mycolor164, line width=1.5pt, forget plot]
  table[row sep=crcr]{%
0.647862893549325	0.455990758240023\\
0.645067566506027	0.469598834672761\\
};
\addplot [color=mycolor165, line width=1.5pt, forget plot]
  table[row sep=crcr]{%
0.650371183424353	0.442326861307556\\
0.647862893549325	0.455990758240023\\
};
\addplot [color=mycolor166, line width=1.5pt, forget plot]
  table[row sep=crcr]{%
0.652591324827422	0.428613197696812\\
0.650371183424353	0.442326861307556\\
};
\addplot [color=mycolor167, line width=1.5pt, forget plot]
  table[row sep=crcr]{%
0.654522334119697	0.414855843278488\\
0.652591324827422	0.428613197696812\\
};
\addplot [color=mycolor168, line width=1.5pt, forget plot]
  table[row sep=crcr]{%
0.656163355763002	0.401060893280592\\
0.654522334119697	0.414855843278488\\
};
\addplot [color=mycolor169, line width=1.5pt, forget plot]
  table[row sep=crcr]{%
0.657513662698864	0.387234459587944\\
0.656163355763002	0.401060893280592\\
};
\addplot [color=mycolor170, line width=1.5pt, forget plot]
  table[row sep=crcr]{%
0.658572656670642	0.373382668034286\\
0.657513662698864	0.387234459587944\\
};

\addplot[area legend, draw=none, fill=mycolor170, forget plot]
table[row sep=crcr] {%
x	y\\
0.624337258467314	0.353949647702269\\
0.657513662698864	0.387234459587944\\
0.700143664576741	0.36745478719702\\
0.633879668583051	0.51989567027944\\
}--cycle;
\addplot [color=mycolor171, line width=1.5pt, forget plot]
  table[row sep=crcr]{%
0.659339868488581	0.359511655688222\\
0.658572656670642	0.373382668034286\\
};
\addplot [color=mycolor172, line width=1.5pt, forget plot]
  table[row sep=crcr]{%
0.659814958237695	0.345627568134172\\
0.659339868488581	0.359511655688222\\
};
\addplot [color=mycolor173, line width=1.5pt, forget plot]
  table[row sep=crcr]{%
0.659997715428361	0.331736556749557\\
0.659814958237695	0.345627568134172\\
};
\addplot [color=mycolor174, line width=1.5pt, forget plot]
  table[row sep=crcr]{%
0.659888059089577	0.317844775979418\\
0.659997715428361	0.331736556749557\\
};
\addplot [color=mycolor175, line width=1.5pt, forget plot]
  table[row sep=crcr]{%
0.659486037804842	0.303958380609673\\
0.659888059089577	0.317844775979418\\
};
\addplot [color=mycolor176, line width=1.5pt, forget plot]
  table[row sep=crcr]{%
0.658791829690624	0.290083523040228\\
0.659486037804842	0.303958380609673\\
};
\addplot [color=mycolor177, line width=1.5pt, forget plot]
  table[row sep=crcr]{%
0.65780574231745	0.276226350559137\\
0.658791829690624	0.290083523040228\\
};
\addplot [color=mycolor178, line width=1.5pt, forget plot]
  table[row sep=crcr]{%
0.656528212573634	0.262393002619035\\
0.65780574231745	0.276226350559137\\
};
\addplot [color=mycolor179, line width=1.5pt, forget plot]
  table[row sep=crcr]{%
0.654959806471712	0.248589608117041\\
0.656528212573634	0.262393002619035\\
};
\addplot [color=mycolor180, line width=1.5pt, forget plot]
  table[row sep=crcr]{%
0.653101218897673	0.234822282679329\\
0.654959806471712	0.248589608117041\\
};
\addplot [color=mycolor181, line width=1.5pt, forget plot]
  table[row sep=crcr]{%
0.650953273303082	0.221097125951593\\
0.653101218897673	0.234822282679329\\
};
\addplot [color=mycolor182, line width=1.5pt, forget plot]
  table[row sep=crcr]{%
0.648516921340251	0.20742021889658\\
0.650953273303082	0.221097125951593\\
};
\addplot [color=mycolor183, line width=1.5pt, forget plot]
  table[row sep=crcr]{%
0.64579324244061	0.193797621099908\\
0.648516921340251	0.20742021889658\\
};
\addplot [color=mycolor184, line width=1.5pt, forget plot]
  table[row sep=crcr]{%
0.642783443336455	0.180235368085351\\
0.64579324244061	0.193797621099908\\
};
\addplot [color=mycolor185, line width=1.5pt, forget plot]
  table[row sep=crcr]{%
0.639488857526311	0.166739468640785\\
0.642783443336455	0.180235368085351\\
};
\addplot [color=mycolor186, line width=1.5pt, forget plot]
  table[row sep=crcr]{%
0.635910944684117	0.153315902155982\\
0.639488857526311	0.166739468640785\\
};
\addplot [color=mycolor187, line width=1.5pt, forget plot]
  table[row sep=crcr]{%
0.632051290012514	0.13997061597343\\
0.635910944684117	0.153315902155982\\
};
\addplot [color=mycolor188, line width=1.5pt, forget plot]
  table[row sep=crcr]{%
0.62791160354052	0.126709522753344\\
0.632051290012514	0.13997061597343\\
};
\addplot [color=mycolor189, line width=1.5pt, forget plot]
  table[row sep=crcr]{%
0.623493719365896	0.113538497854049\\
0.62791160354052	0.126709522753344\\
};

\addplot[area legend, draw=none, fill=mycolor189, forget plot]
table[row sep=crcr] {%
x	y\\
0.584393553020956	0.108968841284565\\
0.62791160354052	0.126709522753344\\
0.659476140333317	0.0918923931034692\\
0.657795387857437	0.258104050549976\\
}--cycle;
\addplot [color=mycolor190, line width=1.5pt, forget plot]
  table[row sep=crcr]{%
0.618799594842545	0.100463376728892\\
0.623493719365896	0.113538497854049\\
};
\addplot [color=mycolor191, line width=1.5pt, forget plot]
  table[row sep=crcr]{%
0.613831309713301	0.087489952340829\\
0.618799594842545	0.100463376728892\\
};
\addplot [color=mycolor192, line width=1.5pt, forget plot]
  table[row sep=crcr]{%
0.608591065188497	0.0746239725958389\\
0.613831309713301	0.087489952340829\\
};
\addplot [color=mycolor193, line width=1.5pt, forget plot]
  table[row sep=crcr]{%
0.603081182970708	0.0618711377962995\\
0.608591065188497	0.0746239725958389\\
};
\addplot [color=mycolor194, line width=1.5pt, forget plot]
  table[row sep=crcr]{%
0.597304104226123	0.0492370981154573\\
0.603081182970708	0.0618711377962995\\
};
\addplot [color=mycolor195, line width=1.5pt, forget plot]
  table[row sep=crcr]{%
0.591262388502971	0.0367274510941034\\
0.597304104226123	0.0492370981154573\\
};
\addplot [color=mycolor196, line width=1.5pt, forget plot]
  table[row sep=crcr]{%
0.584958712597518	0.0243477391605708\\
0.591262388502971	0.0367274510941034\\
};
\addplot [color=mycolor197, line width=1.5pt, forget plot]
  table[row sep=crcr]{%
0.578395869368097	0.0121034471751481\\
0.584958712597518	0.0243477391605708\\
};
\addplot [color=mycolor198, line width=1.5pt, forget plot]
  table[row sep=crcr]{%
0.57157676649773	2.22044604925031e-16\\
0.578395869368097	0.0121034471751481\\
};
\addplot [color=mycolor200, line width=1.5pt, mark size=0.9pt, mark=*, mark options={solid, mycolor200}, forget plot]
  table[row sep=crcr]{%
0	0\\
0	0.33\\
0.57157676649773	2.22044604925031e-16\\
};
\addplot [color=mycolor201, line width=1.5pt, mark size=0.9pt, mark=*, mark options={solid, mycolor201}, forget plot]
  table[row sep=crcr]{%
0	0\\
0	0.33\\
0.649973116988057	0.444607797260174\\
};
\addplot [color=mycolor202, line width=1.5pt, mark size=0.9pt, mark=*, mark options={solid, mycolor202}, forget plot]
  table[row sep=crcr]{%
0	0\\
0	0.33\\
0.424239822393116	0.835589332458526\\
};
\addplot [color=mycolor203, line width=1.5pt, mark size=0.9pt, mark=*, mark options={solid, mycolor203}, forget plot]
  table[row sep=crcr]{%
0	0\\
0	0.33\\
0	0.99\\
};
\addplot [color=mycolor204, line width=1.5pt, mark size=0.9pt, mark=*, mark options={solid, mycolor204}, forget plot]
  table[row sep=crcr]{%
0	0\\
0	0.33\\
-0.424239822393116	0.835589332458526\\
};
\addplot [color=mycolor205, line width=1.5pt, mark size=0.9pt, mark=*, mark options={solid, mycolor205}, forget plot]
  table[row sep=crcr]{%
0	0\\
0	0.33\\
-0.649973116988057	0.444607797260174\\
};
\addplot [color=mycolor206, line width=1.5pt, mark size=0.9pt, mark=*, mark options={solid, mycolor206}, forget plot]
  table[row sep=crcr]{%
0	0\\
0	0.33\\
-0.57157676649773	2.22044604925031e-16\\
};
\addplot [color=black, line width=3.0pt, forget plot]
  table[row sep=crcr]{%
-0.25	0\\
0.25	0\\
};

\addplot[area legend, draw=none, fill=mycolor207, forget plot]
table[row sep=crcr] {%
x	y\\
-0.25	0\\
-0.25	-0.1\\
0.25	-0.1\\
0.25	0\\
}--cycle;
\node[right, align=left]
at (axis cs:-0.15,-0.35) {\small (a)};
\end{axis}

\begin{axis}[%
width=0.302\textwidth,
height=0.264\textwidth,
at={(0.336\textwidth,0\textwidth)},
scale only axis,
xmin=-1,
xmax=1,
xtick={-1,-0.5,0,0.5,1},
xticklabels={\empty},
ymin=-0.5,
ymax=1.25,
ytick={-0.5,0,0.5,1},
yticklabels={\empty},
axis line style={draw=none},
ticks=none,
axis x line*=bottom,
axis y line*=left
]
\addplot [color=mycolor2, line width=1.5pt, forget plot]
  table[row sep=crcr]{%
-0.560468675552731	-0.0934454685921894\\
-0.551314364195506	-0.102039889794608\\
};
\addplot [color=mycolor3, line width=1.5pt, forget plot]
  table[row sep=crcr]{%
-0.569463648750363	-0.0845532149076785\\
-0.560468675552731	-0.0934454685921894\\
};
\addplot [color=mycolor4, line width=1.5pt, forget plot]
  table[row sep=crcr]{%
-0.578290196244506	-0.0753664079558142\\
-0.569463648750363	-0.0845532149076785\\
};
\addplot [color=mycolor5, line width=1.5pt, forget plot]
  table[row sep=crcr]{%
-0.586939303076644	-0.0658886044816366\\
-0.578290196244506	-0.0753664079558142\\
};
\addplot [color=mycolor6, line width=1.5pt, forget plot]
  table[row sep=crcr]{%
-0.595402035451534	-0.0561236370593131\\
-0.586939303076644	-0.0658886044816366\\
};
\addplot [color=mycolor7, line width=1.5pt, forget plot]
  table[row sep=crcr]{%
-0.603669549273176	-0.0460756121223141\\
-0.595402035451534	-0.0561236370593131\\
};
\addplot [color=mycolor8, line width=1.5pt, forget plot]
  table[row sep=crcr]{%
-0.611733098630366	-0.0357489077260493\\
-0.603669549273176	-0.0460756121223141\\
};
\addplot [color=mycolor9, line width=1.5pt, forget plot]
  table[row sep=crcr]{%
-0.619584044223325	-0.0251481710443079\\
-0.611733098630366	-0.0357489077260493\\
};
\addplot [color=mycolor10, line width=1.5pt, forget plot]
  table[row sep=crcr]{%
-0.627213861722984	-0.0142783156011266\\
-0.619584044223325	-0.0251481710443079\\
};
\addplot [color=mycolor11, line width=1.5pt, forget plot]
  table[row sep=crcr]{%
-0.634614150054488	-0.00314451823999418\\
-0.627213861722984	-0.0142783156011266\\
};
\addplot [color=mycolor12, line width=1.5pt, forget plot]
  table[row sep=crcr]{%
-0.641776639596618	0.00824778416744165\\
-0.634614150054488	-0.00314451823999418\\
};
\addplot [color=mycolor13, line width=1.5pt, forget plot]
  table[row sep=crcr]{%
-0.648693200288811	0.019892898270742\\
-0.641776639596618	0.00824778416744165\\
};
\addplot [color=mycolor14, line width=1.5pt, forget plot]
  table[row sep=crcr]{%
-0.655355849637604	0.0317848780453986\\
-0.648693200288811	0.019892898270742\\
};
\addplot [color=mycolor15, line width=1.5pt, forget plot]
  table[row sep=crcr]{%
-0.661756760614348	0.0439175288499597\\
-0.655355849637604	0.0317848780453986\\
};
\addplot [color=mycolor16, line width=1.5pt, forget plot]
  table[row sep=crcr]{%
-0.667888269436185	0.0562844117323311\\
-0.661756760614348	0.0439175288499597\\
};
\addplot [color=mycolor17, line width=1.5pt, forget plot]
  table[row sep=crcr]{%
-0.673742883222326	0.0688788479817597\\
-0.667888269436185	0.0562844117323311\\
};
\addplot [color=mycolor18, line width=1.5pt, forget plot]
  table[row sep=crcr]{%
-0.679313287517823	0.0816939239227776\\
-0.673742883222326	0.0688788479817597\\
};
\addplot [color=mycolor19, line width=1.5pt, forget plot]
  table[row sep=crcr]{%
-0.684592353677102	0.0947224959471047\\
-0.679313287517823	0.0816939239227776\\
};

\addplot[area legend, draw=none, fill=mycolor19, forget plot]
table[row sep=crcr] {%
x	y\\
-0.657816520222726	0.123484371346327\\
-0.679313287517823	0.0816939239227777\\
-0.725937433634245	0.087588115887291\\
-0.616494840464509	-0.0375176745473797\\
}--cycle;
\addplot [color=mycolor20, line width=1.5pt, forget plot]
  table[row sep=crcr]{%
-0.689573146099686	0.107957195779271\\
-0.684592353677102	0.0947224959471047\\
};
\addplot [color=mycolor21, line width=1.5pt, forget plot]
  table[row sep=crcr]{%
-0.694248929310623	0.121390435971488\\
-0.689573146099686	0.107957195779271\\
};
\addplot [color=mycolor22, line width=1.5pt, forget plot]
  table[row sep=crcr]{%
-0.698613174878335	0.135014415623032\\
-0.694248929310623	0.121390435971488\\
};
\addplot [color=mycolor23, line width=1.5pt, forget plot]
  table[row sep=crcr]{%
-0.702659568162677	0.148821126319198\\
-0.698613174878335	0.135014415623032\\
};
\addplot [color=mycolor24, line width=1.5pt, forget plot]
  table[row sep=crcr]{%
-0.706382014886215	0.162802358284616\\
-0.702659568162677	0.148821126319198\\
};
\addplot [color=mycolor25, line width=1.5pt, forget plot]
  table[row sep=crcr]{%
-0.709774647521856	0.176949706745537\\
-0.706382014886215	0.162802358284616\\
};
\addplot [color=mycolor26, line width=1.5pt, forget plot]
  table[row sep=crcr]{%
-0.712831831490152	0.191254578495445\\
-0.709774647521856	0.176949706745537\\
};
\addplot [color=mycolor27, line width=1.5pt, forget plot]
  table[row sep=crcr]{%
-0.715548171159751	0.20570819865813\\
-0.712831831490152	0.191254578495445\\
};
\addplot [color=mycolor28, line width=1.5pt, forget plot]
  table[row sep=crcr]{%
-0.717918515644694	0.220301617642194\\
-0.715548171159751	0.20570819865813\\
};
\addplot [color=mycolor29, line width=1.5pt, forget plot]
  table[row sep=crcr]{%
-0.719937964392406	0.235025718280693\\
-0.717918515644694	0.220301617642194\\
};
\addplot [color=mycolor30, line width=1.5pt, forget plot]
  table[row sep=crcr]{%
-0.721601872556438	0.249871223149482\\
-0.719937964392406	0.235025718280693\\
};
\addplot [color=mycolor31, line width=1.5pt, forget plot]
  table[row sep=crcr]{%
-0.722905856148246	0.264828702057585\\
-0.721601872556438	0.249871223149482\\
};
\addplot [color=mycolor32, line width=1.5pt, forget plot]
  table[row sep=crcr]{%
-0.723845796962466	0.279888579702773\\
-0.722905856148246	0.264828702057585\\
};
\addplot [color=mycolor33, line width=1.5pt, forget plot]
  table[row sep=crcr]{%
-0.724417847270372	0.295041143485326\\
-0.723845796962466	0.279888579702773\\
};
\addplot [color=mycolor34, line width=1.5pt, forget plot]
  table[row sep=crcr]{%
-0.724618434276444	0.310276551472782\\
-0.724417847270372	0.295041143485326\\
};
\addplot [color=mycolor35, line width=1.5pt, forget plot]
  table[row sep=crcr]{%
-0.724444264333178	0.325584840508339\\
-0.724618434276444	0.310276551472782\\
};
\addplot [color=mycolor36, line width=1.5pt, forget plot]
  table[row sep=crcr]{%
-0.723892326909498	0.340955934455395\\
-0.724444264333178	0.325584840508339\\
};
\addplot [color=mycolor37, line width=1.5pt, forget plot]
  table[row sep=crcr]{%
-0.722959898308402	0.35637965257056\\
-0.723892326909498	0.340955934455395\\
};
\addplot [color=mycolor38, line width=1.5pt, forget plot]
  table[row sep=crcr]{%
-0.721644545129665	0.371845717997352\\
-0.722959898308402	0.35637965257056\\
};

\addplot[area legend, draw=none, fill=mycolor38, forget plot]
table[row sep=crcr] {%
x	y\\
-0.684872771675834	0.383910059121767\\
-0.722959898308402	0.35637965257056\\
-0.761863927673703	0.382743055217861\\
-0.720917641476567	0.221645129574288\\
}--cycle;
\addplot [color=mycolor39, line width=1.5pt, forget plot]
  table[row sep=crcr]{%
-0.719944127473723	0.387343766372636\\
-0.721644545129665	0.371845717997352\\
};
\addplot [color=mycolor40, line width=1.5pt, forget plot]
  table[row sep=crcr]{%
-0.71785680188307	0.402863354537747\\
-0.719944127473723	0.387343766372636\\
};
\addplot [color=mycolor41, line width=1.5pt, forget plot]
  table[row sep=crcr]{%
-0.715381024017778	0.418393969346123\\
-0.71785680188307	0.402863354537747\\
};
\addplot [color=mycolor42, line width=1.5pt, forget plot]
  table[row sep=crcr]{%
-0.712515551061994	0.433925036559159\\
-0.715381024017778	0.418393969346123\\
};
\addplot [color=mycolor43, line width=1.5pt, forget plot]
  table[row sep=crcr]{%
-0.709259443858565	0.449445929821879\\
-0.712515551061994	0.433925036559159\\
};
\addplot [color=mycolor44, line width=1.5pt, forget plot]
  table[row sep=crcr]{%
-0.705612068769143	0.464945979709964\\
-0.709259443858565	0.449445929821879\\
};
\addplot [color=mycolor45, line width=1.5pt, forget plot]
  table[row sep=crcr]{%
-0.701573099257476	0.480414482839563\\
-0.705612068769143	0.464945979709964\\
};
\addplot [color=mycolor46, line width=1.5pt, forget plot]
  table[row sep=crcr]{%
-0.697142517193772	0.495840711031221\\
-0.701573099257476	0.480414482839563\\
};
\addplot [color=mycolor47, line width=1.5pt, forget plot]
  table[row sep=crcr]{%
-0.692320613878376	0.511213920519252\\
-0.697142517193772	0.495840711031221\\
};
\addplot [color=mycolor48, line width=1.5pt, forget plot]
  table[row sep=crcr]{%
-0.687107990783237	0.526523361197755\\
-0.692320613878376	0.511213920519252\\
};
\addplot [color=mycolor49, line width=1.5pt, forget plot]
  table[row sep=crcr]{%
-0.681505560009904	0.541758285894462\\
-0.687107990783237	0.526523361197755\\
};
\addplot [color=mycolor50, line width=1.5pt, forget plot]
  table[row sep=crcr]{%
-0.675514544463122	0.556907959663575\\
-0.681505560009904	0.541758285894462\\
};
\addplot [color=mycolor51, line width=1.5pt, forget plot]
  table[row sep=crcr]{%
-0.669136477739324	0.571961669088677\\
-0.675514544463122	0.556907959663575\\
};
\addplot [color=mycolor52, line width=1.5pt, forget plot]
  table[row sep=crcr]{%
-0.662373203729645	0.586908731586822\\
-0.669136477739324	0.571961669088677\\
};
\addplot [color=mycolor53, line width=1.5pt, forget plot]
  table[row sep=crcr]{%
-0.655226875937328	0.601738504704855\\
-0.662373203729645	0.586908731586822\\
};
\addplot [color=mycolor54, line width=1.5pt, forget plot]
  table[row sep=crcr]{%
-0.647699956509695	0.616440395399063\\
-0.655226875937328	0.601738504704855\\
};
\addplot [color=mycolor55, line width=1.5pt, forget plot]
  table[row sep=crcr]{%
-0.639795214985152	0.631003869289184\\
-0.647699956509695	0.616440395399063\\
};
\addplot [color=mycolor56, line width=1.5pt, forget plot]
  table[row sep=crcr]{%
-0.631515726755943	0.645418459877913\\
-0.639795214985152	0.631003869289184\\
};
\addplot [color=mycolor57, line width=1.5pt, forget plot]
  table[row sep=crcr]{%
-0.622864871247679	0.659673777726962\\
-0.631515726755943	0.645418459877913\\
};

\addplot[area legend, draw=none, fill=mycolor57, forget plot]
table[row sep=crcr] {%
x	y\\
-0.585160955160859	0.653150707311076\\
-0.631515726755943	0.645418459877913\\
-0.654647324017635	0.686326670644493\\
-0.689573662589421	0.523817314378557\\
}--cycle;
\addplot [color=mycolor58, line width=1.5pt, forget plot]
  table[row sep=crcr]{%
-0.613846329816959	0.673759519580849\\
-0.622864871247679	0.659673777726962\\
};
\addplot [color=mycolor59, line width=1.5pt, forget plot]
  table[row sep=crcr]{%
-0.604464083368649	0.687665477429592\\
-0.613846329816959	0.673759519580849\\
};
\addplot [color=mycolor60, line width=1.5pt, forget plot]
  table[row sep=crcr]{%
-0.594722409694684	0.701381547501531\\
-0.604464083368649	0.687665477429592\\
};
\addplot [color=mycolor61, line width=1.5pt, forget plot]
  table[row sep=crcr]{%
-0.584625880536536	0.71489773917756\\
-0.594722409694684	0.701381547501531\\
};
\addplot [color=mycolor62, line width=1.5pt, forget plot]
  table[row sep=crcr]{%
-0.57417935837377	0.728204183818162\\
-0.584625880536536	0.71489773917756\\
};
\addplot [color=mycolor63, line width=1.5pt, forget plot]
  table[row sep=crcr]{%
-0.563387992941376	0.741291143494641\\
-0.57417935837377	0.728204183818162\\
};
\addplot [color=mycolor64, line width=1.5pt, forget plot]
  table[row sep=crcr]{%
-0.552257217478825	0.754149019616096\\
-0.563387992941376	0.741291143494641\\
};
\addplot [color=mycolor65, line width=1.5pt, forget plot]
  table[row sep=crcr]{%
-0.540792744714104	0.766768361443745\\
-0.552257217478825	0.754149019616096\\
};
\addplot [color=mycolor66, line width=1.5pt, forget plot]
  table[row sep=crcr]{%
-0.529000562586218	0.779139874484314\\
-0.540792744714104	0.766768361443745\\
};
\addplot [color=mycolor67, line width=1.5pt, forget plot]
  table[row sep=crcr]{%
-0.516886929709903	0.791254428754319\\
-0.529000562586218	0.779139874484314\\
};
\addplot [color=mycolor68, line width=1.5pt, forget plot]
  table[row sep=crcr]{%
-0.504458370586586	0.803103066907197\\
-0.516886929709903	0.791254428754319\\
};
\addplot [color=mycolor69, line width=1.5pt, forget plot]
  table[row sep=crcr]{%
-0.491721670565867	0.814677012215347\\
-0.504458370586586	0.803103066907197\\
};
\addplot [color=mycolor70, line width=1.5pt, forget plot]
  table[row sep=crcr]{%
-0.478683870562022	0.825967676399317\\
-0.491721670565867	0.814677012215347\\
};
\addplot [color=mycolor71, line width=1.5pt, forget plot]
  table[row sep=crcr]{%
-0.465352261530333	0.836966667296457\\
-0.478683870562022	0.825967676399317\\
};
\addplot [color=mycolor72, line width=1.5pt, forget plot]
  table[row sep=crcr]{%
-0.451734378708216	0.847665796361585\\
-0.465352261530333	0.836966667296457\\
};
\addplot [color=mycolor73, line width=1.5pt, forget plot]
  table[row sep=crcr]{%
-0.43783799562643	0.858057085992306\\
-0.451734378708216	0.847665796361585\\
};
\addplot [color=mycolor74, line width=1.5pt, forget plot]
  table[row sep=crcr]{%
-0.423671117895826	0.868132776671835\\
-0.43783799562643	0.858057085992306\\
};
\addplot [color=mycolor75, line width=1.5pt, forget plot]
  table[row sep=crcr]{%
-0.409241976775351	0.877885333922325\\
-0.423671117895826	0.868132776671835\\
};
\addplot [color=mycolor76, line width=1.5pt, forget plot]
  table[row sep=crcr]{%
-0.394559022527225	0.887307455061886\\
-0.409241976775351	0.877885333922325\\
};

\addplot[area legend, draw=none, fill=mycolor76, forget plot]
table[row sep=crcr] {%
x	y\\
-0.364203017062605	0.864467263718543\\
-0.409241976775351	0.877885333922325\\
-0.412050413800992	0.924796581842978\\
-0.514818283493113	0.794152389630146\\
}--cycle;
\addplot [color=mycolor77, line width=1.5pt, forget plot]
  table[row sep=crcr]{%
-0.379630917565443	0.896392075758702\\
-0.394559022527225	0.887307455061886\\
};
\addplot [color=mycolor78, line width=1.5pt, forget plot]
  table[row sep=crcr]{%
-0.364466529403939	0.905132376375786\\
-0.379630917565443	0.896392075758702\\
};
\addplot [color=mycolor79, line width=1.5pt, forget plot]
  table[row sep=crcr]{%
-0.349074923410974	0.91352178810017\\
-0.364466529403939	0.905132376375786\\
};
\addplot [color=mycolor80, line width=1.5pt, forget plot]
  table[row sep=crcr]{%
-0.333465355376492	0.921553998850533\\
-0.349074923410974	0.91352178810017\\
};
\addplot [color=mycolor81, line width=1.5pt, forget plot]
  table[row sep=crcr]{%
-0.317647263899374	0.92922295895743\\
-0.333465355376492	0.921553998850533\\
};
\addplot [color=mycolor82, line width=1.5pt, forget plot]
  table[row sep=crcr]{%
-0.301630262601715	0.936522886610579\\
-0.317647263899374	0.92922295895743\\
};
\addplot [color=mycolor83, line width=1.5pt, forget plot]
  table[row sep=crcr]{%
-0.285424132177413	0.943448273067842\\
-0.301630262601715	0.936522886610579\\
};
\addplot [color=mycolor84, line width=1.5pt, forget plot]
  table[row sep=crcr]{%
-0.269038812282539	0.949993887620783\\
-0.285424132177413	0.943448273067842\\
};
\addplot [color=mycolor85, line width=1.5pt, forget plot]
  table[row sep=crcr]{%
-0.252484393275089	0.956154782311926\\
-0.269038812282539	0.949993887620783\\
};
\addplot [color=mycolor86, line width=1.5pt, forget plot]
  table[row sep=crcr]{%
-0.235771107811911	0.961926296399059\\
-0.252484393275089	0.956154782311926\\
};
\addplot [color=mycolor87, line width=1.5pt, forget plot]
  table[row sep=crcr]{%
-0.218909322310717	0.967304060562236\\
-0.235771107811911	0.961926296399059\\
};
\addplot [color=mycolor88, line width=1.5pt, forget plot]
  table[row sep=crcr]{%
-0.201909528285223	0.972284000849276\\
-0.218909322310717	0.967304060562236\\
};
\addplot [color=mycolor89, line width=1.5pt, forget plot]
  table[row sep=crcr]{%
-0.184782333561615	0.976862342355932\\
-0.201909528285223	0.972284000849276\\
};
\addplot [color=mycolor90, line width=1.5pt, forget plot]
  table[row sep=crcr]{%
-0.167538453384636	0.981035612637084\\
-0.184782333561615	0.976862342355932\\
};
\addplot [color=mycolor91, line width=1.5pt, forget plot]
  table[row sep=crcr]{%
-0.150188701421706	0.984800644845588\\
-0.167538453384636	0.981035612637084\\
};
\addplot [color=mycolor92, line width=1.5pt, forget plot]
  table[row sep=crcr]{%
-0.132743980673596	0.988154580595718\\
-0.150188701421706	0.984800644845588\\
};
\addplot [color=mycolor93, line width=1.5pt, forget plot]
  table[row sep=crcr]{%
-0.115215274300263	0.99109487254836\\
-0.132743980673596	0.988154580595718\\
};
\addplot [color=mycolor94, line width=1.5pt, forget plot]
  table[row sep=crcr]{%
-0.0976136363705509	0.99361928671541\\
-0.115215274300263	0.99109487254836\\
};
\addplot [color=mycolor95, line width=1.5pt, forget plot]
  table[row sep=crcr]{%
-0.0799501825445227	0.995725904481129\\
-0.0976136363705509	0.99361928671541\\
};

\addplot[area legend, draw=none, fill=mycolor95, forget plot]
table[row sep=crcr] {%
x	y\\
-0.0629524754595391	0.961883558831825\\
-0.0976136363705508	0.99361928671541\\
-0.0796587189394079	1.0370493850349\\
-0.229153832225937	0.96438336062564\\
}--cycle;
\addplot [color=mycolor96, line width=1.5pt, forget plot]
  table[row sep=crcr]{%
-0.0622360806972782	0.997413124338431\\
-0.0799501825445227	0.995725904481129\\
};
\addplot [color=mycolor97, line width=1.5pt, forget plot]
  table[row sep=crcr]{%
-0.0444825414931492	0.998679663338427\\
-0.0622360806972782	0.997413124338431\\
};
\addplot [color=mycolor98, line width=1.5pt, forget plot]
  table[row sep=crcr]{%
-0.026700808919232	0.999524558251764\\
-0.0444825414931492	0.998679663338427\\
};
\addplot [color=mycolor99, line width=1.5pt, forget plot]
  table[row sep=crcr]{%
-0.00890215078724677	0.999947166440646\\
-0.026700808919232	0.999524558251764\\
};
\addplot [color=mycolor100, line width=1.5pt, forget plot]
  table[row sep=crcr]{%
0.00890215078724688	0.999947166440646\\
-0.00890215078724677	0.999947166440646\\
};
\addplot [color=mycolor101, line width=1.5pt, forget plot]
  table[row sep=crcr]{%
0.0267008089192321	0.999524558251764\\
0.00890215078724688	0.999947166440646\\
};
\addplot [color=white!1!mycolor100, line width=1.5pt, forget plot]
  table[row sep=crcr]{%
0.0444825414931493	0.998679663338427\\
0.0267008089192321	0.999524558251764\\
};
\addplot [color=mycolor102, line width=1.5pt, forget plot]
  table[row sep=crcr]{%
0.0622360806972783	0.997413124338431\\
0.0444825414931493	0.998679663338427\\
};
\addplot [color=mycolor103, line width=1.5pt, forget plot]
  table[row sep=crcr]{%
0.0799501825445228	0.995725904481129\\
0.0622360806972783	0.997413124338431\\
};
\addplot [color=mycolor104, line width=1.5pt, forget plot]
  table[row sep=crcr]{%
0.097613636370551	0.99361928671541\\
0.0799501825445228	0.995725904481129\\
};
\addplot [color=mycolor105, line width=1.5pt, forget plot]
  table[row sep=crcr]{%
0.115215274300263	0.99109487254836\\
0.097613636370551	0.99361928671541\\
};
\addplot [color=mycolor106, line width=1.5pt, forget plot]
  table[row sep=crcr]{%
0.132743980673596	0.988154580595718\\
0.115215274300263	0.99109487254836\\
};
\addplot [color=mycolor107, line width=1.5pt, forget plot]
  table[row sep=crcr]{%
0.150188701421706	0.984800644845588\\
0.132743980673596	0.988154580595718\\
};
\addplot [color=mycolor108, line width=1.5pt, forget plot]
  table[row sep=crcr]{%
0.167538453384636	0.981035612637084\\
0.150188701421706	0.984800644845588\\
};
\addplot [color=mycolor109, line width=1.5pt, forget plot]
  table[row sep=crcr]{%
0.184782333561615	0.976862342355932\\
0.167538453384636	0.981035612637084\\
};
\addplot [color=mycolor110, line width=1.5pt, forget plot]
  table[row sep=crcr]{%
0.201909528285223	0.972284000849276\\
0.184782333561615	0.976862342355932\\
};
\addplot [color=mycolor111, line width=1.5pt, forget plot]
  table[row sep=crcr]{%
0.218909322310718	0.967304060562236\\
0.201909528285223	0.972284000849276\\
};
\addplot [color=mycolor112, line width=1.5pt, forget plot]
  table[row sep=crcr]{%
0.235771107811911	0.961926296399059\\
0.218909322310718	0.967304060562236\\
};
\addplot [color=mycolor113, line width=1.5pt, forget plot]
  table[row sep=crcr]{%
0.252484393275089	0.956154782311926\\
0.235771107811911	0.961926296399059\\
};

\addplot[area legend, draw=none, fill=mycolor113, forget plot]
table[row sep=crcr] {%
x	y\\
0.253124982859919	0.918252556911604\\
0.235771107811911	0.961926296399059\\
0.270866919226646	0.993180680429806\\
0.104646891655057	0.992974685040832\\
}--cycle;
\addplot [color=mycolor114, line width=1.5pt, forget plot]
  table[row sep=crcr]{%
0.269038812282539	0.949993887620783\\
0.252484393275089	0.956154782311926\\
};
\addplot [color=mycolor115, line width=1.5pt, forget plot]
  table[row sep=crcr]{%
0.285424132177413	0.943448273067842\\
0.269038812282539	0.949993887620783\\
};
\addplot [color=mycolor116, line width=1.5pt, forget plot]
  table[row sep=crcr]{%
0.301630262601715	0.936522886610579\\
0.285424132177413	0.943448273067842\\
};
\addplot [color=mycolor117, line width=1.5pt, forget plot]
  table[row sep=crcr]{%
0.317647263899375	0.92922295895743\\
0.301630262601715	0.936522886610579\\
};
\addplot [color=mycolor118, line width=1.5pt, forget plot]
  table[row sep=crcr]{%
0.333465355376493	0.921553998850533\\
0.317647263899375	0.92922295895743\\
};
\addplot [color=mycolor119, line width=1.5pt, forget plot]
  table[row sep=crcr]{%
0.349074923410974	0.91352178810017\\
0.333465355376493	0.921553998850533\\
};
\addplot [color=mycolor120, line width=1.5pt, forget plot]
  table[row sep=crcr]{%
0.364466529403939	0.905132376375786\\
0.349074923410974	0.91352178810017\\
};
\addplot [color=mycolor121, line width=1.5pt, forget plot]
  table[row sep=crcr]{%
0.379630917565443	0.896392075758702\\
0.364466529403939	0.905132376375786\\
};
\addplot [color=mycolor122, line width=1.5pt, forget plot]
  table[row sep=crcr]{%
0.394559022527225	0.887307455061886\\
0.379630917565443	0.896392075758702\\
};
\addplot [color=mycolor123, line width=1.5pt, forget plot]
  table[row sep=crcr]{%
0.409241976775351	0.877885333922325\\
0.394559022527225	0.887307455061886\\
};
\addplot [color=mycolor124, line width=1.5pt, forget plot]
  table[row sep=crcr]{%
0.423671117895826	0.868132776671835\\
0.409241976775351	0.877885333922325\\
};
\addplot [color=mycolor125, line width=1.5pt, forget plot]
  table[row sep=crcr]{%
0.43783799562643	0.858057085992306\\
0.423671117895826	0.868132776671835\\
};
\addplot [color=mycolor126, line width=1.5pt, forget plot]
  table[row sep=crcr]{%
0.451734378708216	0.847665796361585\\
0.43783799562643	0.858057085992306\\
};
\addplot [color=mycolor127, line width=1.5pt, forget plot]
  table[row sep=crcr]{%
0.465352261530333	0.836966667296457\\
0.451734378708216	0.847665796361585\\
};
\addplot [color=mycolor128, line width=1.5pt, forget plot]
  table[row sep=crcr]{%
0.478683870562022	0.825967676399317\\
0.465352261530333	0.836966667296457\\
};
\addplot [color=mycolor129, line width=1.5pt, forget plot]
  table[row sep=crcr]{%
0.491721670565867	0.814677012215347\\
0.478683870562022	0.825967676399317\\
};
\addplot [color=mycolor130, line width=1.5pt, forget plot]
  table[row sep=crcr]{%
0.504458370586586	0.803103066907197\\
0.491721670565867	0.814677012215347\\
};
\addplot [color=mycolor131, line width=1.5pt, forget plot]
  table[row sep=crcr]{%
0.516886929709903	0.791254428754319\\
0.504458370586586	0.803103066907197\\
};
\addplot [color=mycolor132, line width=1.5pt, forget plot]
  table[row sep=crcr]{%
0.529000562586218	0.779139874484314\\
0.516886929709903	0.791254428754319\\
};

\addplot[area legend, draw=none, fill=mycolor132, forget plot]
table[row sep=crcr] {%
x	y\\
0.513379169930066	0.744390283383609\\
0.516886929709903	0.791254428754319\\
0.562120997499641	0.803999294826325\\
0.412571159685151	0.876552627001076\\
}--cycle;
\addplot [color=mycolor133, line width=1.5pt, forget plot]
  table[row sep=crcr]{%
0.540792744714104	0.766768361443745\\
0.529000562586218	0.779139874484314\\
};
\addplot [color=mycolor134, line width=1.5pt, forget plot]
  table[row sep=crcr]{%
0.552257217478825	0.754149019616096\\
0.540792744714104	0.766768361443745\\
};
\addplot [color=mycolor135, line width=1.5pt, forget plot]
  table[row sep=crcr]{%
0.563387992941376	0.741291143494641\\
0.552257217478825	0.754149019616096\\
};
\addplot [color=mycolor136, line width=1.5pt, forget plot]
  table[row sep=crcr]{%
0.57417935837377	0.728204183818162\\
0.563387992941376	0.741291143494641\\
};
\addplot [color=mycolor137, line width=1.5pt, forget plot]
  table[row sep=crcr]{%
0.584625880536536	0.71489773917756\\
0.57417935837377	0.728204183818162\\
};
\addplot [color=mycolor138, line width=1.5pt, forget plot]
  table[row sep=crcr]{%
0.594722409694684	0.701381547501531\\
0.584625880536536	0.71489773917756\\
};
\addplot [color=mycolor139, line width=1.5pt, forget plot]
  table[row sep=crcr]{%
0.604464083368649	0.687665477429592\\
0.594722409694684	0.701381547501531\\
};
\addplot [color=mycolor140, line width=1.5pt, forget plot]
  table[row sep=crcr]{%
0.613846329816959	0.673759519580849\\
0.604464083368649	0.687665477429592\\
};
\addplot [color=mycolor141, line width=1.5pt, forget plot]
  table[row sep=crcr]{%
0.622864871247679	0.659673777726962\\
0.613846329816959	0.673759519580849\\
};
\addplot [color=mycolor142, line width=1.5pt, forget plot]
  table[row sep=crcr]{%
0.631515726755943	0.645418459877913\\
0.622864871247679	0.659673777726962\\
};
\addplot [color=mycolor143, line width=1.5pt, forget plot]
  table[row sep=crcr]{%
0.639795214985152	0.631003869289184\\
0.631515726755943	0.645418459877913\\
};
\addplot [color=mycolor144, line width=1.5pt, forget plot]
  table[row sep=crcr]{%
0.647699956509695	0.616440395399063\\
0.639795214985152	0.631003869289184\\
};
\addplot [color=mycolor145, line width=1.5pt, forget plot]
  table[row sep=crcr]{%
0.655226875937328	0.601738504704855\\
0.647699956509695	0.616440395399063\\
};
\addplot [color=mycolor146, line width=1.5pt, forget plot]
  table[row sep=crcr]{%
0.662373203729645	0.586908731586822\\
0.655226875937328	0.601738504704855\\
};
\addplot [color=mycolor147, line width=1.5pt, forget plot]
  table[row sep=crcr]{%
0.669136477739324	0.571961669088677\\
0.662373203729645	0.586908731586822\\
};
\addplot [color=mycolor148, line width=1.5pt, forget plot]
  table[row sep=crcr]{%
0.675514544463122	0.556907959663575\\
0.669136477739324	0.571961669088677\\
};
\addplot [color=mycolor149, line width=1.5pt, forget plot]
  table[row sep=crcr]{%
0.681505560009904	0.541758285894462\\
0.675514544463122	0.556907959663575\\
};
\addplot [color=mycolor150, line width=1.5pt, forget plot]
  table[row sep=crcr]{%
0.687107990783237	0.526523361197755\\
0.681505560009904	0.541758285894462\\
};
\addplot [color=mycolor151, line width=1.5pt, forget plot]
  table[row sep=crcr]{%
0.692320613878376	0.511213920519252\\
0.687107990783237	0.526523361197755\\
};

\addplot[area legend, draw=none, fill=mycolor151, forget plot]
table[row sep=crcr] {%
x	y\\
0.663273418951879	0.486020698499905\\
0.687107990783237	0.526523361197755\\
0.733322325958394	0.517991788991046\\
0.631158582423741	0.649108948459155\\
}--cycle;
\addplot [color=mycolor152, line width=1.5pt, forget plot]
  table[row sep=crcr]{%
0.697142517193772	0.495840711031221\\
0.692320613878376	0.511213920519252\\
};
\addplot [color=mycolor153, line width=1.5pt, forget plot]
  table[row sep=crcr]{%
0.701573099257476	0.480414482839563\\
0.697142517193772	0.495840711031221\\
};
\addplot [color=mycolor154, line width=1.5pt, forget plot]
  table[row sep=crcr]{%
0.705612068769143	0.464945979709964\\
0.701573099257476	0.480414482839563\\
};
\addplot [color=mycolor155, line width=1.5pt, forget plot]
  table[row sep=crcr]{%
0.709259443858565	0.449445929821879\\
0.705612068769143	0.464945979709964\\
};
\addplot [color=mycolor156, line width=1.5pt, forget plot]
  table[row sep=crcr]{%
0.712515551061994	0.433925036559159\\
0.709259443858565	0.449445929821879\\
};
\addplot [color=mycolor157, line width=1.5pt, forget plot]
  table[row sep=crcr]{%
0.715381024017778	0.418393969346123\\
0.712515551061994	0.433925036559159\\
};
\addplot [color=mycolor158, line width=1.5pt, forget plot]
  table[row sep=crcr]{%
0.71785680188307	0.402863354537747\\
0.715381024017778	0.418393969346123\\
};
\addplot [color=mycolor159, line width=1.5pt, forget plot]
  table[row sep=crcr]{%
0.719944127473723	0.387343766372636\\
0.71785680188307	0.402863354537747\\
};
\addplot [color=mycolor160, line width=1.5pt, forget plot]
  table[row sep=crcr]{%
0.721644545129665	0.371845717997352\\
0.719944127473723	0.387343766372636\\
};
\addplot [color=mycolor161, line width=1.5pt, forget plot]
  table[row sep=crcr]{%
0.722959898308402	0.35637965257056\\
0.721644545129665	0.371845717997352\\
};
\addplot [color=mycolor162, line width=1.5pt, forget plot]
  table[row sep=crcr]{%
0.723892326909498	0.340955934455395\\
0.722959898308402	0.35637965257056\\
};
\addplot [color=mycolor163, line width=1.5pt, forget plot]
  table[row sep=crcr]{%
0.724444264333178	0.325584840508339\\
0.723892326909498	0.340955934455395\\
};
\addplot [color=mycolor164, line width=1.5pt, forget plot]
  table[row sep=crcr]{%
0.724618434276444	0.310276551472782\\
0.724444264333178	0.325584840508339\\
};
\addplot [color=mycolor165, line width=1.5pt, forget plot]
  table[row sep=crcr]{%
0.724417847270372	0.295041143485326\\
0.724618434276444	0.310276551472782\\
};
\addplot [color=mycolor166, line width=1.5pt, forget plot]
  table[row sep=crcr]{%
0.723845796962466	0.279888579702773\\
0.724417847270372	0.295041143485326\\
};
\addplot [color=mycolor167, line width=1.5pt, forget plot]
  table[row sep=crcr]{%
0.722905856148246	0.264828702057585\\
0.723845796962466	0.279888579702773\\
};
\addplot [color=mycolor168, line width=1.5pt, forget plot]
  table[row sep=crcr]{%
0.721601872556438	0.249871223149482\\
0.722905856148246	0.264828702057585\\
};
\addplot [color=mycolor169, line width=1.5pt, forget plot]
  table[row sep=crcr]{%
0.719937964392406	0.235025718280693\\
0.721601872556438	0.249871223149482\\
};
\addplot [color=mycolor170, line width=1.5pt, forget plot]
  table[row sep=crcr]{%
0.717918515644694	0.220301617642194\\
0.719937964392406	0.235025718280693\\
};

\addplot[area legend, draw=none, fill=mycolor170, forget plot]
table[row sep=crcr] {%
x	y\\
0.680485200875315	0.209490801972538\\
0.719937964392406	0.235025718280693\\
0.757434469241489	0.206696146732526\\
0.724828611062426	0.369686937921498\\
}--cycle;
\addplot [color=mycolor171, line width=1.5pt, forget plot]
  table[row sep=crcr]{%
0.715548171159751	0.20570819865813\\
0.717918515644694	0.220301617642194\\
};
\addplot [color=mycolor172, line width=1.5pt, forget plot]
  table[row sep=crcr]{%
0.712831831490152	0.191254578495445\\
0.715548171159751	0.20570819865813\\
};
\addplot [color=mycolor173, line width=1.5pt, forget plot]
  table[row sep=crcr]{%
0.709774647521856	0.176949706745537\\
0.712831831490152	0.191254578495445\\
};
\addplot [color=mycolor174, line width=1.5pt, forget plot]
  table[row sep=crcr]{%
0.706382014886215	0.162802358284616\\
0.709774647521856	0.176949706745537\\
};
\addplot [color=mycolor175, line width=1.5pt, forget plot]
  table[row sep=crcr]{%
0.702659568162677	0.148821126319198\\
0.706382014886215	0.162802358284616\\
};
\addplot [color=mycolor176, line width=1.5pt, forget plot]
  table[row sep=crcr]{%
0.698613174878335	0.135014415623032\\
0.702659568162677	0.148821126319198\\
};
\addplot [color=mycolor177, line width=1.5pt, forget plot]
  table[row sep=crcr]{%
0.694248929310623	0.121390435971488\\
0.698613174878335	0.135014415623032\\
};
\addplot [color=mycolor178, line width=1.5pt, forget plot]
  table[row sep=crcr]{%
0.689573146099686	0.107957195779271\\
0.694248929310623	0.121390435971488\\
};
\addplot [color=mycolor179, line width=1.5pt, forget plot]
  table[row sep=crcr]{%
0.684592353677102	0.0947224959471047\\
0.689573146099686	0.107957195779271\\
};
\addplot [color=mycolor180, line width=1.5pt, forget plot]
  table[row sep=crcr]{%
0.679313287517823	0.0816939239227776\\
0.684592353677102	0.0947224959471047\\
};
\addplot [color=mycolor181, line width=1.5pt, forget plot]
  table[row sep=crcr]{%
0.673742883222326	0.0688788479817597\\
0.679313287517823	0.0816939239227776\\
};
\addplot [color=mycolor182, line width=1.5pt, forget plot]
  table[row sep=crcr]{%
0.667888269436185	0.0562844117323311\\
0.673742883222326	0.0688788479817597\\
};
\addplot [color=mycolor183, line width=1.5pt, forget plot]
  table[row sep=crcr]{%
0.661756760614348	0.0439175288499597\\
0.667888269436185	0.0562844117323311\\
};
\addplot [color=mycolor184, line width=1.5pt, forget plot]
  table[row sep=crcr]{%
0.655355849637604	0.0317848780453986\\
0.661756760614348	0.0439175288499597\\
};
\addplot [color=mycolor185, line width=1.5pt, forget plot]
  table[row sep=crcr]{%
0.648693200288811	0.019892898270742\\
0.655355849637604	0.0317848780453986\\
};
\addplot [color=mycolor186, line width=1.5pt, forget plot]
  table[row sep=crcr]{%
0.641776639596618	0.00824778416744165\\
0.648693200288811	0.019892898270742\\
};
\addplot [color=mycolor187, line width=1.5pt, forget plot]
  table[row sep=crcr]{%
0.634614150054488	-0.00314451823999418\\
0.641776639596618	0.00824778416744165\\
};
\addplot [color=mycolor188, line width=1.5pt, forget plot]
  table[row sep=crcr]{%
0.627213861722984	-0.0142783156011266\\
0.634614150054488	-0.00314451823999418\\
};
\addplot [color=mycolor189, line width=1.5pt, forget plot]
  table[row sep=crcr]{%
0.619584044223325	-0.0251481710443079\\
0.627213861722984	-0.0142783156011266\\
};

\addplot[area legend, draw=none, fill=mycolor189, forget plot]
table[row sep=crcr] {%
x	y\\
0.580447387717656	-0.0189096662798147\\
0.627213861722984	-0.0142783156011265\\
0.647572782993374	-0.0566347416154408\\
0.69323274356033	0.103191126131379\\
}--cycle;
\addplot [color=mycolor190, line width=1.5pt, forget plot]
  table[row sep=crcr]{%
0.611733098630366	-0.0357489077260493\\
0.619584044223325	-0.0251481710443079\\
};
\addplot [color=mycolor191, line width=1.5pt, forget plot]
  table[row sep=crcr]{%
0.603669549273176	-0.0460756121223141\\
0.611733098630366	-0.0357489077260493\\
};
\addplot [color=mycolor192, line width=1.5pt, forget plot]
  table[row sep=crcr]{%
0.595402035451534	-0.0561236370593131\\
0.603669549273176	-0.0460756121223141\\
};
\addplot [color=mycolor193, line width=1.5pt, forget plot]
  table[row sep=crcr]{%
0.586939303076644	-0.0658886044816366\\
0.595402035451534	-0.0561236370593131\\
};
\addplot [color=mycolor194, line width=1.5pt, forget plot]
  table[row sep=crcr]{%
0.578290196244506	-0.0753664079558142\\
0.586939303076644	-0.0658886044816366\\
};
\addplot [color=mycolor195, line width=1.5pt, forget plot]
  table[row sep=crcr]{%
0.569463648750363	-0.0845532149076785\\
0.578290196244506	-0.0753664079558142\\
};
\addplot [color=mycolor196, line width=1.5pt, forget plot]
  table[row sep=crcr]{%
0.560468675552731	-0.0934454685921894\\
0.569463648750363	-0.0845532149076785\\
};
\addplot [color=mycolor197, line width=1.5pt, forget plot]
  table[row sep=crcr]{%
0.551314364195506	-0.102039889794608\\
0.560468675552731	-0.0934454685921894\\
};
\addplot [color=mycolor198, line width=1.5pt, forget plot]
  table[row sep=crcr]{%
0.542009866196698	-0.110333478262244\\
0.551314364195506	-0.102039889794608\\
};
\addplot [color=mycolor200, line width=1.5pt, forget plot]
  table[row sep=crcr]{%
0	0\\
0.00237853874415817	0.0365644563222274\\
0.00947407971887648	0.07251261737133\\
0.0211670271894913	0.107238573597533\\
0.0372602957112342	0.140157015791846\\
0.0574826320181432	0.170713100527005\\
0.0814931870175079	0.198391802095412\\
0.108887260828536	0.222726593316318\\
0.139203124032154	0.243307308896375\\
0.171929800159218	0.259787058805903\\
0.206515678242634	0.271888075145203\\
0.242377810268097	0.27940639395141\\
0.278911736814097	0.282215293033551\\
0.315501675269171	0.280267427890703\\
0.351530898903097	0.273595629712101\\
0.38639213185185	0.262312352008755\\
0.419497784807907	0.246607775203575\\
0.450289858892402	0.226746601127244\\
0.47824935077845	0.203063591448827\\
0.502905000541411	0.175957925241209\\
0.523841234790238	0.145886470785039\\
0.540705171197681	0.11335608501549\\
0.553212566367284	0.0789150704052904\\
0.561152606785309	0.0431439332790067\\
0.564391462105564	0.00664559932798381\\
0.562874540876154	-0.0299647487556984\\
0.556627410687597	-0.0660700402386724\\
0.545755367233058	-0.101061717129393\\
0.542009866196698	-0.110333478262244\\
};
\addplot [color=mycolor200, line width=3.0pt, only marks, mark size=0.8pt, mark=*, mark options={solid, mycolor200}, forget plot]
  table[row sep=crcr]{%
0	0\\
0.542009866196698	-0.110333478262244\\
};
\addplot [color=mycolor201, line width=1.5pt, forget plot]
  table[row sep=crcr]{%
0	0\\
0.0022158566983056	0.0432581507502345\\
0.00884023616302865	0.0860634687303619\\
0.0198037933748254	0.127967860950881\\
0.0349917600998327	0.168532665453092\\
0.054245146301513	0.20733324329346\\
0.0773624044734897	0.243963423732856\\
0.104101539470873	0.278039756097699\\
0.134182641754024	0.30920552380384\\
0.167290817526365	0.337134478523724\\
0.203079485093073	0.361534255406971\\
0.241174002933858	0.382149433603285\\
0.281175591510567	0.398764210049637\\
0.322665507755519	0.41120465853209\\
0.365209428541325	0.419340550374027\\
0.408361997245367	0.423086717691518\\
0.451671485814725	0.422403944945058\\
0.494684523528339	0.417299379454738\\
0.536950842954933	0.407826456581524\\
0.578027993425269	0.394084340357847\\
0.61748597267733	0.376216885423058\\
0.654911728189615	0.354411131130385\\
0.689913481081883	0.328895343589306\\
0.72212482732002	0.299936626139553\\
0.724477166926653	0.297574892690243\\
};
\addplot [color=mycolor201, line width=3.0pt, only marks, mark size=0.8pt, mark=*, mark options={solid, mycolor201}, forget plot]
  table[row sep=crcr]{%
0	0\\
0.724477166926653	0.297574892690243\\
};
\addplot [color=mycolor202, line width=1.5pt, forget plot]
  table[row sep=crcr]{%
0	0\\
0.00236756868655486	0.0632744545482202\\
0.00945703524215302	0.126195050924859\\
0.0212287522946962	0.18840990981439\\
0.0376168872927363	0.249571098691569\\
0.0585297906696778	0.309336577611303\\
0.0838505083879491	0.367372112044653\\
0.11343743599677	0.423353142063562\\
0.147125110545741	0.476966597421005\\
0.184725135925535	0.527912648375867\\
0.226027236460772	0.575906382471145\\
0.270800432862942	0.620679397888228\\
0.318794333966926	0.661981304466509\\
0.369740537027172	0.699581123993986\\
0.42335412874245	0.733268581937807\\
0.479335278614815	0.76285528339085\\
0.524998084936599	0.783207056183072\\
};
\addplot [color=mycolor202, line width=3.0pt, only marks, mark size=0.8pt, mark=*, mark options={solid, mycolor202}, forget plot]
  table[row sep=crcr]{%
0	0\\
0.524998084936599	0.783207056183072\\
};
\addplot [color=mycolor203, line width=1.5pt, forget plot]
  table[row sep=crcr]{%
0	0\\
0	0.999999999999996\\
};
\addplot [color=mycolor203, line width=3.0pt, only marks, mark size=0.8pt, mark=*, mark options={solid, mycolor203}, forget plot]
  table[row sep=crcr]{%
0	0\\
0	0.999999999999996\\
};
\addplot [color=mycolor204, line width=1.5pt, forget plot]
  table[row sep=crcr]{%
0	0\\
-0.00236756868655486	0.0632744545482202\\
-0.00945703524215302	0.126195050924859\\
-0.0212287522946962	0.18840990981439\\
-0.0376168872927363	0.249571098691569\\
-0.0585297906696778	0.309336577611303\\
-0.0838505083879491	0.367372112044653\\
-0.11343743599677	0.423353142063562\\
-0.147125110545741	0.476966597421005\\
-0.184725135925535	0.527912648375867\\
-0.226027236460772	0.575906382471145\\
-0.270800432862942	0.620679397888228\\
-0.318794333966926	0.661981304466509\\
-0.369740537027172	0.699581123993986\\
-0.42335412874245	0.733268581937807\\
-0.479335278614815	0.76285528339085\\
-0.524998084936599	0.783207056183072\\
};
\addplot [color=mycolor204, line width=3.0pt, only marks, mark size=0.8pt, mark=*, mark options={solid, mycolor204}, forget plot]
  table[row sep=crcr]{%
0	0\\
-0.524998084936599	0.783207056183072\\
};
\addplot [color=mycolor205, line width=1.5pt, forget plot]
  table[row sep=crcr]{%
0	0\\
-0.0022158566983056	0.0432581507502345\\
-0.00884023616302865	0.0860634687303619\\
-0.0198037933748254	0.127967860950881\\
-0.0349917600998327	0.168532665453092\\
-0.054245146301513	0.20733324329346\\
-0.0773624044734897	0.243963423732856\\
-0.104101539470873	0.278039756097699\\
-0.134182641754024	0.30920552380384\\
-0.167290817526365	0.337134478523724\\
-0.203079485093073	0.361534255406971\\
-0.241174002933858	0.382149433603285\\
-0.281175591510567	0.398764210049637\\
-0.322665507755519	0.41120465853209\\
-0.365209428541325	0.419340550374027\\
-0.408361997245367	0.423086717691518\\
-0.451671485814725	0.422403944945058\\
-0.494684523528339	0.417299379454738\\
-0.536950842954933	0.407826456581524\\
-0.578027993425269	0.394084340357847\\
-0.61748597267733	0.376216885423058\\
-0.654911728189615	0.354411131130385\\
-0.689913481081883	0.328895343589306\\
-0.72212482732002	0.299936626139553\\
-0.724477166926653	0.297574892690243\\
};
\addplot [color=mycolor205, line width=3.0pt, only marks, mark size=0.8pt, mark=*, mark options={solid, mycolor205}, forget plot]
  table[row sep=crcr]{%
0	0\\
-0.724477166926653	0.297574892690243\\
};
\addplot [color=mycolor206, line width=1.5pt, forget plot]
  table[row sep=crcr]{%
0	0\\
-0.00237853874415817	0.0365644563222274\\
-0.00947407971887648	0.07251261737133\\
-0.0211670271894913	0.107238573597533\\
-0.0372602957112342	0.140157015791846\\
-0.0574826320181432	0.170713100527005\\
-0.0814931870175079	0.198391802095412\\
-0.108887260828536	0.222726593316318\\
-0.139203124032154	0.243307308896375\\
-0.171929800159218	0.259787058805903\\
-0.206515678242634	0.271888075145203\\
-0.242377810268097	0.27940639395141\\
-0.278911736814097	0.282215293033551\\
-0.315501675269171	0.280267427890703\\
-0.351530898903097	0.273595629712101\\
-0.38639213185185	0.262312352008755\\
-0.419497784807907	0.246607775203575\\
-0.450289858892402	0.226746601127244\\
-0.47824935077845	0.203063591448827\\
-0.502905000541411	0.175957925241209\\
-0.523841234790238	0.145886470785039\\
-0.540705171197681	0.11335608501549\\
-0.553212566367284	0.0789150704052904\\
-0.561152606785309	0.0431439332790067\\
-0.564391462105564	0.00664559932798381\\
-0.562874540876154	-0.0299647487556984\\
-0.556627410687597	-0.0660700402386724\\
-0.545755367233058	-0.101061717129393\\
-0.542009866196698	-0.110333478262244\\
};
\addplot [color=mycolor206, line width=3.0pt, only marks, mark size=0.8pt, mark=*, mark options={solid, mycolor206}, forget plot]
  table[row sep=crcr]{%
0	0\\
-0.542009866196698	-0.110333478262244\\
};
\addplot [color=black, line width=3.0pt, forget plot]
  table[row sep=crcr]{%
-0.25	0\\
0.25	0\\
};

\addplot[area legend, draw=none, fill=mycolor207, forget plot]
table[row sep=crcr] {%
x	y\\
-0.25	0\\
-0.25	-0.1\\
0.25	-0.1\\
0.25	0\\
}--cycle;
\node[right, align=left]
at (axis cs:-0.15,-0.35) {\small (b)};
\end{axis}

\begin{axis}[%
width=0.302\textwidth,
height=0.264\textwidth,
at={(0.673\textwidth,0\textwidth)},
scale only axis,
xmin=-1,
xmax=1,
xtick={-1,-0.5,0,0.5,1},
xticklabels={\empty},
ymin=-0.5,
ymax=1.25,
ytick={-0.5,0,0.5,1},
yticklabels={\empty},
axis line style={draw=none},
ticks=none,
axis x line*=bottom,
axis y line*=left
]
\addplot [color=mycolor2, dashed, line width=1.5pt, forget plot]
  table[row sep=crcr]{%
-0.403439798477933	-0.0392041657339158\\
-0.393232167048833	-0.0370504627483919\\
};
\addplot [color=mycolor3, dashed, line width=1.5pt, forget plot]
  table[row sep=crcr]{%
-0.41399972851392	-0.0408128979794126\\
-0.403439798477933	-0.0392041657339158\\
};
\addplot [color=mycolor4, dashed, line width=1.5pt, forget plot]
  table[row sep=crcr]{%
-0.424880403613673	-0.0418533712059382\\
-0.41399972851392	-0.0408128979794126\\
};
\addplot [color=mycolor5, dashed, line width=1.5pt, forget plot]
  table[row sep=crcr]{%
-0.436049019899179	-0.0423045150277248\\
-0.424880403613673	-0.0418533712059382\\
};
\addplot [color=mycolor6, dashed, line width=1.5pt, forget plot]
  table[row sep=crcr]{%
-0.447471693612574	-0.0421475255033336\\
-0.436049019899179	-0.0423045150277248\\
};
\addplot [color=mycolor7, dashed, line width=1.5pt, forget plot]
  table[row sep=crcr]{%
-0.459113631184928	-0.0413659005108951\\
-0.447471693612574	-0.0421475255033336\\
};
\addplot [color=mycolor8, dashed, line width=1.5pt, forget plot]
  table[row sep=crcr]{%
-0.470939297956461	-0.0399454623381632\\
-0.459113631184928	-0.0413659005108951\\
};
\addplot [color=mycolor9, dashed, line width=1.5pt, forget plot]
  table[row sep=crcr]{%
-0.482912584646416	-0.0378743679392679\\
-0.470939297956461	-0.0399454623381632\\
};
\addplot [color=mycolor10, dashed, line width=1.5pt, forget plot]
  table[row sep=crcr]{%
-0.494996970733852	-0.0351431073642371\\
-0.482912584646416	-0.0378743679392679\\
};
\addplot [color=mycolor11, dashed, line width=1.5pt, forget plot]
  table[row sep=crcr]{%
-0.507155683975761	-0.0317444909147072\\
-0.494996970733852	-0.0351431073642372\\
};
\addplot [color=mycolor12, dashed, line width=1.5pt, forget plot]
  table[row sep=crcr]{%
-0.519351855355404	-0.0276736256197841\\
-0.507155683975761	-0.0317444909147072\\
};
\addplot [color=mycolor13, dashed, line width=1.5pt, forget plot]
  table[row sep=crcr]{%
-0.531548668821005	-0.0229278816599364\\
-0.519351855355404	-0.0276736256197841\\
};
\addplot [color=mycolor14, dashed, line width=1.5pt, forget plot]
  table[row sep=crcr]{%
-0.543709505242376	-0.0175068493943393\\
-0.531548668821005	-0.0229278816599364\\
};
\addplot [color=mycolor15, dashed, line width=1.5pt, forget plot]
  table[row sep=crcr]{%
-0.555798080080269	-0.0114122876683828\\
-0.543709505242376	-0.0175068493943393\\
};
\addplot [color=mycolor16, dashed, line width=1.5pt, forget plot]
  table[row sep=crcr]{%
-0.567778574329541	-0.00464806409349416\\
-0.555798080080269	-0.0114122876683828\\
};
\addplot [color=mycolor17, dashed, line width=1.5pt, forget plot]
  table[row sep=crcr]{%
-0.579615758362443	0.00277991199881467\\
-0.567778574329541	-0.00464806409349416\\
};
\addplot [color=mycolor18, dashed, line width=1.5pt, forget plot]
  table[row sep=crcr]{%
-0.591275108361836	0.0108637632222622\\
-0.579615758362443	0.00277991199881467\\
};
\addplot [color=mycolor19, dashed, line width=1.5pt, forget plot]
  table[row sep=crcr]{%
-0.602722915095701	0.0195937237145619\\
-0.591275108361836	0.0108637632222622\\
};

\addplot[area legend, draw=none, fill=mycolor19, forget plot]
table[row sep=crcr] {%
x	y\\
-0.594056402445217	0.0577766282298671\\
-0.591275108361836	0.0108637632222623\\
-0.634402323816952	-0.00780695682512741\\
-0.476503834515596	-0.0597415991782757\\
}--cycle;
\addplot [color=mycolor20, dashed, line width=1.5pt, forget plot]
  table[row sep=crcr]{%
-0.613926384843658	0.0289582239288165\\
-0.602722915095701	0.0195937237145619\\
};
\addplot [color=mycolor21, dashed, line width=1.5pt, forget plot]
  table[row sep=crcr]{%
-0.624853732342893	0.0389439799460869\\
-0.613926384843658	0.0289582239288165\\
};
\addplot [color=mycolor22, dashed, line width=1.5pt, forget plot]
  table[row sep=crcr]{%
-0.635474265674948	0.0495360866309359\\
-0.624853732342893	0.0389439799460869\\
};
\addplot [color=mycolor23, dashed, line width=1.5pt, forget plot]
  table[row sep=crcr]{%
-0.645758463065791	0.0607181139685224\\
-0.635474265674948	0.0495360866309359\\
};
\addplot [color=mycolor24, dashed, line width=1.5pt, forget plot]
  table[row sep=crcr]{%
-0.655678041619557	0.0724722059417484\\
-0.645758463065791	0.0607181139685224\\
};
\addplot [color=mycolor25, dashed, line width=1.5pt, forget plot]
  table[row sep=crcr]{%
-0.665206018050991	0.0847791813296985\\
-0.655678041619557	0.0724722059417484\\
};
\addplot [color=mycolor26, dashed, line width=1.5pt, forget plot]
  table[row sep=crcr]{%
-0.674316761523067	0.0976186358337697\\
-0.665206018050991	0.0847791813296985\\
};
\addplot [color=mycolor27, dashed, line width=1.5pt, forget plot]
  table[row sep=crcr]{%
-0.682986038734199	0.110969044965155\\
-0.674316761523067	0.0976186358337697\\
};
\addplot [color=mycolor28, dashed, line width=1.5pt, forget plot]
  table[row sep=crcr]{%
-0.691191051434201	0.124807867156317\\
-0.682986038734199	0.110969044965155\\
};
\addplot [color=mycolor29, dashed, line width=1.5pt, forget plot]
  table[row sep=crcr]{%
-0.698910466579312	0.139111646589578\\
-0.691191051434201	0.124807867156317\\
};
\addplot [color=mycolor30, dashed, line width=1.5pt, forget plot]
  table[row sep=crcr]{%
-0.70612443936461	0.153856115267451\\
-0.698910466579312	0.139111646589578\\
};
\addplot [color=mycolor31, dashed, line width=1.5pt, forget plot]
  table[row sep=crcr]{%
-0.712814629396678	0.169016293881762\\
-0.70612443936461	0.153856115267451\\
};
\addplot [color=mycolor32, dashed, line width=1.5pt, forget plot]
  table[row sep=crcr]{%
-0.718964210290872	0.184566591071505\\
-0.712814629396678	0.169016293881762\\
};
\addplot [color=mycolor33, dashed, line width=1.5pt, forget plot]
  table[row sep=crcr]{%
-0.724557872995703	0.200480900692624\\
-0.718964210290872	0.184566591071505\\
};
\addplot [color=mycolor34, dashed, line width=1.5pt, forget plot]
  table[row sep=crcr]{%
-0.729581823162181	0.216732696756156\\
-0.724557872995703	0.200480900692624\\
};
\addplot [color=mycolor35, dashed, line width=1.5pt, forget plot]
  table[row sep=crcr]{%
-0.734023772888208	0.233295125724282\\
-0.729581823162181	0.216732696756156\\
};
\addplot [color=mycolor36, dashed, line width=1.5pt, forget plot]
  table[row sep=crcr]{%
-0.737872927177751	0.250141095886536\\
-0.734023772888208	0.233295125724282\\
};
\addplot [color=mycolor37, dashed, line width=1.5pt, forget plot]
  table[row sep=crcr]{%
-0.741119965461353	0.267243363570551\\
-0.737872927177751	0.250141095886536\\
};
\addplot [color=mycolor38, dashed, line width=1.5pt, forget plot]
  table[row sep=crcr]{%
-0.743757018529044	0.284574615973195\\
-0.741119965461353	0.267243363570551\\
};

\addplot[area legend, draw=none, fill=mycolor38, forget plot]
table[row sep=crcr] {%
x	y\\
-0.71051994731275	0.302911155614707\\
-0.741119965461353	0.267243363570551\\
-0.785107158828874	0.283786619587681\\
-0.707652027414058	0.136715743417334\\
}--cycle;
\addplot [color=mycolor39, dashed, line width=1.5pt, forget plot]
  table[row sep=crcr]{%
-0.745777641228751	0.30210755042843\\
-0.743757018529044	0.284574615973195\\
};
\addplot [color=mycolor40, dashed, line width=1.5pt, forget plot]
  table[row sep=crcr]{%
-0.747176781283267	0.319814949957786\\
-0.745777641228751	0.30210755042843\\
};
\addplot [color=mycolor41, dashed, line width=1.5pt, forget plot]
  table[row sep=crcr]{%
-0.747950744576728	0.337669754977761\\
-0.747176781283267	0.319814949957786\\
};
\addplot [color=mycolor42, dashed, line width=1.5pt, forget plot]
  table[row sep=crcr]{%
-0.748097157257582	0.355645131065591\\
-0.747950744576728	0.337669754977761\\
};
\addplot [color=mycolor43, dashed, line width=1.5pt, forget plot]
  table[row sep=crcr]{%
-0.747614924999429	0.373714532710811\\
-0.748097157257582	0.355645131065591\\
};
\addplot [color=mycolor44, dashed, line width=1.5pt, forget plot]
  table[row sep=crcr]{%
-0.746504189753923	0.391851763004455\\
-0.747614924999429	0.373714532710811\\
};
\addplot [color=mycolor45, dashed, line width=1.5pt, forget plot]
  table[row sep=crcr]{%
-0.744766284321383	0.410031029240921\\
-0.746504189753923	0.391851763004455\\
};
\addplot [color=mycolor46, dashed, line width=1.5pt, forget plot]
  table[row sep=crcr]{%
-0.742403685055021	0.428226994429158\\
-0.744766284321383	0.410031029240921\\
};
\addplot [color=mycolor47, dashed, line width=1.5pt, forget plot]
  table[row sep=crcr]{%
-0.73941996300385	0.446414824730003\\
-0.742403685055021	0.428226994429158\\
};
\addplot [color=mycolor48, dashed, line width=1.5pt, forget plot]
  table[row sep=crcr]{%
-0.735819733787622	0.464570232855212\\
-0.73941996300385	0.446414824730003\\
};
\addplot [color=mycolor49, dashed, line width=1.5pt, forget plot]
  table[row sep=crcr]{%
-0.731608606484601	0.482669517480908\\
-0.735819733787622	0.464570232855212\\
};
\addplot [color=mycolor50, dashed, line width=1.5pt, forget plot]
  table[row sep=crcr]{%
-0.726793131799801	0.500689598743916\\
-0.731608606484601	0.482669517480908\\
};
\addplot [color=mycolor51, dashed, line width=1.5pt, forget plot]
  table[row sep=crcr]{%
-0.721380749767672	0.518608049903681\\
-0.726793131799801	0.500689598743916\\
};
\addplot [color=mycolor52, dashed, line width=1.5pt, forget plot]
  table[row sep=crcr]{%
-0.715379737229088	0.536403125265317\\
-0.721380749767672	0.518608049903681\\
};
\addplot [color=mycolor53, dashed, line width=1.5pt, forget plot]
  table[row sep=crcr]{%
-0.70879915530814	0.554053784470735\\
-0.715379737229088	0.536403125265317\\
};
\addplot [color=mycolor54, dashed, line width=1.5pt, forget plot]
  table[row sep=crcr]{%
-0.701648797099699	0.57153971327486\\
-0.70879915530814	0.554053784470735\\
};
\addplot [color=mycolor55, dashed, line width=1.5pt, forget plot]
  table[row sep=crcr]{%
-0.693939135764078	0.588841340932653\\
-0.701648797099699	0.57153971327486\\
};
\addplot [color=mycolor56, dashed, line width=1.5pt, forget plot]
  table[row sep=crcr]{%
-0.685681273210491	0.605939854330151\\
-0.693939135764078	0.588841340932653\\
};
\addplot [color=mycolor57, dashed, line width=1.5pt, forget plot]
  table[row sep=crcr]{%
-0.676886889536508	0.622817208998981\\
-0.685681273210491	0.605939854330151\\
};

\addplot[area legend, draw=none, fill=mycolor57, forget plot]
table[row sep=crcr] {%
x	y\\
-0.639927449232391	0.61667018943121\\
-0.685681273210491	0.605939854330151\\
-0.711424172116742	0.645257225248137\\
-0.735708585890114	0.480820589282536\\
}--cycle;
\addplot [color=mycolor58, dashed, line width=1.5pt, forget plot]
  table[row sep=crcr]{%
-0.667568193376342	0.639456137158839\\
-0.676886889536508	0.622817208998981\\
};
\addplot [color=mycolor59, dashed, line width=1.5pt, forget plot]
  table[row sep=crcr]{%
-0.657737873296699	0.655840152936504\\
-0.667568193376342	0.639456137158839\\
};
\addplot [color=mycolor60, dashed, line width=1.5pt, forget plot]
  table[row sep=crcr]{%
-0.64740905036508	0.671953554912815\\
-0.657737873296699	0.655840152936504\\
};
\addplot [color=mycolor61, dashed, line width=1.5pt, forget plot]
  table[row sep=crcr]{%
-0.636595232002019	0.6877814261511\\
-0.64740905036508	0.671953554912815\\
};
\addplot [color=mycolor62, dashed, line width=1.5pt, forget plot]
  table[row sep=crcr]{%
-0.625310267215643	0.703309631861531\\
-0.636595232002019	0.6877814261511\\
};
\addplot [color=mycolor63, dashed, line width=1.5pt, forget plot]
  table[row sep=crcr]{%
-0.61356830330436	0.718524814856196\\
-0.625310267215643	0.703309631861531\\
};
\addplot [color=mycolor64, dashed, line width=1.5pt, forget plot]
  table[row sep=crcr]{%
-0.60138374410132	0.733414388949031\\
-0.61356830330436	0.718524814856196\\
};
\addplot [color=mycolor65, dashed, line width=1.5pt, forget plot]
  table[row sep=crcr]{%
-0.58877120982273	0.747966530453526\\
-0.60138374410132	0.733414388949031\\
};
\addplot [color=mycolor66, dashed, line width=1.5pt, forget plot]
  table[row sep=crcr]{%
-0.57574549857096	0.762170167929198\\
-0.58877120982273	0.747966530453526\\
};
\addplot [color=mycolor67, dashed, line width=1.5pt, forget plot]
  table[row sep=crcr]{%
-0.562321549532874	0.776014970325251\\
-0.57574549857096	0.762170167929198\\
};
\addplot [color=mycolor68, dashed, line width=1.5pt, forget plot]
  table[row sep=crcr]{%
-0.548514407903875	0.789491333666839\\
-0.562321549532874	0.776014970325251\\
};
\addplot [color=mycolor69, dashed, line width=1.5pt, forget plot]
  table[row sep=crcr]{%
-0.534339191558703	0.802590366425774\\
-0.548514407903875	0.789491333666839\\
};
\addplot [color=mycolor70, dashed, line width=1.5pt, forget plot]
  table[row sep=crcr]{%
-0.519811059481271	0.815303873713584\\
-0.534339191558703	0.802590366425774\\
};
\addplot [color=mycolor71, dashed, line width=1.5pt, forget plot]
  table[row sep=crcr]{%
-0.504945181957582	0.827624340430509\\
-0.519811059481271	0.815303873713584\\
};
\addplot [color=mycolor72, dashed, line width=1.5pt, forget plot]
  table[row sep=crcr]{%
-0.489756712528083	0.83954491349939\\
-0.504945181957582	0.827624340430509\\
};
\addplot [color=mycolor73, dashed, line width=1.5pt, forget plot]
  table[row sep=crcr]{%
-0.474260761688798	0.851059383308477\\
-0.489756712528083	0.83954491349939\\
};
\addplot [color=mycolor74, dashed, line width=1.5pt, forget plot]
  table[row sep=crcr]{%
-0.458472372324034	0.862162164482135\\
-0.474260761688798	0.851059383308477\\
};
\addplot [color=mycolor75, dashed, line width=1.5pt, forget plot]
  table[row sep=crcr]{%
-0.442406496847489	0.87284827609305\\
-0.458472372324034	0.862162164482135\\
};
\addplot [color=mycolor76, dashed, line width=1.5pt, forget plot]
  table[row sep=crcr]{%
-0.426077976023236	0.883113321424163\\
-0.442406496847489	0.87284827609305\\
};

\addplot[area legend, draw=none, fill=mycolor76, forget plot]
table[row sep=crcr] {%
x	y\\
-0.39749422398008	0.859012082762392\\
-0.442406496847489	0.87284827609305\\
-0.444778703894106	0.919783605363526\\
-0.548756661399473	0.790100436243505\\
}--cycle;
\addplot [color=mycolor77, dashed, line width=1.5pt, forget plot]
  table[row sep=crcr]{%
-0.409501519433089	0.892953467383034\\
-0.426077976023236	0.883113321424163\\
};
\addplot [color=mycolor78, dashed, line width=1.5pt, forget plot]
  table[row sep=crcr]{%
-0.392691687552561	0.902365423665742\\
-0.409501519433089	0.892953467383034\\
};
\addplot [color=mycolor79, dashed, line width=1.5pt, forget plot]
  table[row sep=crcr]{%
-0.375662875393679	0.911346421761811\\
-0.392691687552561	0.902365423665742\\
};
\addplot [color=mycolor80, dashed, line width=1.5pt, forget plot]
  table[row sep=crcr]{%
-0.358429297669535	0.919894193886073\\
-0.375662875393679	0.911346421761811\\
};
\addplot [color=mycolor81, dashed, line width=1.5pt, forget plot]
  table[row sep=crcr]{%
-0.341004975432494	0.928006951917778\\
-0.358429297669535	0.919894193886073\\
};
\addplot [color=mycolor82, dashed, line width=1.5pt, forget plot]
  table[row sep=crcr]{%
-0.323403724135423	0.93568336642176\\
-0.341004975432494	0.928006951917778\\
};
\addplot [color=mycolor83, dashed, line width=1.5pt, forget plot]
  table[row sep=crcr]{%
-0.305639143063211	0.942922545821011\\
-0.323403724135423	0.93568336642176\\
};
\addplot [color=mycolor84, dashed, line width=1.5pt, forget plot]
  table[row sep=crcr]{%
-0.287724606080086	0.949724015784685\\
-0.305639143063211	0.942922545821011\\
};
\addplot [color=mycolor85, dashed, line width=1.5pt, forget plot]
  table[row sep=crcr]{%
-0.26967325363687	0.956087698890323\\
-0.287724606080086	0.949724015784685\\
};
\addplot [color=mycolor86, dashed, line width=1.5pt, forget plot]
  table[row sep=crcr]{%
-0.251497985981294	0.96201389461402\\
-0.26967325363687	0.956087698890323\\
};
\addplot [color=mycolor87, dashed, line width=1.5pt, forget plot]
  table[row sep=crcr]{%
-0.233211457513733	0.96750325969727\\
-0.251497985981294	0.96201389461402\\
};
\addplot [color=mycolor88, dashed, line width=1.5pt, forget plot]
  table[row sep=crcr]{%
-0.214826072230375	0.972556788934478\\
-0.233211457513733	0.96750325969727\\
};
\addplot [color=mycolor89, dashed, line width=1.5pt, forget plot]
  table[row sep=crcr]{%
-0.196353980195601	0.977175796420488\\
-0.214826072230375	0.972556788934478\\
};
\addplot [color=mycolor90, dashed, line width=1.5pt, forget plot]
  table[row sep=crcr]{%
-0.177807074985557	0.981361897293057\\
-0.196353980195601	0.977175796420488\\
};
\addplot [color=mycolor91, dashed, line width=1.5pt, forget plot]
  table[row sep=crcr]{%
-0.159196992045151	0.985116990000907\\
-0.177807074985557	0.981361897293057\\
};
\addplot [color=mycolor92, dashed, line width=1.5pt, forget plot]
  table[row sep=crcr]{%
-0.140535107901323	0.988443239123978\\
-0.159196992045151	0.985116990000907\\
};
\addplot [color=mycolor93, dashed, line width=1.5pt, forget plot]
  table[row sep=crcr]{%
-0.121832540176149	0.991343058768561\\
-0.140535107901323	0.988443239123978\\
};
\addplot [color=mycolor94, dashed, line width=1.5pt, forget plot]
  table[row sep=crcr]{%
-0.103100148344248	0.993819096556363\\
-0.121832540176149	0.991343058768561\\
};
\addplot [color=mycolor95, dashed, line width=1.5pt, forget plot]
  table[row sep=crcr]{%
-0.0843485351800605	0.995874218223028\\
-0.103100148344248	0.993819096556363\\
};

\addplot[area legend, draw=none, fill=mycolor95, forget plot]
table[row sep=crcr] {%
x	y\\
-0.0687433871283057	0.961754077318091\\
-0.103100148344248	0.993819096556363\\
-0.0847316391405903	1.03707589220323\\
-0.234913324393247	0.965839655534865\\
}--cycle;
\addplot [color=mycolor96, dashed, line width=1.5pt, forget plot]
  table[row sep=crcr]{%
-0.0655880488417482	0.997511492838365\\
-0.0843485351800605	0.995874218223028\\
};
\addplot [color=mycolor97, dashed, line width=1.5pt, forget plot]
  table[row sep=crcr]{%
-0.0468287855397956	0.998734178657425\\
-0.0655880488417482	0.997511492838365\\
};
\addplot [color=mycolor98, dashed, line width=1.5pt, forget plot]
  table[row sep=crcr]{%
-0.0280805927398396	0.999545709608665\\
-0.0468287855397956	0.998734178657425\\
};
\addplot [color=mycolor99, dashed, line width=1.5pt, forget plot]
  table[row sep=crcr]{%
-0.00935307285075715	0.999949682422693\\
-0.0280805927398396	0.999545709608665\\
};
\addplot [color=mycolor100, dashed, line width=1.5pt, forget plot]
  table[row sep=crcr]{%
0.00934441264935082	0.99994984440262\\
-0.00935307285075715	0.999949682422693\\
};
\addplot [color=mycolor101, dashed, line width=1.5pt, forget plot]
  table[row sep=crcr]{%
0.028002738694987	0.999550081834595\\
0.00934441264935082	0.99994984440262\\
};
\addplot [color=white!1!mycolor100, dashed, line width=1.5pt, forget plot]
  table[row sep=crcr]{%
0.0466130117668974	0.998754409035007\\
0.028002738694987	0.999550081834595\\
};
\addplot [color=mycolor102, dashed, line width=1.5pt, forget plot]
  table[row sep=crcr]{%
0.065166564719773	0.997566958028781\\
0.0466130117668974	0.998754409035007\\
};
\addplot [color=mycolor103, dashed, line width=1.5pt, forget plot]
  table[row sep=crcr]{%
0.0836549513444937	0.995991968851392\\
0.065166564719773	0.997566958028781\\
};
\addplot [color=mycolor104, dashed, line width=1.5pt, forget plot]
  table[row sep=crcr]{%
0.102069940703602	0.994033780465489\\
0.0836549513444937	0.995991968851392\\
};
\addplot [color=mycolor105, dashed, line width=1.5pt, forget plot]
  table[row sep=crcr]{%
0.120403511276893	0.991696822281572\\
0.102069940703602	0.994033780465489\\
};
\addplot [color=mycolor106, dashed, line width=1.5pt, forget plot]
  table[row sep=crcr]{%
0.138647844952207	0.988985606270697\\
0.120403511276893	0.991696822281572\\
};
\addplot [color=mycolor107, dashed, line width=1.5pt, forget plot]
  table[row sep=crcr]{%
0.156795320894732	0.985904719656025\\
0.138647844952207	0.988985606270697\\
};
\addplot [color=mycolor108, dashed, line width=1.5pt, forget plot]
  table[row sep=crcr]{%
0.174838509326315	0.982458818168873\\
0.156795320894732	0.985904719656025\\
};
\addplot [color=mycolor109, dashed, line width=1.5pt, forget plot]
  table[row sep=crcr]{%
0.192770165244568	0.97865261985396\\
0.174838509326315	0.982458818168873\\
};
\addplot [color=mycolor110, dashed, line width=1.5pt, forget plot]
  table[row sep=crcr]{%
0.210583222109788	0.974490899407729\\
0.192770165244568	0.97865261985396\\
};
\addplot [color=mycolor111, dashed, line width=1.5pt, forget plot]
  table[row sep=crcr]{%
0.228270785526025	0.969978483032787\\
0.210583222109788	0.974490899407729\\
};
\addplot [color=mycolor112, dashed, line width=1.5pt, forget plot]
  table[row sep=crcr]{%
0.245826126940985	0.965120243790981\\
0.228270785526025	0.969978483032787\\
};
\addplot [color=mycolor113, dashed, line width=1.5pt, forget plot]
  table[row sep=crcr]{%
0.263242677387796	0.959921097437005\\
0.245826126940985	0.965120243790981\\
};

\addplot[area legend, draw=none, fill=mycolor113, forget plot]
table[row sep=crcr] {%
x	y\\
0.265015946222635	0.922221486979884\\
0.245826126940985	0.965120243790981\\
0.279565356271115	0.997834413568142\\
0.113503505411535	0.990581711375822\\
}--cycle;
\addplot [color=mycolor114, dashed, line width=1.5pt, forget plot]
  table[row sep=crcr]{%
0.280514021290123	0.954385998714044\\
0.263242677387796	0.959921097437005\\
};
\addplot [color=mycolor115, dashed, line width=1.5pt, forget plot]
  table[row sep=crcr]{%
0.297633890350559	0.948519938092589\\
0.280514021290123	0.954385998714044\\
};
\addplot [color=mycolor116, dashed, line width=1.5pt, forget plot]
  table[row sep=crcr]{%
0.314596157540745	0.942327938933268\\
0.297633890350559	0.948519938092589\\
};
\addplot [color=mycolor117, dashed, line width=1.5pt, forget plot]
  table[row sep=crcr]{%
0.331394831210219	0.93581505505441\\
0.314596157540745	0.942327938933268\\
};
\addplot [color=mycolor118, dashed, line width=1.5pt, forget plot]
  table[row sep=crcr]{%
0.348024049329635	0.928986368684806\\
0.331394831210219	0.93581505505441\\
};
\addplot [color=mycolor119, dashed, line width=1.5pt, forget plot]
  table[row sep=crcr]{%
0.364478073882635	0.921846988782226\\
0.348024049329635	0.928986368684806\\
};
\addplot [color=mycolor120, dashed, line width=1.5pt, forget plot]
  table[row sep=crcr]{%
0.380751285419387	0.914402049698136\\
0.364478073882635	0.921846988782226\\
};
\addplot [color=mycolor121, dashed, line width=1.5pt, forget plot]
  table[row sep=crcr]{%
0.396838177783585	0.906656710169129\\
0.380751285419387	0.914402049698136\\
};
\addplot [color=mycolor122, dashed, line width=1.5pt, forget plot]
  table[row sep=crcr]{%
0.412733353023509	0.898616152615717\\
0.396838177783585	0.906656710169129\\
};
\addplot [color=mycolor123, dashed, line width=1.5pt, forget plot]
  table[row sep=crcr]{%
0.428431516496667	0.890285582729276\\
0.412733353023509	0.898616152615717\\
};
\addplot [color=mycolor124, dashed, line width=1.5pt, forget plot]
  table[row sep=crcr]{%
0.443927472176425	0.881670229328041\\
0.428431516496667	0.890285582729276\\
};
\addplot [color=mycolor125, dashed, line width=1.5pt, forget plot]
  table[row sep=crcr]{%
0.459216118168059	0.872775344463399\\
0.443927472176425	0.881670229328041\\
};
\addplot [color=mycolor126, dashed, line width=1.5pt, forget plot]
  table[row sep=crcr]{%
0.474292442440693	0.863606203757846\\
0.459216118168059	0.872775344463399\\
};
\addplot [color=mycolor127, dashed, line width=1.5pt, forget plot]
  table[row sep=crcr]{%
0.489151518780658	0.854168106956348\\
0.474292442440693	0.863606203757846\\
};
\addplot [color=mycolor128, dashed, line width=1.5pt, forget plot]
  table[row sep=crcr]{%
0.503788502970985	0.844466378673093\\
0.489151518780658	0.854168106956348\\
};
\addplot [color=mycolor129, dashed, line width=1.5pt, forget plot]
  table[row sep=crcr]{%
0.518198629200901	0.834506369315985\\
0.503788502970985	0.844466378673093\\
};
\addplot [color=mycolor130, dashed, line width=1.5pt, forget plot]
  table[row sep=crcr]{%
0.532377206708474	0.824293456171525\\
0.518198629200901	0.834506369315985\\
};
\addplot [color=mycolor131, dashed, line width=1.5pt, forget plot]
  table[row sep=crcr]{%
0.546319616658781	0.813833044633135\\
0.532377206708474	0.824293456171525\\
};
\addplot [color=mycolor132, dashed, line width=1.5pt, forget plot]
  table[row sep=crcr]{%
0.560021309259379	0.803130569556311\\
0.546319616658781	0.813833044633135\\
};

\addplot[area legend, draw=none, fill=mycolor132, forget plot]
table[row sep=crcr] {%
x	y\\
0.548556325796407	0.766891063078948\\
0.546319616658781	0.813833044633135\\
0.589660591501515	0.832002040193747\\
0.432375406707882	0.885765509617073\\
}--cycle;
\addplot [color=mycolor133, dashed, line width=1.5pt, forget plot]
  table[row sep=crcr]{%
0.573477801114171	0.792191496724393\\
0.560021309259379	0.803130569556311\\
};
\addplot [color=mycolor134, dashed, line width=1.5pt, forget plot]
  table[row sep=crcr]{%
0.586684672816199	0.781021324409098\\
0.573477801114171	0.792191496724393\\
};
\addplot [color=mycolor135, dashed, line width=1.5pt, forget plot]
  table[row sep=crcr]{%
0.599637566779363	0.76962558501042\\
0.586684672816199	0.781021324409098\\
};
\addplot [color=mycolor136, dashed, line width=1.5pt, forget plot]
  table[row sep=crcr]{%
0.612332185308545	0.758009846760812\\
0.599637566779363	0.76962558501042\\
};
\addplot [color=mycolor137, dashed, line width=1.5pt, forget plot]
  table[row sep=crcr]{%
0.624764288907151	0.746179715479048\\
0.612332185308545	0.758009846760812\\
};
\addplot [color=mycolor138, dashed, line width=1.5pt, forget plot]
  table[row sep=crcr]{%
0.636929694820665	0.734140836359511\\
0.624764288907151	0.746179715479048\\
};
\addplot [color=mycolor139, dashed, line width=1.5pt, forget plot]
  table[row sep=crcr]{%
0.648824275814405	0.721898895783091\\
0.636929694820665	0.734140836359511\\
};
\addplot [color=mycolor140, dashed, line width=1.5pt, forget plot]
  table[row sep=crcr]{%
0.660443959183303	0.709459623136254\\
0.648824275814405	0.721898895783091\\
};
\addplot [color=mycolor141, dashed, line width=1.5pt, forget plot]
  table[row sep=crcr]{%
0.671784725991197	0.696828792625259\\
0.660443959183303	0.709459623136254\\
};
\addplot [color=mycolor142, dashed, line width=1.5pt, forget plot]
  table[row sep=crcr]{%
0.68284261053683	0.684012225072905\\
0.671784725991197	0.696828792625259\\
};
\addplot [color=mycolor143, dashed, line width=1.5pt, forget plot]
  table[row sep=crcr]{%
0.69361370004347	0.67101578968555\\
0.68284261053683	0.684012225072905\\
};
\addplot [color=mycolor144, dashed, line width=1.5pt, forget plot]
  table[row sep=crcr]{%
0.704094134568787	0.657845405778595\\
0.69361370004347	0.67101578968555\\
};
\addplot [color=mycolor145, dashed, line width=1.5pt, forget plot]
  table[row sep=crcr]{%
0.714280107131459	0.644507044448946\\
0.704094134568787	0.657845405778595\\
};
\addplot [color=mycolor146, dashed, line width=1.5pt, forget plot]
  table[row sep=crcr]{%
0.72416786405071	0.631006730183405\\
0.714280107131459	0.644507044448946\\
};
\addplot [color=mycolor147, dashed, line width=1.5pt, forget plot]
  table[row sep=crcr]{%
0.733753705494843	0.61735054239229\\
0.72416786405071	0.631006730183405\\
};
\addplot [color=mycolor148, dashed, line width=1.5pt, forget plot]
  table[row sep=crcr]{%
0.743033986234652	0.603544616857962\\
0.733753705494843	0.61735054239229\\
};
\addplot [color=mycolor149, dashed, line width=1.5pt, forget plot]
  table[row sep=crcr]{%
0.752005116597482	0.58959514708833\\
0.743033986234652	0.603544616857962\\
};
\addplot [color=mycolor150, dashed, line width=1.5pt, forget plot]
  table[row sep=crcr]{%
0.760663563617544	0.575508385565732\\
0.752005116597482	0.58959514708833\\
};
\addplot [color=mycolor151, dashed, line width=1.5pt, forget plot]
  table[row sep=crcr]{%
0.769005852378022	0.561290644881975\\
0.760663563617544	0.575508385565732\\
};

\addplot[area legend, draw=none, fill=mycolor151, forget plot]
table[row sep=crcr] {%
x	y\\
0.745459586040897	0.531040514401836\\
0.760663563617544	0.575508385565732\\
0.807649502002473	0.576443315556525\\
0.68120866159684	0.684340738498489\\
}--cycle;
\addplot [color=mycolor152, dashed, line width=1.5pt, forget plot]
  table[row sep=crcr]{%
0.777028567540389	0.546948298750696\\
0.769005852378022	0.561290644881975\\
};
\addplot [color=mycolor153, dashed, line width=1.5pt, forget plot]
  table[row sep=crcr]{%
0.784728355056312	0.532487782888505\\
0.777028567540389	0.546948298750696\\
};
\addplot [color=mycolor154, dashed, line width=1.5pt, forget plot]
  table[row sep=crcr]{%
0.792101924057398	0.51791559575678\\
0.784728355056312	0.532487782888505\\
};
\addplot [color=mycolor155, dashed, line width=1.5pt, forget plot]
  table[row sep=crcr]{%
0.799146048918068	0.503238299156296\\
0.792101924057398	0.51791559575678\\
};
\addplot [color=mycolor156, dashed, line width=1.5pt, forget plot]
  table[row sep=crcr]{%
0.805857571486725	0.48846251866724\\
0.799146048918068	0.503238299156296\\
};
\addplot [color=mycolor157, dashed, line width=1.5pt, forget plot]
  table[row sep=crcr]{%
0.812233403480405	0.473594943927495\\
0.805857571486725	0.48846251866724\\
};
\addplot [color=mycolor158, dashed, line width=1.5pt, forget plot]
  table[row sep=crcr]{%
0.818270529038047	0.458642328742432\\
0.812233403480405	0.473594943927495\\
};
\addplot [color=mycolor159, dashed, line width=1.5pt, forget plot]
  table[row sep=crcr]{%
0.823966007427543	0.443611491019796\\
0.818270529038047	0.458642328742432\\
};
\addplot [color=mycolor160, dashed, line width=1.5pt, forget plot]
  table[row sep=crcr]{%
0.829316975901682	0.428509312523611\\
0.823966007427543	0.443611491019796\\
};
\addplot [color=mycolor161, dashed, line width=1.5pt, forget plot]
  table[row sep=crcr]{%
0.834320652698119	0.41334273844137\\
0.829316975901682	0.428509312523611\\
};
\addplot [color=mycolor162, dashed, line width=1.5pt, forget plot]
  table[row sep=crcr]{%
0.838974340178532	0.398118776759178\\
0.834320652698119	0.41334273844137\\
};
\addplot [color=mycolor163, dashed, line width=1.5pt, forget plot]
  table[row sep=crcr]{%
0.843275428102099	0.382844497439779\\
0.838974340178532	0.398118776759178\\
};
\addplot [color=mycolor164, dashed, line width=1.5pt, forget plot]
  table[row sep=crcr]{%
0.847221397028473	0.367527031398848\\
0.843275428102099	0.382844497439779\\
};
\addplot [color=mycolor165, dashed, line width=1.5pt, forget plot]
  table[row sep=crcr]{%
0.85080982184544	0.352173569275236\\
0.847221397028473	0.367527031398848\\
};
\addplot [color=mycolor166, dashed, line width=1.5pt, forget plot]
  table[row sep=crcr]{%
0.854038375416464	0.33679135999124\\
0.85080982184544	0.352173569275236\\
};
\addplot [color=mycolor167, dashed, line width=1.5pt, forget plot]
  table[row sep=crcr]{%
0.856904832343353	0.321387709099352\\
0.854038375416464	0.33679135999124\\
};
\addplot [color=mycolor168, dashed, line width=1.5pt, forget plot]
  table[row sep=crcr]{%
0.859407072839305	0.305969976912304\\
0.856904832343353	0.321387709099352\\
};
\addplot [color=mycolor169, dashed, line width=1.5pt, forget plot]
  table[row sep=crcr]{%
0.861543086707584	0.290545576413642\\
0.859407072839305	0.305969976912304\\
};
\addplot [color=mycolor170, dashed, line width=1.5pt, forget plot]
  table[row sep=crcr]{%
0.863310977421166	0.27512197094643\\
0.861543086707584	0.290545576413642\\
};

\addplot[area legend, draw=none, fill=mycolor170, forget plot]
table[row sep=crcr] {%
x	y\\
0.829648972742323	0.256030101704883\\
0.861543086707584	0.290545576413642\\
0.904890511348218	0.272391974098273\\
0.832909810019151	0.422218268973958\\
}--cycle;
\addplot [color=mycolor171, dashed, line width=1.5pt, forget plot]
  table[row sep=crcr]{%
0.864708966298638	0.259706671678125\\
0.863310977421166	0.27512197094643\\
};
\addplot [color=mycolor172, dashed, line width=1.5pt, forget plot]
  table[row sep=crcr]{%
0.86573539677169	0.244307234840039\\
0.864708966298638	0.259706671678125\\
};
\addplot [color=mycolor173, dashed, line width=1.5pt, forget plot]
  table[row sep=crcr]{%
0.866388738739586	0.228931258740294\\
0.86573539677169	0.244307234840039\\
};
\addplot [color=mycolor174, dashed, line width=1.5pt, forget plot]
  table[row sep=crcr]{%
0.866667593005932	0.21358638054954\\
0.866388738739586	0.228931258740294\\
};
\addplot [color=mycolor175, dashed, line width=1.5pt, forget plot]
  table[row sep=crcr]{%
0.86657069579313	0.198280272859226\\
0.866667593005932	0.21358638054954\\
};
\addplot [color=mycolor176, dashed, line width=1.5pt, forget plot]
  table[row sep=crcr]{%
0.866096923329903	0.183020640012635\\
0.86657069579313	0.198280272859226\\
};
\addplot [color=mycolor177, dashed, line width=1.5pt, forget plot]
  table[row sep=crcr]{%
0.865245296507217	0.167815214209396\\
0.866096923329903	0.183020640012635\\
};
\addplot [color=mycolor178, dashed, line width=1.5pt, forget plot]
  table[row sep=crcr]{%
0.864014985597983	0.152671751384648\\
0.865245296507217	0.167815214209396\\
};
\addplot [color=mycolor179, dashed, line width=1.5pt, forget plot]
  table[row sep=crcr]{%
0.862405315035844	0.137598026864571\\
0.864014985597983	0.152671751384648\\
};
\addplot [color=mycolor180, dashed, line width=1.5pt, forget plot]
  table[row sep=crcr]{%
0.860415768248356	0.122601830800484\\
0.862405315035844	0.137598026864571\\
};
\addplot [color=mycolor181, dashed, line width=1.5pt, forget plot]
  table[row sep=crcr]{%
0.858045992539808	0.107690963384277\\
0.860415768248356	0.122601830800484\\
};
\addplot [color=mycolor182, dashed, line width=1.5pt, forget plot]
  table[row sep=crcr]{%
0.855295804018882	0.0928732298484348\\
0.858045992539808	0.107690963384277\\
};
\addplot [color=mycolor183, dashed, line width=1.5pt, forget plot]
  table[row sep=crcr]{%
0.852165192566299	0.07815643525455\\
0.855295804018882	0.0928732298484348\\
};
\addplot [color=mycolor184, dashed, line width=1.5pt, forget plot]
  table[row sep=crcr]{%
0.848654326837488	0.0635483790747149\\
0.852165192566299	0.07815643525455\\
};
\addplot [color=mycolor185, dashed, line width=1.5pt, forget plot]
  table[row sep=crcr]{%
0.844763559295265	0.0490568495708354\\
0.848654326837488	0.0635483790747149\\
};
\addplot [color=mycolor186, dashed, line width=1.5pt, forget plot]
  table[row sep=crcr]{%
0.840493431267354	0.0346896179774718\\
0.844763559295265	0.0490568495708354\\
};
\addplot [color=mycolor187, dashed, line width=1.5pt, forget plot]
  table[row sep=crcr]{%
0.83584467802351	0.0204544324944421\\
0.840493431267354	0.0346896179774718\\
};
\addplot [color=mycolor188, dashed, line width=1.5pt, forget plot]
  table[row sep=crcr]{%
0.830818233866807	0.00635901209605694\\
0.83584467802351	0.0204544324944421\\
};
\addplot [color=mycolor189, dashed, line width=1.5pt, forget plot]
  table[row sep=crcr]{%
0.82541523723352	-0.00758895983551933\\
0.830818233866807	0.00635901209605694\\
};

\addplot[area legend, draw=none, fill=mycolor189, forget plot]
table[row sep=crcr] {%
x	y\\
0.786531408083375	-0.00936452319432738\\
0.830818233866807	0.00635901209605698\\
0.860750603367538	-0.0298708893124722\\
0.866704374573561	0.136242603843341\\
}--cycle;
\addplot [color=mycolor190, dashed, line width=1.5pt, forget plot]
  table[row sep=crcr]{%
0.819637035795878	-0.021381842044635\\
0.82541523723352	-0.00758895983551933\\
};
\addplot [color=mycolor191, dashed, line width=1.5pt, forget plot]
  table[row sep=crcr]{%
0.813485191561695	-0.035012042095421\\
0.819637035795878	-0.021381842044635\\
};
\addplot [color=mycolor192, dashed, line width=1.5pt, forget plot]
  table[row sep=crcr]{%
0.806961485964735	-0.0484720232144963\\
0.813485191561695	-0.035012042095421\\
};
\addplot [color=mycolor193, dashed, line width=1.5pt, forget plot]
  table[row sep=crcr]{%
0.800067924939359	-0.0617543112729536\\
0.806961485964735	-0.0484720232144963\\
};
\addplot [color=mycolor194, dashed, line width=1.5pt, forget plot]
  table[row sep=crcr]{%
0.792806743972776	-0.0748515018967794\\
0.800067924939359	-0.0617543112729536\\
};
\addplot [color=mycolor195, dashed, line width=1.5pt, forget plot]
  table[row sep=crcr]{%
0.785180413127917	-0.0877562676942029\\
0.792806743972776	-0.0748515018967794\\
};
\addplot [color=mycolor196, dashed, line width=1.5pt, forget plot]
  table[row sep=crcr]{%
0.777191642029587	-0.100461365587747\\
0.785180413127917	-0.0877562676942029\\
};
\addplot [color=mycolor197, dashed, line width=1.5pt, forget plot]
  table[row sep=crcr]{%
0.768843384806275	-0.112959644238109\\
0.777191642029587	-0.100461365587747\\
};
\addplot [color=mycolor198, dashed, line width=1.5pt, forget plot]
  table[row sep=crcr]{%
0.760138844979552	-0.125244051546268\\
0.768843384806275	-0.112959644238109\\
};
\addplot [color=mycolor200, line width=1.5pt, forget plot]
  table[row sep=crcr]{%
0	0\\
0.00264021577995177	0.0231378214737454\\
0.0102631417989769	0.0451472285119818\\
0.0222847694430903	0.0651018309544962\\
0.0379968172620689	0.0823067542179795\\
0.0566359397938552	0.0962932745894058\\
0.0805466265238515	0.108008395257336\\
0.106225117202154	0.115069723026911\\
0.132741207093284	0.117600353937441\\
0.162624283637937	0.115401605719808\\
0.191770702534025	0.108422201394225\\
0.222616260800076	0.0958709602196777\\
0.251438939245415	0.0791750767840287\\
0.280670347428889	0.0570773779823271\\
0.312122424673883	0.027301620803161\\
0.347279823439911	-0.0127792590797687\\
0.458370210742145	-0.145635561719264\\
0.491133619620525	-0.173953806384472\\
0.521588735155465	-0.194324857333091\\
0.551477605479126	-0.209019526848055\\
0.57995644214099	-0.218369015127812\\
0.60950605117782	-0.223361830453355\\
0.639468010038559	-0.223384130600758\\
0.665710775782082	-0.218827680436184\\
0.690743063360053	-0.209740424113097\\
0.713610242223189	-0.196101620712517\\
0.731015133243047	-0.180606581482571\\
0.745222937519324	-0.162140013732459\\
0.7555248756372	-0.141245341904206\\
0.760138844979552	-0.125244051546268\\
};
\addplot [color=mycolor200, line width=3.0pt, only marks, mark size=0.8pt, mark=*, mark options={solid, mycolor200}, forget plot]
  table[row sep=crcr]{%
0	0\\
0.760138844979552	-0.125244051546268\\
};
\addplot [color=mycolor201, line width=1.5pt, forget plot]
  table[row sep=crcr]{%
0	0\\
0.00213031492208926	0.0265551777771054\\
0.00933718053628541	0.0556424375785352\\
0.0211142806560269	0.0832027181078452\\
0.0368650592179941	0.108706008476903\\
0.0582876796470185	0.134205680554551\\
0.0830657537410866	0.15646622359411\\
0.113328042574722	0.17712275339337\\
0.145924921346387	0.193866197912668\\
0.183333994028494	0.207968954503608\\
0.225251673884441	0.218879046116537\\
0.274611364327196	0.226736451746837\\
0.331106054864164	0.230958032224097\\
0.411090687884118	0.231771737115337\\
0.554391422300454	0.232478110805829\\
0.614132839506093	0.237840566680578\\
0.663328917162639	0.246663047736642\\
0.708191932038907	0.259428127923688\\
0.748271105762285	0.275837931742913\\
0.783206595731728	0.295267694154409\\
0.812808502009745	0.316865247563946\\
0.83937319318509	0.342098667917517\\
0.85023668049314	0.354734672435774\\
};
\addplot [color=mycolor201, line width=3.0pt, only marks, mark size=0.8pt, mark=*, mark options={solid, mycolor201}, forget plot]
  table[row sep=crcr]{%
0	0\\
0.85023668049314	0.354734672435774\\
};
\addplot [color=mycolor202, line width=1.5pt, forget plot]
  table[row sep=crcr]{%
0	0\\
0.00218576804371173	0.0399214423970282\\
0.00912717797930851	0.0826755157573078\\
0.0212789715553238	0.127712307687179\\
0.0389219517863875	0.174476559446967\\
0.0621620782841388	0.222461702978127\\
0.0909411003228642	0.271259962193472\\
0.127041480556769	0.323281135686182\\
0.172729526242734	0.380627697744907\\
0.23038075514427	0.445326311783739\\
0.32295019482742	0.541281081406607\\
0.438134582196551	0.661722864952192\\
0.499142565215606	0.732341143011342\\
0.545844587185883	0.793126556119589\\
0.555481102856292	0.806724620456398\\
};
\addplot [color=mycolor202, line width=3.0pt, only marks, mark size=0.8pt, mark=*, mark options={solid, mycolor202}, forget plot]
  table[row sep=crcr]{%
0	0\\
0.555481102856292	0.806724620456398\\
};
\addplot [color=mycolor203, line width=1.5pt, forget plot]
  table[row sep=crcr]{%
0	0\\
1.11022302462516e-16	0.999999999999996\\
};
\addplot [color=mycolor203, line width=3.0pt, only marks, mark size=0.8pt, mark=*, mark options={solid, mycolor203}, forget plot]
  table[row sep=crcr]{%
0	0\\
1.11022302462516e-16	0.999999999999996\\
};
\addplot [color=mycolor204, line width=1.5pt, forget plot]
  table[row sep=crcr]{%
0	0\\
-0.00244445517808556	0.046582587474534\\
-0.00996463542974668	0.0959951059326183\\
-0.0227431636431499	0.147757685585055\\
-0.0408593997477095	0.201434579004565\\
-0.0643302616934541	0.256638734810082\\
-0.0947720172137531	0.315933266611754\\
-0.130953444857747	0.375841931347293\\
-0.172899434876565	0.435978408986465\\
-0.218573988813712	0.493335496585096\\
-0.267728220677032	0.547740101725958\\
-0.320224948044985	0.598925832813246\\
-0.373385567259196	0.644447267535572\\
-0.429478075572959	0.686299838936463\\
-0.485568226767759	0.722305980350095\\
-0.541179066894698	0.752583957462027\\
-0.571314356415634	0.766825461721247\\
};
\addplot [color=mycolor204, line width=3.0pt, only marks, mark size=0.8pt, mark=*, mark options={solid, mycolor204}, forget plot]
  table[row sep=crcr]{%
0	0\\
-0.571314356415634	0.766825461721247\\
};
\addplot [color=mycolor205, line width=1.5pt, forget plot]
  table[row sep=crcr]{%
0	0\\
-0.00226422567624684	0.0332325115786813\\
-0.00954535532188494	0.0691431594933618\\
-0.0211808600039777	0.10389345972452\\
-0.0381217430135031	0.14010568393563\\
-0.0607146466078048	0.177057910222641\\
-0.0870322696376045	0.211461531054269\\
-0.11889925821416	0.245527345569079\\
-0.1538896477086	0.276379179382527\\
-0.194359706783195	0.305707628026261\\
-0.237494574064394	0.330953618757523\\
-0.279822704768816	0.350561982529285\\
-0.323831458446348	0.366029467305102\\
-0.369172949427641	0.376981708930331\\
-0.412102241818463	0.382742365745675\\
-0.455398276924607	0.383893277362487\\
-0.495239328939602	0.380560276204929\\
-0.534446073356936	0.372748302017573\\
-0.572417730579457	0.360254575151938\\
-0.605541230507726	0.344581868453558\\
-0.636421334499417	0.324858497468734\\
-0.664379262829407	0.301179041674025\\
-0.686648629646902	0.276407876655144\\
-0.705337422610355	0.248838033075218\\
-0.719883761908817	0.21887822740185\\
-0.725435111830283	0.203166818455402\\
};
\addplot [color=mycolor205, line width=3.0pt, only marks, mark size=0.8pt, mark=*, mark options={solid, mycolor205}, forget plot]
  table[row sep=crcr]{%
0	0\\
-0.725435111830283	0.203166818455402\\
};
\addplot [color=mycolor206, line width=1.5pt, forget plot]
  table[row sep=crcr]{%
0	0\\
-0.00244132122163232	0.0298699753706927\\
-0.00934885432397414	0.0590388826022058\\
-0.0215317496896024	0.0900370911682223\\
-0.0377397076804156	0.119139601464972\\
-0.0594555561029869	0.148654037594975\\
-0.0845641894905176	0.175346756725888\\
-0.115161183281524	0.201075034524389\\
-0.148529106941221	0.223093877055891\\
-0.184174121044666	0.241196440720399\\
-0.221649199484264	0.255118296148386\\
-0.260501267378047	0.264528830248949\\
-0.296889947143374	0.268859989997866\\
-0.333532366237037	0.268745241147014\\
-0.369843580839088	0.263856419651783\\
-0.401960314216581	0.25502362700157\\
-0.432549221147487	0.241848923882736\\
-0.460838034667605	0.224278898087155\\
-0.483605609182077	0.204784479856492\\
-0.503095309543871	0.182017721072385\\
-0.518572265420365	0.156358609433815\\
-0.52837062876152	0.131586286228611\\
-0.533946814929776	0.105539815629045\\
-0.534900290490844	0.0789235505684848\\
-0.53094975940109	0.0525884904153856\\
-0.523377101005397	0.0305461900166584\\
-0.512023133454969	0.0101946384309025\\
-0.497106081746473	-0.00770684198864124\\
-0.479022286453358	-0.0223981319548366\\
-0.458361621574028	-0.0331621328267624\\
-0.435912551750024	-0.039377276322536\\
-0.412653039443358	-0.040576662540678\\
-0.389723683452895	-0.0365102839040596\\
-0.383406967612843	-0.0343772331452579\\
};
\addplot [color=mycolor206, line width=3.0pt, only marks, mark size=0.8pt, mark=*, mark options={solid, mycolor206}, forget plot]
  table[row sep=crcr]{%
0	0\\
-0.383406967612843	-0.0343772331452579\\
};
\addplot [color=black, line width=3.0pt, forget plot]
  table[row sep=crcr]{%
-0.25	0\\
0.25	0\\
};

\addplot[area legend, draw=none, fill=mycolor207, forget plot]
table[row sep=crcr] {%
x	y\\
-0.25	0\\
-0.25	-0.1\\
0.25	-0.1\\
0.25	0\\
}--cycle;
\node[right, align=left]
at (axis cs:-0.15,-0.35) {\small (c)};
\end{axis}

\begin{axis}[%
width=0.999\textwidth,
height=0.297\textwidth,
at={(-0.012\textwidth,-0.03\textwidth)},
scale only axis,
xmin=0,
xmax=1,
ymin=0,
ymax=1,
axis line style={draw=none},
ticks=none,
axis x line*=bottom,
axis y line*=left
]
\draw[-{Stealth}, color=black] (axis cs:0.948,0.908) -- (axis cs:0.948,0.753);
\node[below right, align=left, draw=none]
at (rel axis cs:0.957,0.871) {$g$};
\end{axis}
\end{tikzpicture}%
%     \vspace{-8mm}
%     \caption{Stages of kinematic complexity in robotics from rigid to soft. (a) Standard 1-DOF revolute joint manipulator with analytic workspace. (b) Ideal soft manipulator deformed under uniform curvature strain (\ie, PCC model). Workspace can be analytically expressed for homogeneous deformation cases, \eg, stiffness dominates gravity, and no contact. (c) Truly underactuated soft manipulator, whose workspace can often not be derived analytically and depends highly on the initial conditions and actuation distribution.}
%     \label{fig:C0:contribTwo}
%     \vspace{-4mm}
%   \end{figure}

Another common approach for modeling soft robots is based on the Elastica theory, where the volumetric soft body is represented by a smooth backbone which captures its geometric features. The method, often referred to as \emph{"soft beam"} models, are applicable to a specific subclass of soft robots, including soft manipulators \cite{Falkenhahn2015,Marchese2016,Jones2006,Webster2010}, fins \cite{Katzschmann2019,Marchese2014}, and soft legs \cite{Drotman2021Feb,vanLaake2022Sep}. Elastica is a general mathematical framework developed by Euler in the 1800s to describe the behavior of elastic rods subjected to external forces \cite{Levien2008,Antman2005}. Early approaches include the seminal work of Chirikjian et al. \cite{Chirikjian1989,Chirikjian1991,Chirikjian1992}, presenting hyper-redundant continuum robot model represented by a spatial differentiable curve. Using a modal parametrization, their work presented inverse kinematics solutions, path planning, and grasping strategies \cite{Chirikjian1992}. Later, Mochiyama et al. \cite{Mochiyama1992,Mochiyama1999} extended their work by developing dynamic formulations that led to shape regulation controllers derived from Lyapunov stability theory \cite{Khalil2014Feb}. 

A modern approach for soft beams is the "\emph{Piecewise Constant Curvature}" model (PCC) \cite{Webster2010}. This generalizable modeling structure discretizes the one-dimensional backbone curve into finite segments of constant curvature. Hence, all strains except curvature are neglected. Following either Lagrangian or Newton-Euler formulations, the approach leads to soft-bodied formulations which are often synonymous with those of rigid robot models. The PCC approximation exhibits modeling structures that closely resemble those of classical rigid serial-link manipulators. With slight modifications involving compliance, similar to rigid robots with joint compliance \cite{DeLuca2016Jul,Lynch2017}, it enables the direct implementation of a collection of classic control approaches \cite{Katzschmann2018,DellaSantina2020,Franco2020,Franco2022May,Jones2006,Kazemipour2022May,Godage2015,Godage2016} applied to soft robots.

However, despite the dimensional advantages over FEM, the PCC approach presents some limitations. First, one such limitation is that the model introduces kinematic singularities and discontinuities for the linear velocity components \cite{Jones2006, Jones2007Apr}. These artifacts are byproducts of the bending parametrization, which can potentially lead to destabilization in closed-loop when approaching zero-curvature. Some solutions have been proposed by lifting the state parametrization \cite{DellaSantina2020Jan}, or singularity avoidance by state jumping \cite{Falkenhahn2015, Tatlicioglu2007}. Second, although PCC allows for analytic closed-form dynamics, the resulting models are highly nonlinear, complex, and large. A solution is found in augmented rigid body models, which explore PCC kinematics to describe a lumped-mass model of the continuum robot. These mitigate expensive spatial integration required for computation of the inertial forces. Although lumped-mass models are seemingly an oversimplification compared to FEM, such models have proven to be rather effective \cite{DellaSantina2020, Kazemipour2022May, Katzschmann2019, Franco2020, Franco2022}. For example, Kazemipour et al. \cite{Kazemipour2022May} propose an adaptive sliding mode control scheme, which is robust against model parameter uncertainties and unknown input disturbances. The last and perhaps largest limitation is that they are only applicable in restrictive settings where constant curvature holds. They are unfit to capture important continuum phenomena, \eg, gravitational deflection, buckling, self-contact, environmental interaction, or wave propagation.

In contrast to PCC surrogate models, Cosserat beam models have demonstrated an ability to capture a broad range of continuum phenomena \cite{Gazzola2018, Renda2017Aug, Renda2018, Boyer2010, Till2019}. Unlike PCC models, they provide truncatable models derived from continuum mechanics. They also provide a precise representation of the hyper-flexible nature of materials under large deformations and even allow for self-collision \cite{Gazzola2018}. Seminal work by Renda et al. \cite{Renda2018, Renda2020} and Boyer et al. \cite{Boyer2021} proposed computational models of Cosserat beams using Geometrically-Exact finite elements on the Lie group $\textrm{SE}(3)$ \cite{Simo1986}. While these models have gained popularity in the soft robotics community, literature on model-based control for Cosserat beam models is slowly emerging when compared to PCC, especially those that consider under-actuation and hyper-redundancy that are often identified with these systems.