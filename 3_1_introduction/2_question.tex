\section{A biometic perspective on soft robotics}
A remarkable studycase, common to soft robotics, is the tentacle of an octopus. The octopus' tentacles can move in any direction, using a virtually infinite number of Degrees of Freedom (DoF) \cite{Sumbre2001Sep,Kier1985,Kennedy2020Nov}. The exceptional dexterity of the octopus arms results from them behaving like a muscular hydrostat. In fact, these arms are composed of densely packed muscle fibers whose orientation can be grouped into transverse, longitudinal, and oblique axes \cite{Kier2007Oct}. Moreover, each flexible arm of the organism is governed by a sophisticated peripheral nervous system that runs axially along each tentacle, composed of approximately 30,000 nerve fibers. Through appropriate communication between the muscle fiber network, the embedded nervous system, and the central motor cortex, complex yet coordinated motions of bending, twisting, and extension can be facilitated. An example of the amazing behavioral motor abilities of the octopus is given in Figure \ref{fig:C0:octopus} provided by Sumbre et al. \cite{Sumbre2001Sep}. The figure presents a visual segment of the bending propagation in an octopus tentacle. The purpose of this motion is to reduce the proximinal distance between the tentacle and an object of interest, such as food particle in this case. Interestingly, the octopus also employs a simulatenous bending and twisting propegation motion to orient the bottom side of its tentacles towards the direction of the objective. This leads to an efficient tranversal wave of bending and twist with speeds of about 12.5 $\pm$ 4.7 \si{\centi \meter \per \second} \cite{Sumbre2001Sep}. Such coordination enables an octopus to orient the tentacle's suckers towards a target object thus enabling effective grasping. The octopus is known for its remarkable ability to accomplish a hierarchy of tasks, which can be attributed to hyper-redundancies in its soft arm. Specifically, these organisms possess more degrees of freedom (DoFs) than are strictly necessary to complete a given task. Consequently, additional DoFs can be assigned to subtasks that run parallel to the main task. This ability allows the organism to passively adapt to uncertainties in its environment without affecting the primary motion. These attributes are highly desirable in modern robotic systems as they enable improved robustness in unstructured environments, allow for less conservative safety requirements regarding human-robot interactions, and greatly improve environmental durability, especially during impact.

\begin{figure}
  \centering
  %% This file was created by matlab2tikz.
%
%The latest updates can be retrieved from
%  http://www.mathworks.com/matlabcentral/fileexchange/22022-matlab2tikz-matlab2tikz
%where you can also make suggestions and rate matlab2tikz.
%
\begin{tikzpicture}

\begin{axis}[%
width=0.434\textwidth,
height=0.256\textwidth,
at={(0\textwidth,0.261\textwidth)},
scale only axis,
axis on top,
xmin=0.5,
xmax=829.5,
tick align=outside,
y dir=reverse,
ymin=0.5,
ymax=489.5,
axis line style={draw=none},
ticks=none
]
\addplot [forget plot] graphics [xmin=0.5, xmax=829.5, ymin=0.5, ymax=489.5] {./fig/fig_octopus_grasp-1.png};
\node[right, align=left, inner sep=0, font=\color{white}]
at (axis cs:650,50) {\scriptsize $t = 0$ s};
\end{axis}

\begin{axis}[%
width=0.434\textwidth,
height=0.256\textwidth,
at={(0.455\textwidth,0.261\textwidth)},
scale only axis,
axis on top,
xmin=0.5,
xmax=829.5,
tick align=outside,
y dir=reverse,
ymin=0.5,
ymax=489.5,
axis line style={draw=none},
ticks=none
]
\addplot [forget plot] graphics [xmin=0.5, xmax=829.5, ymin=0.5, ymax=489.5] {./fig/fig_octopus_grasp-2.png};
\node[right, align=left, inner sep=0, font=\color{white}]
at (axis cs:600,50) {\scriptsize $t = 0.36$ s};
\end{axis}

\begin{axis}[%
width=0.434\textwidth,
height=0.256\textwidth,
at={(0\textwidth,0\textwidth)},
scale only axis,
axis on top,
xmin=0.5,
xmax=829.5,
tick align=outside,
y dir=reverse,
ymin=0.5,
ymax=489.5,
axis line style={draw=none},
ticks=none
]
\addplot [forget plot] graphics [xmin=0.5, xmax=829.5, ymin=0.5, ymax=489.5] {./fig/fig_octopus_grasp-3.png};
\node[right, align=left, inner sep=0, font=\color{white}]
at (axis cs:600,50) {\scriptsize $t = 0.68$ s};
\end{axis}

\begin{axis}[%
width=0.435\textwidth,
height=0.256\textwidth,
at={(0.454\textwidth,0\textwidth)},
scale only axis,
axis on top,
xmin=0.5,
xmax=829.5,
tick align=outside,
y dir=reverse,
ymin=0.5,
ymax=487.5,
axis line style={draw=none},
ticks=none
]
\addplot [forget plot] graphics [xmin=0.5, xmax=829.5, ymin=0.5, ymax=487.5] {./fig/fig_octopus_grasp-4.png};
\node[right, align=left, inner sep=0, font=\color{white}]
at (axis cs:600,50) {\scriptsize $t = 0.92$ s};
\end{axis}
\end{tikzpicture}%
  \includegraphics*[width = .75\textwidth]{./pdf/thesis-figure-1-1.pdf}
  \caption{Recording by Sumbre et al. \cite{Sumbre2001Sep} of an octopus extending its tentacle towards an object of interest using coordinated activation of a tightly packed network of muscle fibers. The highly flexible appendage allows for traveling bending wave propagation, while the octopus orients its suckers towards the object to ensure secure grip. \label{fig:C0:octopus}}
  \vspace{-4mm}
\end{figure}

As illustrated by Figure \ref{fig:C0:octopus}, a key observation to be made is that the success of these biological system cannot be limited to morphological design problem alone. To effectively implement biomimetic design, it is crucial to tailor the problem towards the entire embodiment of the biological system which encompasses both its design and control. In light of the multidisciplinary nature of the soft robotics community, it is crucial to clarify these fundamental aspects:

\terminology{\textbf{Design} is the process of developing mechanical structures that enable a robot to perform specific tasks within a predefined workspace. \textbf{Control}, on the other hand, refers to the process of finding control laws that steers a dynamical system towards a desired state or desired behavior.}{}

% Generally speaking, for rigid robotics, the design and control synthesis are often mutally exclusive problems. This implies  may first design the rigid robot. Furthermore, unlike rigid robotics, the synthesis of control and design are not mutually independent problems; and thus should be considered as a collective \textit{design parameter} of the system. 
\begin{figure}
  \centering
  %% This file was created by matlab2tikz.
%
%The latest updates can be retrieved from
%  http://www.mathworks.com/matlabcentral/fileexchange/22022-matlab2tikz-matlab2tikz
%where you can also make suggestions and rate matlab2tikz.
%
\begin{tikzpicture}

\begin{axis}[%
width=0.434\textwidth,
height=0.256\textwidth,
at={(0\textwidth,0.261\textwidth)},
scale only axis,
axis on top,
xmin=0.5,
xmax=829.5,
tick align=outside,
y dir=reverse,
ymin=0.5,
ymax=489.5,
axis line style={draw=none},
ticks=none
]
\addplot [forget plot] graphics [xmin=0.5, xmax=829.5, ymin=0.5, ymax=489.5] {./fig/fig_octopus_grasp-1.png};
\node[right, align=left, inner sep=0, font=\color{white}]
at (axis cs:650,50) {\scriptsize $t = 0$ s};
\end{axis}

\begin{axis}[%
width=0.434\textwidth,
height=0.256\textwidth,
at={(0.455\textwidth,0.261\textwidth)},
scale only axis,
axis on top,
xmin=0.5,
xmax=829.5,
tick align=outside,
y dir=reverse,
ymin=0.5,
ymax=489.5,
axis line style={draw=none},
ticks=none
]
\addplot [forget plot] graphics [xmin=0.5, xmax=829.5, ymin=0.5, ymax=489.5] {./fig/fig_octopus_grasp-2.png};
\node[right, align=left, inner sep=0, font=\color{white}]
at (axis cs:600,50) {\scriptsize $t = 0.36$ s};
\end{axis}

\begin{axis}[%
width=0.434\textwidth,
height=0.256\textwidth,
at={(0\textwidth,0\textwidth)},
scale only axis,
axis on top,
xmin=0.5,
xmax=829.5,
tick align=outside,
y dir=reverse,
ymin=0.5,
ymax=489.5,
axis line style={draw=none},
ticks=none
]
\addplot [forget plot] graphics [xmin=0.5, xmax=829.5, ymin=0.5, ymax=489.5] {./fig/fig_octopus_grasp-3.png};
\node[right, align=left, inner sep=0, font=\color{white}]
at (axis cs:600,50) {\scriptsize $t = 0.68$ s};
\end{axis}

\begin{axis}[%
width=0.435\textwidth,
height=0.256\textwidth,
at={(0.454\textwidth,0\textwidth)},
scale only axis,
axis on top,
xmin=0.5,
xmax=829.5,
tick align=outside,
y dir=reverse,
ymin=0.5,
ymax=487.5,
axis line style={draw=none},
ticks=none
]
\addplot [forget plot] graphics [xmin=0.5, xmax=829.5, ymin=0.5, ymax=487.5] {./fig/fig_octopus_grasp-4.png};
\node[right, align=left, inner sep=0, font=\color{white}]
at (axis cs:600,50) {\scriptsize $t = 0.92$ s};
\end{axis}
\end{tikzpicture}%
  %\includegraphics*[width = .95\textwidth]{./pdf/thesis-figure-1-octopus.pdf}
  \input{./pdf/thesis-figure-1-octopus.pdf_tex}
  \caption{A schematic representation of the control architecture of a octopus-inspired soft robots with embodied intelligence. The architecture shows how information flows between important biological components such as the body (\eg, soft deformable arm), actuators (\eg, muscle fibers network), sensors (\eg, the nervous system), and the brain (\eg, the motor cortex) that coordinates information throughout the system. \label{fig:C0:biometic} }
  \vspace{-4mm}
\end{figure}

Recognizing the significance of both design and control in soft robot biomimicry, we hypothesize a systematic deconstruction of the fundamental principles that underlie the morphological grasping behavior of the octopus. This is illustrated in Figure \ref{fig:C0:biometic}, which presents a schematic representation of the biological system $\Sigma_{\textrm{bio}}$ seen in Figure \ref{fig:C0:biometic}. 

The objective is to reduce the distance between the prey and one of the tentacles, denoted by $\mathcal{B}$. The soft body consists of a discrete bundles of muscle fibers for motion and nerve fibers for sensing. Due to physical design limitations, there is only a finite number of actuators and sensors that can be accommodated within the soft body, therefore, there exist a region called the "\textit{reachable workspace}" $\mathcal{W}$ in which the system can operate. Although the body has virtually infinite DoFs, it can only be controlled via a finite set of actuators and thus this region is finite. Part of the design problem is therefore finding an appriopriate compostion of actuators and sensor such that the systems' reachability space $\mathcal{W}$ coincides with the desired task. Given the continuum nature of soft robots, as well as its distributed actuation and sensing, such design problems are not a trivial and has thus sparked an active discipline within the field.

Next, the control problem entails finding suitable control action $\textbf{u}$ that steers the arm towards the prey. Using its sensory system, a measurement $\textbf{y}$ of the soft body can be made. Since control law inside the motor cortex must be computable, this involves encoding the virtually infinite DoFs of the soft body onto a smaller finite joint respresentation $\q$ belong to a configuration space $\mathcal{Q}$.
