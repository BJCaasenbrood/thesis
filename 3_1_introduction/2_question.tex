\begin{figure}
  \centering
  %% This file was created by matlab2tikz.
%
%The latest updates can be retrieved from
%  http://www.mathworks.com/matlabcentral/fileexchange/22022-matlab2tikz-matlab2tikz
%where you can also make suggestions and rate matlab2tikz.
%
\begin{tikzpicture}

\begin{axis}[%
width=0.434\textwidth,
height=0.256\textwidth,
at={(0\textwidth,0.261\textwidth)},
scale only axis,
axis on top,
xmin=0.5,
xmax=829.5,
tick align=outside,
y dir=reverse,
ymin=0.5,
ymax=489.5,
axis line style={draw=none},
ticks=none
]
\addplot [forget plot] graphics [xmin=0.5, xmax=829.5, ymin=0.5, ymax=489.5] {./fig/fig_octopus_grasp-1.png};
\node[right, align=left, inner sep=0, font=\color{white}]
at (axis cs:650,50) {\scriptsize $t = 0$ s};
\end{axis}

\begin{axis}[%
width=0.434\textwidth,
height=0.256\textwidth,
at={(0.455\textwidth,0.261\textwidth)},
scale only axis,
axis on top,
xmin=0.5,
xmax=829.5,
tick align=outside,
y dir=reverse,
ymin=0.5,
ymax=489.5,
axis line style={draw=none},
ticks=none
]
\addplot [forget plot] graphics [xmin=0.5, xmax=829.5, ymin=0.5, ymax=489.5] {./fig/fig_octopus_grasp-2.png};
\node[right, align=left, inner sep=0, font=\color{white}]
at (axis cs:600,50) {\scriptsize $t = 0.36$ s};
\end{axis}

\begin{axis}[%
width=0.434\textwidth,
height=0.256\textwidth,
at={(0\textwidth,0\textwidth)},
scale only axis,
axis on top,
xmin=0.5,
xmax=829.5,
tick align=outside,
y dir=reverse,
ymin=0.5,
ymax=489.5,
axis line style={draw=none},
ticks=none
]
\addplot [forget plot] graphics [xmin=0.5, xmax=829.5, ymin=0.5, ymax=489.5] {./fig/fig_octopus_grasp-3.png};
\node[right, align=left, inner sep=0, font=\color{white}]
at (axis cs:600,50) {\scriptsize $t = 0.68$ s};
\end{axis}

\begin{axis}[%
width=0.435\textwidth,
height=0.256\textwidth,
at={(0.454\textwidth,0\textwidth)},
scale only axis,
axis on top,
xmin=0.5,
xmax=829.5,
tick align=outside,
y dir=reverse,
ymin=0.5,
ymax=487.5,
axis line style={draw=none},
ticks=none
]
\addplot [forget plot] graphics [xmin=0.5, xmax=829.5, ymin=0.5, ymax=487.5] {./fig/fig_octopus_grasp-4.png};
\node[right, align=left, inner sep=0, font=\color{white}]
at (axis cs:600,50) {\scriptsize $t = 0.92$ s};
\end{axis}
\end{tikzpicture}%
  \includegraphics*[width = .75\textwidth]{./pdf/thesis-figure-1-1.pdf}
  \caption{\small Recording of an octopus extending its tentacle towards an object of interest, using coordinated activation of a tightly packed network of muscle fibers. The highly flexible appendage allows for traveling bending wave propagation, while orienting its suckers towards the object to ensure a secure grip \cite{Sumbre2001Sep}. Reproduced with permission. \label{fig:C0:octopus}}
  \vspace{-8mm}
\end{figure}

\vspace*{-3mm}
\section{Soft robotics from a biometic perspective}
The octopus is a fascinating subject of study in the field of soft robotics \cite{Renda2018, Wehner2016, Kim2019Aug, Laschi2014, Rus2015, Mochiyama1999, Chang2022}. Unlike animals with rigid skeletons, the octopus has compact arrays of muscle tissue that stiffen and soften when they move. Its eight soft appendages have virtually infinite degrees of freedom (DoF) \cite{Sumbre2001Sep, Kier1985, Kennedy2020Nov}. The exceptional dexterity of the octopus arms results from their behavior as muscular hydrostats. These arms are composed of densely packed muscle fibers whose orientation can be grouped into transverse, longitudinal, and oblique axes \cite{Kier2007Oct}. Each arm can control itself semi-independently from the central brain and is controlled by a decentralized peripheral nervous system that runs axially along each tentacle. Motor control in the octopus arises from approximately 500 million neurons in its body, two-thirds of which are distributed among its limbs, enabling semi-independent control from the central brain. When a tentacle receives a command from the central brain, such as `\textit{find food},' it gathers its own sensory and position data, processes it, and issues its own motor coordination on how to start motion by stiffening or relaxing its muscular network \cite{Kier1985, Sumbre2001Sep}. This remarkable ability to coordinate movement and control each arm independently has inspired soft robotics researchers to develop new approaches to robotic design and control. An example of the amazing behavioral motor abilities of the octopus is shown in Figure \ref{fig:C0:octopus} (obtained from \cite{Sumbre2001Sep}). The figure presents a visual of an octopus attempting to grasp a food particle, which denotes a possible inspiration for many soft robotic systems. Interestingly, the octopus exhibits a remarkable behavior of simultaneous bending and twisting propagation motion to orient the bottom side of its tentacles towards the target.
%This behavior serves not only to align the suckers for successful grasping but also to enhance sensory perception, as suckers are also highly sensitive proprioceptive sensor due dense network of nerves.

This exemplifies the remarkable ability of animals to accomplish a hierarchy of tasks, attributed to the hyper-redundancies in their morphological structure. Hyper-redundancy \cite{Chirikjian1989, Chirikjian1991, Rus2015} implies that the system possesses many additional DoFs than are strictly necessary to complete a given task. Consequently, free joints can be assigned to sub-tasks that run in parallel. This ability allows many organisms to passively adapt to their environment without affecting the primary motion. These attributes are highly sought after in modern robotic systems \cite{Spong1996, Murray1994, DeLuca2016Jul} as they enable robustness in unstructured environments and environmental durability, especially regarding impact.

As shown in Figure \ref{fig:C0:octopus}, the success of biological systems cannot be attributed to their morphological design alone; rather, it is the interplay between physical structure and coordinated motor control that enables their functionality. To effectively implement embodied intelligence in soft robotics, it is crucial to consider the entire embodiment of the bio-inspired system, which thus encompasses both design and control \cite{Rus2015, Hawkes2017}. Given the multidisciplinary nature of soft robotics, it is important to clarify the distinction between these two aspects:

\terminology{\textbf{Design} is the process of developing the structure of the soft robot that enables it to perform specific tasks within a predefined workspace. \textbf{Control}, on the other hand, refers to the process of finding control laws that steer the robotic system towards a desired static or dynamic behavior.}{}

\begin{figure}
  \centering
  %% This file was created by matlab2tikz.
%
%The latest updates can be retrieved from
%  http://www.mathworks.com/matlabcentral/fileexchange/22022-matlab2tikz-matlab2tikz
%where you can also make suggestions and rate matlab2tikz.
%
\begin{tikzpicture}

\begin{axis}[%
width=0.434\textwidth,
height=0.256\textwidth,
at={(0\textwidth,0.261\textwidth)},
scale only axis,
axis on top,
xmin=0.5,
xmax=829.5,
tick align=outside,
y dir=reverse,
ymin=0.5,
ymax=489.5,
axis line style={draw=none},
ticks=none
]
\addplot [forget plot] graphics [xmin=0.5, xmax=829.5, ymin=0.5, ymax=489.5] {./fig/fig_octopus_grasp-1.png};
\node[right, align=left, inner sep=0, font=\color{white}]
at (axis cs:650,50) {\scriptsize $t = 0$ s};
\end{axis}

\begin{axis}[%
width=0.434\textwidth,
height=0.256\textwidth,
at={(0.455\textwidth,0.261\textwidth)},
scale only axis,
axis on top,
xmin=0.5,
xmax=829.5,
tick align=outside,
y dir=reverse,
ymin=0.5,
ymax=489.5,
axis line style={draw=none},
ticks=none
]
\addplot [forget plot] graphics [xmin=0.5, xmax=829.5, ymin=0.5, ymax=489.5] {./fig/fig_octopus_grasp-2.png};
\node[right, align=left, inner sep=0, font=\color{white}]
at (axis cs:600,50) {\scriptsize $t = 0.36$ s};
\end{axis}

\begin{axis}[%
width=0.434\textwidth,
height=0.256\textwidth,
at={(0\textwidth,0\textwidth)},
scale only axis,
axis on top,
xmin=0.5,
xmax=829.5,
tick align=outside,
y dir=reverse,
ymin=0.5,
ymax=489.5,
axis line style={draw=none},
ticks=none
]
\addplot [forget plot] graphics [xmin=0.5, xmax=829.5, ymin=0.5, ymax=489.5] {./fig/fig_octopus_grasp-3.png};
\node[right, align=left, inner sep=0, font=\color{white}]
at (axis cs:600,50) {\scriptsize $t = 0.68$ s};
\end{axis}

\begin{axis}[%
width=0.435\textwidth,
height=0.256\textwidth,
at={(0.454\textwidth,0\textwidth)},
scale only axis,
axis on top,
xmin=0.5,
xmax=829.5,
tick align=outside,
y dir=reverse,
ymin=0.5,
ymax=487.5,
axis line style={draw=none},
ticks=none
]
\addplot [forget plot] graphics [xmin=0.5, xmax=829.5, ymin=0.5, ymax=487.5] {./fig/fig_octopus_grasp-4.png};
\node[right, align=left, inner sep=0, font=\color{white}]
at (axis cs:600,50) {\scriptsize $t = 0.92$ s};
\end{axis}
\end{tikzpicture}%
  %\includegraphics*[width = .95\textwidth]{./pdf/thesis-figure-1-octopus.pdf}
  \input{./pdf/thesis-figure-1-octopus.pdf_tex}
  \caption{\small A schematic representation of the control architecture of an octopus-inspired soft robot with embodied intelligence. The architecture illustrates the flow of information among significant biological components, including the body (\eg, soft, deformable arm), actuators (\eg, a network of muscle fibers), sensors, and decentralized controller (\ie, peripheral nervous system), and the brain responsible for coordination. \label{fig:C0:biometic} }
  \vspace{-3mm}
\end{figure}

Recognizing the fact that the body and brain are equal partners in supporting intelligent behavior, we hypothesize a deconstruction of the morphological behavior that underlies the grasping example in Figure \ref{fig:C0:octopus}. This is illustrated in Figure \ref{fig:C0:biometic}, which presents a schematic representation of a biological system $\Sigma_{\textrm{bio}}$. The control objective is to reduce the distance between the prey and one tentacle, denoted by the continuum body $\mathcal{B} \subset \mathbb{R}^3$. Such a soft appendage consists of discrete bundles of muscle fibers for motion and nerve fibers for sensing, which are denoted by the inputs $\uB$ (\eg, muscle activation) and outputs $\yB$ (\eg, nerve potentials) as shown in Figure \ref{fig:C0:biometic}, respectively. Due to physical design limitations, there is only a finite number of actuators and sensors that can be composed within the body. This yields a compact region called the `\emph{operational workspace}' $\mathcal{W}$ \cite{Spong1996, Murray1994, Ortega1998} in which the system can operate. Even though the body has virtually infinite DoF, it can only be controlled via a finite number of inputs $\uB$, and thus soft robots are inherently underactuated \cite{Spong1996,Russ2022}. This is further emphasized by possible input saturations, such as the inability for uniaxial extension of the muscle fiber network, which is often resolved using antagonistic design. Part of the soft robot design problem is therefore finding an appropriate composition of the inputs $\uB$ such that the system's reachability space $\mathcal{W}$ coincides with the desired task with sufficient kinematic redundancies. A variation where the structural deformations of $\mathcal{B}$ are tuned is called \textit{`optimal shape design'} \cite{Bendsoe2003}. This involves altering the shape of $\mathcal{B}$ to achieve the desired deformation but fixing $\uB$ in space and time. Given the continuum nature of soft robots, as well as their distributed actuation, sensing, and mechanical saturations, such design problems are not trivially solved \cite{Xavier2022Jun,Bern2019,Coevoet2017Feb,Tian2020May,Smith2022}.

On the other hand, control involves determining a control law for the network of actuators $\textbf{u}$ that steers the octopus' arm towards its desired goal. Regarding Figure \ref{fig:C0:biometic}, two mechanisms of closed-loop control can be observed. The first mechanism employs proprioceptive sensors, which are strain-sensing neurons responsible for controlling muscle contractions and measuring joint position. The second mechanism utilizes visual feedback to inform the soft appendage about the desired setpoint relative to the task. To coordinate motion, the octopus must possess a fundamental understanding of the relationship between the inputs and outputs of its soft arm. From an engineering perspective, the dynamics associated with the motion of the continuum body $\mathcal{B}$ can be modeled via Partial Differential Equations (PDEs) that characterize the deformation in both space and time \cite{Armanini2023,DellaSantina2021,Rus2015}. This leads to the well-known issue of infinite-dimensionality \cite{DellaSantina2020,Holzapfel2002,Mochiyama1992}, which infers that such models often lack closed-form solutions. However, this raises a fundamental question: ``\emph{How do octopi and other invertebrate animals accurately predict their continuous motion without any apparent difficulty?'' } 

A solution might be found in dimensional reduction. Assuming that the motor neurons controlling the movement have limited memory and cognitive capabilities, the octopus likely perceives its soft arm as having only a finite number of DOFs to enable online motion prediction. This may suggests that the continuum joints of the arms, processed by the decentralized controllers, are possibly represented by a reduced state $\q$ belonging to a finite-dimensional configuration space $\mathcal{Q}$. As such, biological systems may be able to unconsciously identify internal dynamics models based on a reduced representation alone, \eg, $\ddq = \fB(\q,\dq,\uB)$, that predict how their continuum appendages will evolve over time given the inputs $\uB$ and the initial conditions $\q_0$ and $\dq_0$. Notably, these dynamic models can serve as a control framework for introducing stabilizing feedback terms to the input $\uB$ that ensures convergences towards a desired setpoint. This, however, highlights the main challenge in modeling soft robots; namely $(i)$ identifying an appropriate dimensional reduction that preserves both accuracy and computational tractability \cite{DellaSantina2021}, and $(ii)$ finding control structures applicable to such reduced-order models.

In the field of soft robotics, the aforementioned paradigms have attracted a large audience from the robotics control community. Researchers explore various controllers, including model-based approaches \cite{Armanini2023,DellaSantina2021,Milana2021,Franco2020}, data-driven methods \cite{Bruder2019,Alora2022}, and machine learning techniques \cite{Thuruthel2017Oct,Thuruthel2018Nov,Kim2021Feb,Schegg2022}. The use of dynamical models, whether data-driven or physics-based, in open-loop or closed-loop systems; is essential in enabling effective control strategies that strive for superior performance and robustness. The absence of such models would pose significant challenges in developing efficient control strategies.
