%!TEX root = ../../thesis.tex
%\vspace{-3mm}
\section{Outline of the thesis}
\label{sec:intro:outline}
This thesis discusses the design, modeling and control of soft robotic systems. Including this introductory materials, the thesis consists of seven chapters. 

Chapter 2 presents a historical overview of the field of soft robotics, and is complementarity to introduction presented earlier. Chapter 3 presents the design algorithm for soft actuators that aim to solve the inverse design problem. The chapter start with a brief introduction into continuum mechanics applied to three-dimensional deformation of hyper-elastic materials, followed by its numerical implementation using finite elements. From here, numerical optimization procedures are introduced that solve the inverse design within the context of (fluidic) soft actuation. Chapter 4 follows with the second objective of the thesis, namely modeling for control. Instead of volumetric soft robotic models, lower-dimensional soft beam models are introduced that are tailored for fast and accurate model-based controllers. The chapter focuses primarily on PCC soft beam models. Chapter 5 address the limitation of the PCC model, and instead extends upon it. The chapter formulates a finite-dimensional port-Hamiltonian modeling approach for soft beams, where spatial shape functions are used to discretize the modal flexibilities of the soft robot. From here, energy-shaping controller are introduced that allow for shape control for underactuated soft robots. Chapter 6 presents the culmination of all theoretical material presented in the thesis into a concise, user-friendly, toolkit called \texttt{Sorotoki}. The chapter presents an overview of the included programming tools for the design, modeling and control of soft robots. It also presents the Data-driven Variable Strain approach that lead to efficient low-dimensional models. Finally, Chapter 7 closes the main body of the thesis by summarizing the research deliverable of prior chapters, and provides a list of recommendations that could sculp future work.\\

\textbf{Note for the reader.} Chapters 3-6 are all based on published or submitted researcher articles and can therefore be read independently. A reference to the corresponding research paper is provided at every beginning of these chapters. 
% This thesis, however, provide some minor modifications to these works, either in the context of mathematical or material improvement, or connection between other chapters. An overview of these modifications can be found as a supplementary chapter at the end of the thesis in the chapter named \textit{Modifications}. 

\vfill
% \afterpage{
%   \hspace{-10mm}
% \begin{tabular}{m{0.5cm}|c|c|}
%   & Chapter 1 & \textbf{Introduction} \\ \hline
%   \;\trot{Part \RNum{1}\;} & Chapter 2  & \Centerstack{\\ \textbf{Contribution \RNum{1}}: \\ Development of efficient algorithms, applicable to the general \\ design of soft actuators, that solve the inverse design problem: \\Given  a desired morphological motion, what is the according \\ (soft) material distribution within the  design domain to realize \\the desired joint motion or displacement?\\ \\ \textbf{Contribution \RNum{2}}:} \\ \hline
%   \;\trot{Part \RNum{2}\;} & \Centerstack{Chapter 3\\ Chapter 4\\ Chapter 5} &  \\ \hline
% \end{tabular}
% }

% \\ of soft actuators, that solve the inverse design problem: Given \\ a desired morphological motion, what is the \\according (soft) material distrubution withing the \\design domain to realize the desired joint motion?