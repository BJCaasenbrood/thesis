%!TEX root = ../../thesis.tex
\section{Soft robots: what are they?}
\label{sec:intro:history}
Biological systems have long served as a source of inspiration for roboticists in their pursuit of developing more robust and capable machines. One might marvel at the ease with which we interact with a diverse array of objects in our daily lives. In contrast, conventional (rigid) robots require precise knowledge of an object's weight, shape, and orientation in order to interact with it in a safe and reliable manner. However, the methodologies often found in nature have slowly been adopted into a new robotics discipline known as "\emph{soft robotics}". Unlike rigid robots, these machines are fashioned from less rigid materials characterized by low elasticity and their enveloping properties. An examination of common materials employed in robotics reveals a conspicuous gap in comparison to natural materials and their corresponding elastic moduli, as illustrated in Figure \ref{fig:1:1}. The absence of material diversity in robotics is believed to be a crucial missing element that could enable modern machines to achieve bio-analogous performance.

In engineering, it is common to use the Young's modulus as a measure of elasticity. While limited to homogenous materials subject to small deformations, it can still be applied to classify rigidity in robotics. To aid the reader's understanding, we have included a spectrum of different materials in Figure \ref{fig:1:1}. An observation of this spectrum reveals that biological organisms are primarily composed of low-modulus materials in the range of $10^4$ to $10^7$ \si{Pa}, such as muscle and skin tissue, with rigid materials (such as bone) being much less prevalent. In contrast, classic robotics predominantly rely on hard materials such as metals and hard plastics. Furthermore, it is worth noting that materials which undergo repeated deformation during motion possess correspondingly low elastic moduli, as opposed to the use of rigid materials in classic robotics. The concept of exploring low-elasticity materials, referred to as \emph{"soft materials"}, has fostered a new direction in robotics aimed at unifying robots and biology. Although there exist many definitions on soft materials, we propose the following description before proceeding:
%
\begin{figure}[!t]
    \centering
    %\input{./fig/fig_0_stiffness_spectrum.tex}
    %\include{./pdf/thesis-figure-1-0.pdf}
    %\includepdf[pages=-]{./pdf/thesis-figure-1-0.pdf}
    \includegraphics*[width=\textwidth]{./pdf/thesis-figure-1-0.pdf}
    \caption{Young's modulus spectrum in (\si{Pa}) of rigid and soft materials, where (\ldata{Matlab8}) are the organic (\ie, biological) materials and (\ldata{Matlab7}) inorganic materials. \label{fig:1:1}}
    \vspace{-4mm}
\end{figure}
%
\terminology{\textbf{Soft materials} are a class of homogenous materials with a Young's modulus (\ie, the modulus of elasticity) typically lower than $E \le 10^9$ \si{\pascal}. Following, the word 'soft' or 'softness' refers to the collection of mechanical properties that are often associated with these low moduli materials.}{}\
%
Now, despite the fact that the words \emph{"soft"} and \emph{"robotics"} have a clear definition independently, the collocation of the two sparked many vivid discussions and new idealogies within the robotics community for the past two decades. Throughout its young academic life, several definition have been coined. Throughout its relatively brief academic existence, various definitions have been proposed. Initially, soft robotics referred to robots with variable joint stiffness \cite{AlbuSchaffer2004} or artificial compliance achieved through control \cite{AlbuSchaffer2011}. The term was also used to underline the shift from rigid-linked robots to \textit{"bio-inspired continuum robots are inherently compliant and that exhibit large strains in normal operations"} \cite{Trivedi2008}. Paraphrasing the work of Robison et al. \cite{Robinson1999}: \textit{"soft robotic manipulators are continuum robots made of soft materials that undergo continuous elastic deformation and produce motion through the generation of a smooth backbone curve"}. Alternatively, a broader definition was coined in a review by Kim et al. \cite{Kim2013} simply referring to soft-bodied robots as \textit{"an analogy to soft-bodied animals"}. A concise (but generic) definition was proposed by Laschi et al. \cite{Laschi2014}, as soft robots being \textit{"any robot built by soft materials"}. Rus et al. \cite{Rus2015} defined soft robots in terms of their structural elasticity: \textit{"Systems that are capable of autonomous behavior, and that are primarily composed of materials with moduli in the range of that of soft biological materials"}.

\par The ongoing debate regarding the precise terminology for soft robotics may never reach a definitive conclusion. However, it is crucial to establish consistent terminology not only for the purpose of this thesis, but also proper communication towards a broader scientific community. Previous definitions coined by the scientific community have placed great emphasis on the natural motion that arises from soft materials with high similarities to nature. Our definition of \textit{"soft robots"} is based on the historical development of soft robotics and current scientific trends in literature (discussed in Chapter \ref{chap: history}) with particular focus on design and control. Given the interdisciplinary nature of the field, the terms used in this thesis may differ from those used in existing literature. We will consistently refer to the following definition when discussing "\emph{soft robotics}":
%
\terminology{\textbf{Soft robotics} is a subclass of robotics with purposefully designed compliant actuators embedded into their soft material body whose goal is to enable the robot control over its ability to perform bio-inspired behavior.}{}
%
The formulation above, modified from an early definition proposed in \cite{DellaSantina2020Springer}, emphasizes the significance of soft materials in replicating biological motion, also known as "\textit{bio-mimicry}" or "\textit{bio-mimetics}". Despite the prominent role of classical robotics is automation, robotics orginally owes its origins to bio-mimicry [????]. In response, soft robotics represents a significant advancement in the pursuit of harmonizing robotics and biological principles. The field not only explores the use of soft materials from a design perspective but also considers their implications for control in order to recover biological morphologies.

%In this section, we will present a short historical overview of soft robotics. Hereby showing that the current trends of bio-mimicry and elasticity in robotics find roots in a periods way before the soft robotic boom in the early 2010's. To guide the reader, in Figure \ref{fig:C0:timeline}, we provide a graphical, historic overview of soft robotic systems from 1960 to 2022. We will discuss the inception of soft actuation, early soft robotic designs, and modeling and control strategies for these continuum robotics.