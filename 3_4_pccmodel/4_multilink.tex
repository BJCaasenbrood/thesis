%!TEX root = ../../thesis.tex
\section{Extension to multi-link systems}
\label{sec: chap2 section header}
\noindent We previously expressed the position and velocity kinematics as explicit functions of the generalized coordinates (i.e., Bishop parameters) and their time-derivatives. This explicit dependency stems from the PCC conditions inferring the curvature is non-varying along the spatial domain $[0,L]$, \ie, $\kappa(\sigma,\q) = \kappa(\q)$. Although sufficient for some cases, the condition is generally restrictive, and to some extent inconvenient, since the inclusion of multiple links demands piece-wise integration of the kinematics \eqref{eq:C2:pos_exact}, \eqref{eq:C2:phi_rodr}, \eqref{eq:C2:vel_cont}, and \eqref{eq:C2:acceleration}. Rather than separation of integration, we can extend this PCC description by using piece-wise continuous spatial function to distinguishes multiple soft-bodied links along the continuous body of the soft robot. The idea of parametrization through shapes functions has been explored earlier by Chirikjian et al.
\cite{Chirikjian1994,Chirikjian1992}, and later by Boyer et al. \cite{Boyer2021}, Della Santina et al. \cite{DellaSantina2020}. A similar discontinuous shape function series was used by Berthet-Rayne et al. \cite{Berthet2021} to pursue multi-body dynamics for growing continuum robots; and proposed by Chirikjian \cite{Chirikjian1992} for hyper-redundant robots earlier.

Following the aforementioned works, let us parameterize the geometric strains $\vec{\Gamma}$ and $\vec{U}$ for  a multi-link soft robot with $N$ number of links  through the product of a basis of orthonormal functions $\!\{\theta_i\}_{i=1}^{N}$ and the Bishop parametrization. Contrary to \eqref{eq:C2:Gamma} and \eqref{eq:C2:U}
%
\begin{align} \vec{\Gamma}(\sigma,\q) & \simeq \sum^N_{i=1} \theta_i(\sigma) \ceil{\JB^\star}_3
\,\vec{z}_i, \label{eq:C2:theta_extent} \\ \vec{U}(\sigma,\q) & \simeq \sum^N_{i=1} \theta_i(\sigma)
\floor{\JB^\star}_3\,\vec{z}_i + \config{\vec{U}}, \label{eq:C2:h_extent} \end{align}
%
where $\vec{J}^\star$ is the joint-axis matrix as in \eqref{eq:C2:joint-axis-matrix}, the mathematical operators $\ceil{\cdot}_3$ and $\floor{\cdot}_3$  extract the first or last three rows of a matrix, respectively;  $\tilde{q}_i$ the joint variables of the $i$-th link, and $\theta_i: [0,L] \mapsto \{0,1\}$ is a piece-wise constant shape function, whose purpose is to be non-zero for a given interval on
$\Xs$.
The new generalized coordinate vector becomes the aggregate of all joint variables of the multi-body soft robotic system $\q =  \left(\vec{z}_1^\top,\,\vec{z}_2^\top,...,\,\vec{z}_N^\top \right)^\top$ with the vector $\vec{z}_i = (\varepsilon_{i},\, \kappa_{x,i},\,\kappa_{y,i})^\top$ relating to the Bishop parametrization of the $i$-th link.

Given \eqref{eq:C2:theta_extent} and \eqref{eq:C2:h_extent}, we may now rewrite the velocity-twist as
%
\begin{equation} \etaB(\sigma,\q,\dq) = \left[\Ad_{\gB}^{-1}
\int_0^\sigma \Ad_{\gB} \JB^\star \ThetaB(s) \; ds \right]\dq := \JB(\sigma,\q) \dq
\label{eq:C2:vel_vec_dis} \end{equation}
%
where $\ThetaB(\sigma) = (\theta_1,\,\theta_2,\,...,\theta_n) \otimes \vec{I}_n$ is an unitary selection matrix derived from the basis of piece-wise continuous shape functions $\!\{\theta_i\}_{i=1}^n$. Substitution of the discontinuous variation of the geometric Jacobian in \eqref{eq:C2:vel_vec_dis} into
\eqref{eq:C2:kinetic_energy} leads to the dynamic model of a $N$-link soft robot manipulator in the
Lagrangian form similar to \eqref{eq:C2:dynamic_model}.
%
\begin{example}[Piece-wise selection for two-link system]
To reduce ambiguity on the selection matrix $\ThetaB(\sigma)$, lets consider a spatial coordinate $\sigma_2 \in [L_1,L_1+L_2)$ that lies on the spatial interval of the second link. Consequently, the operation $\ThetaB(\sigma_2) \q = \vec{z}_2$ returns the corresponding joint variable of the second link. This selection of generalized coordinates follows analogously for other links along the serial-chain of the soft manipulator.
\end{example}
%
%We provided a small library of piece-wise continuous shape functions upto $1 \le N \le 8$ links under \texttt{./src/pwf} on the open \sorotoki repository \cite{Caasenbrood2021}.
